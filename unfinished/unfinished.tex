\chapter{More on sheaves}
Previously, we defined a sheaf so that we could equip
it onto a scheme and thus produce a ringed space.
However, we will now proceed to define some additional sheaves on a scheme $X$,
and study the relations between them.

\section{Sheaves of $\OO_X$-modules}
\prototype{Twisting sheaves.}
Up until now we have mostly been working with sheaves of rings,
but now we want to instead think about sheaves of modules related
to a scheme $X$. 
\begin{definition}
	Let $(X, \OO_X)$ be a ringed space.
	A \vocab{sheaf of $\OO_X$-modules} is a sheaf $\SF$
	of abelian groups such that $\SF(U)$ is also an $\OO_X(U)$-module.
	The module structures are also required to be compatible
	with the restriction maps.
\end{definition}
\begin{example}
	[Twisting sheaves]
	Let $X = \Proj S$ be a projective scheme for a graded ring $S$
	(e.g.\ a projective subscheme).
	For any integer $m$ we consider a pre-sheaf
	\[ \SF^n = \left\{ \frac fg \mid g(\pp)\neq 0 \;\forall \pp \in X 
		\text{ and } \deg f - \deg g = n \right\}.
	\]
	Intuitively, these are degree $n$ ``functions''.
	Then $\OO_X(n) = (\SF^n)\sh$;
	we call these the \vocab{twisting sheaves}.

	For example, if $S = \CC[x_0, \dots, x_n]$ then
	\begin{enumerate}[(a)]
		\ii $\OO_X(0) \cong \OO_X$, by definition.
		\ii A degree $n$ polynomial defines a global section of $\OO_X(n)$.
		\ii $\OO_X(m)$ has no nonzero global sections if $m < 0$.
		\ii Suppose $X = \CP^1$ has coordinates $(s:t)$. Then
		\[ \frac 1s \in \OO_X(-1)(U_0) \]
		is an example of a (non-global) section of $\OO_X(-1)$.
		\ii Suppose $X = \CP^2$ has coordinates $(x:y:z)$. Then
		\[ \frac{1}{xyz} \in \OO_X(-1)(U_0 \cap U_1 \cap U_2) \]
		is an example of a (non-global) section of $\OO_X(-3)$.
	\end{enumerate}
\end{example}

Observe that the stalks of such a sheaf are abelian groups.

\section{Sheaves of $\OO_X$-modules make an abelian category}
I won't take the time to do this properly,
since I won't really use much of it.
See \cite{ref:vakil} for a proper exposition.

Fix a ringed space $(X, \OO_X)$.
It's easy to see there is a category of sheaves of $\OO_X$-modules.
(the objects are sheaves and the morphisms are, well, morphisms of sheaves).
To see that it is abelian, we need to do the following:
\begin{itemize}
	\ii Identify the zero object of the category.
	This is constant sheaf giving the trivial group everywhere.
	(This more or less follows from the fact that it's 
	the zero object of the abelian category $\catname{AbGrp}$.)
	\ii Show that we can add two morphisms.
	This is done in the obvious manner.
	\ii Show that any morphism has a ``kernel sheaf'' and ``cokernel sheaf''.
	The definition is not too bad: given $\alpha : \SF \to \SG$ we define the
	pre-sheaves $\ker\alpha$ and $\coker'\alpha$ by
	\begin{align*}
		\ker(\alpha)(U) &= \ker(\alpha_U) \\
		\coker'(\alpha)(U) &= \coker(\alpha_U).
	\end{align*}
	It turns out that $\ker(\alpha)$ is already a sheaf.
	But $\coker'(\alpha)$ is only a pre-sheaf and in general not a sheaf.
	(Again, see \cite{ref:vakil} or \cite{ref:gathmann} for why.)
	Thus we have to define $\coker(\alpha) = \coker'(\alpha) \sh$.
	
	The tricky part, done in \cite{ref:vakil} and the one which I'm omitting,
	is showing that this kernel and cokernel sheaf satisfy
	the correct universal properties.
\end{itemize}

From this we deduce
\begin{theorem}
	[Sheaves form an abelian category]
	Let $(X, \OO_X)$ be a ringed space.
	The category of sheaves of $\OO_X$-modules forms an abelian category.
\end{theorem}

In particular, we can talk about whether a sequence of sheaves
$\SF \to \SG \to \SH$ is exact: just use the categorical definition.
However, if possible we want to avoid unwinding all these definitions
(take a moment and think what doing that would entail).
So, I will just quote the relevant result.
The best one is:
\begin{theorem}
	[Exactness can be checked at stalks]
	A sequence of sheaves of $\OO_X$-modules
	$\SF \to \SG \to \SH$ is exact if and only if
	for every point $p \in X$,
	the induced map of stalks $\SF_p \to \SG_p \to \SH_p$ is exact.
\end{theorem}


Thus, we \emph{really} like stalks; they behave as nicely as possible.
\begin{ques}
	Deduce that $\SF \to \SG$ is injective / surjective / an isomorphism
	if and only if $\SF_p \to \SG_p$ is injective / surjective / an isomorphism
	for every point $p$.
\end{ques}

\begin{remark}
	In fact, one can show that the section functor is \vocab{left exact}:
	if $0 \to \SF \to \SG \to \SH \to 0$ is exact then
	$0 \to \SF(U) \to \SG(U) \to \SH(U)$ is exact but the
	last arrow need not be surjective.
\end{remark}

We also have the following result:
\begin{theorem}
	[Monic/epic on sections]
	Let $\alpha : \SF \to \SG$ be a morphism of $\OO_X$-modules.
	\begin{enumerate}[(a)]
		\ii The map $\alpha$ is monic if and only if $\SF(U) \to \SG(U)$
		is injective for every open $U$.
		\ii The map $\alpha$ is epic if $\SF(U) \to \SG(U)$
		is surjective for every open $U$,
		but the converse is not true in general.
	\end{enumerate}
\end{theorem}

\begin{example}
	[Exponential exact sequence]
	Here is an example of an exact sequence of sheaves:
	we write
	\[ 0 \to \ZZ \taking{\cdot 2\pi i} \SF \taking{\exp} \SF^\ast \to 0 \]
	where
	\begin{itemize}
		\ii $\ZZ$ is the sheaf of locally constant integer-valued functions.
		\ii $\SF$ is the sheaf of holomorphic functions on $\CC$.
		\ii $\SF^\ast$ is the sheaf of nonvanishing
			holomorphic functions on $\CC$.
	\end{itemize}
	You should convince yourself that this sequence is exact
	(try taking the stalk at $0$).
	But if we pick $U = \CC \setminus \{0\}$, then we only have exactness of
	\[ 0 \to \ZZ(U) \taking{\cdot 2\pi i} \SF(U) \taking{\exp} \SF^\ast(U) \]
	with the last arrow not being surjective:
	take the identity function $\id \in \SF^\ast(U)$.
\end{example}
\begin{exercise}
	Compute $\coker'(\exp)$ and $\coker(\exp) = \coker'(\exp)\sh$
	in the above example.
\end{exercise}

In addition, we can do the following constructions.
\begin{definition}
	If $\SF$ and $\SG$ are sheaves of $\OO_X$-modules, then
	\begin{enumerate}[(a)]
		\ii The \vocab{direct sum} $\SF \oplus \SG$ is defined
		on open sets by \[ U \mapsto \SF(U) \oplus \SG(U). \]
		(This is already a sheaf, so we don't need to sheafify.)
		\ii The \vocab{tensor product} $\SF \otimes \SG$ is defined
		by taking the sheafification of the pre-sheaf
		\[ U \mapsto \SF(U) \otimes_{\OO_X(U)} \SG(U). \]
		\ii The \vocab{dual} $\SF^\vee$ of $\SF$ is defined
		by taking the sheafification of the pre-sheaf
		\[ U \mapsto \Hom_{\OO_X(U) \text{ module}} (\SF(U), \OO_X(U)). \]
	\end{enumerate}
\end{definition}
\begin{example}[More on twisting sheaves]
	Consider the twisting sheaves of $X = \CP^N$.
	\begin{enumerate}[(a)]
		\ii $\OO_X(n) \otimes \OO_X(m) \cong \OO_X(m+n)$.
		What do you think the isomorphism is?
		\ii $\OO_X(n)^\vee \cong \OO_X(-n)$.
	\end{enumerate}
\end{example}

\section{Sheaves associated to modules}
As the previous section shows, sheaves are rather than complicated objects
and perhaps not the easiest things to work with.
So we want a way to simplify our work a little,
and specialize our attention to a ``simpler'' class of sheaves.

We're going to work instead with general $R$-modules.
For this to work, we have to understand ``localization of a module'',
generalizing the localization of $R$ from before.
\begin{definition}
	Let $S \subseteq R$, where $R$ is a ring,
	and assume $S$ is closed under multiplication.
	Let $M$ be an $R$-module.
	Then the \vocab{localization of $M$ at $S$}, denoted $S\inv M$,
	is defined as the set of fractions
	\[ \left\{ m/s \mid m \in M, s \in S \right\} \]
	where we declare two fractions $m_1 / s_1 = m_2 / s_2$ 
	to be equal if 
	\[ \exists s \in S : \quad s(s_2m_1 - s_1m_2) = 0. \]
	Finally, we define $M_\pp = (R\setminus\pp)\inv M$
	and $M_f = \left\{ 1, f, f^2, \dots \right\}\inv M$.
\end{definition}
Now for every module $M$ we can construct a sheaf of $\OO_X$-modules.
\begin{definition}
	Let $M$ be an $R$-module.
	Then we construct the sheaf $\wt M$ on $X = \Spec R$ as follows:
	it is the sheafication of the pre-sheaf
	\[ U \mapsto 
		\left\{ \frac{m}{g} \mid m \in M, f \in R,
			\text{ and }
			g(\pp) \neq 0 \;\forall \pp \in U \right\}.
	\]
	Thus, in terms of germs,
	\[
		\wt M(U)
		= \left\{ (m_\pp \in M_\pp)_{\pp \in U} \text{ which are locally $m/f$} \right\}.
	\]
	This becomes a sheaf of $\OO_X$ modules.
	We say $\wt M$ is the \vocab{sheaf associated to $M$}.
\end{definition}
\begin{example}
	[Examples of sheaves associated to modules]
	\listhack
	\begin{enumerate}[(a)]
		\ii If we set $M = R$ then $\wt M = \OO_X$.
		\ii If we let $M = R \oplus R$ then $\wt M = \OO_X^{\oplus 2}$.
		\ii Suppose $R = \CC[x]$, so $X = \Spec R = \Aff^1$.
		Then we define a module $M$ as follows:
		\begin{itemize}
			\ii As a set $M = \{ \lambda \in \CC \}$.
			\ii The action of $R$ on $M$ is
			given by $f \cdot \lambda = f(2016)\lambda$.
		\end{itemize}
		Then we obtain skyscraper sheaf at $(2016)$ (see \Cref{prob:module_sky})
		\[ 
			\wt M(U) 
			=
			\begin{cases}
				\CC & (2016) \in U \\
				0 & (2016) \notin U.
			\end{cases}
		\]
	\end{enumerate}
\end{example}
These sheaves have the usual properties we like:
\begin{theorem}
	[Properties of sheaves associated to modules]
	Let $X = \Spec R$ and a let $M$ be an $R$-module.
	\begin{enumerate}[(a)]
		\ii If $\pp \in X$ then $\wt M_\pp \cong M_\pp$;
		the stalk of $\wt M$ is the localization of $M$.
		\ii For $f \in R$, we have $\wt M(D(f)) = M_f$.
	\end{enumerate}
\end{theorem}
The proof is identical to \Cref{thm:affine_struct_master}.

\section{Quasi-coherent sheaves}
Then we define:
\begin{definition}
	A sheaf $\SF$ on a scheme $X$ is \vocab{quasi-coherent} if
	on any affine open set $U = \Spec R$, the restriction of $\SF$ to $U$
	is isomorphic to $\wt{M_U}$ for some module $M_U$.
\end{definition}
We like this definition because, in practice, most sheaves which occur
are quasi-coherent: which is good because modules are easier to think about
than entire sheaves. The condition is hard to check as stated,
but in fact we have the following nice result (whose proof we omit).
\begin{theorem}
	[Quasi-coherence can be checked on any particular cover]
	Let $\SF$ be a sheaf on a scheme $X$,
	and suppose there exists some \emph{particular} affine open cover
	$\{U_i\}_i$ of $X$ such that $\SF$ is isomorphic to some $\wt M_{U_i}$
	when restricted to any $U_i$.
	Then $\SF$ is quasi-coherent.
\end{theorem}

\begin{example}
	[Examples of quasi-coherent sheaves]
	\listhack
	\begin{enumerate}[(a)]
		\ii If $X$ is a scheme, then $\OO_X$ is quasi-coherent,
		almost by definition.
		\ii The twisting sheaves $\OO_X(n)$ are always quasi-coherent
		when $X$ is a projective.
	\end{enumerate}
\end{example}

Kahler differential module, show the Euler sequence

\section{Locally free sheaves}
\section\problemhead
\begin{problem}
	\label{prob:module_sky}
	Suppose $R = \CC[x]$, so $X = \Spec R = \Aff^1$.
	Then we define a $R$-module $M = \CC$ with action
	$f \cdot \lambda = f(2016)\lambda$ for $f \in R$, $\lambda \in M$.
	\begin{enumerate}[(a)]
		\ii Let $\pp \in X$.  Show that $M_\pp = \CC$
		if $\pp = (x-2016)$ and $M_\pp = 0$ otherwise.
		\ii Show that $\wt M$ is a skyscraper sheaf.
	\end{enumerate}
\end{problem}


\chapter{Divisors and line bundles}
$K(X)$ 


\chapter{grp-classify}
\section{Semidirect Products}
\prototype{$D_{12} \cong C_6 \rtimes C_2$.}
(Semidirect products are a little heavy and I won't use them other than to illustrate a point in this section.
So you might skip this section on a first read.)

Let $G$ be a group and $N$ a normal subgroup.
Let $Q = G/N$.
Is there a way we can recover $G$ as a ``product'' of $Q$ and $N$?

Our first guess would be $G \cong N \times Q$, the Cartesian product.
Unfortunately, this fails.
For example, let $G = D_{12}$, and let $N = \left\{ 1,r,\dots,r^5 \right\}$.
You can check that $N$ is indeed normal in $D_{12}$; however, $D_{12}/N \cong C_2$ while $N \cong C_6$.
And it's certainly not the case that $D_{12} \cong C_6 \times C_2$, since
$D_{12}$ is not abelian!

In fact, we have that
\[ C_{12} / C_6 \cong C_2 \text{ and } D_{12} / C_6 \cong C_2 \]
so we can't even determine $G$ uniquely.
But can we at least find all possible $G$?

You can define another type of product, called a semidirect product, as follows.
\begin{definition}
	An \vocab{automorphism} is an isomorphism from a group to itself.
	Given a group $G$, let $\Aut(G)$ be the set of automorphisms of $G$.
	Then $\Aut(G)$ forms a group, called the \vocab{automorphism group} of $G$.
\end{definition}
\begin{definition}
	Let $H = (H, \star)$ and $K = (K, \ast)$ be groups.
	Let $\phi : K \to \opname{Aut}(H)$ be a homomorphism.
	Then we define the \vocab{semidirect product} $H \rtimes_\phi K$ as follows.
	\begin{itemize}
		\ii The elements of $H \rtimes K$ are ordered pairs $(h,k) \in H \times K$.
		\ii The group operation will be
		\[ (h_1, k_1)(h_2, k_2)
			= \left( h_1 \star \phi_{k_1}(h_2), k_1 \ast k_2 \right). \]
	\end{itemize}
\end{definition}
In the above, I've let $\phi_k$ denote the image of $k \in K$ under $\phi$.
It's not hard to check that this actually makes $H \rtimes_\phi K$ into a group.
\begin{ques}
	Convince yourself that $H' = \left\{ (1,k) \mid k \in K \right\}$
	is a subgroup of $H \rtimes_\phi K$ which is isomorphic to $H$.
	Also, show that $H' \normalin H \rtimes_\phi K$.
\end{ques}
\begin{example}[Dihedral Group]
	Let $H = C_n = \left<x \mid x^n=1\right>$ and $K = C_2 = \left<y \mid y^2=1\right>$.
	Let $\phi : C_2 \to \Aut(C_n)$ by sending $x$ to the map $y \mapsto y\inv$.
	In other words, $\phi_x(y) = y\inv$.
	Then \[ D_{2n} = C_n \rtimes_\phi C_2. \]
\end{example}

Unfortunately, even this isn't enough to capture all possible values of $G$.
This only works if there's a natural way that you can view $K$ as a subgroup of $G$, since this construction also causes $K$ to appear as a subgroup of $G$.
Alas, this is not always possible.
