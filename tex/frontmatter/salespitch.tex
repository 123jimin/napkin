\chapter{Sales pitches}
\label{ch:sales}
\newcommand{\pitch}[1]{\ii[\textsf{\color{blue}\ref{#1}}.] \textsf{\color{blue} \textbf{\nameref{#1}.}} \\[1ex]} % for now. . .
\newcommand{\buzzword}[1]{\textbf{\color{green!40!black} #1}}

This chapter contains a pitch for each part,
to help you decide what you want to read
and to elaborate more on how they are interconnected.

For convenience, here is again the dependency plot
that appeared in the frontmatter.
%\chapter*{Graph of Chapter Dependencies}
%\addcontentsline{toc}{chapter}{Graph of Chapter Dependencies}

\bgroup
\renewcommand{\href}[1]{} % temp disable links
\renewcommand{\solidwidth}{0.7pt}
\renewcommand{\boldwidth}{1.5pt}

\setcounter{diagheight}{50}
\begin{chart}
\halfcourse 10,45:{Chapters 1,3}{Groups}{}
\halfcourse 48,45:{Chapter 2}{Spaces}{}
\halfcourse 48,35:{Chapters 4-5}{Topology}{}
\halfcourse 63,45:{Chapter 6}{Modules}{}
\halfcourse 63,35:{Chapters 7-9}{Lin Alg}{}
\halfcourse 5,35:{Chapter 10}{Grp Actions}{}
\halfcourse 5,20:{Chapter 11}{Grp Classif}{}
\halfcourse 30,35:{Chapter 12-13}{Ideals}{}
\reqhalfcourse 55,10:{Chapters 14-16}{Cmplx Ana}{}
\halfcourse 58,19:{Chapters 17-19}{Quantum}{}
\reqhalfcourse 30,10:{Chapters 20-22}{Alg Top 1}{}
\reqhalfcourse 30,22:{Chapters 23-25}{Cat Th}{}
\halfcourse 65,25:{Chapters 26-29}{Diff Geo}{}
\reqhalfcourse 20,0:{Chapters 30-35}{Alg Top 2}{}
\reqhalfcourse 42,10:{Chapters 36-40}{Alg NT 1}{}
\reqhalfcourse 42,0:{Chapter 41-45}{Alg NT 2}{}
\reqhalfcourse 35,45:{Chapters 46-49}{Rep Theory}{}
\reqhalfcourse 16,10:{Chapters 50-53}{Alg Geom 1}{}
\halfcourse 6,0:{Chapters 54-56}{Alg Geom 2}{}
\reqhalfcourse 65,0:{Chapters 57-63}{Set Theory}{}

\prereqc 30,35,35,45;0:   % Ideals -> Rep Th
\prereqc 63,35,35,45;0:   % Lin Alg -> Rep Th
\coreqc  30,22,35,45;50: % Cats -> Rep Th
\prereqc 63,35,58,19;30:  % Lin Alg -> Quantum
\prereqc 48,45,65,25;-20: % Spaces -> Diff Geo

\prereqc 10,45,30,22;0:   % Gp -> Cat Th
\coreqc  30,35,30,22;0:   % Ideals -> Cat Th
\coreqc  48,45,30,22;0:   % Space -> Cat Th
\coreqc  63,35,30,22;10:  % Lin Alg -> Cat Th
\prereqc 63,35,65,25;0:   % Lin Alg -> Diff Geo
\coreqc  48,35,30,22;-10: % Top -> Cat Th
\coreqc  30,22,16,10;0:   % Cat Th -> AG2
\prereqc 63,45,30,35;0:   % Module -> Ideal

\prereqc 30,10,20,0;0:    % AT1 -> AT2
\prereqc 30,22,20,0;20:   % Cat -> AT2
\coreqc  30,22,30,10;0:   % Cat -> AT1

\prereqc 10,45,5,35;0:    % Grp -> Grp Act
\prereq   5,35,5,20:      % Grp Act -> Grp Class
\prereq  48,45,48,35:     % Space -> Top
\prereqc 48,35,30,10;0:   % Top -> AT1
\prereqc  5,35,30,10;0:   % Grp Act -> AT1
\prereqc 48,35,55,10;-30: % Top -> Cmplx Ana
\coreqc  10,45,30,35;0:   % Grp -> Ideals
\prereqc 30,35,42,10;0:   % Ideals -> ANT1
\prereq  63,45,63,35:     % Modules -> Linalg
\prereq  63,45,42,10:     % Modules -> ANT1
\coreqc  63,35,42,10;0:   % Lin Alg -> ANT1
\prereqc 30,35,16,10;0:   % Ideals -> AG1
\prereqc 48,35,16,10;-30: % Top -> AG1
\prereqc 42,10,42,0;0:    % ANT1 -> ANT2
\prereqc 16,10,6,0;0:     % AG1 -> AG2
\prereqc 30,22,6,0;80:     % Cat -> AG2
\prereqc  5,35,42,0;-30:  % Grp Act -> ANT2
\end{chart}
\egroup


\section{The basics}
\begin{itemize}
\pitch{part:startout}
I made a design decision that the first part
should have a little bit both of algebra and topology:
so this first chapter begins by defining a \buzzword{group},
while the second chapter begins by defining a \buzzword{metric space}.
The intention is so that newcomers get to see two different
examples of ``sets with additional structure''
in somewhat different contexts,
and to have a minimal amount of literacy as these sorts
of definitions appear over and over.\footnote{In particular,
	I think it's easier to learn
	what a homeomorphism is after seeing group isomorphism,
	and what a homomorphism is after seeing continuous map.}

\pitch{part:absalg}
The algebraically inclined can then delve into
further types of algebraic structures:
some more details of \buzzword{groups},
and then \buzzword{rings} and \buzzword{fields} ---
which will let you generalize $\ZZ$, $\QQ$, $\RR$, $\CC$.
So you'll learn to become familiar with all sorts of other nouns
that appear in algebra, unlocking a whole host of objects
that one couldn't talk about before.

We'll also come into \buzzword{ideals},
which generalize the GCD in $\ZZ$ that you might know of.
For example, you know in $\ZZ$ that any integer
can be written in the form $3a+5b$ for $a,b \in \ZZ$,
since $\gcd(3,5)=1$.
We'll see that this statement is really
a statement of ideals: ``$(3,5)=1$ in $\ZZ$'',
and thus we'll understand in what situations
it can be generalized, e.g.\ to polynomials.

\pitch{part:basictop}
The more analytically inclined can instead move into topology,
learning more about spaces.
We'll find out that ``metric spaces'' are actually too specific,
and that it's better to work with \buzzword{topological spaces},
which are based on the so-called \buzzword{open sets}.
You'll then get to see the buddings of some geometrical ideals,
ending with the really great notion of \buzzword{compactness},
a powerful notion that makes real analysis tick.

One example of an application of compactness to tempt you now:
a continuous function $f \colon [0,1] \to \RR$
always achieves a \emph{maximum} value.
(In contrast, $f \colon (0,1) \to \RR$ by $x \mapsto 1/x$ does not.)
We'll see the reason is that $[0,1]$ is compact.
\end{itemize}

\section{Abstract algebra}
\begin{itemize}
\pitch{part:linalg}
In high school, linear algebra is often really unsatisfying.
You are given these arrays of numbers,
and they're manipulated in some ways that don't really make sense.
For example, the determinant is defined as this
funny-looking sum with a bunch of products that seems
to come out of thin air. Where does it come from?
Why does $\det(AB) = \det A \det B$ with such a bizarre formula?

Well, it turns out that you \emph{can} explain all of these things!
The trick is to not think of linear algebra
as the study of matrices,
but instead as the study of \emph{linear maps}.
In earlier chapters we saw that we got great generalizations
by speaking of ``sets with enriched structure'' and ``maps between them''.
This time, our sets are \buzzword{vector space}
and our maps are \buzzword{linear maps}.
We'll find out that a matrix is actually just
a way of writing down a linear map as an array of numbers,
but using the ``intrinsic'' definitions
we'll de-mystify all the strange formulas from high school
and show you where they all come from.

In particular, we'll see \emph{easy} proofs
that column rank equals row rank,
determinant is multiplicative, trace is the sum of the diagonal entries,
how the dot product works,
and all the words starting with ``eigen-''.
We'll even have a bonus chapter for Fourier analysis
showing that you can also explain all the big buzz-words
by just being comfortable with vector spaces.

\pitch{part:groups}
Some of you might be interested in more about groups,
and this chapter will give you a way to play further.
It starts with an exploration of \buzzword{group actions},
then goes into a bit on \buzzword{Sylow theorems},
which are the tools that let us try to \emph{classify all groups}.

\pitch{part:repth}
If $G$ is a group, we can try to understand
it by implementing it as a \emph{matrix},
i.e.\ considering embeddings $G \injto \GL_n(\CC)$.
These are called \buzzword{representations} of $G$;
it turns out that they can be decomposed into \buzzword{irreducible} ones.
Astonishingly we will find that we can
\emph{basically characterize all of them}:
the results turn out to be short and completely unexpected.

For example, we will find out that there are finitely
many irreducible representations of a given finite group $G$;
if we label them $V_1$, $V_2$, \dots, $V_r$,
then we will find that $r$ is the number
of conjugacy classes of $G$, and moreover that
\[ |G| = (\dim V_1)^2 + \dots + (\dim V_r)^2 \]
which comes out of nowhere!

The last chapter of this part will show you some
unexpected corollaries.
Here is one of them:
let $G$ be a finite group and create variables $x_g$
for each $g \in G$.
A $|G| \times |G|$ matrix $M$ is defined by setting
the $(g,h)$th entry to be the variable $x_{g \cdot h}$.
Then this determinant will turn out to \emph{factor},
and the factors will correspond to the $V_i$ we described above:
there will be an irreducible factor of degree $\dim V_i$
appearing $\dim V_i$ times.
This result, called the \buzzword{Frobenius determinant},
is said to have given birth to representation theory.

\pitch{part:quantum}
If you ever wondered what \buzzword{Shor's algorithm} is,
this chapter will use the built-up linear algebra to tell you!
\end{itemize}

\section{Real and complex analysis}
\begin{itemize}
\pitch{part:calc}
In this part, we'll use our built-up knowledge of
metric and topological spaces to give short, rigorous definitions
and theorems typical of high school calculus.
That is, we'll really define and prove most everything you've seen about
\buzzword{limits}, \buzzword{series}, \buzzword{derivatives}, and \buzzword{integrals}.

Although this might seem intimidating,
it turns out that actually, by the time we start this chapter,
\emph{the hard work has already been done}:
the notion of limits, open sets, and compactness
will make short work of what was swept under the rug in AP calculus.
Most of the proofs will thus actually be quite short,
instead, we sit back and watch all the pieces slowly come together,
the reward for our careful study of topology beforehand.

That said, if you are willing to suspend belief,
you can actually read most of the other parts
without knowing the exact details of all the calculus here,
so in some sense this part is ``optional''.

\pitch{part:cmplxana}
It turns out that \buzzword{holomorphic functions}
(complex-differentiable functions)
are close to the nicest things ever:
they turn out to be given by a Taylor series
(i.e.\ are basically polynomials).
This means we'll be able to prove unreasonably nice results
about holomorphic functions $\CC \to \CC$, like
\begin{itemize}
	\ii they are determined by just a few inputs,
	\ii their contour integrals are all zero,
	\ii they can't be bounded unless they are constant,
	\ii \dots.
\end{itemize}
We then introduce \buzzword{meromorphic functions},
which are like quotients of holomorphic functions,
and find that we can detect their zeros by simply drawing
loops in the plane and integrating over them:
the famous \buzzword{residue theorem} appears.
(In the practice problems, you will see this even gives
us a way to evaluate real integrals that can't be evaluated otherwise.)

\pitch{part:measure}
Measure theory is the upgraded version of integration.
The Riemann integration is for a lot of purposes not really sufficient;
for example, if $f$ is the function equalling $1$ at rational numbers
but $0$ at irrational numbers,
we would hope that $\int_0^1 f(x) \; dx = 0$,
but the Riemann integral is not capable of handling this function $f$.

The \buzzword{Lebesgue integral} will handle these mistakes
by assigning a \emph{measure} to a generic space $\Omega$,
making it into a \buzzword{measure space}.
This will let us develop a richer theory of integration
where the above integral \emph{does} work out to zero
because the ``rational numbers have measure zero''.
Even the development of the measure will be an achievement,
because it means we've developed a rigorous, complete way
of talking about what notions like area and volume mean ---
on any space, not just $\RR^n$!
So for example the Lebesgue integral will let us
integrate functions over any \buzzword{measure space}.

\pitch{part:prob}
Using the tools of measure theory, we'll be able to start
giving rigorous definitions of \buzzword{probability}, too.
We'll see that a \buzzword{random variable} is actually
a function from a measure space of worlds to $\RR$,
giving us a rigorous way to talk about its probabilities.
We can then start actually stating results like
the \buzzword{law of large numbers} and \buzzword{central limit theorem}
in ways that make them both easy to state and straightforward to prove.

\pitch{part:diffgeo}
Multivariable calculus is often confusing
because of all the partial derivatives.
But we'll find out that, armed with our good understanding
of linear algebra, that we're really looking at a \buzzword{total derivative}:
at every point of a function $f \colon \RR^n \to \RR$
we can associate a \emph{linear map} $Df$ which
captures in one object the notion of partial derivatives.
Set up this way, we'll get to see versions of \buzzword{differential forms}
and \buzzword{Stoke's theorem},
and we finally will know what the notation $dx$ really means.
In the end, we'll say a little bit about manifolds in general.
\end{itemize}

\section{Algebraic number theory}
\begin{itemize}
\pitch{part:algnt1}
Why is $3+\sqrt5$ the conjugate of $3-\sqrt5$?
How come the norm $\norm{a+b\sqrt5} = a^2-5b^2$ used in Pell equations
just happens to be multiplicative?
Why is it we can do factoring into primes in $\ZZ[i]$
but not in $\ZZ[\sqrt{-5}]$?
All these questions and more will be answered in this part,
when we learn about \buzzword{number fields},
a generalization of $\QQ$ and $\ZZ$ to things like $\QQ(\sqrt5)$
and $\ZZ[\sqrt{5}]$.
We'll find out that we have unique factorization into prime ideals,
that there is a real \emph{multiplicative norm} in play here,
and so on.
We'll also see that Pell's equation falls out of this theory.

\pitch{part:algnt2}
All the big buzz-words come out now:
\buzzword{Galois groups}, the \buzzword{Frobenius}, and friends.
We'll see quadratic reciprocity is just a shadow of
the behavior of the Frobenius element,
and meet the \buzzword{Chebotarev density theorem},
which generalizes greatly the Dirichlet theorem on the infinitude
of primes which are $a \pmod n$.
Towards the end, we'll also state \buzzword{Artin reciprocity},
one of the great results of \buzzword{class field theory},
and how it generalizes quadratic reciprocity and cubic reciprocity.
\end{itemize}

\section{Algebraic topology}
\begin{itemize}
\pitch{part:algtop1}
What's the difference between an annulus and disk?
Well, one of them has a ``hole'' in it,
but if we are just given intrinsic topological spaces
it's hard to make this notion precise.
The \buzzword{fundamental group} $\pi_1(X)$
and more general \buzzword{homotopy group}
will make this precise --- we'll find a way to define an abelian group
$\pi_1(X)$ for every topological space $X$ which captures the idea
there is a hole in the space, by throwing lassos into the space
and seeing if we can reel them in.

Amazingly, the fundamental group $\pi_1(X)$ will, under mild conditions,
tell you about ways to cover $X$ with a so-called
\buzzword{covering projection}.
One picture is that one can wrap a real line $\RR$ into a helix shape
and then project it down into the circle $S^1$.
This will turn out to correspond to the fact that $\pi_1(S^1) = \ZZ$
which has only one subgroup.
More generally the subgroups of $\pi_1(X)$ will be in
bijection with ways to cover the space $X$!

\pitch{part:cats}
What do fields, groups, manifolds, metric spaces, measure spaces,
modules, representations, rings, topological spaces, vector spaces,
all have in common?
Answer: they are all ``objects with additional structure'',
with maps between them.

The notion of \buzzword{category} will appropriately generalize all of them.
We'll see all sorts of constructions and ideas
can be abstracted into the framework of a category,
in which we \emph{only} think about objects and arrows between them,
without probing too hard into the details of what those objects are.
This results in drawing many \buzzword{commutative diagrams}.

For example, any way of taking an objection in one category
and getting another one (for example $\pi_1$ as above,
from the category of spaces into the category of groups)
will probably be a \buzzword{functor}.
We'll unify $G \times H$, $X \times Y$, $R \times S$,
and anything with the $\times$ symbol into the notion of a product,
and then even more generally into a \buzzword{limit}.
Towards the end, we talk about \buzzword{abelian categories}
and talk about the famous
\buzzword{snake lemma}, \buzzword{five lemma}, and so on.

\pitch{part:algtop2}
Using the language of category theory,
we then resume our adventures in algebraic topology,
in which we define the \buzzword{homology groups}
which give a different way of noticing holes in a space,
in a way that is longer to define but easier to compute in practice.
We'll then reverse the construction to get so-called
\buzzword{cohomology rings} instead,
which give us an even finer invariant for telling spaces apart.
\end{itemize}

\section{Algebraic geometry}
\begin{itemize}
\pitch{part:ag1}
We begin with a classical study of classical \buzzword{complex varieties}:
the study of intersections of polynomial equations over $\CC$.
This will naturally lead us into the geometry of rings,
giving ways to draw pictures of ideals,
and motivating \buzzword{Hilbert's nullstellensatz}.
The \buzzword{Zariski topology} will show its face,
and then we'll play with \buzzword{projective varities}
and \buzzword{quasi-projective varieties},
with a bonus detour into \buzzword{Bezout's theorem}.
All this prepares us for our journey into schemes.

\pitch{part:ag2}
We now get serious and delve into Grothendiek's definition of
an \buzzword{affine scheme}:
a generalization of our classical varieties
that lets us start with any ring $A$
and construct a space $\Spec A$ on it.
We'll equip it with its own Zariski topology
and then a sheaf of functions on it,
making it into a \buzzword{locally ringed space};
we will find that the sheaf can be understood
effectively in terms of \buzzword{localization} on it.
We'll find that the language of commutative algebra provides
elegant generalizations of what's going on geometrically:
prime ideals correspond to irreducible closed subsets,
radical ideals correspond to closed subsets,
maximal ideals correspond to closed points, and so on.
We'll draw lots of pictures of spaces and examples to accompany this.

\pitch{part:ag3}
Not yet written! Wait for v2.
\end{itemize}

\section{Set theory}
\begin{itemize}
\pitch{part:st1}
Why is \buzzword{Russell's paradox} such a big deal
and how is it resolved?
What is this \buzzword{Zorn's lemma}
that everyone keeps talking about?
In this part we'll learn the answers to these questions
by giving a real description of the \buzzword{Zermelo-Frankel}
axioms, and the \buzzword{axiom of choice},
delving into the details of how math is built axiomatically
at the very bottom foundations.
We'll meet the \buzzword{ordinal numbers} and \buzzword{cardinal numbers}
and learn how to do \buzzword{transfinite induction} with them.

\pitch{part:st2}
The \buzzword{continuum hypothesis}
states that there are no cardinalities
between the size of the natural numbers and the size of the real numbers.
It was shown to be \emph{independent} of the axioms ---
one cannot prove or disprove it.
How could a result like that possibly be proved?
Using our understanding of the ZF axioms,
we'll develop a bit of \buzzword{model theory}
and then use \buzzword{forcing} in order to show
how to construct entire models of the universe
in which the continuum hypothesis is true or false.
\end{itemize}
