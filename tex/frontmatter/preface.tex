\chapter*{Preface: Why this exists}
\addcontentsline{toc}{chapter}{Preface}
I'll be eating a quick lunch with some friends of mine who are still in high school.
They'll ask me what I've been up to the last few weeks,
and I'll tell them that I've been learning category theory.
They'll ask me what category theory is about.
I tell them it's about abstracting things by looking at just the
structure-preserving morphisms between them, rather than the objects themselves.
I'll try to give them the standard example $\mathbf{Gp}$,
but then I'll realize that they don't know what a homomorphism is.
So then I'll start trying to explain what a homomorphism is,
but then I'll remember that they haven't learned what a group is.
So then I'll start trying to explain what a group is,
but by the time I finish writing the group axioms on my napkin,
they've already forgotten why I was talking about groups in the first place.
And then it's 1PM, people need to go places, and I can't help but think:
\begin{quote}
	``Man, if I had forty hours instead of forty minutes, I bet I could actually have explained this all''.
\end{quote}
This book is my attempt at those forty hours.

\section*{Olympians}
What do you do if you're a talented high school student
who wants to learn higher math?

To my knowledge, there aren't actually that many possible things to do.
One obvious route is to try to negotiate with your school to
let you take math classes from a local college.
But this option isn't available to many people,
and even when it is, more than likely you'll still be the
best student in your class at said local college.
Plus it's a huge logistical pain in the rare cases where it's at all possible.

Then you have another very popular route --- the world of math contests.
Suddenly you're connected to this peer group of other kids who are insanely smart.
You start training for these contests,
you get really good at solving hard problems,
and before you know it you're in Lincoln, Nebraska, taking the
IMO\footnote{IMO is short for ``International Mathematical Olympiad'',
	the premier high school mathematical olympiad.
	See \url{imo-official.org} for more details.} Team Selection Test.
Finally you're feeling challenged, you have clear goals to work towards,
and a group of like-minded peers to support and encourage you.
You gradually learn all about symmedians, quadratic reciprocity, Muirhead, \dots

And yet math olympiads have a weird quirk to them:
they restrict themselves to only elementary problems,
leaving analysis, group theory, linear algebra, etc.\ all off limits.

So now suppose you're curious what category theory is all about.
Not too many of your friends know, since it's not a contest topic.
You could try taking a course from a local college, but that's just going back to square one again.
You could try chats with friends or talks or whatever, but that's like the napkin at lunch again:
I can tell you category theory is about looking at arrows instead of the objects themselves,
but if I don't tell you what a functor is, or about the Yoneda lemma,
or what a limit is, I haven't really showed you anything.

So you finally do what any sensible person would do --- search ``category theory'' on Wikipedia.
You scroll through the page, realize that you don't know what half the words are,
give up, and go back to doing geometry problems from the IMO Shortlist.

\section*{Verbum sapienti satis est}
Higher math for high school students typically comes in two flavors:
\begin{itemize}
	\ii Someone tells you about the hairy ball theorem in the form
	``you can't comb the hair on a spherical cat''
	then doesn't tell you anything about why it should be true,
	what it means to actually ``comb the hair'', 
	or any of the underlying theory,
	leaving you with just some vague notion in your head.

	\ii You take a class and prove every result in full detail,
	and at some point
	you stop caring about what the professor is saying.
\end{itemize}
Presumably you already know how unsatisfying the first approach is.
So the second approach seems to be the default,
but in general I find it to be terribly inefficient.

I was talking to a friend of mine one day who described briefly
what the Israel IMO training looked like.
It turns out that rather than actually preparing for the IMO,
the students would, say, get taught an entire semester's worth of
undergraduate algebra in the couple weeks.
Seeing as a semester is twenty weeks or so, this is an improvement
by a factor of ten.

This might sound ludicrous, but I find it totally believable.
The glaring issue with classes is that they are not designed for the top students.
Olympiad students have a huge advantage here --- push them in the right direction,
and they can figure out the rest of the details themselves
without you having to spell out all the specificities.
So it's easily possible for these top students to learn subjects
four or five or six times faster than the average student.

Along these lines, often classes like to prove things for the sake of proving them.
I personally find that many proofs don't really teach you anything,
and that it is often better to say ``you could work this out if you wanted to, but it's not worth your time''.
Unlike some of the classes you'll have to take in college,
it is not the purpose of this book to train you to solve exercises or write proofs,\footnote{%
	Which is not to say problem-solving isn't valuable;
	that's why we do contest math.
	It's just not the point of this book.}
but rather to just teach you interesting math.
Indeed, most boring non-instructive proofs fall into two categories:
\begin{enumerate}[(i)]
	\ii Short proofs (often ``verify all details'')
	which one could easily work out themselves.
	\ii Long, less easy proofs that no one remembers
	two weeks later anyways.
	(Sometimes so technical as to require lengthy detours.)
\end{enumerate}
The things that are presented should be memorable, or something worth caring about.

In particular, I place a strong emphasis over explaining why a theorem \emph{should}
be true rather than writing down its proof.
I find the former universally more enlightening.
This is a recurrent theme of this book:
\begin{moral}
	Natural explanations supersede proofs.
\end{moral}

\section*{Obligatory disclaimer}
Apparently this wasn't obvious, so let me add:

This book is not primarily intended as a replacement for standard coursework.
For space constraints it's obviously impossible to have 600 pages cover
what would otherwise be many, many semesters of higher math.
(I probably have accidentally implied this in earlier versions of this draft,
so let me set that record straight now.)

This is especially true if you plan on doing serious research
in any of the subjects covered here.
If you're just an ex-IMO medalist curious ``what is algebraic NT about?'',
then this Napkin may fit you quite well.
If you're currently taking an algebraic number theory class and are confused
about some point, then this Napkin may also be a helpful second reference.
But if you're an undergraduate at Harvard hoping that these notes
will cover the equivalent of Math 223a, you will probably be more disappointed.

\section*{Presentation of specific topics}
See \Cref{ch:refs} for some general comments on why I chose
to present certain topics in the way that I did.
At the end of said appendix are pointers to
several references for further reading.

\section*{Acknowledgements}
%% TODO: add more
I am indebted to countless people for this work.
Here is a partial (surely incomplete) list.

Thanks to all my teachers and professors for teaching me much of the
material covered in these notes,
as well as the authors of all the references I have cited here.
A special call-out to \cite{ref:55a}, \cite{ref:msci},
\cite{ref:manifolds}, \cite{ref:gathmann}, \cite{ref:18-435},
\cite{ref:etingof}, \cite{ref:145a}, which were especially influential.

Thanks also to everyone who read through preview copies of my draft,
and pointed out errors and gave other suggestions.
Special mention to Andrej Vukovi\'c and Alexander Chua for catching several hundred errors.
Thanks also to Brian Gu and Tom Tseng for many corrections.
I'd also like to express my gratitude for the many kind words I received
during the development of this project;
these generous comments led me to keep working on this.

Finally, a huge thanks to the math olympiad community.
All the enthusiasm, encouragement, and thank-you notes I have received
over the years led me to begin writing this in the first place.
I otherwise would never have the arrogance to dream a project like this
was at all possible.
And of course I would be nowhere near where I am today were it not for the
life-changing journey I took in chasing my dreams to the IMO.
Forever TWN2!
