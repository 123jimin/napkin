\chapter{More on sheaves (in progress)}
Previously, we defined a sheaf so that we could equip
it onto a scheme and thus produce a ringed space.
However, we will now proceed to define some additional sheaves on a scheme $X$,
and study the relations between them.

\section{Sheaves of $\OO_X$-modules}
\prototype{Twisting Sheaves}
Up until now we have mostly been working with sheaves of rings,
but now we want to instead think about sheaves of modules related
to a scheme $X$. 
\begin{definition}
	Let $(X, \OO_X)$ be a ringed space.
	A \vocab{sheaf of $\OO_X$-modules} is a sheaf $\SF$
	of abelian groups such that $\SF(U)$ is also an $\OO_X(U)$-module.
	The module structures are also required to be compatible
	with the restriction maps.
\end{definition}
\begin{example}
	[Twisting sheaves]
	Let $X = \Proj S$ be a projective scheme for a graded ring $S$
	(e.g.\ a projective subscheme).
	For any integer $m$ we consider a pre-sheaf
	\[ \SF^n = \left\{ \frac fg \mid g(\pp)\neq 0 \;\forall p \in X 
		\text{ and } \deg f - \deg g = n \right\}.
	\]
	Intuitively, these are degree $n$ ``functions''.
	Then $\OO_X(n) = (\SF^n)\sh$;
	we call these the \vocab{twisting sheaves}.

	For example, if $S = \CC[x_0, \dots, x_n]$ then
	\begin{enumerate}[(a)]
		\ii $\OO_X(0) \cong \OO_X$, by definition.
		\ii A degree $n$ polynomial defines a global section of $\OO_X(n)$.
		\ii $\OO_X(m)$ has no nonzero global sections if $m < 0$.
		\ii Suppose $X = \CP^1$ has coordinates $(s:t)$. Then
		\[ \frac 1s \in \OO_X(-1)(U_0) \]
		is an example of a (non-global) section of $\OO_X(-3)$.
		\ii Suppose $X = \CP^2$ has coordinates $(x:y:z)$. Then
		\[ \frac{1}{xyz} \in \OO_X(-1)(U_0 \cap U_1 \cap U_2) \]
		is an example of a (non-global) section of $\OO_X(-3)$.
	\end{enumerate}
\end{example}

Observe that the stalks of such a sheaf are abelian groups.

\section{Sheaves of $\OO_X$-modules make an abelian category}
I won't take the time to do this properly,
since I won't really use much of it.
See \cite{ref:vakil} for a proper exposition.

Fix a ringed space $(X, \OO_X)$.
It's easy to see there is a category of sheaves of $\OO_X$-modules.
(the objects are sheaves and the morphisms are, well, morphisms of sheaves).
To see that it is abelian, we need to:
\begin{itemize}
	\ii Identify the zero object of the category.
	This is constant sheaf $0$, which gives $\SF(U) = \{0\}$ for every $U$.
	(This more or less follows from the fact that $\{0\}$
	is the zero object of the abelian category $\catname{AbGrp}$.)
	\ii Show that we can add two morphisms.
	This is done in the obvious manner.
	\ii Show that any morphism has a ``kernel sheaf'' and ``cokernel sheaf''.
	The definition is not too bad: given $\alpha : \SF \to \SG$ we define the
	pre-sheaves $\ker\alpha$ and $\coker'\alpha$ by
	\begin{align*}
		\ker(\alpha)(U) &= \ker(\alpha_U) \\
		\coker'(\alpha)(U) &= \coker(\alpha_U).
	\end{align*}
	It turns out that $\ker(\alpha)$ is already a sheaf.
	But $\coker'(\alpha)$ is only a pre-sheaf and in general not a sheaf.
	(Again, see \cite{ref:vakil} or \cite{ref:gathmann} for why.)
	Thus we have to define $\coker(\alpha) = \coker'(\alpha) \sh$.
	
	The tricky part, done in \cite{ref:vakil} and the one which I'm omitting,
	is showing that this kernel and cokernel sheaf satisfy
	the correct universal properties.
\end{itemize}

From this we deduce
\begin{theorem}
	[Sheaves form an abelian category]
	Let $(X, \OO_X)$ be a ringed space.
	The category of sheaves of $\OO_X$-modules forms an abelian category.
\end{theorem}

In particular, we can talk about whether a sequence of sheaves
$\SF \to \SG \to \SH$ is exact: just use the categorical definition.
However, if possible we want to avoid using the long-winded definition.
So, I will just quote the relevant results:
\begin{theorem}
	[Exactness can be checked at stalks]
	A sequence of sheaves of $\OO_X$-modules
	$\SF \to \SG \to \SH$ is exact if and only if
	for every point $p \in X$,
	the induced map of stalks $\SF_p \to \SG_p \to \SH_p$ is exact.
\end{theorem}
Thus, we \emph{really} like stalks; they behave as nicely as possible.
\begin{ques}
	Deduce that $\SF \to \SG$ is injective / surjective / an isomorphism
	if and only if $\SF_p \to \SG_p$ is injective / surjective / an isomorphism
	for every point $p$.
\end{ques}

If we insist on sections, the only useful result we have is:
\begin{theorem}
	[Monic/epic on sections]
	Let $\alpha : \SF \to \SG$ be a morphism of $\OO_X$-modules.
	\begin{enumerate}[(a)]
		\ii The map $\alpha$ is monic if and only if $\SF(U) \to \SG(U)$
		is injective for every open $U$.
		\ii The map $\alpha$ is epic if $\SF(U) \to \SG(U)$
		is surjective for every open $U$,
		but the converse is not true in general.
	\end{enumerate}
\end{theorem}

In addition, we can do the following constructions.
\begin{definition}
	If $\SF$ and $\SG$ are sheaves of $\OO_X$-modules, then
	\begin{enumerate}[(a)]
		\ii The \vocab{direct sum} $\SF \oplus \SG$ is defined
		on open sets by \[ U \mapsto \SF(U) \oplus \SG(U). \]
		(This is already a sheaf, so we don't need to sheafify.)
		\ii The \vocab{tensor product} $\SF \otimes \SG$ is defined
		by taking the sheafification of the pre-sheaf
		\[ U \mapsto \SF(U) \otimes_{\OO_X(U)} \SG(U). \]
		\ii The \vocab{dual} $\SF^\vee$ of $\SF$ is defined
		by taking the sheafification of the pre-sheaf
		\[ U \mapsto \Hom_{\OO_X(U) \text{ module}} (\SF(U), \OO_X(U)). \]
	\end{enumerate}
\end{definition}
\begin{example}[More on twisting sheaves]
	Consider the twisting sheaves of $X = \CP^N$.
	\begin{enumerate}[(a)]
		\ii $\OO_X(n) \otimes \OO_X(m) \cong \OO_X(m+n)$.
		What do you think the isomorphism is?
		\ii $\OO_X(n)^\vee \cong \OO_X(-n)$.
	\end{enumerate}
\end{example}

\section{Quasi-coherent sheaves}

