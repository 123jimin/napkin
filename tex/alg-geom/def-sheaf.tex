\chapter{Sheaves and Ringed Spaces}
Most of the complexity of the affine variety $V$ earlier comes from $\OO_V$.
This is a type of object called a ``sheaf''.
The purpose of this chapter is to completely define what this sheaf is,
and just what it is doing.

The typical example to keep in mind is a sheaf of
``functions with property $P$'' on a topological space $X$:
for every open set $U$, $\SF(U)$ gives us the ring of functions on $X$.
However, we will work very abstractly and only assume $\SF(U)$
is a ring, without an interpretation as ``functions''.

The payoff for this abstraction is that it will
allow us to define an arbitrary scheme in the next chapter.
Varieties use $\CC[x_1, x_2, \dots, x_n] / I$ as their ``ring of functions'',
and by using the fully general sheaf we will be replace this
with \emph{any} commutative ring.
In particular, we can use the case where $I$ is not semiprime, such as
$\CC[x] / (x^2)$; this gives the ``multiplicity''
behavior that we sought after all along.  

\section{Pre-Sheaves}
\prototype{The sheaf of holomorphic (or regular, continuous,
differentiable, constant, whatever) functions.}

The proper generalization of our $\OO_V$ is a so-called sheaf of rings.
Recall that $\OO_V$ was a took \emph{open sets of $V$} to \emph{rings},
with the interpretation that $\OO_V(U)$ was a ``ring of functions''.

In light of this, we first make the following definition
for the general construct, in the language of categories.

\begin{definition}
	For a topological space $X$ let $\Opens(X)$ denote
	its open sets of $X$.
\end{definition}
\begin{definition}
	A \vocab{pre-sheaf} of rings on a space $X$ is a function
	\[ \SF : \Opens(X) \to \catname{Rings} \]
	meaning each open set gets associated with a ring $\SF(U)$.
	Each individual element of $\SF(U)$ is called a \vocab{section}.
	An element of $\SF(X)$ is called a \vocab{global section}.

	It is also equipped with the following information:
	for any $U_1 \subseteq U_2$ there is a \vocab{restriction map}
	\[ \res_{U_1,U_2} : \SF(U_2) \to \SF(U_1) \]
	such that $\res_{U,U}$ is the identity and
	whenever $U_1 \subseteq U_2 \subseteq U_3$ the diagram
	\begin{diagram}
		\SF(U_3) & \rTo^{\res_{U_2,U_3}} & \SF(U_2) \\
		& \rdTo_{\res_{U_1,U_3}} & \dTo_{\res_{U_1,U_2}} \\
		&& \SF(U_1)
	\end{diagram}
	commutes. (Restricting ``big to medium to small''
	is the same as ``big to small''.)
\end{definition}

\begin{abuse}
	If $s \in \mathscr F(U_2)$ is some section and $U_1 \subseteq U_2$,
	then rather than write $\res_{U_1,U_2}(s)$
	I will write $s\restrict{U_1}$ instead:
	``$s$ restricted to $U_1$''.
	This is abuse of notation because the section $s$ is just
	an element of some ring, and in the most abstract of cases
	may not have a natural interpretation as function.
\end{abuse}

\begin{example}[Examples of Sheaves]
	\listhack
	\begin{enumerate}[(a)]
		\ii For an affine variety $V$, $\OO_V$ is of course a sheaf,
		with $\OO_V(U)$ being the ring of regular functions on $U$.
		The restriction map just says that if $U_1 \subseteq U_2$,
		then a function $s \in \OO_V(U_2)$ can also be thought of as
		a function $s \restrict{U_1} \in \OO_V(U_1)$,
		hence the name ``restriction''.
		The commutativity of the diagram then follows.
		
		\ii Let $X \subseteq \RR^n$ be an open set.
		Then there is a sheaf of smooth/differentiable/etc.\ functions on $X$.
		In fact, one can do the same construction for any manifold $M$.

		\ii Similarly, if $X \subseteq \CC$ is open,
		we can construct a sheaf of holomorphic functions on $X$.
	\end{enumerate}
	In all these examples, the sections $s \in \SF(U)$
	are really functions on the space, but in general they need not be.
\end{example}

Now, we give a second, equivalent and far shorter definition of pre-sheaf:
\begin{abuse}
	By abuse of notation, $\Opens(X)$ will also be thought of as a
	posetal category by inclusion. Thus $\varnothing$ is an initial object
	and the entire space $X$ is a terminal object.
\end{abuse}
\begin{definition}
	A \vocab{pre-sheaf} of rings on $X$ is a contravariant functor
	\[ \SF : \Opens(X)\op \to \catname{Rings}. \]
\end{definition}
\begin{ques}
	Check that these definitions are equivalent.
\end{ques}
It is now clear that we can actually replace $\catname{Rings}$
with any category we want.

So we should think:
\begin{moral}
	Pre-sheaves should be thought of as
	``returning the ring of functions with a property $P$''.
\end{moral}

\section{Sheaves}
\prototype{Constant functions aren't pre-sheaves,
	but locally constant ones are.}

Now, the main idea next is that
\begin{moral}
	Sheaves are pre-sheaves for which $P$ is a \emph{local} property.
\end{moral}

The formal definition doesn't illuminate this as much as the examples
do, but I have to give it first for the examples to make sense.

\begin{definition}
	A \vocab{sheaf} $\mathscr F$ is a pre-sheaf satisfying the following
	two additional axioms:
	Suppose $U$ is covered by open sets $U_\alpha \subseteq U$. Then:
	\begin{enumerate}
		\ii (Identity) If $s, t \in \mathscr F(U)$ are sections,
		and $s\restrict{U_\alpha} = t\restrict{U_\alpha}$
		for all $\alpha$, then $s = t$.
		\ii (Collation) Consider sections
		$s_\alpha \in \mathscr(U_\alpha)$ for each $\alpha$.
		Suppose that 
		\[ s_\alpha \restrict{U_\alpha \cap U_\beta}
			= s_\beta \restrict{U_\alpha \cap U_\beta} \]
		for each $U_\alpha$ and $U_\beta$.
		Then we can find $s \in U$ such that
		$s \restrict{U_\alpha}  = s_\alpha$.
	\end{enumerate}
\end{definition}
This is best illustrated by picture: consider an open cover $U_1 \cup U_2$.
\begin{center}
	\begin{asy}
		size(4cm);
		filldraw(shift(-0.5,0)*unitcircle, lightred+opacity(0.3), red);
		filldraw(shift( 0.5,0)*unitcircle, lightblue+opacity(0.3), blue);
		label("$U_1$", (-0.5,0)+dir(135), dir(135), red);
		label("$U_2$", ( 0.5,0)+dir(45), dir(45), blue);
	\end{asy}
\end{center}
Then for a sheaf of functions, the axioms are saying that:
\begin{itemize}
	\ii If $s$ and $t$ are functions (with property $P$)
	on the whole space $U = U_1 \cup U_2$,
	and $s \restrict{U_1} = t \restrict{U_1}$,
	$s \restrict{U_2} = t \restrict{U_2}$,
	then $s = t$ on the entire union.
	This is clear.

	\ii If $s_1$ is a function with property $P$ on $U_1$
	and $s_2$ is a function with property $P$ on $U_2$,
	and the two functions agree on the overlap,
	then one can collate them to obtain a function $s$
	on the whole space:
	this is obvious, but \textbf{the catch is that the collated function
	needs to have property $P$ as well.}
	That's why it matters that $P$ is local.
\end{itemize}

The reason we need these axioms is that in our abstract definition of a sheaf,
the output of the sheaf is an abstract ring, which need not actually
have a concrete interpretation as ``functions on $X$'', even though
our examples will usually have this property.

Now for the examples, which are more enlightening:
\begin{example}
	[Examples and Non-Examples of Sheaves]
	\listhack
	\begin{enumerate}[(a)]
		\ii Pre-sheaves of arbitrary / continuous / differentiable / smooth
		/ holomorphic functions are still sheaves.
		This is because to verify a function is continuous,
		one only needs to look at small neighborhoods at once.
		
		\ii For a complex variety $V$, $\OO_V$ is a sheaf,
		precisely because our definition was \emph{locally} quotients
		of polynomials.

		\ii The pre-sheaf of \emph{constant} real functions on a space $X$
		is \emph{not} a sheaf, because it fails the collation axiom.
		Namely, suppose that $U_1 \cap U_2 = \varnothing$.
		Then if $s_1$ is the constant function $1$ on $U_1$
		while $s_2$ is the constant function $2$ on $U_2$,
		then we cannot collate these to a constant function on $U_1 \cup U_2$.

		\ii On the other hand, \emph{locally constant} functions
		do produce a sheaf. (A function is locally constant
		if for every point it is constant on some neighborhood.)
	\end{enumerate}
	In fact, the sheaf in (c) is what is called a \emph{sheafification}
	of the pre-sheaf constant functions, which we define momentarily.
\end{example}

\section{Stalks}
\prototype{Germs of real smooth functions tell you the derivatives,
but germs of holomorphic functions determine the entire function.}
Let $\SF$ be a sheaf.
If we have a function $s \in \SF(U)$ and a point $p \in U$,
then in general it doesn't make sense to ask what $s(p)$ is
(even though all our examples look like this),
because $\SF(U)$ is an arbitrary ring.
So, we will replace the notion of $s(p)$ with a so-called \emph{germ}.

\begin{definition}
	Let $\SF$ be a pre-sheaf of rings.
	For every point $p$ we define the \vocab{stalk} $\SF_p$ to be the set
	\[ \left\{ (s, U) \mid s \in \SF(U), p \in U \right\} \]
	modulo the relation $\sim$ that 
	\[ (s_1,U_1) \sim (s_2, U_2) \text{ if }
		s_1 \restrict{U_1 \cap U_2} = s_2 \restrict{U_1 \cap U_2}. \]
	The equivalence classes themselves are called \vocab{germs}.
\end{definition}
So what's happening is:
\begin{moral}
	We consider functions $s$ defined near $p$
	and, since we only care about local behavior,
	we identify any two functions agreeing on a neighborhood of $p$,
	no matter how small.
\end{moral}

Notice that the stalk is itself a ring as well:
for example, addition is done by
\[ 
	\left( s_1, U_1 \right) + \left( s_2, U_2 \right)
	=
	\left( s_1 \restrict{U_1 \cap U_2} + s_2 \restrict{U_1 \cap U_2},
	U_1 \cap U_2 \right).
\]

\begin{definition}
	The germ of a given $s \in \SF(U)$ at a point $p$
	is the equivalence class for $(s,U) \in \SF_p$.
	We denote this by $[s]_p$.
\end{definition}

So the germ $[s]_p$ now plays the role of the ``value'' $s(p)$.
But actually, it carries much more information than that.

\begin{example}
	[Germs of Real Smooth Functions]
	Let $X = \RR$ and let $\SF$ be the sheaf on $X$ of smooth functions
	(i.e.\ $\SF(U)$ is the set of smooth real-valued functions on $U$).

	Consider a global section, $s : \RR \to \RR$ (thus $s \in \SF(X)$)
	and its germ at $0$.
	\begin{enumerate}[(a)]
		\ii From the germ we can read off $s(0)$, obviously.
		\ii We can also find $s'(0)$, because the germ carries enough
		information to compute the limit $\lim_{h \to 0} \frac1h[s(h)-s(0)]$.
		\ii Similarly, we can compute the second derivative and so on.
		\ii However, we can't read off, say, $s(3)$ from the germ.
		For example, take
		\[
			s(x) = \begin{cases}
				e^{-\frac{1}{x-1}} & x > 1 \\
				0 & x \le 1.
			\end{cases}
		\]
		Note $s(3) = e^{-\half}$, but $[\text{zero function}]_0 = [s]_0$.
		So germs can't distinguish between the zero function and $s$.
	\end{enumerate}
\end{example}
\begin{example}
	[Germs of Holomorphic Functions]
	Holomorphic functions are very strange in this respect.
	Consider the sheaf $\SF$ on $\CC$ of holomorphic functions $\CC \to \CC$.

	Take $s : \CC \to \CC$ a global section.
	Given the germ of $s$ at $0$, we can read off $s(0)$, $s'(0)$, et cetera.
	The miracle of complex analysis is that just knowing
	the derivatives of $s$ at zero is enough to reconstruct all of $s$:
	we can compute the Taylor series of $s$ now.
	\textbf{Thus germs of holomorphic functions determine the function};
	they carry much more information than their real counterparts.
	
	In particular, we can concretely describe the sheaf:
	\[
		\SF_p = \left\{
			\sum_{k \ge 0} c_k (z-p)^k,
			\text{ convergent near $p$}
		\right\}.
	\]
	In particular, this includes germs of meromorphic functions,
	so long as there is no pole at $p$ itself.
\end{example}

And of course, our algebraic geometry example.
\begin{abuse}
	Rather than writing $(\OO_V)_p$ we will write $\OO_{V,p}$.
\end{abuse}
\begin{example}
	[Stalks of the Sheaf on an Affine Variety]
	Let $V \subseteq \Aff^n$ be a variety, and assume $p \in V$.
	Then, a regular function $\varphi$ on $U \subseteq V$
	is supposed to be a function on $U$ that ``locally'' is a quotient
	of two functions in $\CC[V]$.

	However, \textbf{as far as the germ is concerned, we only care about
	whichever quotient applies near the point $p$}.
	In light of this, we only think about the representation at $p$,
	and ignore the local clause completely:
	thus the entire stalk can be thought of as
	\[
		\OO_{V,p} = 
		\left\{ \left( \tfrac fg , U \right) \mid 
			p \in U, \; f,g \in \CC[V], \;
			\text{$g \neq 0$ on $U$} \right\}
	\]
	modulo the usual relations.

	Now, since we happen to be working with complex polynomials,
	we know that a rational function is determined by its
	behavior on any neighborhood of $p$
	(complex analysis forever!).
	Thus 
	\[
		\OO_{V,p} =
		\left\{ \tfrac fg \mid f,g \in \CC[V], \; g(p) \neq 0 \right\}.
	\]
	which don't vanish on $p$.
\end{example}

\begin{remark}
	[For category lovers]
	You may notice that $\SF_p$ seems to be
	``all the $\SF_p(U)$ coming together'', where $p \in U$.
	And in fact, $\SF_p(U)$ is the categorical \emph{limit}
	of the diagram formed by all the $\SF(U)$ such that $p \in U$.
	This is often written
	\[ \SF_p = \varinjlim_{U \ni p} \SF(U) \]
	Thus we can define stalks in any category with limits,
	though to be able to talk about germs the category needs
	to be concrete.
\end{remark}

\section{Sheafification}
\prototype{The pre-sheaf of locally constant functions
	becomes the sheaf of constant functions.}
Now that we have the language of germs,
we can define the so-called sheafification.
The idea is that if $\SF$ is the pre-sheaf of ``functions with property $P$''
then we want to associate a sheaf $\SF\sh$ of
``functions which are locally $P$'', which makes them into a sheaf.
We have already seen two examples of this:
\begin{example}
	[Sheafification]
	\listhack
	\begin{enumerate}[(a)]
		\ii If $X$ is a topological space,
		and $\SF$ is the pre-sheaf of constant functions on $X$,
		then $\SF\sh$ is the sheaf of locally constant functions.

		\ii If $V$ is an affine variety,
		and $\SF$ is the pre-sheaf of rational functions,
		then $\SF\sh$ is the sheaf of regular functions
		(which are locally rational).
	\end{enumerate}
\end{example}

So how do we encode ``locally $P$''?
The way we do this is by considering a sequence of germs,
one at every point, such that at any point they can locally be put
together to form a section.

\begin{definition}
	Let $\SF$ be pre-sheaf.
	For an open set $U$, a sequence $(g_p)_{p \in U}$ of germs
	($g_p \in \SF_p$) is said to be \vocab{compatible} if
	they can be ``locally collated'':
	\begin{quote}
		For any $p \in U$ there exists a $s \in \SF(U_p)$
		such that $[s]_q = g_q$ for each $q \in U_p$.
		(Here $U_p$ is any neighborhood of $p$.)
	\end{quote}
	Intuitively, the germs should ``collate together'' to some section.
	Then we define the \vocab{sheafification} $\SF\sh$ by setting
	\[ \SF\sh(U) =
		\left\{ \text{sequences of compatible
		germs $(g_p)_{p \in U}$} \right\}.  \]
\end{definition}
\begin{exercise}
	Reconcile this definition with the two examples we gave.
\end{exercise}
The construction is contrived so that given a section
$(g_p)_{p \in U} \in \SF\sh(U)$ the germ at a point $p$ is $g_p$:
\begin{lemma}
	[Pre-Sheaves and Sheaves Have the Same Stalk]
	Let $\SF$ be a pre-sheaf and $\SF\sh$ its sheafification.
	Let $q \in U$ with $U$ an open set.
	Then there is an isomorphism
	\[ (\SF\sh)_q \cong \SF_q. \]
\end{lemma}
\begin{proof}
	A germ in $(\SF\sh)_q$ looks like 
	$\left( U, (g_p)_{p \in U} \right)$, where $g_p$ are germs of $\SF_p$.
	Then the isomorphism is given by
	\[ \left( U, (g_p)_{p \in U} \right) \mapsto g_q \in \SF_q. \]
	The inverse map is given by for each $g = (s,U) \in \SF_q$ by
	\[ g \mapsto \left( (g)_{p \in U}, U \right) \]
	i.e.\ the sequence of germs is the constant sequence.
\end{proof}

\section{Local Rings}
\prototype{Smooth functions $f$ on $X \subseteq \RR^n$ have invertible
germs at $p$ unless $f(p) = 0$.}
The stalks of the examples we produced above are special types
of rings, called \emph{local rings}.
Algebraically, the definition of these is:
\begin{definition}
	A \vocab{local ring} $R$ is a ring with exactly one maximal ideal.
\end{definition}

% Wikipedia has a good explanatio nhere
To see why this definition applies to the stalks above,
we need to identify what the maximal ideal is.
Let's go back to the example of $X = \RR$ and $\SF(U)$ the smooth functions,
and consider the stalk $\SF_{p}$, where $p \in X$.
Define the ideal $\mm_p$ to be the set of germs $(s,U)$ for which $s(p) = 0$.

Then $\mm_p$ is maximal: we have an exact sequence
\[ 0 \to \mm_p \to \SF(U) \taking{(s,U) \mapsto s(p)} \RR \to 0 \]
and so $\SF(U) / \mm_p \cong \RR$, which is a field.

It remains to check there are no nonzero maximal ideals.
Now note that if $s \notin \mm_p$,
then $s$ is nonzero in some neighborhood of $p$,
then one can construct the function $1/s$ in a neighborhood of $p$.
So \textbf{every element of $\SF_p \setminus \mm_p$ is a unit};
$\mm_p$ is in fact the only maximal ideal!

More generally, any time a sheaf $\SF$ is of the form
``nice field-valued functions'',
the stalk $\SF_p$ usually has a maximal ideal
consisting of the germs vanishing at $p$.
The discussion above implies, for example, the following.
\begin{proposition}
	[Stalks Are Often Local Rings]
	The stalks of each of the following types of sheaves are local rings:
	\begin{enumerate}[(a)]
		\ii Sheaves of continuous real/complex functions on a topological space
		\ii Sheaves of smooth functions on any manifold
		\ii Regular functions on an algebraic variety $V$.
	\end{enumerate}
\end{proposition}

\section{Locally Ringed Space}
We can now define:
\begin{definition}
	A \vocab{ringed space} is a topological space $X$ equipped
	with a sheaf $\OO_X$ of rings.
	Suppose that for every point $p$, the stalk $\OO_{X,p}$
	is a local ring.
	Then we say that $\OO_X$ is a \vocab{locally ringed space}.
	We denote the maximal ideals by $\mm_{X,p}$.
\end{definition}

In particular, in the previous chapter we showed that every
affine variety could be built into a locally ringed space. Hooray!
\begin{abuse}
	A ringed space $(X, \OO_X)$ is abbreviated to just $X$,
	while $p \in X$ means ``$p$ is in the topological space $X$''.
\end{abuse}


\section{(Optional) Defining the Morphisms}
Finally, it remains to define a morphism of locally ringed space.
To do this we have to build up in several steps.

\begin{remark}
	This is marked as optional because it involves a \emph{lot}
	of definitions, and because we will more or less
	ignore the actual definition later on when we start applying
	schemes to things. One can probably get away with just
	suspending belief and knowing ``there is a reasonable definition
	of morphisms of locally ringed spaces''.
\end{remark}

\subsection*{Morphisms of Pre-Sheaves and Sheaves}
First, recall that a sheaf is a contravariant functor (pre-sheaf)
with extra conditions. In light of this, it is not hard to guess
the definition of a morphism of pre-sheaves:
\begin{definition}
	A \vocab{morphism of (pre-)sheaves} $\phi : \SF \to \SG$ on the same
	space $X$ is a \textbf{natural transformation} of the underlying functors.
	Isomorphism of sheaves is defined in the usual way.
\end{definition}
\begin{ques}
	Show that this amounts to: for each $U \subseteq X$ we need to specify
	a morphism $\phi : \SF(U) \to \SG(U)$ such that the diagram
	\begin{diagram}
		\SF(U_2) & \rTo^{\phi_{U_2}} & \SG(U_2) \\
		\dTo^{\res_{U_1, U_2}} && \dTo_{\res_{U_1, U_2}} \\
		\SF(U_1) & \rTo_{\phi_{U_1}} & \SG(U_1)
	\end{diagram}
	commutes any time that $U_1 \subseteq U_2$.
\end{ques}

%Now, note that
%\begin{proposition}
%	[Morphisms of Sheaves Induce Stalk Morphisms]
%	Given a morphism $\phi : \SF \to \SG$ of sheaves on $X$, we get
%	for every point $p \in X$ an induced morphism
%	\[ \phi_p : \SF_p \to \SG_p. \]
%\end{proposition}
%\begin{proof}
%	We have $\phi_U : \SF(U) \to \SG(U)$ for every $U$
%	and hence construct the map $(U, f) \mapsto (U, \phi_U(f))$.
%\end{proof}

\subsection*{Morphisms of Ringed Spaces}
This is more involved.
Suppose we have ringed spaces $X = (X, \OO_X)$ and $Y = (Y, \OO_Y)$,
and we want to define a map
\[ f : X \to Y. \]

On the level of spaces, $f : X \to Y$ should of course be a continuous map.
But we also want to $f$ to do something with $\OO_X$ and $\OO_Y$.
However, we only defined how to take a morphism of sheaves
that act on the \emph{same} space!
Thus we impose the following definition,
which lets us \emph{push} the sheaf on $X$ to a sheaf on $Y$.
Picture:
\begin{diagram}
	X && \rTo^f & Y && X && \rTo^f & Y \\
	\Opens(X)\op && \lTo^{f\pre} & \Opens(Y)\op
	&& f\pre(U) && \lMapsto^{f\pre} & U & \\
	& \rdMapsto(3,2)_{\SF} && \dMapsto_{f_\ast \SF}
		&& & \rdMapsto(3,2)_{\SF} && \dMapsto_{f_\ast \SF} & \\
	&&& \catname{Rings}
		&& && \SF(f\pre(U)) & = (f_\ast \SF)(U) & \in \catname{Rings}
\end{diagram}

\begin{definition}
	Let $\SF$ be a sheaf on $X$, and $f : X \to Y$ a continuous map.
	The \vocab{pushforward sheaf} $f_\ast \SF$ on $Y$ is defined by
	\[ (f_\ast \SF)(U) = \SF(f\pre(U)) \qquad \forall U \subseteq Y. \]
	This makes sense, since $f\pre(U)$ is open in $X$.
\end{definition}
As $f\pre$ is a functor $\Opens(Y)\op \to \Opens(X)\op$,
\textbf{the pushforward $f_\ast\SF$ is
just the composition of these two functors}.

\begin{ques}
	Technically $f_\ast\SF$ is supposed to be a
	functor $\Opens(Y)\op \to \catname{Rings}$, so it also needs
	to come with restriction arrows. What are they?
\end{ques}

I haven't actually checked that $f_\ast \SF$ is a sheaf
(as opposed to a pre-sheaf), but this isn't hard to do.

Now we can define a morphism of ringed spaces.
In addition to the topological spaces,
we also include a natural transformation between
the two structure sheafs, after pushing the one on $X$ forward.
\begin{definition}
	A \vocab{morphism of ringed spaces} $f : (X, \OO_X) \to (Y, \OO_Y)$
	consists of a continuous map of topological spaces $f : X \to Y$,
	as well as a morphism of sheaves $f^\ast : \OO_Y \to (f_\ast \OO_X)$.
\end{definition}
The latter is thus a map
$f^\ast_U : \OO_Y(U) \to (f_\ast \OO_X)(U) = \OO_X(f\pre(U))$
for every $U \subseteq Y$, satisfying
the axioms for a natural transformation.

\subsection*{Morphisms of Locally Ringed Spaces}
Last step!
Suppose now that $(X, \OO_X)$ and $(Y, \OO_Y)$ are locally ringed spaces.
Thus we need to deal with some information about the stalks.

Given a morphism of ringed spaces $f : (X, \OO_X) \to (Y, \OO_Y)$,
we can actually use the $f^\ast_U$ above to induce maps on the stalks,
as follows. For a point $p \in X$, construct the map
\begin{diagram}
	f^\ast_P :& \OO_{Y, f(p)} & \rTo && \OO_{X, f(p)} \\
	& (s, U) & \rMapsto && (f^\ast_U(s), f\pre(U)).
\end{diagram}
where $s \in \OO_Y(U)$.

\begin{definition}
	Let $R$ and $S$ be local rings with maximal ideals $\mm_R$ and $\mm_S$.
	A \vocab{morphism of local rings} is a homomorphism of rings
	$\psi : R \to S$ such that $\psi\pre(\mm_S) = \mm_R$.
\end{definition}
\begin{definition}
	A \vocab{morphism of locally ringed spaces}
	is a morphism of ringed spaces $f : (X, \OO_X) \to (Y, \OO_Y)$ such that
	for every point $p$ the induced map of stalks is a morphism of local rings.
\end{definition}
Recalling that $\mm_{X,p}$ is the maximal ideal of $\OO_X$ at $p$,
the new condition is saying that
\[ (f^\ast_P)\pre (\mm_{X,p}) = \mm_{Y,f(p)}. \]
Concretely, if $g$ is a function which vanishes on $f(p)$,
then its ``pullback'' $f^\ast_U(g)$ vanishes on $p$.

This completes the definition of a morphism of locally ringed spaces.
Isomorphisms of (locally) ringed spaces are defined in the usual way.

\section\problemhead
\begin{problem}
	Prove that if $\SF$ is a sheaf, then $\SF \cong \SF\sh$.
\end{problem}


