\chapter{Morphisms of varieties}
Now that we've taken all the time to define affine and projective varieties,
we want to take the time to define the \emph{morphisms} between these objects.
Fortunately, we know both affine and projective varieties are special
cases of baby ringed spaces, so in fact we will just define a morphism
between \emph{any} two baby ringed spaces.

\section{Defining morphisms of baby ringed spaces}
\prototype{See next section.}
Let $(X, \OO_X)$ and $(Y, \OO_Y)$ be baby ringed spaces, and think
about how to define a morphism between them.

The guiding principle in algebra is that we want morphisms
to be functions on underlying structure, but also respect
the enriched additional data on top.
To give some examples from the very beginning of time:
\begin{example}
	[How to define a morphism]
	\listhack
	\begin{itemize}
	\ii Consider groups.
	A group $G$ has an underlying set (of elements),
	which we then enrich with a multiplication operation.
	So a homomorphism is a map of the underlying sets,
	plus it has to respect the group multiplication.
	\ii Consider $R$-modules.
	Each $R$-module has an underlying abelian group,
	which we then enrich with scalar multiplication.
	So we require that a linear map respects the scalar multiplication as well,
	in addition to being a homomorphism of abelian groups.
	\ii Consider topological spaces.
	A space $X$ has an underlying set (of points),
	which we then enrich with a topology of open sets
	So we consider amps of the set of points
	which respect the topology (pre-images of open sets are open).
	\end{itemize}
\end{example}
This time, the ringed spaces $(X, \OO_X)$ have an underlying
\emph{topological space}, which we have enriched with a structure sheaf.
So, we want a continuous map $f : X \to Y$ of these topological spaces,
which we then need to respect the sheaf of regular functions.

How might we do this? Well, if we let $\psi : Y \to \CC$
be a regular function, then there's a natural way to get
then composition gives us a way to write a map $X \to Y \to \CC$.
We then want to require that this is also a regular function.

More generally, we can take any regular function on $Y$
and obtain some function on $X$, which we call a pullback.
We then require that all the pullbacks are regular on $X$.
\begin{definition}
	Let $(X, \OO_X)$ and $(Y, \OO_Y)$ be baby ringed spaces.
	Given a map $f : X \to Y$ and a regular function $\phi \in \OO_Y(U)$,
	we define the \vocab{pullback} of $\phi$, denoted $f^\ast\phi$,
	to be the composed function
	\begin{diagram}
		f\pre(U) &\rTo^f & U & \rTo^\phi & \CC.
	\end{diagram}
\end{definition}
The use of the word ``pullback'' is the same as in our study
of differential forms.

\begin{definition}
	Let $(X, \OO_X)$ and $(Y, \OO_Y)$ be baby ringed spaces.
	A continuous map of topological spaces $f: X \to Y$
	is a \vocab{morphism} if every pullback of a regular function on $Y$
	is a regular function on $X$.

	Two baby ringed spaces are \vocab{isomorphic}
	if there are mutually inverse morphisms between them,
	which we then call \vocab{isomorphisms}.
\end{definition}

In particular, the pullback gives us a (reversed) \emph{ring homomorphism}
\[ f^\ast : \OO_Y(U) \to \OO_X(f\pre(U)) \] for \emph{every} $U$;
thus our morphisms package a lot of information.
Here's a picture of a morphism $f$,
and the pullback of $\phi : U \to \CC$ (where $U \subseteq Y$).
\begin{center}
	\begin{asy}
		size(13cm);
		bigblob("$X$");
		pair p = (0.5,0);
		filldraw(CR(p, 1), opacity(0.2)+lightgreen, deepgreen+dashed);
		label("$f^{\text{pre}}(U)$", p+dir(135), dir(135), deepgreen);

		transform t = scale(0.8) * shift(14*dir(180));
		add(t * CC());
		p = t*p;

		bigblob("$Y$");
		pair q = (0,0.5);
		filldraw(CR(q, 1.2), opacity(0.2)+lightred, red+dashed);
		label("$U$", q+1.2*dir(45), dir(45), red);
		
		draw( (-9,0.5)--(-3,0.5), EndArrow);
		label("$f$", (-6,0.5), dir(90));
	
		draw(q--(0,-8), red, EndArrow);
		label("$\phi \in \mathcal O_Y(U)$", (0,-6), dir(0), red);
		label(scale(2.718)*"$\mathbb C$", (0,-9), black);

		draw( p--(-0.5,-8), deepgreen, EndArrow);
		label("$f^\ast \phi \in \mathcal O_X(f^{\text{pre}}(U))$",
			(p+(-0.5,-8))/2, dir(225), deepgreen);
	\end{asy}
\end{center}

On a more philosophical point, we like this definition because
it adheres to our philosophy of treating our
varieties as intrinsic objects, rather than embedded ones,
However, it is somewhat of a nuisance to actually verify it,
and it what follows I will wave my hands a lot in claiming
that something is a morphism, since doing so is mostly detail checking.
The theorems which follow will give us alternative definitions
of morphism which are more coordinate-based
and easier to use for actual computations.

\section{Examples}
\begin{example}
	[The pullback of $\frac{1}{x-25}$ under $t \mapsto t^2$]
	The map 
	\[ f : X = \Aff^1 \to Y = \Aff^1 \quad\text{by}\quad t \mapsto t^2 \]
	is a morphism of varieties.
	For example, consider the regular function $\varphi = \frac{1}{x-25}$ on
	the open set $Y \setminus \{25\} \subseteq Y$.
	The $f$-inverse image is $X \setminus \{\pm5\}$.
	Thus the pullback is
	\[ f^\ast\varphi : X \setminus \{\pm5\} \to Y \setminus \{25\}
		\quad\text{by}\quad x \mapsto \frac{1}{x^2-25} \]
	which is regular on $X \setminus \{\pm5\}$.

	More generally, given any polynomial $P(t)$,
	the map $t \mapsto P(t)$ will work.
\end{example}
\begin{exercise}
	Let $X = \Aff^1$, $Y = \Aff^1$.
	By considering $\id \in \OO_Y(Y)$, show that no other
	regular functions exist.
\end{exercise}

In fact, let's generalize the previous exercise:
\begin{theorem}
	[Regular maps of affine varieties are globally polynomials]
	Let $X \subseteq \Aff^m$ and $Y \subseteq \Aff^n$ be affine varieties.
	Every morphism $f : X \to Y$ of varieties is given by
	\[
		x = \left( x_1, \dots, x_m \right)
		\xmapsto{f} \left( P_1(x), \dots, P_n(x) \right)
	\]
	where $P_1$, \dots, $P_n$ are polynomials.
	\label{thm:affine_global_polynomial}
\end{theorem}
\begin{proof}
	It's not too hard to see that all such functions work,
	so let's go the other way.
	Let $f : X \to Y$ be a morphism.

	First, remark that $f\pre(Y) = X$.
	Now consider the regular function $\pi_1 \in \OO_Y(Y)$,
	given by the projection $(y_1, \dots, y_n) \mapsto y_1$.
	Thus we need $f \circ \pi_1$ to be regular on $X$.

	But for affine varieties $\OO_X(X)$ is just the coordinate ring $\CC[X]$
	and so we know there is a polynomial $P_1$ such that $f \circ \pi_1 = P_1$.
	Similarly for the other coordinates.
\end{proof}
Unfortunately, the situation is a little weirder in the projective setting.
If $X \subseteq \CP^m$ and $Y \subseteq \CP^n$ are projective varieties,
then every function
\[
	x = \left( x_0 : x_1 : \dots : x_m \right)
	\mapsto \left( P_0(x) : P_1(x) : \dots : P_n(x) \right)
\]
is a valid morphism, provided the $P_i$ are homogeneous
of the same degree and don't all vanish simultaneously.
However if we try to repeat the proof for affine varieties
we run into an issue: there is no $\pi_1$ morphism.
(Would we send $(1:1) = (2:2)$ to $1$ or $2$?)

And unfortunately, there is no way to repair this.
Counterexample:
\begin{example}
	[Projective map which is not globally polynomial]
	Let $V = \Vp(xy-z^2) \subseteq \CP^2$.
	Then the map
	\[
		V \to \CP^1
		\quad\text{by}\quad
		(x:y:z)
		\mapsto
		\begin{cases}
			(x:z) & x \neq 0 \\
			(z:y) & y \neq 0
		\end{cases}
	\]
	turns out to be a morphism of projective varieties.
	This is well defined just because $(x:z) = (z:y)$ if $x,y \neq 0$;
	this should feel reminiscent of the definition of regular function.
\end{example}
The good news is that ``local'' issues are the only limiting factor.
\begin{theorem}
	[Regular maps of projective varieties are locally polynomials]
	Let $X \subseteq \CP^m$ and $Y \subseteq \CP^n$ be projective varieties
	and let $f : X \to Y$ be a morphism.
	Then at every point $p \in X$ there exists an neighborhood $U_p \ni p$
	and polynomials $P_0$, $P_1$, \dots, $P_n$ (which depend on $U$) so that
	\[ f(x) = \left( P_0(x) : P_1(x) : \dots : P_n(x) \right)
		\quad \forall x = (x_0 : \dots : x_n) \in U_p. \]
	Of course the polynomials $P_i$ must be homogeneous of the same degree
	and cannot vanish simultaneously on any point of $U_p$.
\end{theorem}
\begin{example}
	[Example of an isomorphism]
	In fact, the map $V = \Vp(xy-z)^2 \to \CP^1$ is an isomorphism.
	The inverse map $\CP^1 \to V$ is given by 
	\[ (s:t) \mapsto (s^2:st:t^2). \]
	Thus actually $V \cong \CP^1$.
\end{example}

\begin{remark}
Note also that if $X \cong Y$ as quasi-projective varieties,
then we get a lot of information.
For one thing, we know that $X$ and $Y$ are homeomorphic topological spaces.
But moreover, we get bijections between all the rings $\OO_X(f\pre(U))$
and $\OO_Y(U)$; in particular, we have $\OO_X(X) \cong \OO_Y(Y)$.
\end{remark}

One last useful example:
we can have maps from affine spaces to projective spaces.
\begin{example}
	[Embedding $\Aff^1 \injto \CP^1$]
	Consider a morphism
	\[ f : \Aff^1 \injto \CP^1 \quad\text{by}\quad t \mapsto (t:1).  \]
	This is also a morphism of varieties.
	(Can you see what the pullbacks look like?)
	This reflects the fact that $\CP^1$ is ``$\Aff^1$ plus a point at infinity''.
\end{example}

\section{Quasi-projective varieties}
\prototype{$\Aff^1 \setminus \{0\}$ might be best.}
In light of the previous example, we saw $\Aff^1$ is in some
sense a subset of $\CP^1$.
However, $\Aff^1$ is not itself a projective variety,
and so we want to loosen our definition a little:
\begin{definition}
	A \vocab{quasi-projective variety} is
	an open set $X$ of a projective variety $V$.
	It is a baby ringed space $(X, \OO_X)$ too,
	because for any open set $U \subseteq X$
	we simply define $\OO_X(U) = \OO_V(U)$.
\end{definition}
Every projective variety is quasi-projective by taking $X = V$.
But since we have a definition of morphism, we can now show:
\begin{proposition}
	[Affine $\subseteq$ quasi-projective]
	Every affine variety is isomorphic to a quasi-projective one.
\end{proposition}
\begin{proof}
	[Sketch of proof]
	We'll just do the example $V = \VV(y-x^2)$.
	We take the projective variety $W = \Vp(zy-x^2)$,
	and look at the corresponding standard open set $D(z)$.
	Then there is an isomorphism $V \to D(z)$ by $(x,y) \mapsto (x:y:1)$,
	with inverse map given by $(x:y:z) \mapsto (x/z, y/z)$.
\end{proof}

In summary, we have three different types of varieties:
\begin{itemize}
	\ii Affine varieties,
	\ii Projective varieties, and
	\ii Quasi-projective varieties.
\end{itemize}
These are baby ringed spaces, and so the notion of isomorphism
shows that the last type subsumes the first two.

The reason quasi-projective varieties are interesting is that
almost all varieties which occur in real life are quasi-projective.
In fact, it is tricky even to exhibit a single example
of a variety which is not quasi-projective.
I certainly don't know one.

\section{Some applications}
Natural question: are there quasi-projective varieties
which are neither affine nor projective?

Our first guess might be to take the simplest projective variety,
say $\CP^1$, and delete a point (to get an open set).
This is quasi-projective, but it's isomorphic to $\Aff^1$.
So instead we start with the simplest affine variety,
say $\Aff^1$, and delete a point.

Surprisingly, this doesn't work.
\begin{example}
	[$\Aff^1 \setminus \{0\}$ is affine]
	Let $X = \Aff^1 \setminus \{0\}$ be an quasi-projective variety.
	We claim that in fact we have an isomorphism
	\[ X \cong V = \VV(xy-1) \subseteq \Aff^2 \]
	which shows that $X$ is still isomorphic to an affine variety.
	The maps are
	\begin{align*}
		X & \leftrightarrow V \\
		t &\mapsto (t, 1/t) \\
		x &\mapsfrom (x, y).
	\end{align*}
	Intuitively, the ``hyperbola $y=1/x$'' in $\Aff^2$ can be projected
	onto the $x$-axis.
\end{example}
In fact, deleting any number of points from $\Aff^1$ fails
if we delete $\{1,2,3\}$, the resulting open set
is isomorphic as a baby ringed space to $\VV(y(x-1)(x-2)(x-3)-1)$,
which colloquially might be called $y = \frac{1}{(x-1)(x-2)(x-3)}$.

More generally, we have:
\begin{theorem}
	[Distinguished open subsets of affines are affine]
	Consider $X = D(f) \subseteq V = \VV(f_1, \dots, f_m) \subseteq \Aff^n$,
	where $V$ is an affine variety,
	and the distinguished open set $X$ is thought of as
	a quasi-projective variety.
	Define
	\[ W = \VV(f_1, \dots, f_m, y \cdot f-1) \subseteq \Aff^{n+1} \]
	where $y$ is the $(n+1)$st coordinate of $\Aff^{n+1}$.

	Then $X \cong W$.
\end{theorem}

Fortunately, the true example is not too far away.
First, check:
\begin{ques}
	Let $f : X \to Y$ be a morphism of quasi-projective varieties.
	Take a point $p \in X$ and let $q = f(p) \in Y$.
	Let $\varphi$ be a regular function on $Y$.

	Prove that if $\varphi(q) = 0$ then $(f^\ast\varphi)(p) = 0$.
\end{ques}
\begin{example}[Punctured plane is not affine]
	We claim that $X = \Aff^2 \setminus \{ (0,0) \}$ is not affine.
	This is confusing, so you will have to pay attention.

	Assume for contradiction there is an affine variety $V$
	and an isomorphism
	\[ f : X \to V. \]
	Then taking the pullback we get a ring isomorphism
	\[ f^\ast : \OO_V(V) \to \OO_X(X) = \CC[x,y]. \]
	Now let $\OO_V(V) = \CC[a,b]$ where $f^\ast(a) = x$, $f^\ast(b) = y$.
	In particular, we actually have to have $V \cong \Aff^2$.

	Now in the \emph{affine} variety $V$ we can take $\VV(a)$ and $\VV(b)$.
	These have nonempty intersection since $(a,b)$
	is a maximal ideal in $\OO_V(V)$.
	Call this point $q$.  Now let $p = f\inv(q)$.
	Then by the previous question, $p$ should have vanish
	on both the regular functions $x$ and $y$ in $\OO_X(X)$.
	But there is no point in $X$ with this property, which is a contradiction.
\end{example}

\section\problemhead
\begin{problem}
	Consider the map 
	\[ \Aff^1 \to \VV(y^2-x^3) \subseteq \Aff^2
		\quad\text{by}\quad t \mapsto (t^2, t^3). \]
	Show that it is a morphism of varieties, but it is not an isomorphism.
\end{problem}

\begin{dproblem}
	Show that every projective variety has a neighborhood
	which is isomorphic to an affine variety.
	In this way, ``projective varieties are locally affine''.
	\begin{hint}
		Use the standard affine charts.
	\end{hint}
\end{dproblem}

\begin{problem}
	Let $V$ be a affine variety
	and let $W$ be a irreducible projective variety.
	Prove that $V \cong W$ if and only
	if $V$ and $W$ are a single point.
	\begin{hint}
		Examine the global regular functions.
	\end{hint}
	\begin{sol}
		If they were isomorphic, we would have $\OO_V(V) \cong \OO_W(W)$.
		For irreducible projective varieties, $\OO_W(W) \cong \CC$,
		while for affine varieties $\OO_V(V) \cong \CC[V]$.
		Thus we conclude $V$ must be a single point.
	\end{sol}
\end{problem}
