\chapter{Locally ringed spaces and schemes}

\section{Morphisms of sheaves}
First, recall that a sheaf is a contravariant functor (pre-sheaf)
with extra conditions. In light of this, it is not hard to guess
the definition of a morphism of pre-sheaves:
\begin{definition}
	A \vocab{morphism of (pre-)sheaves} $\alpha : \SF \to \SG$ on the same
	space $X$ is a \textbf{natural transformation} of the underlying functors.
	Isomorphism of sheaves is defined in the usual way.
\end{definition}
\begin{ques}
	Show that this amounts to: for each $U \subseteq X$ we need to specify
	a morphism $\alpha_U : \SF(U) \to \SG(U)$ such that the diagram
	\begin{diagram}
		\SF(U_2) & \rTo^{\alpha_{U_2}} & \SG(U_2) \\
		\dTo^{\res_{U_1, U_2}} && \dTo_{\res_{U_1, U_2}} \\
		\SF(U_1) & \rTo_{\alpha_{U_1}} & \SG(U_1)
	\end{diagram}
	commutes any time that $U_1 \subseteq U_2$.
\end{ques}
However, in the sheaf case we like stalks more than sections because
they are theoretically easier to think about.
And in fact:
\begin{proposition}
	[Morphisms determined by stalks]
	A morphism of sheaves $\alpha : \SF \to \SG$ induces a morphism of stalks
	\[ \alpha_p : \SF_p \to \SG_p \]
	for every point $p \in X$.
	Moreover, the sequence $(\alpha_p)_{p \in X}$ determines $\alpha$ uniquely.
\end{proposition}
\begin{proof}
	The morphism $\alpha_p$ itself is just
	$(s, U) \xmapsto{\alpha_p} (\alpha_U(s), U)$.
	\begin{ques}
		Show this is well-defined.
	\end{ques}
	Now suppose $\alpha , \beta : \SF \to \SG$ satisfy $\alpha_p = \beta_p$
	for every $p$. We want to show $\alpha_U(s) = \beta_U(s)$
	for every $s \in \SF(U)$.
	\begin{ques}
		Verify this using the description of sections
		as sequences of germs. \qedhere
	\end{ques}
\end{proof}
Thus a morphism of sheaves can be instead modelled as a morphism
of all the stalks. We will see later on that this viewpoint is quite useful.

\section{Local rings, and locally ringed spaces}
\prototype{Smooth functions $f$ on $X \subseteq \RR^n$ have invertible
germs at $p$ unless $f(p) = 0$.}

\section{Morphisms of (locally) ringed spaces}
Finally, it remains to define a morphism of locally ringed space.
To do this we have to build up in several steps.

\begin{remark}
	I always secretly felt that one can probably get away with just
	suspending belief and knowing ``there is a reasonable definition
	of morphisms of locally ringed spaces''.
	Daring readers are welcome to try this approach.
\end{remark}

\subsection*{Morphisms of ringed spaces}
We talked about the morphisms of locally ringed spaces
when we discussed projective varieties.
We want to copy the same construction here.

The idea was as follows: to ``preserve'' the structure via $f : X \to Y$,
we took every regular function on $Y$ and made it into a function on $X$
by composing it with $f$.
This gave us a ring homomorphism $f^\ast : \OO_Y(U) \to \OO_X(f\pre(U))$
for every open set $U$ in $Y$.

Unfortunately, in the general case we don't have a notion of function,
so ``composing by $f$'' doesn't make sense.
The solution to this is to \emph{include all the ring homomorphisms $f^\ast$}
in the data for a morphism $f : X \to Y$;
we package this information using the morphism of sheaves.

Here are the full details.
Up above we only defined a morphism of sheaves on the same space,
while right now we have a sheaf on $X$ and $Y$.
So we have to first ``push'' the sheaf on $X$ into a sheaf on $Y$ by
\[ U \mapsto \OO_X\left( f\pre(U) \right). \]
This is important enough to have a name:
\begin{definition}
	Let $\SF$ be a sheaf on $X$, and $f : X \to Y$ a continuous map.
	The \vocab{pushforward sheaf} $f_\ast \SF$ on $Y$ is defined by
	\[ (f_\ast \SF)(U) = \SF(f\pre(U)) \qquad \forall U \subseteq Y. \]
	This makes sense, since $f\pre(U)$ is open in $X$.
\end{definition}
Picture:
\begin{diagram}
	X && \rTo^f & Y && X && \rTo^f & Y \\
	\Opens(X)\op && \lTo^{f\pre} & \Opens(Y)\op
	&& f\pre(U) && \lMapsto^{f\pre} & U & \\
	& \rdMapsto(3,2)_{\SF} && \dMapsto_{f_\ast \SF}
		&& & \rdMapsto(3,2)_{\SF} && \dMapsto_{f_\ast \SF} & \\
	&&& \catname{Rings}
		&& && \SF(f\pre(U)) & = (f_\ast \SF)(U) & \in \catname{Rings}
\end{diagram}
I haven't actually checked that $f_\ast \SF$ is a sheaf
(as opposed to a pre-sheaf), but this isn't hard to do.
Also, as $f\pre$ is a functor $\Opens(Y)\op \to \Opens(X)\op$,
the pushforward $f_\ast\SF$ is
just the composition of these two functors.
\begin{ques}
	Technically $f_\ast\SF$ is supposed to be a
	functor $\Opens(Y)\op \to \catname{Rings}$, so it also needs
	to come with restriction arrows. What are they?
\end{ques}

Now that we moved both sheaves onto $Y$,
we mimic the construction with quasi-projective varieties
by asking for a sheaf morphism from $\OO_Y$ to $f_\ast \OO_X$:
\begin{definition}
	A \vocab{morphism of ringed spaces} $f : (X, \OO_X) \to (Y, \OO_Y)$
	consists of a continuous map of topological spaces $f : X \to Y$,
	plus additionally a morphism of sheaves $f^\ast : \OO_Y \to f_\ast \OO_X$.
\end{definition}
\begin{exercise}
	(Mandatory) Check that this coincides with our work
	on baby ringed spaces if we choose $f^\ast$ to be the pullback.
\end{exercise}

\subsection*{Morphisms of locally ringed spaces}
Last step!
Suppose now that $(X, \OO_X)$ and $(Y, \OO_Y)$ are locally ringed spaces.
Thus we need to deal with some information about the stalks.

Given a morphism of ringed spaces $f : (X, \OO_X) \to (Y, \OO_Y)$,
we can actually use the $f^\ast_U$ above to induce maps on the stalks,
as follows. For a point $p \in X$, construct the map
\begin{diagram}
	f^\ast_p :& \OO_{Y, f(p)} & \rTo && \OO_{X, f(p)} \\
	& (s, U) & \rMapsto && (f^\ast_U(s), f\pre(U)).
\end{diagram}
where $s \in \OO_Y(U)$.

\begin{definition}
	Let $R$ and $S$ be local rings with maximal ideals $\mm_R$ and $\mm_S$.
	A \vocab{morphism of local rings} is a homomorphism of rings
	$\psi : R \to S$ such that $\psi\pre(\mm_S) = \mm_R$.
\end{definition}
\begin{definition}
	A \vocab{morphism of locally ringed spaces}
	is a morphism of ringed spaces $f : (X, \OO_X) \to (Y, \OO_Y)$ such that
	for every point $p$ the induced map of stalks is a morphism of local rings.
\end{definition}
Recalling that $\mm_{X,p}$ is the maximal ideal of $\OO_X$ at $p$,
the new condition is saying that
\[ (f^\ast_p)\pre (\mm_{X,p}) = \mm_{Y,f(p)}. \]
Concretely, if $g$ is a function vanishing on $f(p)$,
then its pullback $f^\ast_U(g)$ vanishes on $p$.

This completes the definition of a morphism of locally ringed spaces.
Isomorphisms of (locally) ringed spaces are defined in the usual way.


\section{Schemes}
\begin{definition}
	A \vocab{scheme} is a locally ringed space $(X, \OO_X)$
	with an open cover $\{U_\alpha\}$ of $X$
	such that each pair $(U_\alpha, \OO_X \restrict{U_\alpha})$
	is isomorphic to an affine scheme.
\end{definition}
Hooray!


\section{Projective scheme}
\prototype{Projective varieties, in the same way.}
The most important class of schemes which are not affine are
\emph{projective} schemes.
The complete the obvious analogy:
\[
	\frac{\text{Affine variety}}{\text{Projective variety}}
	= 
	\frac{\text{Affine scheme}}{\text{Projective scheme}}.
\]
Let $S$ be \emph{any} (commutative) graded ring, like $\CC[x_0, \dots, x_n]$.
\begin{definition}
	We define $\Proj S$, the \vocab{projective scheme over $S$}:
	\begin{itemize}
		\ii As a set, $\Proj S$ consists of \emph{homogeneous prime ideals}
		$\pp$ which do not contain $S^+$.
		\ii If $I \subseteq S$ is homogeneous, then
		we let $\Vp(I) = \{ \pp \in \Proj S \mid I \subseteq \pp \}$.
		Then the \vocab{Zariski topology} is imposed by declaring 
		sets of the form $\Vp(I)$ to be closed.
		\ii We now define a pre-sheaf $\SF$ on $\Proj S$ by
		\[ \SF(U) = 
			\left\{ \frac{f}{g} \mid 
			g(\pp) \neq 0 \; \forall \pp \in U \text{ and }
			\deg f = \deg g \right\}.
		\]
		In other words, the rational functions are quotients $f/g$
		where $f$ and $g$ are \emph{homogeneous of the same degree}.
		Then we let \[ \OO_{\Proj S} = \SF\sh \] be the sheafification.
	\end{itemize}
\end{definition}
\begin{definition}
	The \vocab{distinguished open sets} $D(f)$ of the $\Proj S$
	are defined as $\left\{ \pp \in \Proj S : f(\pp) \neq 0 \right\}$,
	as before; these form a basis for the Zariski topology of $\Proj S$.
\end{definition}
Now, we want analogous results as we did for affine structure sheaf.
So, we define a slightly modified localization:
\begin{definition}
	Let $S$ be a graded ring.
	\begin{enumerate}[(i)]
		\ii For a prime ideal $\pp$, let
		\[ S_{(\pp)} = \left\{ \frac fg \mid g(\pp) \neq 0 \text{ and }
			\deg f = \deg g \right\} \]
		denote the elements of $S_\pp$ with ``degree zero''.
		\ii For any homogeneous $g \in S$ of degree $d$, let
		\[ S_{(g)} = \left\{ \frac{f}{g^r} \mid 
			\deg f = r \deg g \right\} \]
		denote the elements of $S_g$ with ``degree zero''.
	\end{enumerate}
\end{definition}

\begin{theorem}
	[On the projective structure sheaf]
	Let $S$ be a graded ring and let $\Proj S$ the associated projective scheme.
	\begin{enumerate}[(a)]
		\ii Let $\pp \in \Proj S$.
		Then $\OO_{\Proj S, \pp} \cong S_{(\pp)}$.
		\ii Suppose $g$ is homogeneous with $\deg g > 0$. Then
		\[ D(g) \cong \Spec S_{(g)} \]
		as locally ringed spaces.
		In particular, $\OO_{\Proj S}(D(g)) \cong S_{(g)}$.
	\end{enumerate}
\end{theorem}
\begin{ques}
	Conclude that $\Proj S$ is a scheme.
\end{ques}

Of course, the archetypal example is that
\[ \Proj \CC[x_0, x_1, \dots, x_n] / I \]
corresponds to the projective subvariety of $\CP^n$
cut out by $I$ (when $I$ is radical).
In the general case of an arbitrary ideal $I$, we
call such schemes \vocab{projective subscheme} of $\CP^n$.
For example, the ``double point'' in  $\CP^1$
is given by $\Proj[x_0,x_1]/(x_0^2)$.

\begin{remark}
	No comment yet on what the global sections of $\OO_{\Proj S}(\Proj S)$ are.
	(The theorem above requires $\deg g > 0$, so we cannot just take $g=1$.)
	One might hope that in general $\OO_{\Proj S}(\Proj S) \cong S^0$
	in analogy to our complex projective varieties, but
	one needs some additional assumptions on $S$ for this to hold.
\end{remark}


\section{Where to go from here}
This chapter concludes the long setup for the definition of a scheme.
Unfortunately, with respect to algebraic geometry this is as much as I have
the patience or knowledge to write about.
So, if you want to actually see how schemes are used in ``real life'',
you'll have to turn elsewhere.

A good reference I happily recommend is \cite{ref:gathmann}.
More intense is \cite{ref:vakil}.
See \Cref{ch:refs} for further remarks.


\section{\problemhead}

\begin{problem}
	Given an affine scheme $X = \Spec R$,
	show that there is a unique morphism of schemes $X \to \Spec \ZZ$,
	and describe where it sends points of $X$.
	\begin{hint}
		Use the proof that $\catname{AffSch} \simeq \catname{CRing}$.
	\end{hint}
	\begin{sol}
		$\pp$ gets sent to the characteristic of the field $\OO_{X,\pp} / \mm_{X,\pp}$.
	\end{sol}
\end{problem}


