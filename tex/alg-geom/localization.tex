\chapter{Localization}
Before we proceed on to defining an affine scheme,
we will take the time to properly cover one more algebraic construction
that of a \emph{localization}.
This is mandatory because when we define a scheme,
we will find that all the sections and stalks
are actually obtained using this construction.

One silly slogan might be:
\begin{moral}
	Localization is the art of adding denominators.
\end{moral}
You may remember that when we were working with affine varieties,
there were constantly expressions of the form
$\left\{ \frac{f}{g} \mid g(p) \ne 0 \right\}$
and the like.
The point is that we introduced a lot of denominators.
Localization will give us a concise way of doing this in general.

\emph{Notational note}:
moving forward we'll prefer to denote rings by $A$, $B$, \dots,
rather than $R$, $S$, \dots.

\section{Spoilers}
Here is a preview of things to come,
so that you know what you are expecting.
Some things here won't make sense,
but that's okay, it is just foreshadowing.

Let $V \subseteq \Aff^n$, and for brevity let $R = \CC[V]$ be its coordinate ring.
We saw in previous sections how to compute $\OO_V(D(g))$
and $\OO_{V, p}$ for $p \in V$ a point.
For example, if we take $\Aff^1$ and consider a point $p$, then
$\OO_{\Aff^1}(D(x-p)) = \left\{ \frac{f(x)}{(x-p)^n} \right\}$
and $\OO_{\Aff^1, p} = \left\{ \frac{f(x)}{g(x)} \mid g(p) \ne 0 \right\}$.
More generally, we had
\begin{align*}
	\OO_{V}(D(g)) &= \left\{ \frac{f}{g^n} \mid f \in R \right\}
		\quad\text{by \Cref{thm:reg_func_distinguish_open}} \\
	\OO_{V,p} &= \left\{ \frac{f}{g} \mid f,g \in R, g(p) \ne 0 \right\}
		\quad\text{by \Cref{thm:stalks_affine_var}}.
\end{align*}

We will soon define something called a localization,
which will give us a nice way of expressing the above:
if $R = \CC[V]$ is the coordinate ring, then
the above will become abbreviated to just
\begin{align*}
	\OO_{V}(D(g)) &= R_{g} \\
	\OO_{V, p} &= R_{\mm} \quad \text{where } \{p\} = \VV(\mm).
\end{align*}
The former will be pronounced 
``$R$ localized away from $g$''
while the latter will be pronounced
``$R$ localized at $\mm$''.

Even more generally,
next chapter we will throw out the coordinate ring $R$
altogether and replace it with a general commutative ring $A$
(which are still viewed as functions).
We will construct a ringed space called $X = \Spec A$,
whose elements are prime ideals of $A$
and is equipped with the Zariski topology and a sheaf $\OO_X$.
It will turn out that, in analogy to what we had before,
\begin{align*}
	\OO_X(D(f)) &= A_f \\
	\OO_{X,\pp} &= A_\pp
\end{align*}
for any element $f \in A$ and prime ideal $\pp \in \Spec A$.
Thus just as with complex affine varieties,
localizations will give us a way to more or less
describe the sheaf $\OO_X$ completely.

\section{The definition}
\begin{definition}
	Let $A$ be a ring and $S \subset A$ any subset
	closed under multiplication.
	Then the \vocab{localization of $A$ at $S$}, denoted $S\inv A$,
	is defined as the set of fractions
	\[ \left\{ a/s \mid a \in A, s \in S \right\} \]
	where we declare two fractions $a_1 / s_1 = a_2 / s_2$ 
	to be equal if
	\[ \exists s \in S : \quad s(a_1s_2 - a_2s_1) = 0. \]
	Addition and multiplication in this ring
	are defined in the obvious way.
\end{definition}
In particular, if $0 \in S$ then $S\inv A$ is the zero ring.
So we usually only take situations where $0 \notin S$.

We give in brief now two examples which will be
motivating forces for the construction of the affine scheme.
\begin{example}
	[Localizations of {$\CC[x]$}]
	Let $A = \CC[x]$.
	\begin{enumerate}[(a)]
		\ii Suppose we let $S = \left\{ 1, x, x^2, x^3, \dots \right\}$
		be the powers of $x$.
		Then
		\[ S\inv A = \left\{ \frac{f(x)}{x^n}
			\mid f \in \CC[x], n \in \ZZ_{\ge 0} \right\}.  \]
		In other words, we get the Laurent polynomials in $x$.

		You might recognize this as
		\[ \OO_V(U) \text{ where } V = \Aff^1, \; U = V \setminus \{0\}. \]
		i.e.\ the sections of the punctured line.
		In line with the ``hyperbola effect'',
		this is also expressible as $\CC[x,y] / (xy-1)$.

		\ii Let $p \in \CC$.
		Suppose we let $S = \left\{ g(x) \mid g(p) \ne 0 \right\}$,
		i.e.\ we allow any denominators where $g(p) \ne 0$.
		Then
		\[ S\inv A = \left\{ \frac{f(x)}{g(x)}
			\mid f,g \in \CC[x], g(p) \ne 0 \right\}.  \]
		You might recognize this is as the stalk $\OO_{\Aff^1, p}$.
		This will be important later on.
	\end{enumerate}
\end{example}

\begin{remark}
	[Why the extra $s$?]
	The reason we need the condition $s(a_1s_2 - a_2s_1) = 0$
	rather than the simpler $a_1s_2 - a_2s_1 = 0$ is that
	otherwise the equivalence relation may fail to be transitive.
	Here is a counterexample: take
	\[ A = \Zc{12} \qquad S = \{ 2, 4, 8 \}. \]
	Then we have for example
	\[ \frac12 = \frac24 = \frac64 = \frac32. \]
	So we need to have $\frac12=\frac32$ which is only true
	with the first definition.

	Of course, if $A$ is an integral domain (and $0\notin S$)
	then this is a moot point.
\end{remark}

\begin{example}
	[Field of fractions]
	Let $A$ be an integral domain and $S = A \setminus \{0\}$.
	Then $S\inv A$ is the field of fractions of $A$.
\end{example}


Here are the two special cases of localization we will need the most.

\section{Localization away from an element}
\prototype{$\CC[x]_{x}$ is Laurent polynomials.}
\begin{definition}
	For $g \in R$, we define the \vocab{localization of $R$ away from $g$},
	denoted $R_g$ to be $\{1, g, g^2, g^3, \dots\}\inv R$.
	(Note that $\left\{ 1, g, g^2, \dots \right\}$ is multiplicatively closed.)
\end{definition}

\begin{example}
	[Localization at an element, algebraic geometry flavored]
	We saw that if $A$ is the coordinate ring of a variety,
	then $A_g$ is interpreted geometrically as $\OO_V(D(g))$.
	Here are some special cases:
	\begin{enumerate}[(a)]
		\ii As we saw, if $A = \CC[x]$,
		then $A_{x} = \left\{ \frac{f(x)}{x^n} \right\}$
		consists of Laurent polynomials.

		\ii Let $A = \CC[x,y,z]$.
		Then \[ A_x = \left\{ \frac{f(x,y,z)}{x^n} \mid
			f \in \CC[x,y,z], \; n \ge 0 \right\} \]
		is rational functions whose denominators are powers of $x$.

		\ii Let $A = \CC[x,y]$.
		If we localize away from $y-x^2$ we get
		\[ A_{y-x^2} = \left\{ \frac{f(x,y)}{(y-x^2)^n} \mid
			f \in \CC[x,y], \; n \ge 0 \right\} \]
		By now you should recognize this as $\OO_{\Aff^2}(D(y-x^2))$.
	\end{enumerate}
\end{example}

\begin{example}
	[An example with zero-divisors]
	Let $A = \CC[x,y] / (xy)$
	(which intuitively is the coordinate ring of two axes).
	Suppose we localize at $x$:
	equivalently, allowing denominators of $x$.
	Since $xy = 0$ in $A$, we now have $0 = x\inv (xy) = y$,
	so $y = 0$ in $A$, and thus $y$ just goes away completely.
	From this we get a ring isomorphism
	\[ A_{(x)} \cong \CC[x]_{(x)}. \]
\end{example}
\todo{commute with quotient}

\begin{example}
	[Some arithmetic examples of localizations]
	\listhack
	\begin{enumerate}[(a)]
		\ii Let $A = \ZZ$. We localize away from $6$:
		\[ A_{6} = \left\{ \frac{m}{6^n} \mid m \in \ZZ,
			n \in \ZZ_{\ge 0} \right\}.  \]
		So $A_6$ consist of those rational numbers whose
		denominators have only powers of $2$ and $3$.
		For example, it contains $\frac{5}{12} = \frac{15}{36}$.

		\ii Here is a more confusing example:
		if we localize $A = \Zc{60}$ away from the element $5$,
		we get $A_5 \cong \Zc{12}$.
		You should try to think about why this is the case.
	\end{enumerate}
\end{example}

\section{Localization at a prime ideal}
\label{sec:localize_prime_ideal}
\begin{definition}
	If $A$ is a ring and $\pp$ is a prime ideal, then we define
	\[ A_\pp \defeq \left( A \setminus \pp \right)^{-1} A. \]
	This is called the \vocab{localization at $\pp$}.
\end{definition}
\begin{ques}
	Why is $S = A \setminus \pp$ multiplicatively closed
	in the above definition?
\end{ques}

Warning: this notation sort of conflicts with the previous one.
The ring $A_\pp$ is the localization with multiplicative set $A \setminus \pp$,
while $A_f$ is the localization with multiplicative set $\{1,f,f^2,\dots\}$;
these two are quite different beasts!
In $A_\pp$ the subscript denotes the \emph{forbidden} denominators;
in $A_f$ the subscript denotes the \emph{allowed} denominators.

This special case is important because we will see that
stalks of schemes will all be of this shape.
In fact, the same was true for affine varieties too.
\begin{example}
	[Relation to affine varieties]
	Let $V \subseteq \Aff^n$, let $A = \CC[V]$
	and let $p = (a_1, \dots, a_n)$ be a point.
	Consider the maximal (hence prime) ideal
	\[ \mm = (x_1 - a_1, \dots, x_n - a_n). \]
	Observe that a function $f \in A$ vanishes at $p$
	if and only if $f \pmod{\mm} = 0$, equivalently $f \in \mm$.
	Thus, by \Cref{thm:stalks_affine_var} we can write
	\begin{align*}
		\OO_{V,p} &= \left\{ \frac{f}{g} \mid f,g \in A, g(p) \ne 0 \right\} \\
		&= \left\{ \frac{f}{g} \mid f \in A, g \in A \setminus \mm \right\} \\
		&= \left( A \setminus \mm \right)\inv A = A_\mm.
	\end{align*}
	So, we can also express $\OO_{V,p}$ concisely as a localization.
\end{example}
Consequently, we give several examples in this vein.

\begin{example}
	[Geometric examples of localizing at a prime]
	\listhack
	\begin{enumerate}[(a)]
		\ii We let $\mm$ be the maximal ideal $(x)$ of $A = \CC[x]$.
		Then \[ A_\mm = \left\{ \frac{f(x)}{g(x)} \mid g(0) \ne 0 \right\} \]
		consists of the Laurent polynomials.

		\ii We let $\mm$ be the maximal ideal $(x,y)$ of $A = \CC[x,y]$.
		Then \[ A_\mm = \left\{ \frac{f(x,y)}{g(x,y)} \mid g(0,0) \ne 0 \right\}. \]

		\ii Let $\pp$ be the prime ideal $(y-x^2)$ of $A = \CC[x,y]$.
		Then
		\[ A_\pp = \left\{ \frac{f(x,y)}{g(x,y)} \mid g \notin (y-x^2) \right\}. \]
		This is a bit different from what we've seen before:
		the polynomials in the denominator are allowed to vanish
		at a point like $(1,1)$, as long as they don't vanish on
		\emph{every} point on the parabola.
		This doesn't correspond to any stalk we're familiar with right now,
		but it will later
		(it will be the ``stalk at the generic point of the parabola'').

		\ii Let $A = \CC[x]$ and localize at the prime ideal $(0)$.
		This gives \[ A_{(0)} = \left\{ \frac{f(x)}{g(x)} \mid g(x) \ne 0 \right\}. \]
		This is all rational functions, period.
	\end{enumerate}
\end{example}

\begin{example}
	[Arithmetic examples]
	We localize $\ZZ$ at a few different primes.
	\begin{enumerate}[(a)]
		\ii If we localize $\ZZ$ at $(0)$:
		\[ \ZZ_{(0)} = \left\{ \frac mn \mid n \ne 0 \right\}
			\cong \QQ. \]
		\ii If we localize $\ZZ$ at $(3)$, we get
		\[ \ZZ_{(3)} = \left\{ \frac mn \mid n \ne 3 \right\} \]
		which is the ring of rational numbers
		whose denominators are relatively prime to $3$.
	\end{enumerate}
\end{example}

\begin{example}
	[Field of fractions]
	If $A$ is an integral domain,
	the localization $A_{(0)}$
	is the field of fractions of $A$.
\end{example}


\section{Two mandatory algebra exercises}
\todo{ideals bijection}
\todo{write this}
The previous example illustrates an important observation:
\begin{exercise}
	Let $I \subsetneq R$ be a proper ideal.
	Construct a bijection between the maximal ideals of $R$ containing $I$,
	and the maximal ideals of $R/I$.
\end{exercise}
(\cite{ref:vakil} labels this exercise as
``Essential Algebra Exercise (Mandatory if you haven't done it before)''.)


\section{\problemhead}

\begin{problem}
	Let $A = \Zc{2016}$. Compute $A_{60}$.
\end{problem}

\begin{problem}
	[Injectivity of localizations]
	Let $A$ be a ring and $S \subseteq A$ a multiplicatively closed set.
	Find necessary and sufficient conditions
	for the map $A \mapsto S\inv A$ to be injective.
	\begin{hint}
		Consider zero divisors.
	\end{hint}
	\begin{sol}
		If and only if $S$ has no zero divisors.
	\end{sol}
\end{problem}

\begin{problem}
	Let $A$ be a ring such that $A_\pp$ is an integral domain
	for every prime ideal $\pp$ of $A$.
	Must $A$ be an integral domain?
	\begin{hint}
		No. Imagine two axes.
	\end{hint}
	\begin{sol}
		Take $A = \CC[x,y] / (xy)$.
	\end{sol}
\end{problem}
\todo{problems}
