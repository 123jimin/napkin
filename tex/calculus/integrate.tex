\chapter{Riemann integrals}
\begin{quote}
	\footnotesize
	``Trying to Riemann integrate discontinuous
	functions is kind of outdated.'' \\
	--- Dennis Gaitsgory, \cite{ref:55b}
\end{quote}

We will go ahead and define the Riemann integral,
but we won't do very much with it.
The reason is that the Lebesgue integral is basically better,
so we will define it, check the fundamental theorem of calculus
(or rather, leave it as a problem at the end of the chapter),
and then always use Lebesgue integrals forever after.

\section{Uniform continuity}
\prototype{$f(x) = x^2$ is not uniformly continuous on $\RR$,
but functions on compact sets are always uniformly continuous.}
\begin{definition}
	Let $f \colon M \to N$ be a continuous
	map between two metric spaces.
	We say that $f$ is \vocab{uniformly continuous} if
	for all $\eps > 0$ there exists a $\delta > 0$ such that
	\[ d_M(p,q) < \delta \implies d_N(f(p), f(q)) < \eps. \]
\end{definition}
This difference is that given an $\eps > 0$ we must specify a $\delta > 0$
which works for \emph{every} choice $p$ and $q$ of inputs;
whereas usually $\delta$ is allowed to depend on $p$ and $q$.
(Also, this definition can't be ported
to a general topological space.)

\begin{example}
	[Uniform continuity failure]
	\listhack
	\begin{enumerate}[(a)]
		\ii The function $f \colon \RR \to \RR$ by $x \mapsto x^2$
		is not uniformly continuous.
		Suppose we take $\eps = 0.1$ for example.
		There is no $\delta$ such that if $|x-y| < \delta$
		then $|x^2-y^2| < 0.1$,
		since as $x$ and $y$ get large,
		the function $f$ becomes increasingly sensitive
		to small changes.

		\ii The function $(0,1) \to \RR$ by $x \mapsto x\inv$
		is not uniformly continuous.

		\ii The function $\RR_{>0} \to \RR$ by $x \mapsto \sqrt x$
		does turn out to be uniformly continuous
		(despite having unbounded derivatives!).
		Indeed, you can check that the assertion
		\[ \left\lvert x-y  \right\rvert < \eps^2
			\implies \left\lvert \sqrt x - \sqrt y \right\rvert
			< \eps \]
		holds for any $x, y, \eps > 0$.
	\end{enumerate}
\end{example}

The good news is that in the compact case all is well.
\begin{theorem}
	[Uniform continuity free for compact spaces]
	Let $M$ be a compact metric space.
	Then any continuous map $f \colon M \to N$ is
	also uniformly continuous.
\end{theorem}
\begin{proof}
	Assume for contradiction there is some bad $\eps > 0$.
	Then taking $\delta = 1/n$,
	we find that for each integer $n$
	there exists points $p_n$ and $q_n$
	which are within $1/n$ of each other,
	but are mapped more than $\eps$ away from each other by $f$.
	In symbols, $d_M(p_n, q_n) < 1/n$ but $d_N(f(p_n), f(q_n)) > 1/n$.

	By compactness of $M$,
	we can find a convergent subsequences
	$p_{i_1}$, $p_{i_2}$, \dots\ converging to some $x \in M$..
	Since the $q_{i_n}$ is within $1/i_n$ of $p_{i_n}$,
	it ought to converge as well, to the same point $x \in M$.
	Then the sequences $f(p_{i_n})$ and $f(q_{i_n})$
	should both converge to $f(x) \in N$,
	but this is impossible as they are always $\eps$
	away from  each other.
\end{proof}
This means for example that $x^2$ viewed
as a continuous function $[0,1] \to \RR$ is automatically
uniformly continuous.
Man, isn't compactness great?

\section{Dense sets and extension}
\prototype{Functions from $\QQ \to N$ extend to $\RR \to N$
if they're uniformly continuous and $N$ is complete.
See also counterexamples below.}

\begin{definition}
	Let $S$ be a subset (or subspace) of a topological space $X$.
	Then we say that $S$ is \vocab{dense}
	if every open subset of $X$ contains a point of $S$.
\end{definition}

\begin{example}
	[Dense sets]
	\listhack
	\begin{enumerate}[(a)]
		\ii $\QQ$ is dense in $\RR$.
		\ii In general, any metric space $M$ is dense
		in its completion $\ol M$.
	\end{enumerate}
\end{example}

Dense sets lend themselves to having functions completed.
The idea is that if I have a continuous function
$f \colon \QQ \to N$, for some metric space $N$,
then there should be \emph{at most} one way to extend it to a function
$\wt f \colon \RR \to N$.
For we can approximate each rational number by real numbers:
if I know $f(1)$, $f(1.4)$, $f(1.41)$, \dots\
$\wt f(\sqrt2)$ had better be the limit of this sequence.
So it is certainly unique.

However, there are two ways this could go wrong:
\begin{example}
	[Non-existence of extension]
	\listhack
	\begin{enumerate}[(a)]
		\ii It could be than $N$ is not complete,
		so the limit may not even exist in $N$.
		For example if $N = \QQ$,
		then certainly there is no way to
		extend even the identify function $f \colon \QQ \to N$
		to a function $\wt f \colon \QQ \to N$.

		\ii Even if $N$ was complete, we might run into issues
		where $f$ explodes.
		For example, let $N = \RR$ and define
		\[ f(x) = \frac{1}{x-\sqrt2} \qquad f \colon \QQ \to \RR. \]
		There is also no way to extend this
		due to the explosion of $f$ near $\sqrt2 \notin \QQ$,
		which would cause $\wt f(\sqrt2)$ to be undefined.
	\end{enumerate}
\end{example}
However, the way to fix this is to require $f$ to be uniformly continuous,
and in that case we do get a unique extension.

\begin{theorem}
	[Extending uniformly continuous functions]
	Let $M$ be a metric space, $N$ a \emph{complete} metric space,
	and $S$ a dense subspace of $M$.
	Suppose $\psi \colon S \to N$ is a \emph{uniformly} continuous function.
	Then there exist as unique continuous function $\wt \psi \colon M \to N$
	such that the digaram
	\begin{center}
	\begin{tikzcd}
		M \ar[r, "\wt \psi"] & N \\
		S \ar[u, hook] \ar[ru, "\psi"'] &
	\end{tikzcd}
	\end{center}
	commutes.
\end{theorem}
\begin{proof}
	[Outline of proof]
	As mentioned in the discussion,
	each $x \in M$ can be approximated by a
	sequence $x_1$, $x_2$, \dots\ in $S$ with $x_i \to x$.
	The two main hypotheses, completeness and uniform continuity,
	are now used:
	\begin{exercise}
		Prove that $\psi(x_1)$, $\psi(x_2)$, \dots\ converges in $N$
		by using uniform continuity to show that it is Cauchy,
		and then appealing to completeness of $N$.
	\end{exercise}
	Hence we define $\wt \psi(x)$ to be the limit of that sequence;
	this doesn't depend on the choice of sequence,
	and one can use sequential continuity to show $\wt \psi$ is continuous.
\end{proof}

\section{Defining the Riemann integral}
Extensions will allow us to define the Riemann integral.
I need to introduce a bit of notation so bear with me.

\begin{definition}
	Let $[a,b]$ be a closed interval.
	\begin{itemize}
		\ii We let $C^0([a,b])$ denote the set of
		continuous functions on $[a,b] \to \RR$.
		\ii We let $R([a,b])$ denote the set of
		\textbf{rectangle functions} on $[a,b] \to \RR$.
		These functions which are constant on the intervals
		$[t_0,t_1)$, $[t_1, t_2)$, $[t_2, t_3)$, \dots, $[t_{n-2}, t_{n-1})$, %chktex 9
		and also $[t_{n-1}, t_n]$,
		for some $a = t_0 < t_1 < t_2 < \dots < t_n = b$.
		\ii We let $M([a,b]) = C^0([a,b]) \cup R([a,b])$.
	\end{itemize}
\end{definition}
Warning: only $C^0([a,b])$ is common notation,
and the other two are made up.

See picture below for a typical a rectangle function.
(It is irritating that we have to officially assign a single
value to each $t_i$,
even though there are naturally two values we want to use,
and so we use the convention of letting
the left endpoint be closed).
\begin{center}
\begin{asy}
	import graph;
	graph.xaxis();
	graph.yaxis();
	label("$a=t_0$", (-3,0), dir(-90), red);
	label("$b=t_4$", ( 3,0), dir(-90), red);
	pen b = blue + 1.2;
	draw( (-3,3)--(-1.2,3), b );
	draw( (-1.2,1)--(0.8,1), b );
	draw( (0.8,4)--(2,4), b );
	draw( (2,2.5)--(3,2.5), b );

	draw( (-3,3)--(-3,0), red );
	draw( (-1.2,3)--(-1.2,0), dotted+heavycyan );
	draw( (0.8,4)--(0.8,0), dotted+heavycyan );
	draw( (2,4)--(2,0), dotted+heavycyan );
	draw( (3,2.5)--(3,0), red );
	label("$t_1$", (-1.2,0), dir(-90), heavycyan);
	label("$t_2$", (0.8,0), dir(-90), heavycyan);
	label("$t_3$", (2,0), dir(-90), heavycyan);

	dotfactor *= 2;
	dot( (-3,3), blue ); // x = a
	opendot( (-1.2, 3), blue );
	opendot( (0.8, 1), blue );
	opendot( (2, 4), blue );
	dot( (-1.2,1), blue );
	dot( (0.8,4), blue );
	dot( (2,2.5), blue );
	dot( (3, 2.5), blue ); // x = b
\end{asy}
\end{center}

\begin{definition}
	We can impose a metric on $M([a,b])$
	by defining
	\[ d(f,g) = \sup_{x \in [a,b]} \left\lvert f(x) - g(x) \right\rvert. \]
\end{definition}

Now, there is a natural notion of integral
for rectangle functions: just sum up the obvious rectangles!
Officially, this is the expression
\[ f(a)(t_1-a) + f(t_1)(t_2-t_1) +
	+ f(t_2) \left( t_3 - t_2 \right)
	+ \dots + f(t_n) \left( b - t_n \right).  \]
We denote this function by
\[ \Sigma : R([a,b]) \to \RR. \]

\begin{theorem}
	[The Riemann integral]
	\label{thm:gaitsgory_riemann}
	There exists a unique continuous map
	\[ \textstyle{\int_a^b} \colon M([a,b]) \to \RR \]
	such that the diagram
	\begin{center}
	\begin{tikzcd}
		M([a,b]) \ar[r, "\int_a^b"] & \RR \\
		R([a,b]) \ar[u, hook] \ar[ru, "\Sigma"'] &
	\end{tikzcd}
	\end{center}
	commutes.
\end{theorem}
\begin{proof}
	We want to apply the extension theorem,
	so we just have to check a few things:
	\begin{itemize}
		\ii We claim $R([a,b])$ is a dense subset of $M([a,b])$.
		In other words, for any continuous $f \colon [a,b] \to \RR$
		and $\eps > 0$,
		we want there to exist a rectangle function
		that approximates $f$ within $\eps$.

		This follows by uniform continuity.
		We know there exists a $\delta > 0$ such
		that whenever $|x-y| < \delta$ we have $|f(x)-f(y)| < \eps$.
		So as long as we select a rectangle function
		whose rectangles have width less than $\delta$,
		and such that the upper-left corner of each rectangle
		lies on the graph of $f$, then we are all set. 

		\begin{center}
		\begin{asy}
			import graph;
			real f(real x) { return 3 + (x-2)*(x-2)*(x+5) / 35; }
			draw( (5, f(5))--(5, 0), red+dotted );
			for (int i=0; i<10; ++i) {
				filldraw(box( (i-5, 0), (i-4, f(i-5)) ),
					opacity(0.1)+lightgreen, deepgreen);
				dot( (i-5, f(i-5)), deepgreen);
			}
			draw(graph(f,-5.2,5.2,operator ..), blue, Arrows(TeXHead));
			graph.xaxis();
			graph.yaxis();
			label("$a$", (-5,0), dir(-90), blue);
			label("$b$", (5,0), dir(-90), blue);
		\end{asy}
		\end{center}


		\ii The ``add-the-rectangles'' map $\Sigma \colon R([a,b]) \to \RR$
		is \emph{uniformly} continuous.
		Actually this is pretty obvious:
		if two rectangle functions $f$ and $g$
		have $d(f,g) < \eps$,
		then $d(\Sigma f, \Sigma g) < \eps(b-a)$.

		\ii $\RR$ is complete.\qedhere
	\end{itemize}
\end{proof}

\section{Meshes}
The above definition might seem fantastical, overcomplicated,
hilarious, or terrible, depending on your taste.
But if you unravel it, it's really the picture you are used to.
What we have done is taking every continuous function $f \colon [a,b] \to \RR$
and showed that it can be approximated by a rectangle function
(which we phrased as a dense inclusion).
Then we added the area of the rectangles.
Nonetheless, we will give a definition that's
more like what you're used to seeing in other places.
\begin{definition}
	A \emph{tagged partition} $P$ of $[a,b]$
	consists of a partition of $[a,b]$ into $n$ intervals,
	with a point $\xi_i$ in the $n$th interval, denoted
	\[ a = t_0 < t_1 < t_2 < \dots < t_n = b
		\qquad\text{and}\qquad \xi_i \in [t_{i-1}, t_i]
		\quad \forall \; 1 \le i \le n.  \]
	The \emph{mesh} of $P$ is the width
	of the longest interval, i.e.\ $\max_i(t_i - t_{i-1})$.
\end{definition}

Of course the point of this definition
is that we add the rectangles,
but the $\xi_i$ are the sample points.

\begin{center}
\begin{asy}
	import graph;
	real f(real x) { return 3 + (x-2)*(x-2)*(x+5) / 35; }
	real[] t = {-4.4, -3.3, -0.2, 1.3, 3.5, 4.5};
	filldraw(box((-5,0), (-4.1,f(t[0]))), opacity(0.1)+lightcyan, heavycyan );
	filldraw(box((-4.1,0), (-2,f(t[1]))), opacity(0.1)+lightcyan, heavycyan );
	filldraw(box((-2,0), (0.6,f(t[2]))), opacity(0.1)+lightcyan, heavycyan );
	filldraw(box((0.6,0), (1.9,f(t[3]))), opacity(0.1)+lightcyan, heavycyan );
	filldraw(box((1.9,0), (3.7,f(t[4]))), opacity(0.1)+lightcyan, heavycyan );
	filldraw(box((3.7,0), (5,f(t[5]))), opacity(0.1)+lightcyan, heavycyan );
	for (int i=0; i<=5; ++i) { dot( (t[i], f(t[i])), blue ); }
	draw(graph(f,-5.2,5.2,operator ..), blue, Arrows(TeXHead));
	label("$\xi$", (t[1], f(t[1])), dir(90), blue);
	graph.xaxis();
	graph.yaxis();
	label("$a$", (-5,0), dir(-90), blue);
	label("$b$", (5,0), dir(-90), blue);
\end{asy}
\end{center}

\begin{theorem}
	[Riemann integral]
	Let $f \colon [a,b] \to \RR$ be continuous.
	Then
	\[ \int_a^b f(x) \; dx
		= \lim_{\substack{\text{$P$ tagged partition}
				\\ \opname{mesh} P \to 0}}
			\left( \sum_{i=1}^n f(\xi_i) (t_i - t_{i-1}) \right). \]
	Here the limit means that we can take any sequence
	of partitions whose mesh approaches zero.
\end{theorem}
\begin{proof}
	The right-hand side corresponds to the areas
	of some rectangle functions $g_1$, $g_2$, \dots
	with increasingly narrow rectangles.
	As in the proof \Cref{thm:gaitsgory_riemann},
	as the meshes of those rectangles approaches zero,
	by uniform continuity, we have $d(f, g_n) \to 0$ as well.
	Thus by continuity in the diagram of \Cref{thm:gaitsgory_riemann},
	we get $\lim_n \Sigma(g_n) = \int(f)$ as needed.
\end{proof}

Combined with the mean value theorem,
this can be used to give a short proof of
the fundamental theorem of calculus
for functions $f$ with a continuous derivative.
The idea is that for any choice of partition
$a \le t_0 < t_1 < t_2 < \dots < t_n \le b$,
using the Mean Value Theorem it should be possible
to pick $\xi_i$ in each interval to match
with the slope of the secant:
at which point the areas sum to the total change in $f$.
We illustrate this situation with three points,
and invite the reader to fill
in the details as \Cref{thm:FTC}.

\begin{center}
\begin{asy}
	size(9cm);
	import graph;
	graph.xaxis("$x$");
	graph.yaxis("$y$");
	real f(real x) { return x*x/2+0.4; }
	pair P(real x) { return (x, f(x)); }
	draw(graph(f,-2,2,operator ..), blue, Arrows(TeXHead));
	label("$f$", (2, f(2)), dir(-45), blue);
	for (real i=-1.3; i<=2; ++i) {
		if (i < 1) { draw(P(i)--P(i+1), deepgreen, EndArrow(TeXHead), Margins); }
		dot(P(i), deepgreen);
	}
	real m1 = f(-0.3) - f(-1.3);
	real m2 = f(0.7) - f(-0.3);
	real m3 = f(1.7) - f(0.7);
	real r = 0.4;
	void draw_tangent(real m) {
		dot(P(m), red);
		draw((m-r, f(m)-r*m)--(m+r, f(m)+r*m), red);
	}
	draw_tangent(m1);
	draw_tangent(m2);
	draw_tangent(m3);
	draw(P(-1.3)--(-1.3,0), deepgreen+dashed);
	draw(P(-0.3)--(-0.3,0), deepgreen+dashed);
	draw(P( 0.7)--( 0.7,0), deepgreen+dashed);
	draw(P( 1.7)--( 1.7,0), deepgreen+dashed);
	label("$t_0$", (-1.3,0), dir(-90), deepgreen);
	label("$t_1$", (-0.3,0), dir(-90), deepgreen);
	label("$t_2$", ( 0.7,0), dir(-90), deepgreen);
	label("$t_3$", ( 1.7,0), dir(-90), deepgreen);
	draw(P(m1)--(m1,0), red+dashed);
	draw(P(m2)--(m2,0), red+dashed);
	draw(P(m3)--(m3,0), red+dashed);
	label("$\xi_1$", (m1,0), dir(-90), red+fontsize(8pt));
	label("$\xi_2$", (m2,0), dir(-90), red+fontsize(8pt));
	label("$\xi_3$", (m3,0), dir(-90), red+fontsize(8pt));
	draw(P(-1.3)--P(1.7), blue+dashed, EndArrow, Margins);
	label("Net change", 0.3*P(-1.3)+0.7*P(1.7), dir(90), blue);
\end{asy}
\end{center}



\section{\problemhead}
\begin{problem}
	Let $f \colon (a,b) \to \RR$ be differentiable
	and assume $f'$ is bounded.
	Show that $f$ is uniformly continuous.
	\begin{hint}
		Contradiction and mean value theorem (again!).
	\end{hint}
\end{problem}

\begin{sproblem}
	[Fundamental theorem of calculus]
	\label{thm:FTC}
	Let $f \colon [a,b] \to \RR$ be continuous,
	differentiable on $(a,b)$,
	and assume the derivative $f'$ extends to a
	continuous function $f' \colon [a,b] \to \RR$.
	Prove that
	\[ \int_a^b f'(x) \; dx = f(b) - f(a). \]
	\begin{hint}
		For every positive integer $n$,
		take a partition where every rectangle has width $w = \frac{b-a}{n}$.
		Use the mean value theorem to construct a tagged partition
		such that the first rectangle has area $f(a+w)-f(a)$,
		the second rectangle has area $f(a+2w) - f(a+w)$, and so on;
		thus the total area is $f(b) - f(a)$.
	\end{hint}
\end{sproblem}

\begin{problem}
	[Improper integrals]
	For each real number $r > 0$,
	evaluate the limit\footnote{If you are not
		familiar with the notation $\eps \to 0^+$,
		you can replace $\eps$ with $1/M$ for $M > 0$,
		and let $M \to \infty$ instead.}
	\[ \lim_{\eps \to 0^+} \int_\eps^1 \frac{1}{x^r} \; dx \]
	or show it does not exist.
	This can intuitively be thought of as
	the ``improper'' integral $\int_0^1 x^{-r} \; dx$;
	it doesn't make sense in our original definition since
	we did not (and cannot) define the integral
	over the non-compact $(0,1]$ %chktex 9
	but we can still consider the integral over $[\eps,1]$
	for any $\eps > 0$.
\end{problem}

\begin{problem}
	Show that
	\[ \lim_{n \to \infty}
		\left( \frac{1}{n+1} + \frac{1}{n+2} + \dots + \frac{1}{2n} \right)
		= \log 2.  \]
	\begin{hint}
		Write this as $\frac1n \sum_{k=1}^n \frac{1}{1+\frac kn}$.
		Then you can interpret it as a rectangle sum
		of a certain Riemann integral.
	\end{hint}
\end{problem}
