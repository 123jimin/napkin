\chapter{Lebesgue integration}
On any measure space $(\Omega, \SA, \mu)$ we can then,
for a function $f \colon \Omega \to [0,\infty]$
define an integral
\[ \int_\Omega f \; d\mu. \]
This integral may be $+\infty$ (even if $f$ is finite).
As the details of the construction won't matter for us later on,
we will state the relevant definitions,
skip all the proofs,
and also state all the properties that we actually care about.
Consequently, this chapter will be quite short.

\section{The definition}
The construction is done in four steps.
\begin{definition}
	If $A$ is a measurable set of $\Omega$,
	then the \vocab{indicator function}
	$\mathbf{1}_A \colon \Omega \to \RR$ is defined by
	\[ \mathbf{1}_A(\omega) = \begin{cases}
			1 & \omega \in A \\
			0 & \omega \notin A.
		\end{cases} \]
\end{definition}

\begin{step}
	[Indicator functions]
	For an indicator function, we require
	\[ \int_\Omega \mathbf{1}_A \; d\mu \defeq \mu(A) \]
	(which may be infinite).
\end{step}
We extend this linearly now for nonnegative functions
which are sums of indicators:
these functions are called \vocab{simple functions}.
\begin{step}
	[Simple functions]
	Let $A_1$, \dots, $A_n$ be a finite collection of measurable sets.
	Let $c_1$, \dots, $c_n$ be either nonnegative real numbers or $+\infty$.
	Then we define
	\[ \int_\Omega \left( \sum_{i=1}^n c_i \mathbf{1}_{A_i} \right) \; d\mu
		\defeq \sum_{i=1}^n c_i \mu(A_i). \]
	If $c_i = \infty$ and $\mu(A_i) = 0$, we treat $c_i \mu(A_i) = 0$.
\end{step}
One can check the resulting sum does not depend
on the representation of the simple function as $\sum c_i \mathbf{1}_{A_i}$.
In particular, it is compatible with the previous step.

Conveniently, this is already enough to define the integral
for $f \colon \Omega \to [0, +\infty]$.
Note that $[0,+\infty]$ can be thought of as a topological space
where we add new open sets $(a,+\infty]$ %chktex 9
for each real number $a$ to our usual basis of open intervals.
Thus we can equip it with the Borel sigma-algebra.\footnote{We
	\emph{could} also try to define a measure on it,
	but we will not: it is a good enough for us
	that it is a measurable space.}
\begin{step}
	[Nonnegative functions]
	For each measurable function $f \colon \Omega \to [0, +\infty]$, let
	\[ \int_\Omega f \; d\mu \defeq
		\sup_{0 \le s \le f} \left( \int_\Omega s \; d\mu \right) \]
	where the supremum is taken over all \emph{simple} $s$
	such that $0 \le s \le f$.
	As before, this integral may be $+\infty$.
\end{step}
One can check this is compatible with the previous definitions.
At this point, we introduce an important term.
\begin{definition}
	A measurable (nonnegative) function
	$f \colon \Omega \to [0, +\infty]$
	is \vocab{absolutely integrable} or \vocab{Lebesgue integrable}
	if $\int_\Omega f \; d\mu < \infty$.
\end{definition}
Warning: I find ``Lebesgue integrable'' to be \emph{really} confusing terminology.
Indeed, \emph{every} measurable function from $\Omega$ to $[0,+\infty]$
can be assigned a Lebesgue integral, it's just that
this integral may be $+\infty$.
So the definition is far more stringent than the name suggests.
Even constant functions can fail to be Lebesgue integrable:
\begin{example}
	[Lebesgue integrable functions]
	The constant function $1$ is \emph{not}
	Lebesgue integrable on $\RR$,
	since $\int_\RR 1 \; d\mu = \mu(\RR) +\infty$.
\end{example}
For this reason, I will usually prefer the term ``absolutely integrable''.
(If it were up to me, I would call it ``finitely integrable'',
and usually do so privately.)

Finally, this lets us integrate general functions.
\begin{definition}
	In general, a measurable function $f \colon \Omega \to [-\infty, \infty]$
	is \vocab{absolutely integrable} or \vocab{Lebesgue integrable} if $|f|$ is.
\end{definition}
Since we'll be using the first word, this is easy to remember:
``absolutely integrable'' requires taking absolute values.

\begin{step}
	[Absolutely integrable functions]
	If $f \colon \Omega \to [-\infty, \infty]$ is absolutely integrable,
	then we define
	\begin{align*}
		f^+(x) &= \max\left( f(x), 0 \right) \\
		f^-(x) &= \min\left( f(x), 0 \right) \\
	\end{align*}
	and set
	\[ \int_\Omega f \; d\mu = \int_\Omega |f^+| \; d\mu
		- \int_\Omega |f^-| \; d\mu \]
	which in particular is finite.
\end{step}
You may already start to see that we really like nonnegative functions:
with the theory of measures, it is possible to integrate them,
and it's even okay to throw in $+\infty$'s everywhere.
But once we start dealing with functions that can be either positive or negative,
we have to start adding finiteness restrictions ---
actually essentially what we're doing is splitting
the function into its positive and negative part,
requiring both are finite, and then integrating.


To finish this section, we state for completeness
some results that you probably could have guessed were true.
Fix $\Omega = (\Omega, \SA, \mu)$, and
let $f$ and $g$ be measurable real-valued functions
such that $f(x) = g(x)$ almost everywhere.
\begin{itemize}
	\ii (Almost-everywhere preservation)
	The function $f$ is absolutely integrable if and only if $g$ is,
	and if so, their Lebesgue integrals match.
	\ii (Additivity)
	If $f$ and $g$ are absolutely integrable then
	\[ \int_\Omega f+g \; d\mu
		= \int_\Omega f \; d\mu
		+ \int_\Omega g \; d\mu. \]
	The ``absolutely integrable'' hypothesis can be dropped
	if $f$ and $g$ are nonnegative.
	\ii (Scaling) If $f$ is absolutely integrable and $c \in \RR$
	then $cf$ is absolutely integrable and
	\[ \int_\Omega cf \; d\mu = c \int_\Omega f \; d\mu. \]
	The ``absolutely integrable'' hypothesis can be dropped
	if $f$ is nonnegative and $c > 0$.
	\ii (Monotoncity)
	If $f$ and $g$ are absolutely integrable and $f \le g$, then
	\[ \int_\Omega f \; d\mu \le \int_\Omega g \; d\mu. \]
	The ``absolutely integrable'' hypothesis can be dropped
	if $f$ and $g$ are nonnegative.
\end{itemize}
There are more famous results like monotone/dominated convergence
that are also true, but we won't state them here
as we won't really have a use for them in the context of probability.
(They appear later on in a bonus chapter.)

\section{Relation to Riemann integrals (or: actually computing Lebesgue integrals)}
For closed intervals, this actually just works out of the box.
\begin{theorem}
	[Lebesgue integral generalizes Riemann integral]
	Let $f \colon [a,b] \to \RR$ be a Riemann integrable function
	(where $[a,b]$ is equipped with the Borel measure).
	Then $f$ is also Lebesgue integrable and the integrals agree:
	\[ \int_a^b f(x) \; dx = \int_{[a,b]} f \; d\mu. \]
\end{theorem}

Thus in practice, we do all theory with Lebesgue integrals (they're nicer),
but when we actually need to compute $\int_{[1,4]} x^2 \; d\mu$
we just revert back to our usual antics with the
Fundamental Theorem of Calculus.
\begin{example}
	[Integrating $x^2$ over {$[1,4]$}]
	Reprising our old example:
	\[ \int_{[1,4]} x^2 \; d\mu
		= \int_1^4 x^2 \; dx
		= \frac13 \cdot 4^3 - \frac13 \cdot 1^3 = 21.  \]
\end{example}

This even works for \emph{improper} integrals,
if the functions are nonnegative.
The statement is a bit cumbersome to write down, but here it is.
\begin{theorem}
	[Improper integrals are nice Lebesgue ones]
	Let $f \ge 0$ be a \emph{nonnegative}
	continuous function defined on $(a,b) \subseteq \RR$,
	possibly allowing $a = -\infty$ or $b = \infty$.
	Then
	\[ \int_{(a,b)} f \; d\mu
		= \lim_{\substack{a' \to a^+ \\ b' \to b^-}}
		\int_{a'}^{b'} f(x) \; dx \]
	where we allow both sides to be $+\infty$
	if $f$ is not absolutely integrable.
\end{theorem}
The right-hand side makes sense since $[a',b'] \subsetneq (a,b)$
is a compact interval on which $f$ is continuous.
This means that improper Riemann integrals of nonnegative
functions can just be regarded as Lebesgue ones
over the corresponding open intervals.

It's probably better to just look at an example though.
\begin{example}
	[Integrating $1/\sqrt{x}$ on $(0,1)$]
	For example, you might be familiar with improper integrals like
	\[ \int_0^1 \frac{1}{\sqrt x} \; dx
		\defeq \lim_{\eps \to 0^+}
		\int_\eps^1 \frac{1}{\sqrt x} \; dx
		= \lim_{\eps \to 0^+} \left( 2\sqrt{1} - 2\sqrt{\eps} \right) = 2.
	\]
	(Note this appeared before as \Cref{prob:improper}.)
	In the Riemann integration situation, we needed the limit as $\eps \to 0^+$
	since otherwise $\frac{1}{\sqrt x}$ is not defined as a function $[0,1] \to \RR$.
	However, it is a \emph{measurable nonnegative}
	function $(0,1) \to [0,+\infty]$,
	and hence
	\[ \int_{(0,1)} \frac{1}{\sqrt x} \; d\mu = 2. \]
\end{example}

If $f$ is not nonnegative, then all bets are off.
Indeed \Cref{prob:sin_improper} is the famous counterexample.

\section{\problemhead}

\begin{sproblem}
	[The indicator of the rationals]
	\label{prob:1QQ}
	Take the indicator function
	$\mathbf 1_{\QQ} \colon \RR \to \{0,1\} \subseteq \RR$
	for the rational numbers.
		\begin{enumerate}[(a)]
	   \ii Prove that $\mathbf{1}_\QQ$ is not Riemann integrable.
	   \ii Show that $\int_\RR \mathbf{1}_\QQ$ exists
	   and determine its value --- the one you expect!
   \end{enumerate}
\end{sproblem}

\begin{dproblem}
	[An improper Riemann integral with sign changes]
	\label{prob:sin_improper}
	Define $f \colon (1,\infty) \to \RR$ by $f(x) = \frac{\sin(x)}{x}$.
	Show that $f$ is not absolutely integrable,
	but that the improper Riemann integral
	\[ \int_1^\infty f(x) \; dx \defeq
		\lim_{B \to \infty}
		\int_a^b f(x) \; dx \]
	nonetheless exists.
\end{dproblem}
