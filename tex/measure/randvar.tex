\chapter{Random variables (TO DO)}
\todo{write chapter}
Having properly developed the Lebesgue measure
and the integral on it,
we can now proceed to develop random variables.

\section{Random variables}
With all this set-up, random variables are going to be really quick to define.
\begin{definition}
	A (real) \vocab{random variable} $X$ on a probability space
	$\Omega = (\Omega, \SA, \mu)$
	is a measurable function $X \colon \Omega \to \RR$,
	where $\RR$ is equipped with the Borel $\sigma$-algebra.
\end{definition}
In particular, addition of random variables, etc.\
all makes sense, as we can just add.
Also, we can integrate $X$ over $\Omega$, by previous chapter.

\begin{definition}
	[First properties of random variables]
	Given a random variable $X$,
	the \vocab{expected value} of $X$ is defined by
	the Lebesgue integral
	\[ \EE[X] = \int_{\Omega} X(\omega) \; d\mu. \]
	Confusingly, the letter $\mu$ is often used for expected values.

	The \vocab{$k$th moment} of $X$ is defined as $\EE[X^k]$,
	for each positive integer $k \ge 1$.
	The \vocab{variance} of $X$ is then defined as
	\[ \Var(X) = \EE\left[ (X-\EE[X])^2 \right]. \]
\end{definition}
\begin{ques}
	Show that $\mathbf{1}_A$ is a random variable
	(just check that it is Borel measurable),
	and its expected value is $\mu(A)$.
\end{ques}

An important property of expected value you probably already know:
\begin{theorem}
	[Linearity of expectation]
	If $X$ and $Y$ are random variables on $\Omega$ then
	\[ \EE[X+Y] = \EE[X] + \EE[Y]. \]
\end{theorem}
\begin{proof}
	$\EE[X+Y] = \int_\Omega X(\omega) + Y(\omega) \; d\mu
	= \int_\Omega X(\omega) \; d\mu + \int_\Omega Y(\omega) \; d\mu
	= \EE[X] + \EE[Y]$.
\end{proof}
Note that $X$ and $Y$ do not have to be ``independent'' here:
a notion we will define shortly.

\section{Distribution functions}

\section{Examples of random variables}

\section{Characteristic functions}

\section{Independent random variables}

\section{\problemhead}
\begin{problem}
	[Equidistribution]
	Let $X_1$, $X_2$, \dots be i.i.d.\ uniform random variables on $[0,1]$.
	Show that almost surely the $X_i$ are equidistributed,
	meaning that
	\[ \lim_{N \to \infty} \frac{ \# \{1 \le i \le N \mid a \le X_i(\omega) \le b \}}{N}
		= b-a \qquad \forall 0 \le a < b \le 1 \]
	holds for almost all choices of $\omega$.
\end{problem}

\begin{problem}
	[Side length of triangle independent from median]
	Let $X_1$, $Y_1$, $X_2$, $Y_2$, $X_3$, $Y_3$
	be six independent standard Gaussians.
	Define triangle $ABC$ in the Cartesian plane
	by $A = (X_1,Y_1)$, $B = (X_2,Y_2)$, $C = (X_3,Y_3)$.
	Prove that the length of side $BC$
	is independent from the length of the $A$-median.
\end{problem}

% 18.175 has some other good pset problems that could go here
