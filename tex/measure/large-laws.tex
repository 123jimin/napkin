\chapter{Large number laws}

\section{Notions of convergence}
\begin{definition}
	Let $X$, $X_n$ be random variables on a probability space $\Omega$.
	We say $X_n$ \vocab{converges almost surely} to $X$ if
	\[ \mu \left( \omega \in \Omega :
		\lim_n X_n(\omega) = X(\omega) \right) = 1. \]
\end{definition}
This is a very strong notion of convergence:
it says in almost every \emph{world},
the values of $X_n$ converge to $X$.
In fact, it is almost better for me to give a \emph{non-example}.
\begin{example}
	[Non-example of almost sure convergence]
	Imagine an immortal skeleton archer is practicing shots,
	and on the $n$th shot, he scores a bulls-eye with probability
	$1 - \frac 1n$
	(which tends to zero because the archer improves over time).
	Let $X_n \in \{0, 1, \dots, 10\}$ be the score of the $n$th shot.

	Although the skeleton is gradually approaching perfection,
	there is \emph{no particular world} in which the archer
	misses only finitely many shots: that is
	\[ \mu \left( \omega \in \Omega :
		\lim_n X_n(\omega) = 10 \right) = 0. \]
\end{example}

Therefore, for many purposes we need a weaker notion of convergence.
\begin{definition}
	Let $X$, $X_n$ be random variables on a probability space $\Omega$.
	We say $X_n$ \vocab{converges in probability} to $X$ if
	if for every $\eps > 0$ and $\delta > 0$, we have
	\[ \mu \left( \omega \in \Omega :
			\left\lvert X_n(\omega) - X(\omega) \right\rvert < \eps
		\right) \ge 1 - \delta  \]
	for $n$ large enough (in terms of $\eps$ and $\delta$).
\end{definition}
In this sense, our skeleton archer does succeed:
for any $\delta > 0$, if $n > \delta\inv$
then the skeleton archer does hit a bulls-eye
in a $1-\delta$ fraction of the worlds.
In general, you can think of this as saying that for any $\delta > 0$,
the chance of an $\eps$-anomaly event at the $n$th stage
eventually drops below $\delta$.

\begin{remark}
	To mask $\delta$ from the definition,
	this is sometimes written instead as:
	for all $\eps$
	\[ \lim_{n \to \infty} \mu \left( \omega \in \Omega :
		\left\lvert X_n(\omega) - X(\omega) \right\rvert < \eps
		\right) = 1. \]
	I suppose it doesn't make much difference,
	though I personally don't like the asymmetry.
\end{remark}


\section{\problemhead}
\begin{problem}
	[Quantifier hell]
	\gim
	In the definition of convergence in probability
	suppose we allowed $\delta = 0$
	(rather than $\delta > 0$).
	Show that the modified definition is
	equivalent to almost sure convergence.
	\begin{hint}
		This is actually trickier than it appears,
		you cannot just push quantifiers (contrary to the name),
		but have to focus on $\eps = 1/m$ for $m = 1, 2, \dots$.

		The problem is saying for each $\eps > 0$,
		if $n > N_\eps$, we have
		$\mu(\omega : |X(\omega)-X_n(\omega)| \le \eps) = 1$.
		For each $m$ there are some measure zero ``bad worlds'';
		take the union.
	\end{hint}
	\begin{sol}
		For each positive integer $m$,
		consider what happens when $\eps = 1/m$.
		Then, by hypothesis, there is a threshold $N_m$
		such that the \emph{anomaly set}
		\[ A_m \defeq \left\{ \omega :
			|X(\omega)-X_n(\omega)| \ge \frac 1m
			\text{ for some } n > N_m \right\} \]
		has measure $\mu(A_m) = 0$.
		Hence, the countable union $A = \bigcup_{m \ge 1} A_m$ has measure zero too.

		So the complement of $A$ has measure $1$.
		For any world $\omega \notin A$,
		we then have 
		\[ \lim_n \left\lvert X(\omega) - X_n(\omega) \right\rvert = 1 \]
		because when $n > N_m$ that absolute value
		is always at most $1/m$ (as $\omega \notin A_m$).
	\end{sol}
\end{problem}

