\chapter{Bonus: Pontryagin Duality}
\label{ch:pontryagin}

\section{A glimpse of Pontryagin duality}
In fact all the examples we have covered can be subsumed
as special cases of \emph{Pontryagin duality},
where we replace the domain with a general group $G$.
This is mostly beyond the scope of Napkin,
but I'll outline what's going on.

\subsection{LCA groups}
Some hypotheses are necessary here:
\begin{definition}
	A \vocab{locally compact abelian (LCA) group}
	consists of two pieces of data:
	\begin{itemize}
		\ii A Hausdorff topological space $G$
		which is \emph{locally compact}:
		every point of $G$ has a compact neighborhood.
		\ii $G$ should also have an \emph{abelian group} operation on it,
		which is continuous\footnote{I haven't (and will not)
			define what this means, but you can try to guess.}
		with respect to the topology.
	\end{itemize}
\end{definition}
Our previous examples fall into this category:
\begin{example}
	[Examples of locally compact abelian groups]
	\listhack
	\begin{itemize}
		\ii Any finite group $Z$ with the discrete topology is LCA.
		\ii The circle group $\TT$ is LCA and also in fact compact.
		\ii The real numbers $\RR$ are an example of an LCA group
		which is \emph{not} compact.
	\end{itemize}
\end{example}
It turns out that for an LCA group,
it's possible to define an integral $\int_G f$
for functions $f \colon G \to \CC$.
The relevant buzz-word is
\href{https://en.wikipedia.org/wiki/Haar_measure}{Haar measure},
which we won't work in detail here.

\subsection{The Pontryagin dual}
The key definition is:
\begin{definition}
	Let $G$ be an LCA group.
	Then its \vocab{Pontryagin dual} is the abelian group
	\[ \wh G \defeq \left\{ \text{continuous group homomorphisms }
			\xi : G \to \TT \right\}. \]
	The maps $\xi$ are called \vocab{characters}.
	It can itself be made into an LCA group
	using the so-called
	\href{https://en.wikipedia.org/wiki/Compact-open_topology}{compact-open topology},
	which we don't define here.
\end{definition}
\begin{example}
	[Examples of Pontryagin duals]
	\listhack
	\begin{itemize}
		\ii $\wh{\ZZ} \cong \TT$,
		since group homomorphisms $\ZZ \to \TT$ are determined by the image of $1$.
		\ii $\wh{\TT} \cong \ZZ$.
		The characters are given by $\theta \mapsto n\theta$ for $n \in \ZZ$.
		\ii $\wh{\RR} \cong \RR$.
		This is because a nonzero continuous homomorphism
		$\RR \to S^1$ is determined by the fiber above $1 \in S^1$.
		(Algebraic topologists might see covering projections here.)
		\ii $\wh{\ZZ/n\ZZ} \cong \ZZ/n\ZZ$,
		characters $\xi$ being determined by the image $\xi(1) \in \TT$.
		\ii $\wh{G \times H} \cong \wh G \times \wh H$.
	\end{itemize}
\end{example}
\begin{exercise}
	[$\wh Z \cong Z$, for those who read \Cref{sec:FTFGAG}]
	If $Z$ is a finite abelian group, show that $\wh Z \cong Z$,
	using the results of the previous example.
	You may now recognize that the bilinear form
	$\cdot \colon Z \times Z \to \TT$
	is exactly a choice of isomorphism $Z \to \wh Z$.
	It is not ``canonical''.
\end{exercise}


True to its name as the dual,
and in analogy with $(V^\vee)^\vee \cong V$ for vector spaces $V$, we have:
\begin{theorem}
	[Pontryagin duality theorem]
	For any LCA group $G$, there is an isomorphism
	\[ G \cong \wh{\wh G} \qquad \text{by} \qquad
		x \mapsto \left( \xi \mapsto \xi(x) \right). \]
\end{theorem}

\subsection{The orthonormal basis in the compact case}
If $G$ is LCA and compact,
then it turns out it's possible to integrate over $G$
Thus we may let $L^2(G)$ be the space of square-integrable functions to $\CC$, i.e.
\[ L^2(G) = \left\{ f : G \to \CC
	\quad\text{such that}\quad \int_G |f|^2 < \infty \right\}. \]
Thus we can equip it with the inner form
\[ \left< f,g \right> = \int_G f \cdot \ol{g}. \]
In that case, we get all the results we wanted before:
\begin{theorem}
	[Characters of $\wh G$ forms an orthonormal basis]
	\label{thm:god}
	Assume $G$ is LCA and compact.
	Then $\wh G$ is \emph{discrete}, and the characters
	\[ (e_\xi)_{\xi \in \wh G}
		\qquad\text{by}\qquad e_\xi(x) = e(\xi(x)) = \exp(2\pi i \xi(x)) \]
	form an orthonormal basis of $L^2(G)$.
	Thus for each $f \in L^2(G)$ we have
	\[ f = \sum_{\xi \in \wh G} \wh f(\xi) e_\xi \]
	where
	\[ \wh f(\xi) = \left< f, e_\xi \right>
		= \int_G f(x) \exp(-2\pi i \xi(x)) \; d\mu. \]
\end{theorem}
The sum $\sum_{\xi \in \wh G}$ makes sense since $\wh G$ is discrete.
In particular,
\begin{itemize}
	\ii Letting $G = Z$ gives ``Fourier transform on finite groups''.
	\ii The special case $G = \ZZ/n\ZZ$ has its
	\href{https://en.wikipedia.org/wiki/Discrete-time_Fourier_transform}%
	{own Wikipedia page}.
	\ii Letting $G = \TT$ gives the ``Fourier series'' earlier.
\end{itemize}

\subsection{The Fourier transform of the non-compact case}
If $G$ is LCA but not compact, then Theorem~\ref{thm:god} becomes false.
On the other hand, it's still possible to define $\wh G$
and then try to write
\[ \wh f(\xi) = \int_G f \cdot \ol{e_\xi} \]
for functions $f$.
The results are less fun in this case,
though, and we won't discuss them further.

Despite the fact that the $e_\xi$ no longer form an orthonormal basis,
the transformed function $\wh f : \wh G \to \CC$ is still often useful.
In particular, they have special names for a few special $G$:
\begin{itemize}
	\ii If $G = \RR$, then $\wh G = \RR$,
	and this construction gives the poorly named
	``\href{https://en.wikipedia.org/wiki/Fourier_transform}{(continuous) Fourier transform}''.
	\ii If $G = \ZZ$, then $\wh G = \TT$,
	and this construction gives the poorly named
	``\href{https://en.wikipedia.org/wiki/Discrete-time_Fourier_transform}{discrete time Fourier transform}.
\end{itemize}





\section{Summary}
We summarize our various flavors of Fourier analysis
from the previous sections in the following table.
In the first part $G$ is compact,
in the second half $G$ is not.
\[
	\begin{array}{llll}
		\hline
		\text{Name} & \text{Domain }G & \text{Dual }\wh G 
			& \text{Characters} \\ \hline
		\text{Binary Fourier analysis} & \{\pm1\}^n
			& S \subseteq \left\{ 1, \dots, n \right\}
			& \prod_{s \in S} x_s \\
		\text{Fourier transform on finite groups} & Z
			& \xi \in \wh Z \cong Z & e( i \xi \cdot x) \\
		\text{Discrete Fourier transform} & \ZZ/n\ZZ & \xi \in \ZZ/n\ZZ
			& e(\xi x / n) \\
		\text{Fourier series} & \TT \cong [-\pi, \pi]  & n \in \ZZ
			& \exp(inx) \\ \hline
		\text{Continuous Fourier transform} & \RR & \xi \in \RR
		 	& e(\xi x) \\
		\text{Discrete time Fourier transform} & \ZZ & \xi \in \TT \cong [-\pi, \pi]
		 	& \exp(i \xi n) \\
	\end{array}
\]
You might notice that the \textbf{various names are awful}.
This is part of the reason I got confused as a high school student:
every type of Fourier series above has its own Wikipedia article.
If it were up to me, we would just use the term ``$G$-Fourier transform'',
and that would make everyone's lives a lot easier.


\endinput
\section{Peter-Weyl}
In fact, if $G$ is a Lie group, even if $G$ is not abelian
we can still give an orthonormal basis of $L^2(G)$
(the square-integrable functions on $G$).
It turns out in this case the characters are attached to complex
irreducible representations of $G$
(and in what follows all representations are complex).

The result is given by the Peter-Weyl theorem.
First, we need the following result:
\begin{lemma}
	[Compact Lie groups have unitary reps]
	Any finite-dimensional (complex) representation $V$ of a compact Lie group $G$
	is unitary, meaning it can be equipped with a $G$-invariant inner form.
	Consequently, $V$ is completely reducible:
	it splits into the direct sum of irreducible representations of $G$.
\end{lemma}
\begin{proof}
	Suppose $B : V \times V \to \CC$ is any inner product.
	Equip $G$ with a right-invariant Haar measure $dg$.
	Then we can equip it with an ``averaged'' inner form
	\[ \wt B(v,w) = \int_G B(gv, gw) \; dg. \]
	Then $\wt B$ is the desired $G$-invariant inner form.
	Now, the fact that $V$ is completely reducible follows from the fact
	that given a subrepresentation of $V$, its orthogonal complement
	is also a subrepresentation.
\end{proof}

The Peter-Weyl theorem then asserts that the finite-dimensional irreducible
unitary representations essentially give an orthonormal basis for $L^2(G)$,
in the following sense.  Let $V = (V, \rho)$ be such a representation of $G$,
and fix an orthonormal basis of $e_1$, \dots, $e_d$ for $V$ (where $d = \dim V$).
The $(i,j)$th \vocab{matrix coefficient} for $V$ is then given by 
\[ G \taking{\rho} \GL(V) \taking{\pi_{ij}} \CC \]
where $\pi_{ij}$ is the projection onto the $(i,j)$th entry of the matrix.
We abbreviate $\pi_{ij} \circ \rho$ to $\rho_{ij}$.
Then the theorem is:
\begin{theorem}
	[Peter-Weyl]
	Let $G$ be a compact Lie group.
	Let $\Sigma$ denote the (pairwise non-isomorphic) irreducible finite-dimensional
	unitary representations of $G$.
	Then
	\[ \left\{ \sqrt{\dim V} \rho_{ij}
			\; \Big\vert \; (V, \rho) \in \Sigma,
			\text{ and } 1 \le i,j \le \dim V \right\}  \]
	is an orthonormal basis of $L^2(G)$.
\end{theorem}
Strictly, I should say $\Sigma$ is a set of representatives of 
the isomorphism classes of irreducible unitary representations,
one for each isomorphism class.

In the special case $G$ is abelian,
all irreducible representations are one-dimensional.
A one-dimensional representation of $G$ is a map
$G \injto \GL(\CC) \cong \CC^\times$,
but the unitary condition implies it is actually a map $G \injto S^1 \cong \TT$,
i.e.\ it is an element of $\wh G$.
