\chapter{Bonus: Fourier Analysis}
Now that we've worked hard to define abstract inner product spaces,
I want to give an (optional) application:
how to set up Fourier analysis correctly,
using this language.

As a bonus, I also prove a form of Arrow's Impossibility Theorem
using binary Fourier analysis,
and then talk about the fancier generalizations using Pontryagin duality
and the Peter-Weyl theorem.

In what follows, we let $\TT = \RR/\ZZ$ denote the ``circle group'',
thought of as the additive group of ``real numbers modulo $1$''.
There is a canonical map $e : \TT \to \CC$ sending $\TT$ to the
complex unit circle, given by $e(\theta) = \exp(2\pi i \theta)$.

(Disclaimer: I will deliberately be sloppy with convergence issues,
in part because I don't fully understand them myself,
and in part because I don't care.)

\section{Synopsis}
Suppose we have a domain $Z$ and are interested in functions $f \colon Z \to \CC$.
Naturally, the set of such functions form a complex vector space.
We like to equip the set of such functions
with an positive definite \emph{inner product}.
% usually something like $\left< f,g \right> = \EE_{x \in Z} f(x) \ol{g(x)}$.
%which makes the set of such functions into a normed vector space.
%(Here $\EE$ is an ``average'', which is a finite sum if $|Z| < \infty$
%but otherwise is usually an integral.)
%In particular, $\left< f,f\right>$ is the average of $|f(x)|^2$;
%and thus gives a positive definite inner form.

The idea of Fourier analysis is to then select an \emph{orthonormal basis}
for this set of functions, say $(e_\xi)_{\xi}$,
which we call the \vocab{characters};
the indexing $\xi$ are called \vocab{frequencies}.
In that case, since we have a basis, every function $f : Z \to \CC$
becomes a sum
\[ f(x) = \sum_{\xi} \wh f(\xi) e_\xi \]
where $\wh f(\xi)$ are complex coefficients of the basis;
appropriately we call $\wh f$ the \vocab{Fourier coefficients}.
The variable $x \in Z$ is referred to as the \vocab{physical} variable.
This is generally good because the characters are deliberately chosen
to be nice ``symmetric'' functions,
like sine or cosine waves or other periodic functions.
Thus $we$ decompose an arbitrarily complicated function into a sum on nice ones.

\section{Hilbert spaces}
\prototype{Any complex inner product space, really.}
\todo{WRITE ME}
In this case, an orthonormal basis must allow
infinite linear combinations,
as long as the sum of squares is finite.

For convenience, we record a few facts about orthonormal bases.
\begin{proposition}
	[Facts about orthonormal bases]
	\label{prop:orthonormal}
	Let $V$ be a complex Hilbert space
	with inner form $\left< -,-\right>$
	and suppose $x = \sum_\xi a_\xi e_\xi$ and $y = \sum_\xi b_\xi e_\xi$
	where $e_\xi$ are an orthonormal basis.
	Then
	\begin{align*}
		\left< x,x \right> &= \sum_\xi |a_\xi|^2 \\
		a_\xi &= \left< x, e_\xi \right> \\
		\left< x,y \right> &= \sum_\xi a_\xi \ol{b_\xi}.
	\end{align*}
\end{proposition}
\begin{ques}
	Prove all of these.
	(You don't need any of the preceding discussion,
	it's only there to motivate the notation with lots of scary $\xi$'s.)
\end{ques}

\section{Common examples}
\subsection{Binary Fourier analysis on $\{\pm1\}^n$}
Let $Z = \{\pm 1\}^n$ for some positive integer $n$,
so we are considering functions $f(x_1, \dots, x_n)$ accepting binary values.
Then the functions $Z \to \CC$ form a $2^n$-dimensional vector space $\CC^Z$,
and we endow it with the inner form
\[ \left< f,g \right> = \frac{1}{2^n} \sum_{x \in Z} f(x) \ol{g(x)}. \]
In particular,
\[ \left< f,f \right>
	= \frac{1}{2^n} \sum_{x \in Z} \left\lvert f(x) \right\rvert^2 \]
is the average of the squares;
this establishes also that $\left< -,-\right>$ is positive definite.

In that case, the \vocab{multilinear polynomials} form a basis of $\CC^Z$,
that is the polynomials
\[ \chi_S(x_1, \dots, x_n) = \prod_{s \in S} x_s. \]
\begin{exercise}
	Show that they're actually orthonormal under $\left< -,-\right>$.
	This proves they form a basis, since there are $2^n$ of them.
\end{exercise}
Thus our frequency set is actually the subsets $S \subseteq \{1, \dots, n\}$.
Thus, we have a decomposition
\[ f = \sum_{S \subseteq \{1, \dots, n\}} \wh f(S) \chi_S. \]
\begin{example}
	[An example of binary Fourier analysis]
	Let $n = 2$.
	Then binary functions $\{ \pm 1\}^2 \to \CC$ have a basis
	given by the four polynomials
	\[ 1, \quad x_1, \quad x_2, \quad x_1x_2. \]
	For example, consider the function $f$
	which is $1$ at $(1,1)$ and $0$ elsewhere.
	Then we can put
	\[ f(x_1, x_2) = \frac{x_1+1}{2} \cdot \frac{x_2+1}{2}
		= \frac14 \left( 1 + x_1 + x_2 + x_1x_2 \right). \]
	So the Fourier coefficients are $\wh f(S) = \frac 14$
	for each of the four $S$'s.
\end{example}
This notion is useful in particular for
binary functions $f : \{\pm1\}^n \to \{\pm1\}$;
for these functions (and products thereof),
we always have $\left< f,f \right> = 1$.

It is worth noting that the frequency $\varnothing$ plays a special role:
\begin{exercise}
	Show that
	\[ \wh f(\varnothing) = \frac{1}{|Z|} \sum_{x \in Z} f(x). \]
\end{exercise}

\subsection{Fourier analysis on finite groups $Z$}
This time, suppose we have a finite abelian group $Z$,
and consider functions $Z \to \CC$;
this is a $|Z|$-dimensional vector space.
The inner product is the same as before:
\[ \left< f,g \right> = \frac{1}{|Z|} \sum_{x \in Z} f(x) \ol{g(x)}. \]

To proceed, we'll need to be able to multiply two elements of $Z$.
This is a bit of a nuisance since it actually won't really
matter what map I pick, so I'll move briskly;
feel free to skip most or all of the remaining paragraph.
\begin{definition}
We select a \emph{symmetric non-degenerate bilinear form}
\[ \cdot \colon Z \times Z \to \TT \]
satisfying the following properties:
\begin{itemize}
	\ii $\xi \cdot (x_1 + x_2) = \xi \cdot x_1 + \xi \cdot x_2$
	and $(\xi_1 + \xi_2) \cdot x = \xi_1 \cdot x + \xi_2 \cdot x$
	(this is the word ``bilinear'')
	\ii $\cdot$ is symmetric,
	\ii For any $\xi \neq 0$, there is an $x$ with $\xi \cdot x \neq 0$
	(this is the word ``nondegenerate'').
\end{itemize}
\end{definition}
\begin{example}
	If $Z = \Zm3$ then $\xi \cdot x = (\xi x)/3$ satisfies the above.
\end{example}
In general, it turns out finite abelian groups
decompose as the sum of cyclic groups (see \Cref{sec:FTFGAG}),
which makes it relatively easy to find such a $\cdot$;
but as I said the choice won't matter, so let's move on.

Now for the fun part: defining the characters.
the characters are given by
In this way, the set of frequencies is also $Z$,
but the $\xi \in Z$ play very different roles from the ``physical'' $x \in Z$.
\begin{proposition}
	[$e_\xi$ are orthonormal]
	For each $\xi \in Z$ we define the character
	\[ e_\xi(x) = e(\xi \cdot x). \]
	The $|Z|$ characters form an orthonormal basis.
\end{proposition}
We won't prove this here, since it requires too
much detail about the $\cdot$ that
I'm deliberately trying to avoid.
Instead, the following example might be enlightening.
%\begin{proof}
%	Compute
%	\begin{align*}
%		\left< e_{\xi}, e_{\xi'} \right>
%		&= \frac{1}{|Z|} \sum_{x \in Z} e(\xi \cdot x) \ol{e(\xi' \cdot x)} \\
%		&= \frac{1}{|Z|} \sum_{x \in Z} e(\xi \cdot x) e(-\xi' \cdot x) \\
%		&= \frac{1}{|Z|} \sum_{x \in Z} e\left( (\xi-\xi') \cdot x \right).
%	\end{align*}
%\end{proof}

\begin{example}
	[Cube roots of unity filter]
	Suppose $Z = \Zm3$, with the inner form given by $\xi \cdot x = (\xi x)/3$.
	Let $\omega = \exp(\frac 23 \pi i)$ be a primitive cube root of unity.
	Note that
	\[ e_\xi(x) = \begin{cases}
			1 & \xi = 0 \\
			\omega^x & \xi = 1 \\
			\omega^{2x} & \xi = 2.
		\end{cases} \]
	Then given $f \colon Z \to \CC$ with $f(0) = a$, $f(1) = b$, $f(2) = c$,
	we obtain
	\[ f(x) = \frac{a+b+c}{3} \cdot 1
		+ \frac{a + \omega^2 b + \omega c}{3} \cdot \omega^x
		+ \frac{a + \omega b + \omega^2 c}{3} \cdot \omega^{2x}.  \]
	In this way we derive that the transforms are
	\begin{align*}
		\wh f(0) &= \frac{a+b+c}{3} \\
		\wh f(1) &= \frac{a+\omega^2 b+ \omega c}{3} \\
		\wh f(2) &= \frac{a+\omega b+\omega^2c}{3}.
	\end{align*}
\end{example}
\begin{exercise}
	Show that in analogy to $\wh f(\varnothing)$
	for binary Fourier analysis, we now have
	\[ \wh f(0) = \frac{1}{|Z|} \sum_{x \in Z} f(x). \]
\end{exercise}
Olympiad contestants may recognize the previous example
as a ``roots of unity filter'', which is exactly the point.
For concreteness, suppose one wants to compute
\[ \binom{1000}{0} + \binom{1000}{3} + \dots + \binom{1000}{999}. \]
In that case, we can consider the function
\[ w : \ZZ/3 \to \CC. \]
such that $w(0) = 1$ but $w(1) = w(2) = 0$.
By abuse of notation we will also think of $w$
as a function $w : \ZZ \surjto \ZZ/3 \to \CC$.
Then the sum in question is
\begin{align*}
	\sum_n \binom{1000}{n} w(n)
	&= \sum_n \binom{1000}{n} \sum_{k=0,1,2} \wh w(k) \omega^{kn} \\
	&= \sum_{k=0,1,2} \wh w(k) \sum_n \binom{1000}{n} \omega^{kn} \\
	&= \sum_{k=0,1,2} \wh w(k) (1+\omega^k)^n.
\end{align*}
In our situation, we have $\wh w(0) = \wh w(1) = \wh w(2) = \frac13$,
and we have evaluated the desired sum.
More generally, we can take any periodic weight $w$
and use Fourier analysis in order to interchange the order of summation.

\begin{example}
	[Binary Fourier analysis]
	Suppose $Z = \{\pm 1\}^n$, viewed as an abelian group
	under pointwise multiplication
	hence isomorphic to $(\ZZ/2\ZZ)^{\oplus n}$.
	Assume we pick the dot product defined by
	\[ \xi \cdot x = \half \sum_i \xi_i x_i \]
	where $\xi = (\xi_1, \dots, \xi_n)$ and $x = (x_1, \dots, x_n)$.

	We claim this coincides with the first example we gave.
	Indeed, let $S \subseteq \{1, \dots, n\}$
	and let $\xi \in \{\pm1\}^n$ which is $-1$ at positions in $S$,
	and $+1$ at positions not in $S$.
	Then the character $\chi_S$ form the previous example
	coincides with the character $e_\xi$ in the new notation.
	In particular, $\wh f(S) = \wh f(\xi)$.

	Thus Fourier analysis on a finite group $Z$ subsumes
	binary Fourier analysis.
\end{example}

\subsection{Fourier series for functions $L^2([-\pi, \pi])$}
Now we consider the space $L^2([-\pi, \pi])$ of
square-integrable functions $[-\pi, \pi] \to \CC$, with inner form
\[ \left< f,g \right> = \frac{1}{2\pi} \int_{[-\pi, \pi]} f(x) \ol{g(x)}. \]
Sadly, this is infinite-dimensional now,
but it turns out to be a Hilbert space, and we'll treat it this way.

Now, it turns out in this case that
\begin{theorem}
	For each integer $n$, define
	\[ e_n(x) = \exp(inx). \]
	Then $e_n$ form an orthonormal basis
	of the Hilbert space $L^2([-\pi, \pi])$.
\end{theorem}
Thus this time the frequency set $\ZZ$ is infinite.
So every function $f \in L^2([-\pi, \pi])$ decomposes as
\[ f(x) = \sum_n \wh f(n) \exp(inx) \]
for $\wh f(n)$.

This is a little worse than our finite examples:
instead of a finite sum on the right-hand side,
we actually have an infinite sum.
This is because our set of frequencies is now $\ZZ$, which isn't finite.
In this case the $\wh f$ need not be finitely supported,
but do satisfy $\sum_n |\wh f(n)|^2 < \infty$.

Since the frequency set is indexed by $\ZZ$,
we call this a \vocab{Fourier series}
to reflect the fact that the index is $n \in \ZZ$.

\begin{exercise}
	Show once again 
	\[ \wh f(0) = \frac{1}{2\pi} \int_{[-\pi, \pi]} f(x). \]
\end{exercise}

Often we require that the function $f$ satisfies $f(-\pi) = f(\pi)$,
so that $f$ becomes a periodic function,
and we can think of it as $f : \TT \to \CC$.

\subsection{Summary}
We summarize our various flavors of Fourier analysis in the following table.
\[
	\begin{array}{llll}
		\hline
		\text{Type} & \text{Physical var} & \text{Frequency var}
			& \text{Basis functions} \\ \hline
		\text{Binary} & \{\pm1\}^n
			& \text{Subsets } S \subseteq \left\{ 1, \dots, n \right\}
			& \prod_{s \in S} x_s \\
		\text{Finite group} & Z & \xi \in Z, \text{ choice of } \cdot,
			& e(\xi \cdot x) \\
		\text{Fourier series} & \TT \text{ or } [-\pi, \pi] & n \in \ZZ
			& \exp(inx) \\
		% \textbf{Fourier transf} & \RR & \xi \in \RR
		% 	& \exp(2\pi i x \xi)
	\end{array}
\]

\section{Parseval and friends}
The notion of an orthonormal basis makes several ``big-name'' results
in Fourier analysis quite lucid.
Basically, we can take every result from Proposition~\ref{prop:orthonormal},
translate it into the context of our Fourier analysis,
and get a big-name result.

\begin{corollary}
	[Parseval theorem]
	Let $f : Z \to \CC$, where $Z$ is a finite abelian group.
	Then \[ \sum_\xi |\wh f(\xi)|^2 = \frac{1}{|Z|} \sum_{x \in Z} |f(x)|^2. \]
	Similarly, if $f : [-\pi, \pi] \to \CC$ is square-integrable then
	its Fourier series satisfies
	\[ \sum_n |\wh f(n)|^2 = \frac{1}{2\pi} \int_{[-\pi, \pi]} |f(x)|^2. \]
\end{corollary}
\begin{proof}
Recall that $\left< f,f\right>$ is equal to the
square sum of the coefficients.
\end{proof}

\begin{corollary}
	[Formulas for $\wh f$]
	Let $f : Z \to \CC$, where $Z$ is a finite abelian group.
	Then \[ \wh f(\xi) = \frac{1}{|Z|} \sum_{x \in Z} f(x) \ol{e_\xi(x)}. \]
	Similarly, if $f : [-\pi, \pi] \to \CC$ is square-integrable then
	its Fourier series is given by
	\[ \wh f(n) = \frac{1}{2\pi} \int_{[-\pi, \pi]} f(x) \exp(-inx). \]
\end{corollary}
\begin{proof}
Recall that in an orthonormal basis $(e_\xi)_\xi$,
the coefficient of $e_\xi$ in $f$ is $\left< f, e_\xi\right>$.
\end{proof}
\begin{ques}
	What happens when $\xi = 0$ above?
\end{ques}

\begin{corollary}
	[Plancherel theorem]
	Let $f : Z \to \CC$, where $Z$ is a finite abelian group.
	Then \[ \left< f,g \right> = \sum_{\xi \in Z} \wh f(\xi) \ol{\wh g(\xi)}. \]
	Similarly, if $f : [-\pi, \pi] \to \CC$ is square-integrable then
	\[ \left< f,g \right> = \sum_n \wh f(\xi) \ol{\wh g(\xi)}. \]
\end{corollary}
\begin{ques}
	Prove this one in one line (like before).
\end{ques}

\section{(Optional) Arrow's Impossibility Theorem}
As an application, of binary Fourier analysis,
we now prove a form of
\href{https://en.wikipedia.org/wiki/Arrow's_impossibility_theorem}{Arrow's theorem}.
Consider $n$ voters voting among $3$ candidates $A$, $B$, $C$.
Each voter specifies a tuple $v_i = (x_i, y_i, z_i) \in \{\pm1\}^3$ as follows:
\begin{itemize}
	\ii $x_i = 1$ if person $i$ ranks $A$ ahead of $B$, and $x_i = -1$ otherwise.
	\ii $y_i = 1$ if person $i$ ranks $B$ ahead of $C$, and $y_i = -1$ otherwise.
	\ii $z_i = 1$ if person $i$ ranks $C$ ahead of $A$, and $z_i = -1$ otherwise.
\end{itemize}
Tacitly, we only consider $3! = 6$ possibilities for $v_i$:
we forbid ``paradoxical'' votes of the form $x_i = y_i = z_i$
by assuming that people's votes are consistent
(meaning the preferences are transitive).

Then, we can consider a voting mechanism
\begin{align*}
	f \colon \{\pm1\}^n &\to \{\pm1\} \\
	g \colon \{\pm1\}^n &\to \{\pm1\} \\
	h \colon \{\pm1\}^n &\to \{\pm1\}
\end{align*}
such that $f(x_\bullet)$ is the global preference of $A$ vs.\ $B$,
$g(y_\bullet)$ is the global preference of $B$ vs.\ $C$,
and $h(z_\bullet)$ is the global preference of $C$ vs.\ $A$.
We'd like to avoid situations where the global preference
$(f(x_\bullet), g(y_\bullet), h(z_\bullet))$
is itself paradoxical.

In fact, we will prove the following theorem:
\begin{theorem}
	[Arrow Impossibility Theorem]
	Assume that $(f,g,h)$ always avoids paradoxical outcomes,
	and assume $\EE f = \EE g = \EE h = 0$.
	Then $(f,g,h)$ is either a dictatorship or anti-dictatorship:
	there exists a ``dictator'' $k$ such that
	\[ f(x_\bullet) = \pm x_k, \qquad g(y_\bullet) = \pm y_k,
		\qquad h(z_\bullet) = \pm z_k \]
	where all three signs coincide.
\end{theorem}
The assumption $\EE f = \EE g = \EE h = 0$ provides symmetry
(and e.g.\ excludes the possibility that $f$, $g$, $h$
are constant functions which ignore voter input).
Unlike the usual Arrow theorem, we do \emph{not} assume
that $f(+1, \dots, +1) = +1$ (hence possibility of anti-dictatorship).

\begin{proof}
	Suppose the voters each randomly select one of the $3!=6$
	possible consistent votes.
	In \Cref{prob:arrow_lemma} it is shown
	that the exact probability of a paradoxical outcome
	for any functions $f$, $g$, $h$ is given exactly by
	\[ \frac14 + \frac14 \sum_{S \subseteq \{1, \dots, n\}}
		\left( -\frac13 \right)^{\left\lvert S \right\rvert} 
		\left( \wh f(S) \wh g(S) + \wh g(S) \wh h(S) + \wh h(S) \wh f(S) \right).
		\]
	Assume for contradiction this equals $0$.
	Then, we derive
	\[ 1 = \sum_{S \subseteq \{1, \dots, n\}}
		-\left( -\frac13 \right)^{\left\lvert S \right\rvert} 
		\left( \wh f(S) \wh g(S) + \wh g(S) \wh h(S) + \wh h(S) \wh f(S) \right). \]
	But now we can just use weak inequalities.
	We have $\wh f(\varnothing) = \EE f = 0$ and similarly for $\wh g$ and $\wh h$,
	so we restrict attention to $|S| \ge 1$.
	We then combine the famous inequality $|ab+bc+ca| \le a^2+b^2+c^2$
	(which is true across all real numbers) to deduce that
	\begin{align*}
		1 &= \sum_{S \subseteq \{1, \dots, n\}}
		-\left( -\frac13 \right)^{\left\lvert S \right\rvert} 
		\left( \wh f(S) \wh g(S) + \wh g(S) \wh h(S) + \wh h(S) \wh f(S) \right) \\
		&\le \sum_{S \subseteq \{1, \dots, n\}}
		\left( \frac13 \right)^{\left\lvert S \right\rvert} 
		\left( \wh f(S)^2 + \wh g(S)^2 + \wh h(S)^2 \right) \\
		&\le \sum_{S \subseteq \{1, \dots, n\}} \left( \frac13 \right)^1
		\left( \wh f(S)^2 + \wh g(S)^2 + \wh h(S)^2 \right) \\
		&= \frac13 (1+1+1) = 1.
	\end{align*}
	with the last step by Parseval.
	So all inequalities must be sharp, and in particular $\wh f$, $\wh g$, $\wh h$
	are supported on one-element sets, i.e.\ they are linear in inputs.
	As $f$, $g$, $h$ are $\pm 1$ valued, each $f$, $g$, $h$ is itself
	either a dictator or anti-dictator function.
	Since $(f,g,h)$ is always consistent, this implies the final result.
\end{proof}


\section{A glimpse of Pontryagin duality}
In fact all the examples we have covered can be subsumed
as special cases of \emph{Pontryagin duality},
where we replace the domain with a general group $G$.
This is mostly beyond the scope of Napkin,
but I'll outline what's going on.

\subsection{LCA groups}
Some hypotheses are necessary here:
\begin{definition}
	A \vocab{locally compact abelian (LCA) group}
	consists of two pieces of data:
	\begin{itemize}
		\ii A Hausdorff topological space $G$
		which is \emph{locally compact}:
		every point of $G$ has a compact neighborhood.
		\ii $G$ should also have an \emph{abelian group} operation on it,
		which is continuous\footnote{I haven't (and will not)
			define what this means, but you can try to guess.}
		with respect to the topology.
	\end{itemize}
\end{definition}
Our previous examples fall into this category:
\begin{example}
	[Examples of locally compact abelian groups]
	\listhack
	\begin{itemize}
		\ii Any finite group $Z$ with the discrete topology is LCA.
		\ii The circle group $\TT$ is LCA and also in fact compact.
		\ii The real numbers $\RR$ are an example of an LCA group
		which is \emph{not} compact.
	\end{itemize}
\end{example}
It turns out that for an LCA group,
it's possible to define an integral $\int_G f$
for functions $f \colon G \to \CC$.
The relevant buzz-word is
\href{https://en.wikipedia.org/wiki/Haar_measure}{Haar measure},
which we won't work in detail here.

\subsection{The Pontryagin dual}
The key definition is:
\begin{definition}
	Let $G$ be an LCA group.
	Then its \vocab{Pontryagin dual} is the abelian group
	\[ \wh G \defeq \left\{ \text{continuous group homomorphisms }
			\xi : G \to \TT \right\}. \]
	The maps $\xi$ are called \vocab{characters}.
	It can itself be made into an LCA group
	using the so-called
	\href{https://en.wikipedia.org/wiki/Compact-open_topology}{compact-open topology},
	which we don't define here.
\end{definition}
\begin{example}
	[Examples of Pontryagin duals]
	\listhack
	\begin{itemize}
		\ii $\wh{\ZZ} \cong \TT$,
		since group homomorphisms $\ZZ \to \TT$ are determined by the image of $1$.
		\ii $\wh{\TT} \cong \ZZ$.
		The characters are given by $\theta \mapsto n\theta$ for $n \in \ZZ$.
		\ii $\wh{\RR} \cong \RR$.
		This is because a nonzero continuous homomorphism
		$\RR \to S^1$ is determined by the fiber above $1 \in S^1$.
		(Algebraic topologists might see covering projections here.)
		\ii $\wh{\ZZ/n\ZZ} \cong \ZZ/n\ZZ$,
		characters $\xi$ being determined by the image $\xi(1) \in \TT$.
		\ii $\wh{G \times H} \cong \wh G \times \wh H$.
	\end{itemize}
\end{example}
\begin{exercise}
	[$\wh Z \cong Z$, for those knowing \Cref{thm:FTFGAG}]
	If $Z$ is a finite abelian group, show that $\wh Z \cong Z$,
	using the results of the previous example.
	You may now recognize that the bilinear form
	$\cdot : Z \times Z \to \TT$
	is exactly a choice of isomorphism $Z \to \wh Z$.
	It is not ``canonical''.
\end{exercise}


True to its name as the dual,
and in analogy with $(V^\vee)^\vee \cong V$ for vector spaces $V$, we have:
\begin{theorem}
	[Pontryagin duality theorem]
	For any group $G$, there is an isomorphism
	\[ G \cong \wh{\wh G} \qquad \text{by} \qquad
		x \mapsto \left( \xi \mapsto \xi(x) \right). \]
	This is the \vocab{Pontryagin duality theorem}.
\end{theorem}

\subsection{The orthonormal basis in the compact case}
If $G$ is LCA and compact,
then it turns out it's possible to integrate over $G$
Thus we may let $L^2(G)$ be the space of square-integrable functions to $\CC$, i.e.
\[ L^2(G) = \left\{ f : G \to \CC
	\quad\text{such that}\quad \int_G |f|^2 < \infty \right\}. \]
Thus we can equip it with the inner form
\[ \left< f,g \right> = \int_G f \cdot \ol{g}. \]
In that case, we get all the results we wanted before:
\begin{theorem}
	[Characters of $\wh G$ forms an orthonormal basis]
	\label{thm:god}
	Assume $G$ is LCA and compact.
	Then $\wh G$ is \emph{discrete}, and the characters
	\[ (e_\xi)_{\xi \in \wh G}
		\qquad\text{by}\qquad e_\xi(x) = e(\xi(x)) = \exp(2\pi i \xi(x)) \]
	form an orthonormal basis of $L^2(G)$.
	Thus for each $f \in L^2(G)$ we have
	\[ f = \sum_{\xi \in \wh G} \wh f(\xi) e_\xi \]
	where
	\[ \wh f(\xi) = \left< f, e_\xi \right>
		= \int_G f(x) \exp(-2\pi i \xi(x)) \; d\mu. \]
\end{theorem}
The sum $\sum_{\xi \in \wh G}$ makes sense since $\wh G$ is discrete.
In particular,
\begin{itemize}
	\ii Letting $G = Z$ gives ``Fourier transform on finite groups''.
	\ii The special case $G = \ZZ/n\ZZ$ has its
	\href{https://en.wikipedia.org/wiki/Discrete-time_Fourier_transform}%
	{own Wikipedia page}.
	\ii Letting $G = \TT$ gives the ``Fourier series'' earlier.
\end{itemize}

\subsection{The Fourier transform of the non-compact case}
If $G$ is LCA but not compact, then Theorem~\ref{thm:god} becomes false.
On the other hand, it's still possible to define $\wh G$
and then try to write
\[ \wh f(\xi) = \int_G f \cdot \ol{e_\xi} \]
for functions $f$.
The results are less fun in this case,
though, and we won't discuss them further.

Despite the fact that the $e_\xi$ no longer form an orthonormal basis,
the transformed function $\wh f : \wh G \to \CC$ is still often useful.
In particular, they have special names for a few special $G$:
\begin{itemize}
	\ii If $G = \RR$, then $\wh G = \RR$,
	and this construction gives the poorly named
	``\href{https://en.wikipedia.org/wiki/Fourier_transform}{(continuous) Fourier transform}''.
	\ii If $G = \ZZ$, then $\wh G = \TT$,
	and this construction gives the poorly named
	``\href{https://en.wikipedia.org/wiki/Discrete-time_Fourier_transform}{discrete time Fourier transform}.
\end{itemize}

\section{Summary}
We summarize our various flavors of Fourier analysis
from the previous sections in the following table.
In the first part $G$ is compact,
in the second half $G$ is not.
\[
	\begin{array}{llll}
		\hline
		\text{Name} & \text{Domain }G & \text{Dual }\wh G 
			& \text{Characters} \\ \hline
		\text{Binary Fourier analysis} & \{\pm1\}^n
			& S \subseteq \left\{ 1, \dots, n \right\}
			& \prod_{s \in S} x_s \\
		\text{Fourier transform on finite groups} & Z
			& \xi \in \wh Z \cong Z & e( i \xi \cdot x) \\
		\text{Discrete Fourier transform} & \ZZ/n\ZZ & \xi \in \ZZ/n\ZZ
			& e(\xi x / n) \\
		\text{Fourier series} & \TT \cong [-\pi, \pi]  & n \in \ZZ
			& \exp(inx) \\ \hline
		\text{Continuous Fourier transform} & \RR & \xi \in \RR
		 	& e(\xi x) \\
		\text{Discrete time Fourier transform} & \ZZ & \xi \in \TT \cong [-\pi, \pi]
		 	& \exp(i \xi n) \\
	\end{array}
\]
You might notice that the \textbf{various names are awful}.
This is part of the reason I got confused as a high school student:
every type of Fourier series above has its own Wikipedia article.
If it were up to me, we would just use the term ``$G$-Fourier transform'',
and that would make everyone's lives a lot easier.

\section{\problemhead}

\begin{problem}
	\gim
	\label{prob:arrow_lemma}
	Let $f,g,h \colon \{\pm1\}^n \to \{\pm1\}$
	be any three functions.
	For each $i$, we randomly select $(x_i, y_i, z_i) \in \{\pm1\}^3$
	subject to the constraint that not all are equal
	(hence, choosing among $2^3-2=6$ possibilities).
	Prove that the probability that
	\[ f(x_1, \dots, x_n) = g(y_1, \dots, y_n) = h(z_1, \dots, z_n) \]
	is given by the formula
	\[ \frac14 + \frac14 \sum_{S \subseteq \{1, \dots, n\}}
		\left( -\frac13 \right)^{\left\lvert S \right\rvert} 
		\left( \wh f(S) \wh g(S) + \wh g(S) \wh h(S) + \wh h(S) \wh f(S) \right)
		\]
	\begin{hint}
		Define the Boolean function $D : \{\pm 1\}^3 \to \RR$ by
		$D(a,b,c) = ab+bc+ca$.
		Write out the value of $D(a,b,c)$ for each $(a,b,c)$.
		Then, evaluate its expected value.
	\end{hint}
	\begin{sol}
	Define the Boolean function $D : \{\pm 1\}^3 \to \RR$ by
	\[ D(a,b,c) = ab + bc + ca
		= \begin{cases}
			3 & a,b,c \text{ all equal} \\
			-1 & a,b,c \text{ not all equal}.
		\end{cases}.
	\]
	Thus paradoxical outcomes arise when
	$D(f(x_\bullet), g(y_\bullet), h(z_\bullet)) = 3$.
	Now, we compute that for randomly selected
	$x_\bullet$, $y_\bullet$, $z_\bullet$ that
	\begin{align*}
		\EE D(f(x_\bullet), g(y_\bullet), h(z_\bullet))
		&= \EE \sum_S \sum_T
			\left( \wh f(S) \wh g(T) + \wh g(S) \wh h(T) + \wh h(S) \wh f(T) \right)
			\left( \chi_S(x_\bullet)\chi_T(y_\bullet) \right) \\
		&= \sum_S \sum_T
			\left( \wh f(S) \wh g(T) + \wh g(S) \wh h(T) + \wh h(S) \wh f(T) \right)
			\EE\left( \chi_S(x_\bullet)\chi_T(y_\bullet) \right).
	\end{align*}
	Now we observe that:
	\begin{itemize}
		\ii If $S \neq T$, then $\EE \chi_S(x_\bullet) \chi_T(y_\bullet) = 0$,
		since if say $s \in S$, $s \notin T$ then $x_s$ affects
		the parity of the product with 50\% either way,
		and is independent of any other variables in the product.
		\ii On the other hand, suppose $S = T$.
		Then 
		\[ \chi_S(x_\bullet) \chi_T(y_\bullet)
			= \prod_{s \in S} x_sy_s. \]
		Note that $x_sy_s$ is equal to $1$ with probability $\frac13$
		and $-1$ with probability $\frac23$
		(since $(x_s, y_s, z_s)$ is uniform from $3!=6$ choices,
		which we can enumerate).
		From this an inductive calculation on $|S|$ gives that
		\[
			\prod_{s \in S} x_sy_s
			=
			\begin{cases}
				+1 & \text{ with probability } \half(1+(-1/3)^{|S|}) \\
				-1 & \text{ with probability } \half(1-(-1/3)^{|S|}).
			\end{cases}
		\]
		Thus
		\[ \EE \left( \prod_{s \in S} x_sy_s \right) = \left( -\frac13 \right)^{|S|}.  \]
	\end{itemize}
	Piecing this altogether, we now have that
	\[
		\EE D(f(x_\bullet), g(y_\bullet), h(z_\bullet))
		=
		\left( \wh f(S) \wh g(T) + \wh g(S) \wh h(T) + \wh h(S) \wh f(T) \right)
		\left( -\frac13 \right)^{|S|}.
	\]
	Then, we obtain that
	\begin{align*}
		&\EE \frac14 \left( 1 + D(f(x_\bullet), g(y_\bullet), h(z_\bullet)) \right) \\
		=& \frac14 + \frac14\sum_S
		\left( \wh f(S) \wh g(T) + \wh g(S) \wh h(T) + \wh h(S) \wh f(T) \right)
		\wh f(S)^2 \left( -\frac13 \right)^{|S|}.
	\end{align*}
	Comparing this with the definition of $D$ gives the desired result.
	\end{sol}
\end{problem}

\endinput
\section{Peter-Weyl}
In fact, if $G$ is a Lie group, even if $G$ is not abelian
we can still give an orthonormal basis of $L^2(G)$
(the square-integrable functions on $G$).
It turns out in this case the characters are attached to complex
irreducible representations of $G$
(and in what follows all representations are complex).

The result is given by the Peter-Weyl theorem.
First, we need the following result:
\begin{lemma}
	[Compact Lie groups have unitary reps]
	Any finite-dimensional (complex) representation $V$ of a compact Lie group $G$
	is unitary, meaning it can be equipped with a $G$-invariant inner form.
	Consequently, $V$ is completely reducible:
	it splits into the direct sum of irreducible representations of $G$.
\end{lemma}
\begin{proof}
	Suppose $B : V \times V \to \CC$ is any inner product.
	Equip $G$ with a right-invariant Haar measure $dg$.
	Then we can equip it with an ``averaged'' inner form
	\[ \wt B(v,w) = \int_G B(gv, gw) \; dg. \]
	Then $\wt B$ is the desired $G$-invariant inner form.
	Now, the fact that $V$ is completely reducible follows from the fact
	that given a subrepresentation of $V$, its orthogonal complement
	is also a subrepresentation.
\end{proof}

The Peter-Weyl theorem then asserts that the finite-dimensional irreducible
unitary representations essentially give an orthonormal basis for $L^2(G)$,
in the following sense.  Let $V = (V, \rho)$ be such a representation of $G$,
and fix an orthonormal basis of $e_1$, \dots, $e_d$ for $V$ (where $d = \dim V$).
The $(i,j)$th \vocab{matrix coefficient} for $V$ is then given by 
\[ G \taking{\rho} \GL(V) \taking{\pi_{ij}} \CC \]
where $\pi_{ij}$ is the projection onto the $(i,j)$th entry of the matrix.
We abbreviate $\pi_{ij} \circ \rho$ to $\rho_{ij}$.
Then the theorem is:
\begin{theorem}
	[Peter-Weyl]
	Let $G$ be a compact Lie group.
	Let $\Sigma$ denote the (pairwise non-isomorphic) irreducible finite-dimensional
	unitary representations of $G$.
	Then
	\[ \left\{ \sqrt{\dim V} \rho_{ij}
			\; \Big\vert \; (V, \rho) \in \Sigma,
			\text{ and } 1 \le i,j \le \dim V \right\}  \]
	is an orthonormal basis of $L^2(G)$.
\end{theorem}
Strictly, I should say $\Sigma$ is a set of representatives of 
the isomorphism classes of irreducible unitary representations,
one for each isomorphism class.

In the special case $G$ is abelian,
all irreducible representations are one-dimensional.
A one-dimensional representation of $G$ is a map
$G \injto \GL(\CC) \cong \CC^\times$,
but the unitary condition implies it is actually a map $G \injto S^1 \cong \TT$,
i.e.\ it is an element of $\wh G$.


