\chapter{What is a Vector Space?}
This is a pretty light chapter.
The point of it is to just define what a vector space and a basis are.
These are very intuitive concepts that you likely already know.

\section{The Definitions of a Ring and Field}
\prototype{$\ZZ$, $\RR$, or $\CC$ are rings; the latter two are fields.}
I'll very informally define a ring/field here; if you're interested in the actual definition,
you can consult the first part of the chapter on ideals.
For now you can just remember
\begin{quote}
	A commutative ring is a structure with a \emph{commutative}
	addition and multiplication, as well as subtraction, like $\ZZ$. 
	It also has an additive identity $0$ and multiplicative identity $1$.
	(Yes, I'm considering only commutative rings.)

	If the multiplication is invertible like in $\RR$ or $\CC$,
	(meaning $\frac 1x$ makes sense for any $x \neq 0$),
	then the ring is called a \emph{field}.
\end{quote}
In fact, if you replace ``field'' by ``$\RR$'' everywhere in what follows, you probably won't lose much.
It's customary to use the letter $R$ for rings and $k$ for fields.

\section{Modules and Vector Spaces}
\prototype{Polynomials of degree at most $n$.}
You intuitively know already that $\RR^n$ is a ``vector space'':
its elements can be added together,
and there's some scaling by real numbers.
Let's develop this more generally.

Fix a commutative ring $R$.
Then informally,
\begin{moral}
	An $R$-module is any structure where you can add two elements
	and scale by elements of $R$.
\end{moral}
You can think of the $R$-module as consisting of soldiers
being commanded by the ring $R$.
Moreover, a \vocab{vector space} is just a module whose commanding ring
is actually a field.
I'll give you the full definition in a moment,
but first, examples\dots

\begin{example}
	[Quadratic Polynomials, aka My Favorite Example]
	My favorite example of an $\RR$-vector space is the
	set of polynomials of degree at most two, namely
	\[ \left\{ ax^2+bx+c \mid a,b,c \in \RR \right\}. \]
	Indeed, you can add any two quadratics, and multiply by constants.
	You can't multiply two quadratics to get a quadratic,
	but that's irrelevant -- in a vector space there need not
	be a notion of multiplying two vectors together.

	In a sense we'll describe later, this vector space has dimension $3$.
	But I hope you can see why this is kind of true!
	\label{example:quadratic_vector_space}
\end{example}
\begin{example}[All Polynomials]
	The set of \emph{all} polynomials with real coefficients is an
	$\RR$-vector space, because you can \emph{add any two polynomials}
	and \emph{scale by constants}.
\end{example}

\begin{example}
	[Euclidean Space]
	\listhack
	\begin{enumerate}[(a)]
		\ii The complex numbers
		\[ \left\{ a+bi \mid a,b \in \RR \right\} \]
		form a real vector space. As we'll see later,
		it has ``dimension $2$''.
		\ii The real numbers $\RR$ form a real vector space of dimension $1$.
		\ii The set of 3D vectors
		\[ \left\{ (x,y,z) \mid x,y,z \in \RR \right\} \]
		forms a real vector space, because you can add any two triples
		component-wise. Again, we'll later explain
		why it has ``dimension $3$''.
	\end{enumerate}
\end{example}

\begin{example}
	[More Examples of Vector Spaces]
	\listhack
	\begin{enumerate}[(a)]
		\ii The set \[ \QQ[\sqrt 2] = \left\{ a + b \sqrt 2 , a, b \in \QQ \right\} \]
		has a structure of a $\QQ$-vector space in the obvious fashion:
		one can add any two elements, and scale by rational numbers.
		(It is not a real vector space -- why?)
		\ii The set \[ \left\{ (x,y,z) \mid x+y+z = 0 \text{ and } x,y,z \in \RR \right\} \]
		is a $2$-dimensional real vector space.
		\ii The set of all functions $f : \RR \to \RR$ is also a real vector space
		(since the notions $f+g$ and $c \cdot f$ both make sense for $c \in \RR$).
	\end{enumerate}
\end{example}

Now let me write the actual rules for how this multiplication behaves.
\begin{definition}
	Let $R$ be a commutative ring.
	An $R$-\vocab{module} is an additive abelian group $M = (M,+)$
	equipped with a left multiplication by elements of $R$.
	This multiplication must satisfy the following properties
	for every $r_1, r_2 \in R$ and $m \in M$:
	\begin{enumerate}[(i)]
		\ii $r_1 \cdot (r_2 m) = (r_1r_2) \cdot m$.
		\ii Multiplication is distributive, meaning
		\[ (r_1+r_2) \cdot m = r_1 \cdot m + r_2 \cdot m 
			\text{ and }
			r \cdot (m_1 + m_2) = r \cdot m_1 + r \cdot m_2. \]
		\ii $1_R \cdot m = m$.
		\ii $0_R \cdot m = 0_M$.\footnote{This condition is 
			extraneous; you can actually deduce it
			from the first three rules.}
	\end{enumerate}
	If $R$ is a field we say $M$ is an $R$-\vocab{vector space};
	its elements are called \vocab{vectors}
	and the members of $R$ are called \vocab{scalars}.
\end{definition}

\begin{abuse}
	In the above, we're using the same symbol $+$ for the addition of $M$
	and the addition of $R$.
	Sorry about that, but it's kind of hard to avoid, and the point
	of the axioms is that these additions should be related.
	I'll try to remember to put $r \cdot m$ for the multiplication of the module and just $r_1r_2$ for the multiplication of $R$.
\end{abuse}

\begin{ques}
	In \Cref{example:quadratic_vector_space},
	I was careful to say ``degree at most $2$'' instead of ``degree $2$''.
	What's the reason for this?
	In other words, why is
	\[ \left\{ ax^2 + bx + c \mid a,b,c \in \RR, a \neq 0  \right\} \]
	not an $\RR$-vector space?
\end{ques}

A couple less intuitive but somewhat important examples\dots
\begin{example}[Abelian Groups are $\ZZ$-modules]
	\listhack
	\begin{enumerate}[(a)]
		\ii The example of real polynomials
		\[ \left\{ ax^2+bx+c \mid a,b,c \in \RR \right\} \]
		is also a $\ZZ$-module!
		Indeed, we can add any two such polynomials,
		and we can scale them by integers.
		\ii The set of integers modulo $100$, say $\ZZ/100\ZZ$,
		is a $\ZZ$-module as well. Can you see how?
		\ii In fact, \emph{any} abelian group $G = (G,+)$ is a $\ZZ$-module.
		The multiplication can defined by
		\[ n \cdot g = \underbrace{g+\dots+g}_{\text{$n$ times}}. \]
	\end{enumerate}
\end{example}
\begin{example}
	[Every ring is its own module]
	\listhack
	\begin{enumerate}[(a)]
	\ii $\RR$ can be thought of as an $\RR$-vector space over itself.
	Can you see why?

	\ii By the same reasoning,
	we see that \emph{any} commutative ring $R$ can be thought of
	as an $R$-module over itself.
	\end{enumerate}
\end{example}

\section{Direct Sums}
\prototype{$\{ax^2+bx+c\} = \RR \oplus x\RR \oplus x^2\RR$, and
$\RR^2$ is the sum of its axes.}
Let's return to \Cref{example:quadratic_vector_space}, and consider
\[ V = \left\{ ax^2+bx+c \mid a,b,c \in \RR \right\}.  \]
Even though I haven't told you what a dimension is,
you can probably see that this vector space ``should have'' dimension $3$.
We'll get to that in a moment.

The other thing you may have noticed is that somehow
the $x^2$, $x$ and $1$ terms don't ``talk to each other''.
They're totally unrelated.
In other words, we can consider the three sets
\begin{align*}
	x^2\RR &\defeq \left\{ ax^2 \mid a \in \RR \right\} \\
	x\RR &\defeq \left\{ bx \mid b \in \RR \right\} \\
	\RR &\defeq \left\{ c \mid c \in \RR \right\}.
\end{align*}
In an obvious way, each of these can be thought of as a ``copy'' of $\RR$.

Then $V$ quite literally consists of the ``sums of these sets''.
Specifically, every element of $V$ can be written \emph{uniquely}
as the sum of one element from each of these sets.
This motivates us to write
\[ V = x^2\RR \oplus x\RR \oplus \RR. \]
The notion which captures this formally is the \vocab{direct sum}.

\begin{definition}
	Let $M$ be an $R$-module.
	Let $M_1$ and $M_2$ be subsets of $M$ which are themselves $R$-modules.
	Then we write $M = M_1 \oplus M_2$ and say $M$ is a \vocab{direct sum}
	of $M_1$ and $M_2$
	if every element from $M$ can be written uniquely as the sum
	of an element from $M_1$ and $M_2$.
\end{definition}
\begin{example}[Euclidean Plane]
	Take the vector space $\RR^2 = \left\{ (x,y) \mid x \in \RR, y \in \RR \right\}$.
	We can consider it as a direct sum of its $x$-axis and $y$-axis:
	\[ X = \left\{ (x,0) \mid x \in \RR  \right\} 
		\text{ and }
		Y = \left\{ (0,y) \mid y \in \RR \right\}. \]
	Then $\RR^2 = X \oplus Y$.
\end{example}

This gives us a ``top-down'' way to break down modules
into some disconnected components.

By applying this idea in reverse, we can also construct
new vector spaces as follows.
In a very unfortunate accident, the two names and notations for technically
distinct things are exactly the same.
\begin{definition}
	Let $M$ and $N$ be $R$-modules.
	We define the \vocab{direct sum} $M \oplus N$
	to the be $R$-module whose elements are pairs $(m,n) \in M \times N$.
	The operations are given by
	\[ (m_1, n_1) + (m_2, n_2) = (m_1+m_2, n_1+n_2). \]
	and
	\[ r \cdot (m, n) = (r \cdot m, r \cdot n). \]
\end{definition}

For example, while we technically wrote $\RR^2 = X \oplus Y$,
since each of $X$ and $Y$ is just a copy of $\RR$,
we may as well have written $\RR^2 \cong \RR \oplus \RR$.

\begin{abuse}
	The above illustrates an abuse of notation in the way we write a direct sum. The symbol $\oplus$ has two meanings.
	\begin{itemize}
		\ii If $V$ is a \emph{given} space and $W_1$ and $W_2$ are subspaces, then $V = W_1 \oplus W_2$ means that ``$V$ \emph{splits} as a direct sum $W_1 \oplus W_2$'' in the way we defined above.
		\ii If $W_1$ and $W_2$ are two \emph{unrelated} spaces, then $W_1 \oplus W_2$ is \emph{defined} as the vector space whose \emph{elements} are pairs $(w_1, w_2) \in W_1 \times W_2$.
	\end{itemize}
	You can see that these definitions ``kind of'' coincide.
\end{abuse}

In this way, you can see that $V$ should be isomorphic
to $\RR \oplus \RR \oplus \RR$;
we had $V = x^2\RR \oplus x\RR \oplus \RR$,
but the $1$, $x$, $x^2$ don't really talk to each other
and each of the summands is really just a copy of $\RR$ at heart.

\begin{definition}
	We can also define, for every positive integer $n$, the module
	\[ M^{\oplus n}
		\defeq \underbrace{M \oplus M \oplus \dots \oplus M}_{\text{$n$ times}}. \]
\end{definition}

\section{Linear Independence, Spans, and Basis}
\prototype{%
	$\left\{ 1,x,x^2 \right\}$ is a basis of
	$\left\{ ax^2 + bx + c \mid a,b,c \in \RR \right\}$.}

The idea of a basis, the topic of this section,
gives us another way to capture the notion that
\[ V = \left\{ ax^2+bx+c \mid a,b,c \in \RR \right\} \]
is just sums of copies of $\{1,x,x^2\}$.
This section should be very very intuitive, if technical.
If you can't see why the theorems here ``should'' be true,
you're doing it wrong.


Let $M$ be an $R$-module now.
We define three very classical notions that you likely are already familiar with.
If not, fall upon your notion of Euclidean space or $V$ above.
\begin{definition}
	A \vocab{linear combination} of some vectors $v_1, \dots, v_n$
	is a sum of the form $r_1 v_1 + \dots + r_n v_n$,
	where $r_1, \dots, r_n \in R$.
	The linear combination is called \vocab{trivial}
	if $r_1 = r_2 = \dots = r_n = 0_R$, and \vocab{nontrivial} otherwise.
\end{definition}
\begin{definition}
	Consider a finite set of vectors $v_1, \dots, v_n$ in a module $M$.
	\begin{itemize}
		\ii It is called \vocab{linearly independent} if there
		is no nontrivial linear combination with value $0_M$.
		(Observe that $0_M = 0 \cdot v_1 + 0 \cdot v_2 + \dots + 0 \cdot v_n$
		is always true -- the assertion is that there is no other
		way to express $0_M$ in this from.)
		\ii It is called a \vocab{generating set} if every $v \in M$ can be written as
		a linear combination of the $\{v_i\}$.
		If $M$ is a vector space we say it is \vocab{spanning} instead.
		\ii It is called a \vocab{basis} (plural \vocab{bases})
		if every $v \in M$ can be written
		\emph{uniquely} as a linear combination of the $\{v_i\}$.
	\end{itemize}
	The same definitions apply for an infinite set, with the proviso
	that all sums must be finite.

\end{definition}
So by definition, $\left\{ 1,x,x^2 \right\}$ is a basis for $V$.
It's not the only one: $\{2, x, x^2\}$ and $\{x+4, x-2, x^2+x\}$
are other examples of bases, though not as natural.
However, the set $S = \{3+x^2, x+1, 5+2x+x^2\}$ is not a basis:
it fails for the following two reasons.
\begin{itemize}
	\ii Note that
	$0 = (3+x^2) + 2(x+1) - (5+2x+x^2)$.
	So the set $S$ is not linearly independent.
	\ii It's not possible to write $x^2$ as a sum of elements of $S$.
	So $S$ fails to be spanning.
\end{itemize}
With these new terms, we can just say a basis is a linearly independent and spanning set.

\begin{example}[More Example of Bases]
	\listhack
	\begin{enumerate}[(a)]
		\ii Regard $\QQ[\sqrt2]$ as a $\QQ$-vector space. Then $\{1, \sqrt 2\}$ is a basis.
		\ii If $V$ is the set of all real polynomials, there is an infinite basis $\{1, x, x^2, \dots\}$.
		The condition that we only use finitely many terms just says
		that the polynomials must have finite degree (which is good).
		\ii Let $V = \{ (x,y,z) \mid x+y+z=0 \text{ and } x,y,z \in \RR\}$.
		Then we expect there to be a basis of size $2$, but unlike previous examples
		there is no immediately ``obvious'' choice.
		Some working examples include:
		\begin{itemize}
			\ii $(1,-1,0)$ and $(1,0,-1)$,
			\ii $(0,1,-1)$ and $(1,0,-1)$,
			\ii $(5,3,-8)$ and $(2,-1,-1)$.
		\end{itemize}
	\end{enumerate}
\end{example}

\begin{ques}
	Show that a set of vectors is a basis if and only if
	it is linearly independent and spanning.
	(Think about the polynomial example if you get stuck.)
\end{ques}

\begin{figure}[ht]
	\centering
	\snsd[height=7cm]{basis-span-fantasy.jpg}
	\caption{A basis is also a spanning set.}
\end{figure}

Now we state a few results which assert
that bases in vector spaces behave as nicely as possible.
\begin{theorem}[Maximality and Minimality of Bases]
	\label{thm:vector_best}
	Let $V$ be a vector space over some field $k$
	and take $e_1, \dots, e_n \in V$. The following are equivalent:
	\begin{enumerate}[(a)]
		\ii The $e_i$ form a basis.
		\ii The $e_i$ are spanning, but no proper subset is spanning.
		\ii The $e_i$ are linearly independent, but adding any other
		element of $V$ makes them not linearly independent.
	\end{enumerate}
\end{theorem}
\begin{remark}
	If we replace $V$ by a general module $M$ over a commutative ring $R$,
	then (a) $\implies$ (b) and (a) $\implies$ (c) but not conversely.
\end{remark}
\begin{proof}
	Straightforward, do it yourself if you like.
	The key point to notice is that you need to divide by scalars for the converse direction,
	hence $V$ is required to be a vector space instead of just a module
	for the implications (b) $\implies$ (a) and (c) $\implies$ (a).
\end{proof}

\begin{theorem}
	[Dimension Theorem for Vector Spaces]
	Any two bases of a vector space have the same size.
\end{theorem}
\begin{proof}
	We prove something stronger:
	Let $V$ be a vector space, and assume $v_1, \dots, v_n$ is a spanning set
	while $w_1, \dots, w_m$ is linearly independent. We claim that $n \ge m$.
	\begin{ques}
		Show that this claim is enough to imply the theorem.
	\end{ques}

	Let $A_0 = \{v_1, \dots, v_n\}$ be the spanning set.
	Throw in $w_1$: by the spanning condition, so $w_1 = c_1v_1 + \dots + c_nv_n$.
	There's some nonzero coefficient, say $c_n$.
	Thus \[ v_n = \frac{1}{c_n} w_1 - \frac{c_1}{c_n}v_1 - \frac{c_2}{c_n}v_2 - \dots. \]
	Thus $A_1 = \{v_1, \dots, v_{n-1}, w_1\}$ is spanning.
	Now do the same thing, throwing in $w_2$, and deleting some element of the $v_i$ as before to get $A_2$;
	the condition that the $w_i$ are linearly independent ensures that some $v_i$ coefficient
	must always not be zero.
	Since we can eventually get to $A_m$, we have $n \ge m$.
\end{proof}
\begin{remark}
	In fact, this is true for modules over any commutative ring.
	Interestingly, the proof for the general case proceeds by reducing
	to the case of a vector space.
\end{remark}

The Dimension Theorem, true to its name, lets us define the \vocab{dimension} of
a vector space as the size of any finite basis, if one exists.
When it does exist we say $V$ is \vocab{finite-dimensional}.
So for example,
\[ V = \left\{ ax^2 + bx + c \mid a,b,c \in \RR \right\} \]
has dimension three, because $\left\{ 1,x,x^2 \right\}$ is a basis.
That's not the only basis: we could as well have written
\[ \left\{ a(x^2-4x) + b(x+2) + c \mid a,b,c \in \RR \right\} \]
and gotten the exact same vector space.
But the beauty of the theorem is that no matter how we try
to contrive the generating set, we always will get exactly three elements.
That's why it makes sense to say $V$ has dimension three.

On the other hand, the set of all polynomials is \emph{infinite-dimensional}
(which should be intuitively clear).

A basis $e_1, \dots, e_n$ of $V$ is really cool because it means that to specify $v \in V$, I just have to specify $a_1, \dots, a_n \in k$,
and then let $v = a_1e_1 + \dots + a_ne_n$.
You can even think of $v$ as just $\left( a_1, \dots, a_n \right)$.
% In a way I'll make precise in a moment, $V$ is actually isomorphic to just $k^{\oplus n}$.
To put it another way, if $V$ is a $k$-vector space we always have
\[ V = e_1k \oplus e_2k \oplus \dots \oplus e_nk. \]

For the remainder of this chapter,
\textbf{assume all vector spaces are finite-dimensional
unless otherwise specified}.
Infinite-dimensional vector spaces make my eyes glaze over.
\todo{darn it's really annoying to remember FD}

\section{Linear Maps}
We've seen homomorphisms and continuous maps.
Now we're about to see linear maps, the structure preserving maps
between vector spaces. Can you guess the definition?

\begin{definition}
	Let $V$ and $W$ be vector spaces over the same field $k$.
	A \vocab{linear map} is a map $T : V \to W$ which obeys the following rules.
	\begin{enumerate}[(i)]
		\ii $T$ is a homomorphism $(V,+) \to (W,+)$ of abelian groups, meaning $T(v_1 + v_2) = T(v_1) + T(v_2)$.
		\ii For any $a \in k$ and $v \in V$, $T(a \cdot v) = a \cdot T(v)$.
	\end{enumerate}
	If this map is a bijection (equivalently, if it has an inverse),
	it is an \vocab{isomorphism}.
	We say $V$ and $W$ are \vocab{isomorphic} vector spaces and write $V \cong W$.
\end{definition}

\begin{example}[Examples of Linear Maps]
	\listhack
	\begin{enumerate}[(a)]
		\ii For any vector spaces $V$ and $W$ there is a trivial linear map sending everything to $0_W \in W$.
		\ii For any vector space $V$, there is the identity isomorphism $\id : V \to V$.
		\ii The map $\RR^3 \to \RR$ by $(a,b,c) \mapsto 4a+2b+c$ is a linear map.
		\ii Let $V$ be the set of real polynomials of degree at most $2$.
		The map $\RR^3 \to V$ by $(a,b,c) \mapsto ax^2+bx+c$ is an \emph{isomorphism}.
		\ii Let $W$ be the set of functions $\RR \to \RR$.
		The evaluation map $W \to \RR$ by $f \mapsto f(0)$ is a linear map.
		\ii There is a map of $\QQ$-vector spaces $\QQ[\sqrt2] \to \QQ[\sqrt2]$
		called ``multiply by $\sqrt2$''; this map sends $a+b\sqrt2 \mapsto 2b + a\sqrt2$.
		This map is an isomorphism, because it has an inverse ``multiply by $1/\sqrt2$''.
	\end{enumerate}
\end{example}

In the expression $T(a \cdot v) = a \cdot T(v)$, note that the first $\cdot$ is the multiplication of $V$ and the second $\cdot$ is the multiplication of $W$.
Note that this notion of isomorphism really only cares about the size of the basis:
\begin{proposition}[$n$-dimensional vector spaces are isomorphic]
	If $V$ is an $n$-dimensional vector space, then
	$V \cong k^{\oplus n}$.
\end{proposition}
\begin{ques}
	Let $e_1$, \dots, $e_n$ be a basis for $V$.
	What is the isomorphism?
	(Your first guess is probably right.)
\end{ques}
\begin{remark}
	You could technically say that all finite-dimensional vector
	spaces are just $k^{\oplus n}$ and that no other space is worth
	caring about.
	But this seems kind of rude.
	Spaces often are more than just triples: $ax^2+bx+c$ is a polynomial,
	and so it has some ``essence'' to it that you'd lose if you
	just compressed it into $(a,b,c)$.

	Moreover, a lot of spaces, like the set of vectors $(x,y,z)$ with $x+y+z=0$,
	do not have an obvious choice of basis.
	Thus to cast such a space into $k^{\oplus n}$
	would require you to make arbitrary decisions.
\label{rem:vector_spaces_have_essence}
\end{remark}

\section{What is a Matrix?}
% Now I get to tell you what a matrix is: it's a way of writing a linear map in terms of bases.

Suppose we have a vector space $V$ with basis $e_1, \dots, e_m$
and a vector space $W$ with basis $w_1, \dots, w_n$.
I also have a map $T : V \to W$ and I want to tell you what $T$ is.
It would be awfully inconsiderate of me to try and tell you what $T(v)$
is at every point $v$.
In fact, I only have to tell you what $T(e_1)$, \dots, $T(e_m)$ is,
because from there you can work out $T(a_1e_1 + \dots + a_me_m)$ for yourself:
\[ T(a_1e_1 + \dots + a_me_m) = a_1T(e_1) + \dots + a_mT(e_m). \]
Since the $e_i$ are a basis, that tells you all you need to know about $T$.
\begin{example}
	Let $V = \left\{ ax^2+bx+c \mid a,b,c \in \RR \right\}$.
	Then $T(ax^2+bx+c) = aT(x^2) + bT(x) + cT(1)$.
\end{example}

Now I can even be more concrete.
I could tell you what $T(e_1)$ is, but seeing as I have a basis of $W$,
I can actually just tell you what $T(e_1)$ is in terms of this basis.
Specifically, there are unique $a_{11}, a_{21}, \dots, a_{n1} \in k$ such that
\[ T(e_1) = a_{11} w_1 + a_{21} w_2 + \dots + a_{n1} w_n. \]
So rather than telling you the value of $T(e_1)$ in some abstract space $W$,
I could just tell you what $a_{11}, a_{21}, \dots, a_{n1}$ were.
Then I'd just repeat this for $T(e_2)$, $T(e_3)$, all the way up to $T(e_m)$,
and that would tell you everything you need to know about $T$.

That's where the matrix $T$ comes from!
It's just a concise way of writing down all $mn$ numbers I need to tell you.
To be explicit, the matrix for $T$ is defined as the array
\[ T =  \underbrace{%
	\left(
	\begin{array}{cccc}
		\mid & \mid & & \mid \\	
		T(e_1) & T(e_2) & \dots & T(e_{m}) \\
		\mid & \mid & & \mid \\	
	\end{array}
	\right)
	}_m \Bigg\} n
	=
	\left(
	\begin{array}{cccc}
		a_{11} & a_{12} & \dots & a_{1m} \\
		a_{21} & a_{22} & \dots & a_{2m} \\
		\vdots & \vdots & \ddots & \vdots \\
		a_{n1} & a_{n2} & \dots & a_{nm}
	\end{array}
	\right).
	\]
From here you can actually work out for yourself what it means to multiply two matrices.
Suppose we have picked a basis for three spaces $U$, $V$, $W$.
Given maps $T : U \to V$ and $S : V \to W$, we can consider their composition $S \circ T$, i.e.
\[ U \taking{T} V \taking{S} W. \]
Matrix multiplication is defined exactly so that the matrix $ST$
is the same thing we get from interpreting the composed function $S \circ T$ as a matrix.
\begin{exercise}
	Check this for yourself!
	For a concrete example $\RR^2 \taking{T} \RR^2 \taking{S} T$
	by $T(\ee_1) = 2\ee_1+3\ee_2$ and $T(\ee_2) = 4\ee_1+5\ee_2$,
	$S(\ee_1) = 6\ee_1+7\ee_2$ and $S(\ee_2) = 8\ee_1+9\ee_2$.
	Compute $S(T(\ee_1))$ and $S(T(\ee_2))$ and see how it compares
	to multiplying the matrices associated to $S$ and $T$.
\end{exercise}
In particular, since function composition is associative,
it follows that matrix multiplication is as well.
To drive this point home,
\begin{moral}
	A matrix is the laziest possible way to specify
	a linear map from $V$ to $W$.
\end{moral}

This means you can define concepts like the determinant or the trace of a matrix
both in terms of an ``intrinsic'' map $T : V \to W$ and in terms of the entries of the matrix.
Since the map $T$ itself doesn't refer to any basis,
the abstract definition will imply that the numerical definition doesn't depend on the choice of a basis.
\section{Subspaces and Picking Convenient Bases}
\prototype{Any two linearly independent vectors in $\RR^3$.}
% A submodule is exactly what you think it is.
\begin{definition}
	Let $M$ be a left $R$-module. 
	A \vocab{submodule} $N$ of $M$ is a module $N$
	such that every element of $N$ is also an element of $M$.
	If $M$ is a vector space then $N$ is called a \vocab{subspace}.
\end{definition}

\begin{example}[Kernels]
	The \vocab{kernel} of a map $T : V \to W$ is the
	set of $v \in V$ such that $T(v) = 0_W$.
	It is a subspace of $V$, since it's closed under addition and scaling (why?).
\end{example}
\begin{example}[Spans]
	Let $V$ be a vector space and $v_1, \dots, v_m$ be any vectors of $V$.
	The \vocab{span} of these vectors is defined as the set
	\[ \left\{ a_1v_1 + \dots + a_mv_m \mid a_1, \dots, a_m \in k \right\}. \]
	Note that it is a subspace of $V$ as well!
\end{example}
\begin{ques}
	Why is $0_V$ an element of each of the above examples?
	In general, why must any subspace contain $0_V$?
\end{ques}

Subspaces behave nicely with respect to bases.
\begin{theorem}[Basis Completion]
	Let $V$ be an $n$-dimensional space, and $V'$ a subspace of $V$.
	Then 
	\begin{enumerate}[(a)]
		\ii $V'$ is also finite-dimensional.
		\ii If $e_1, \dots, e_m$ is a basis of $V'$, then there exist
		$e_{m+1}, \dots, e_n$ in $V$ such that
		$e_1, \dots, e_n$ is a basis of $V$.
	\end{enumerate}
\end{theorem}
\begin{proof}
	Boring and intuitive.
\end{proof}
This means that given $V'$ a subspace of $V$,
you can always interpret $V'$ as the span of some subset of a basis of $V$.

The same goes for maps, using $\ker T$:
\begin{theorem}[Picking A Basis for Linear Maps]
	Let $T:V\to W$ be a map of finite-dimensional vector spaces. 
	Then there exists a basis $v_1, \dots, v_n$ of $V$
	and a basis $w_1, \dots, w_m$ of $W$,
	as well as a nonnegative integer $k$, such that
	\[
		T(v_i) =
		\begin{cases}
			w_i & \text{if $i \le k$} \\
			0_W & \text{if $i > k$}.
		\end{cases}
	\]
	\label{thm:linear_map_basis}
\end{theorem}
For example, if $V = k^{\oplus 3}$ and $W = k^{\oplus 99}$,
$T$ might be the map which sends $e_3 \in V$ to $0 \in W$
and $e_1, e_2 \in V$ to $f_{13}, f_{37}$ in $W$.
\begin{proof}[Sketch of Proof]
	You might like to try this one yourself before reading on.

	Let $\ker T$ have dimension $n-k$.
	We can pick $v_{k+1}, \dots, v_{n}$ a basis of $\ker T$.
	Then extend it to a basis $v_1, \dots, v_k$ of $V$.
	The map $T$ is injective over the span of $v_1, \dots, v_k$
	(since only $0_V$ is in the kernel) so its images in $W$ are linearly independent.
	Setting $w_i = T(v_i)$ for each $i$,
	we get some linearly independent set in $W$.
	Then extend it again to a basis of $W$.
\end{proof}

The above theorem is quite important: you should keep it in your head.
It gives you a picture of how a linear map behaves:
in particular, for $T : V \to W$, 
one can write $V = \ker T \oplus V'$,
so that $T$ annihilates its kernel while sending $V'$
to an isomorphic copy in $W$.

A corollary of this (which you should have expected anyways) is the so called Rank-Nullity Theorem,
which is just the analog of the First Isomorphism Theorem.
\begin{theorem}
	[Rank-Nullity Theorem]
	\label{thm:rank_nullity}
	If $T : V \to W$, then 
	\[ \dim V = \dim \ker T + \dim \img T. \]
	% (Here $\img T = T``(V) \subseteq W$ is the image of $T$.)
\end{theorem}

\section{A Cute Application: Lagrange Interpolation}
Here's a cute application\footnote{Source: Communicated to me
by Joe Harris at the first Harvard-MIT Undegraduate Math Symposium.}
of linear algebra to a theorem from high school we all know.
\begin{theorem}
	[Lagrange Interpolation]
	Let $x_1, \dots, x_{n+1}$ be distinct real numbers
	and $y_1, \dots, y_{n+1}$ any real numbers.
	Then there exists a \emph{unique} polynomial $P$ of degree at most $n$
	such that \[ P(x_i) = y_i \] for every $i$.
\end{theorem}
\begin{proof}
	The idea is to consider the vector space $V$ of polynomials with degree at most $n$,
	as well as the vector space $W = \RR^{n+1}$.
	\begin{ques}
		Check that $\dim V = \dim W$.
	\end{ques}
	Then consider the linear map $T : V \to W$ given by
	\[ P \mapsto \left( P(x_1), \dots, P(x_{n+1}) \right). \]
	This is indeed a linear map because, well, $T(P+Q) = T(P)+T(Q)$ and $T(cP) = cT(P)$.
	It also happens to be injective: if $P \in \ker T$,
	then $P(x_1) = \dots = P(x_{n+1}) = 0$,
	but $\deg P = n$ and so $P$ can only be the zero polynomial.
	
	So $T$ is an injective map between vector spaces of the same dimension.
	Thus it is actually a bijection, which is exactly what we wanted.
\end{proof}

\section{(Digression) Arrays of Numbers Are Evil}
\label{sec:basis_evil}
As I'll stress repeatedly, a matrix represents a \emph{linear map between two vector spaces}.
Writing it in the form of an $m \times n$ matrix is merely a very convenient way to see the map concretely.
But it obfuscates the fact that this map is, well, a map, not an array of numbers.

If you took high school precalculus, you'll see everything done in terms of matrices.
To any typical high school student, a matrix is an array of numbers.
No one is sure what exactly these numbers represent, but they're told how to magically multiply these arrays to get more arrays. 
They're told that the matrix
\[
	\left(
	\begin{array}{cccc}
		1 & 0 & \dots & 0 \\
		0 & 1 & \dots & 0 \\
		\vdots & \vdots & \ddots & \vdots \\
		0 & 0 & \dots & 1 \\
	\end{array}
	\right)
\]
is an ``identity matrix'', because when you multiply by another matrix it doesn't change.
Then they're told that the determinant is some magical combination of these numbers formed by this weird multiplication rule.
No one knows what this determinant does, other than the fact that $\det(AB) = \det A \det B$, and something about areas and row operations and Cramer's rule.

Then you go into linear algebra in college, and you do more magic
with these arrays of numbers.
You're told that two matrices $T_1$ and $T_2$ are similar if \[ T_2 = ST_1S\inv \] for some invertible matrix $S$.
You're told that the trace of a matrix $\Tr T$ is the sum of the diagonal entries.
Somehow this doesn't change if you look at a similar matrix, but you're not sure why.
Then you define the characteristic polynomial as \[ p_T = \det (XI - T). \]
Somehow this also doesn't change if you take a similar matrix, but now you really don't know why.
And then you have the Cayley-Hamilton Theorem in all its black magic: $p_T(T)$ is the zero map.
Out of curiosity you Google the proof, and you find some ad-hoc procedure which still leaves you with no idea why it's true.

This is terrible. Who gives a damn about $T_2 = ST_1S\inv$?
Only if you know that the matrices are linear maps does this make sense:
$T_2$ is just $T_1$ rewritten with a different choice of basis.

In my eyes, this is \emph{evil}.
Linear algebra is the study of \emph{linear maps},
but it is taught as the study of \emph{arrays of numbers},
and no one knows what these numbers mean.
And for a good reason: the numbers are meaningless.
They are a highly convenient way of encoding the matrix,
but they are not the main objects of study,
any more than the dates of events are the main objects of study in history.

The other huge downside is that people get the impression
that the only (real) vector space in existence is $\RR^{\oplus n}$.
As explained in \Cref{rem:vector_spaces_have_essence},
while you \emph{can} work this way if you're a soulless robot,
it's very unnatural for humans to do so.

When I took Math 55a as a freshman at Harvard,
I got the exact opposite treatment:
we did all of linear algebra without writing down a single matrix.
During all this time I was quite confused.
What's wrong with a basis?
I didn't appreciate until later that this approach was the
morally correct way to treat the subject: it made it clear what was happening.

Throughout this project, I've tried to strike a balance between these two approaches,
using matrices when appropriate to illustrate the maps and to simplify proofs,
but ultimately writing theorems and definitions in their \emph{morally correct} form.
I hope that this has both the advantage of giving the ``right'' definitions
while being concrete enough to be digested.
But I would just like to say for the record that,
if I had to pick between the high school approach and the 55a approach,
I would pick 55a in a heartbeat.

\section{A Word on General Modules}
\prototype{$\ZZ[\sqrt2]$ is a $\ZZ$-module of rank two.}
I focused mostly on vector fields (aka modules over a field) in this chapter
for simplicity, so I want to make a few remarks about
modules over a general commutative ring $R$ before concluding.

Firstly, recall that for general modules,
we say ``generating set'' instead of ``spanning set''.
Shrug.

The main issue with rings is that our key theorem \Cref{thm:vector_best}
fails in spectacular ways.
For example, consider $\ZZ$ as a $\ZZ$-module over itself.
Then $\{2\}$ is linearly independent, but it cannot be extended to a basis.
Similarly, $\{2,3\}$ is spanning, but one cannot cut it down to a basis.
You can see why defining dimension is going to be difficult.

Nonetheless, there are still analogs of some of the definitions above.
\begin{definition}
	An $R$-module $M$ is called \vocab{finitely generated} if it has a finite generating set.
\end{definition}
\begin{definition}
	An $R$-module $M$ is called \vocab{free} if it has a basis.
	As said before, the analogue of the Dimension Theorem holds,
	and we use the word \vocab{rank} to denote the size of the basis.
	As before, there's an isomorphism $M \cong R^{\oplus n}$ where $n$ is the rank.
\end{definition}
\begin{example}[An Example of a $\ZZ$-module]
	The $\ZZ$-module
	\[ \ZZ[\sqrt2] = \left\{ a + b\sqrt 2 \mid a,b \in \ZZ \right\} \]
	has a basis $\{1, \sqrt 2\}$, so we say it is
	a free $\ZZ$-module of rank $2$.
\end{example}

Finally, recall that an abelian group can be viewed a $\ZZ$-module
(and in fact vice-versa!), so we can (and will) apply these words to abelian groups.
Thus:
\begin{abuse}
We'll use the notation $G \oplus H$ for two abelian groups $G$ and $H$
for their Cartesian product, emphasizing the fact that $G$ and $H$ are abelian.
\end{abuse}
This will happen when we study algebraic number theory and homology groups.

\section\problemhead
\begin{problem}[TSTST 2014]
	Let $P(x)$ and $Q(x)$ be arbitrary polynomials with real coefficients, and let $d$ be the degree of $P(x)$. Assume that $P(x)$ is not the zero polynomial.
	Prove that there exist polynomials $A(x)$ and $B(x)$ such that
	\begin{enumerate}[(i)]
		\ii Both $A$ and $B$ have degree at most $d/2$,
		\ii At most one of $A$ and $B$ is the zero polynomial,
		\ii $P$ divides $A+Q \cdot B$.
	\end{enumerate}
	\begin{hint}
		Interpret as $V \oplus V \to W$ for suitable $V$, $W$.
	\end{hint}
	\begin{sol}
		Let $V$ be the space of real polynomials with degree at most $d/2$ (which has dimension $1 + \left\lfloor d/2 \right\rfloor$), and $W$ be the space of real polynomials modulo $P$ (which has dimension $d$).
		Then $\dim (V \oplus V) > \dim W$.
		So the linear map $V \times V \to W$ by $(A,B) \mapsto A + Q \cdot B$
		cannot be injective.
	\end{sol}
\end{problem}

\begin{problem}
	[Putnam 2003]
	Do there exist polynomials $a(x)$, $b(x)$ , $c(y)$, $d(y)$ such that
	\[ 1 + xy + (xy)^2 = a(x)c(y) + b(x)d(y) \]
	holds identically?
	\begin{hint}
		Plug in $y = -1, 0, 1$. Use dimensions of $\RR[x]$.
	\end{hint}
\end{problem}

\begin{sproblem}
	[Idempotents are Projection Maps]
	\label{prob:idempotent}
	Let $TP: V \to V$ be a linear map, where $V$ is a vector space
	(not necessarily finite-dimensional).
	Suppose $P$ is \vocab{idempotent}, meaning $P \circ P = P$,
	or equivalently $P$ is the identity on its image.
	Prove that $V = \ker P \oplus \img P$.
	Thus we can think of $P$ as \emph{projection} onto the subspace $\img P$.
\end{sproblem}

\begin{sproblem}
	\label{prob:endomorphism_eventual_lemma}
	Let $V$ be a finite dimensional vector space.
	Let $T : V \to V$ be a linear map, and let $T^n : V \to V$ denote $T$ applied $n$ times.
	Prove that there exists an integer $N$ such that
	\[ V = \ker T^N \oplus \img T^N. \]
	\begin{hint}
		Use the fact that the infinite chain of subspaces
		\[ \ker T \subseteq \ker T^2 \subseteq \ker T^3 \subseteq \dots \]
		and the similar chain for $\img T$ must eventually stabilize
		(look at the dimension of the spaces).
	\end{hint}
\end{sproblem}

%Now consider the sequences
%\[
%	\{0\} \subsetneq \ker S \subseteq \ker S^2 \subseteq \ker S^3 \subseteq \dots
%	\text{  and  }
%	V \supsetneq \img S \supseteq \img S^2 \supseteq \img S^3 \supseteq \dots.
%\]
%\begin{ques}
%	Show that these spaces are all $S$-invariant.
%	Why do these inclusions happen?
%\end{ques}
%For dimension reasons, these subspaces must eventually stabilize: for some large integer $N$,
%$\ker S^N = \ker S^{N+1} = \dots$ and $\img S^N = \img S^{N+1} = \img S^{N+2} = \dots$.
%When this happens, $\ker S^N \bigcap \img S^N = \{0\}$ (why?), and 
%thus for dimension reasons $V_1 = \ker S^N \oplus \img S^N$.\footnote{%
%	Recall here that for \emph{any} map $T : V \to W$, $\dim \ker T + \dim \img T = \dim V$.}

