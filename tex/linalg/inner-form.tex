\chapter{Inner Product Spaces}
It will often turn out that our vector spaces which look more like $\RR^n$
not only have the notion of addition, but also a notion of \emph{orthogonality}
and the notion of \emph{distance}.
All this is achieved by endowing the vector space with a so-called \textbf{inner form},
which you likely already know as the ``dot product'' for $\RR^n$.
Indeed, in $\RR^n$ you already know that
\begin{itemize}
	\ii $v \cdot w = 0$ if and only if $v$ and $w$ are perpendicular, and
	\ii $|v|^2 = v \cdot v$.
\end{itemize}
The purpose is to quickly set up this structure in full generality.
Some highlights of the chapter:
\begin{itemize}
	\ii We'll see that the high school ``dot product'' formulation is actually very natural:
	it falls out from the two axioms we listed above.
	If you ever wondered why $\sum a_ib_i$ behaves as nicely as it does, now you'll know.
	\ii We show how the inner form can be used to make $V$ into a \emph{metric space},
	giving it more geometric structure.
	\ii We'll identify $V \simeq V^\vee$ in a way that wasn't possible before,
	and as a corollary deduce the following nice result:
	symmetric matrices with complex entries always have real eigenvalues.
\end{itemize}

Throughout this chapter, \emph{all vector spaces are over $\CC$ or $\RR$},
unless otherwise specified.
We'll generally prefer working over $\CC$ instead of $\RR$ since,
like we saw in the chapter on spectral theory, $\CC$ is algebraically closed
in a nice way that $\RR$ is not,
and moreover every real matrix can be thought of as a matrix with complex entries as well.

\section{Defining the Norm}
\prototype{Dot product in $\RR^n$.}

First, let's define the inner form for real spaces.
Rather than the notation $v \cdot w$ it is most customary
to use $\left< v,w \right>$ for general vector spaces.
\begin{definition}
	Let $V$ be a real vector space.
	A \vocab{real inner form}\footnote{%
	Other names include ``inner product'', ``dot product'', ``positive definite symmetric bilinear form'', \dots}
	is a function
	\[ \left< \bullet, \bullet \right> : V \times V \to \RR \]
	which satisfies the following properties:
	\begin{itemize}
			\ii The form is \vocab{symmetric}: for any $v,w \in V$ we have
		\[ \left< v,w \right> = \left< w,v\right>. \]
		Of course, one would expect this property from a product.

		\ii The form is \textbf{linear in the first argument}, meaning that
		$\left< cx, y\right> = c \left< x,y \right>$ and
		$\left< x+y, z\right> = \left< x,z\right> + \left< y,z\right>$.
		This is often summarized by the single equation
		\[ \left< cx+y, z \right> = c \left< x,z \right> + \left< y,z \right>. \]
		Of course by symmetry it is linear in the second argument too.
		This should be thought of as ``distributive over addition''.

		\ii The form is \vocab{positive definite}, meaning $\left<v,v\right> \ge 0$
		is a nonnegative real number, and equality takes place only if $v = 0_V$.
	\end{itemize}
\end{definition}

\begin{example}
	[$\RR^n$]
	As we already know, one can define the inner form on $\RR^n$ as follows.
	Let $\ee_1 = (1, 0, \dots, 0)$, $\ee_2 = (0, 1, \dots, 0)$, \dots, $\ee_n = (0, \dots, 0, 1)$
	be the standard basis.
	Then we let
	\[ 
		\left< a_1 \ee_1 + \dots + a_n \ee_n, b_1 \ee_1 + \dots + b_n \ee_n \right>
		\defeq a_1b_1 + \dots + a_nb_n.
	\]
	It's easy to see this is bilinear (symmetric and linear in both arguments).
	To see it is positive definite, note that if $a_i = b_i$
	then the dot product is $a_1^2 + \dots + a_n^2$, which is zero exactly when all $a_i$ are zero.
\end{example}

The definition for a complex product space is similar, but has a minor difference:
rather than symmetry we instead have \emph{conjugate symmetry}
meaning $\left< v, w \right> = \ol{ \left< w,v \right> }$.
Thus, while we still have linearity in the first argument,
we actually have a different linearity for the second argument.
To be explicit:
\begin{definition}
	Let $V$ be a complex vector space.
	A \vocab{complex inner form} $\left< \bullet, \bullet \right> : V \times V \to \CC$
	is a function which satisfies the following properties:
	\begin{itemize}
		\ii The form has \vocab{conjugate symmetry}, which means that
		for any $v,w \in V$ we have
		\[ \left< v,w \right> = \ol{\left< w,v\right>}. \]

		\ii The form is \textbf{linear in the first argument}, so again we have
		\[ \left< cx+y, z \right> = c \left< x,z \right> + \left< y,z \right>. \]
		However it no longer follows that it is linear in the second argument, see below!

		\ii The form is \vocab{positive definite},
		meaning $\left<v,v\right>$ is a nonnegative real number,
		and equals zero exactly when $v = 0_V$.
	\end{itemize}
\end{definition}
\begin{ques}
	Show that in a complex inner form, we instead have
	\[ \left< x, cy + z \right> = \ol c \left< x, y \right> + \left< x, z\right>.  \]
\end{ques}

\begin{example}
	[$\CC^n$]
	The dot product in $\CC^n$ is defined as follows:
	let $\ee_1$, $\ee_2$, \dots, $\ee_n$ be the standard basis.
	For complex numbers $w_i$, $z_i$ we set
	\[ 
		\left< w_1 \ee_1 + \dots + w_n \ee_n, z_1 \ee_1 + \dots + z_n \ee_n \right>
		\defeq w_1\ol{z_1} + \dots + w_n \ol{z_n}.
	\]
\end{example}
\begin{ques}
	Check that the above is in fact a complex inner form.
\end{ques}

\begin{definition}
	A real or complex vector space with an inner form is called
	an \vocab{inner product space}.
\end{definition}
\begin{remark}
	\label{rem:only_complex_forms}
	Note that we can apply the axioms for a complex inner form
	to those for a real inner form verbatim; in $\RR$, conjugation is the identity.
	Thus for real vector spaces, for example, ``conjugate symmetry'' is just ``symmetry''.
	Consequently, when proving results about inner product spaces it usually
	suffices to do just the complex case.
\end{remark}

The above example explains why we have to require the complex inner form
to satisfy conjugate symmetry rather than just symmetry.
If we had tried to define the dot product as $\sum w_i z_i$,
then we would have lost the condition of being positive definite,
because there is no guarantee that $\left< v,v \right> = \sum z_i^2$ will even be a real number at all.
On the other hand, with conjugate symmetry we actually enforce $\left< v,v \right> = \ol{\left< v,v \right>}$,
id est $\left< v,v \right> \in \RR$ for every $v$.

Now that we have a dot product, we can talk both about the norm and orthogonality.

\section{Norms}
\prototype{$\RR^n$ becomes its usual Euclidean space with the vector norm.}

The inner form equips our vector space with a notion of distance, which we call the norm.
\begin{definition}
	Let $V$ be an inner product space.
	The \vocab{norm} of $v \in V$ is defined by 
	\[ \norm{v} = \sqrt{\left<v,v\right>}. \]
	This definition makes sense because we assumed our form to be positive definite.
\end{definition}

\begin{example}[$\RR^n$ and $\CC^n$ Are Normed Vector Spaces]
	When $V = \RR^n$ or $V = \CC^n$ with the standard dot product norm,
	then the norm of $v$ corresponds to the absolute value that we are used to.
\end{example}

Our goal now is to prove that
\begin{moral}
	With the metric $d(v,w) = \norm{v-w}$, $V$ becomes a metric space.
\end{moral}
To do this we have to establish the triangle inequality.
We use the following lemma as a stepping stone:
Let's now prove something we all know and love:
\begin{lemma}
	[Cauchy-Schwarz]
	Let $V$ be an inner product space.
	For any $v,w \in V$ we have
	\[ \left\lvert \left< v,w\right> \right\rvert
	\le \norm{v} \norm{w} \]
	with equality if and only if $v$ and $w$ are linearly dependent.
\end{lemma}
\begin{proof}
	The key to the proof is to think about the equality case:
	we'll use the inequality $\left< cv-w, cv-w\right> \ge 0$.
	Deferring the choice of $c$ until later, we compute
	\begin{align*}
		0 &\le \left< cv-w, cv-w \right> \\
		&= \left< cv, cv\right> - \left< cv, w\right> - \left< w, cv\right> + \left< w,w \right> \\
		&= |c|^2 \left< v,v \right> - c \left< v,w \right> - \ol c \left< w,v \right> + \left< w,w \right> \\
		&= |c|^2 \norm{v}^2 + \norm{w}^2 - c \left< v,w \right> - \ol{c \left< v,w\right> }. \\
		2 \Re \left[ c \left< v,w \right> \right] &\le |c|^2 \norm{v}^2 + \norm{w}^2  \\
		\intertext{At this point, a good choice of $c$ is} 
		c &= \frac{ \norm w}{\norm v} \cdot \frac{|\left< v,w\right>|}{\left< v,w\right>}. \\
		\intertext{since then}
		c \left< v,w \right> &= \frac{\norm w}{\norm v} \left\lvert \left< v,w\right> \right\rvert \in \RR \\
		|c| &= \frac{\norm w}{\norm v} \\
		\intertext{whence the inequality becomes}
		 2\frac{\norm w}{\norm v} \left\lvert \left< v,w\right> \right\rvert &\le 2 \norm{w}^2  \\
		 \left\lvert \left< v,w\right> \right\rvert &\le \norm v \norm w. \qedhere
	\end{align*}
\end{proof}
Thus:
\begin{theorem}
	[Triangle Inequality]
	We always have
	\[ \norm v + \norm w \le \norm{v+w} \]
	with quality if and only if $v$ and $w$ are linearly dependent.
\end{theorem}
\begin{exercise}
	Prove this by squaring both sides, and applying Cauchy-Schwarz.
\end{exercise}

In this way, our vector space now has a topological structure of a metric space.

\section{Orthogonality}
\prototype{Still $\RR^n$!}
Our next goal is to give the geometric notion of ``perpendicular''.
The definition is easy enough:
\begin{definition}
	Two nonzero vectors $v$ and $w$ in an inner product space
	are \vocab{orthogonal} if $\left< v,w \right> = 0$.
\end{definition}

As we expect from our geometric intution in $\RR^n$, this implies independence:
\begin{lemma}[Orthogonal Vectors Are Independent]
	Any set of nonzero pairwise orthogonal vectors is linearly independent.
\end{lemma}
\begin{proof}
	Consider a dependence
	\[ a_1 v_1 + \dots + a_n v_n = 0 \]
	for $a_i$ in $\RR$ or $\CC$.
	Then \[ 0_V = \left< v_1, \sum a_i v_i \right> = a_1 \norm{v_i}^2. \]
	Hence $a_1 = 0$, since we assumed $\norm{v_1} \neq 0$.
	Similarly $a_2 = \dots = a_m = 0$.
\end{proof}

In light of this, we can now consider a stronger condition on our bases:
\begin{definition}
	An \vocab{orthonormal} basis of a finite-dimensional inner product space $V$
	is a basis $e_1$, \dots, $e_n$ such that
	$\norm{e_i} = 1$ for every $i$ and $\left< e_i, e_j \right> = 0$ for any $i \neq j$.
\end{definition}
\begin{example}[$\RR^n$ and $\CC^n$ have standard bases]
	In $\RR^n$ and $\CC^n$ equipped with the standard dot product,
	the standard basis $\ee_1$, \dots, $\ee_n$ is also orthonormal.
\end{example}
This is no loss of generality:
\begin{theorem}[Gram-Schmidt]
	Let $V$ be a finite-dimensional inner product space.
	Then it has an orthonormal basis.
\end{theorem}
\begin{proof}[Sketch of Proof]
	One constructs the orthonormal basis explicitly from any basis
	$e_1$, \dots, $e_n$ of $V$.
	Define $\opname{proj}_u(v) = \frac{\left< v,u\right>}{\left< u,u\right>} u$.
	Then recursively define
	\begin{align*}
		u_1 &= e_1 \\
		u_2 &= e_2 - \opname{proj}_{u_1}(e_2) \\
		u_3 &= e_3 - \opname{proj}_{u_1}(e_3) - \opname{proj}_{u_2}(e_3) \\
		&\vdotswithin{=} \\
		u_n &= e_n - \opname{proj}_{u_1}(e_n) - \dots - \opname{proj}_{u_{n-1}}(e_{n-1}).
	\end{align*}
	One can show the $u_i$ are pairwise orthogonal and not zero.
\end{proof}
Thus, we can generally assume our bases are orthonormal.

Worth remarking:
\begin{example}[The Dot Product is the ``Only'' Inner Form]
	Let $V$ be a finite-dimensional inner product space,
	and consider \emph{any} orthonormal basis $e_1, \dots, e_n$.
	Then we have that
	\[ \left< a_1e_1 + \dots + a_ne_n, b_1e_1 + \dots + b_ne_n \right> 
		= \sum_{i,j=1}^n a_i\ol{b_j} \left< e_i, e_j \right>
		= \sum_{i=1}^n a_i \ol{b_i} \]
	owing to the fact that the $\{e_i\}$ are orthonormal.
\end{example}
So the dot product is actually a \emph{highly natural} inner form to consider.
It arises just out of the geometric consideration that in an orthonormal basis
we ought to have norm $1$ and pairwise orthogonality;
this is enough to obtain the dot product by extending linearly.


\section{Identifying with the Dual Space}
Earlier I complained that there was no natural isomorphism $V \simeq V^\vee$.
But in fact, given an inner form we can actually make such an identification:
that is we can naturally associate every linear map $\xi : V \to k$ with a vector $v \in V$.

To see how we might do this, suppose $V = \RR^3$ for now with standard basis $\ee_1$, $\ee_2$, $\ee_3$.
How might we use the standard product to represent a map from $V \to \RR$?
For example, take $\xi \in V^\vee$ by $\xi(\ee_1) = 3$, $\xi(\ee_2) = 4$ and $\xi(\ee_3) = 5$.
Actually, I claim that
\[ \xi(v) = \left< v, 3\ee_1 + 4\ee_2 + 5\ee3 \right> \]
for every $v$.
And this is obvious, once you do the computation.

More generally:
\begin{theorem}[$V$ is isomorphic to $V^\vee$]
	Let $V$ be a finite-dimensional inner product space and $V^\vee$ its dual.
	Then the map $V \to V^\vee$ by
	\[ v \mapsto \left<  - , v \right> \in V^\vee \]
	is an isomorphism.
\end{theorem}
\begin{proof}
	It suffices to show that the map is injective and surjective.
	\begin{itemize}
		\ii Injective: suppose $\left< w, v_1 \right> = \left< w, v_2\right>$
		for every vector $w$.
		This means $\left< w, v_1 - v_2\right> = 0$ for every vector $w \in V$.
		This can only happen if $v_1 - v_2 = 0$; for example, take $w = v_1 - v_2$
		and use positive definiteness.
		\ii Surjective: take an orthonormal basis, and mimic the argument we gave earlier.
	\end{itemize}
	Actually, since we already know $\dim V = \dim V^\vee$ we only had to prove one of the above.
	As a matter of personal taste, I find the proof of injectivity more elegant,
	and the proof of surjectivity more enlightening, so I included both.
\end{proof}

\section{The Transpose of a Matrix}
Let $V$ be a finite-dimensional inner product space and let $T : V \to V$.
The \vocab{adjoint} or \vocab{conjugate transpose} of $T$, denoted $T^\dagger$,
is defined as follows: for every vector $w \in V$, we let $T^\dagger(w)$ be the unique vector with
\[ \left< v, T^\dagger(w) \right> = \left< T(v), w \right> \]
for every $v \in V$.
This is well-defined, because $v \mapsto \left< T(v), w \right>$ is some function in $V^\vee$,
and hence by the isomorphism we described earlier it can be written uniquely in the form $\left< v, x \right>$
for some $x \in V$; we then let $T^\dagger(w) = x$.

Of course, the name conjugate transpose suggests something else:
\begin{theorem}
	[Adjoints are Conjugate Transposes]
	Fix an orthonormal basis of a finite-dimensional inner product $V$.
	If we write $T$ as a matrix in this basis, then $T^\dagger$ is the conjugate transpose.
\end{theorem}
\begin{proof}
	One-line version: take $v$ and $w$ to be basis elements, and this falls right out.

	Full proof: let
	\[ T = \left(
		\begin{array}{ccc}
			a_{11} & \dots & a_{1n} \\
			\vdots & \ddots & \vdots \\
			a_{n1} & \dots & a_{nn}
		\end{array}
		\right)
	\]
	in this basis $e_1$, \dots, $e_n$.
	Then, letting $w = e_i$ and $v = e_j$ we deduce that
	\[ \left< e_i, T^\dagger(e_j) \right> = \left< T(e_i) , e_j\right> = a_{ji}
		\implies
		\left< T^\dagger(e_j), e_i \right> = \ol{a_{ji}} \]
	for any $i$, which is enough to deduce the result.
\end{proof}

In this way we can talk about the transpose of a matrix in a meaningful way.

\section{Spectral Theory of Normal Maps}
The world would be a very beautiful place if it turned out
that we could pick a basis of eigenvectors that was also orthonormal.
This is of course far too much to hope for; even without the orthonormal condition,
we saw in the previous chapter that we could still have $1$'s off the diagonal.
However, it turns out that there is a good characterization of this:

\begin{definition}
	We say $T$ is \vocab{normal} if $TT^\dagger = T^\dagger T$.
\end{definition}
\begin{theorem}
	[Normal $\iff$ Diagonalizable in Inner Form]
	Let $V$ be a finite-dimensional inner product space.
	A linear map $T : V \to V$ is normal
	if and only if one can pick an orthonormal basis of eigenvectors.
\end{theorem}
\begin{proof}
	One direction is trivial (figure out which one).
	For the other, see the last page of 
	\url{math.ucla.edu/~mikehill/Math5651/Lecture28.pdf} until
	I get bored enough to actually write down the proof again.
	\todo{remove snarky note}
\end{proof}
This means that not only can we write
\[ T = \left(
	\begin{array}{cccc}
		\lambda_1 & \dots & \dots & 0 \\
		0 & \lambda_2 & \dots & 0 \\
		\vdots & \vdots & \ddots & \vdots \\
		0 & 0 & \dots & \lambda_n
	\end{array}
	\right)
\]
but moreover that the basis associated with this matrix happens to be orthonormal vectors.


In particular,
\begin{definition}
	A matrix over a complex finite-dimensional inner product space
	is said to be \vocab{self-adjoint} or \vocab{Hermitian} if $T = T^\dagger$;
	i.e.\ as a matrix in any orthonormal basis, $T$ is its own conjugate transpose.
	For real finite-dimensional inner product spaces we say ``self-adjoint'', ``Hermitian'' or \vocab{symmetric}.
\end{definition}

Then we have that
\begin{theorem}[Hermitian matrices have real eigenvalues]
	A Hermitian matrix $T$ is diagonalizable, and all its eigenvalues are real.
\end{theorem}
\begin{proof}
	Obviously Hermitian $\implies$ normal,
	so write it in the orthonormal basis of eigenvectors.
	To see that the eigenvalues are real, note that $T = T^\dagger$
	means $\lambda_i = \ol{\lambda_i}$ for every $i$.
\end{proof}


\section\problemhead
\begin{problem}
	[Pythagorean Theorem]
	Show that if $\left< v,w \right> = 0$ in an inner product space,
	then $\norm{v}^2 + \norm{w}^2 = \norm{v+w}^2$.
\end{problem}

\begin{problem}[Taiwan IMO Camp]
	\gim
	In a town there are $n$ people and $k$ clubs. Each club has an odd number of members, and any two clubs have an even number of common members. Prove that $k \le n$.
	\begin{hint}
		Dot products in $\FF_2$.
	\end{hint}
	\begin{sol}
		Interpret clubs as vectors in the vector space $\mathbb F_2^n$.
		Consider a ``dot product'' to show that all $k$ vectors are linearly independent.
		Thus $k \le \dim \mathbb F_2^n = n$.
	\end{sol}
\end{problem}

\begin{problem}[Putnam 2014]
	\gim
	Let $n$ be a positive integer. What is the largest $k$ for which there exist $n\times n$ matrices $M_1,\dots,M_k$ and $N_1,\dots,N_k$ with real entries such that for all $i$ and $j,$ the matrix product $M_iN_j$ has a zero entry somewhere on its diagonal if and only if $i\ne j?$
	\begin{hint}
		$k = n^n$.
		Endow tensor products with an inner form.
		Note that ``zero entry somewhere on its diagonal''
		is equivalent to the product of those entries being zero.
	\end{hint}
\end{problem}
