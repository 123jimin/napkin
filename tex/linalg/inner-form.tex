\chapter{Inner product spaces}
It will often turn out that our vector spaces which look more like $\RR^n$
not only have the notion of addition, but also a notion of \emph{orthogonality}
and the notion of \emph{distance}.
All this is achieved by endowing the vector space with a so-called \textbf{inner form},
which you likely already know as the ``dot product'' for $\RR^n$.
Indeed, in $\RR^n$ you already know that
\begin{itemize}
	\ii $v \cdot w = 0$ if and only if $v$ and $w$ are perpendicular, and
	\ii $|v|^2 = v \cdot v$.
\end{itemize}
The purpose is to quickly set up this structure in full generality.
Some highlights of the chapter:
\begin{itemize}
	\ii We'll see that the high school ``dot product''
	formulation is actually very natural:
	it falls out from the two axioms we listed above.
	If you ever wondered why $\sum a_i b_i$ behaves as nicely as it does,
	now you'll know.
	\ii We show how the inner form can be used to make $V$ into a \emph{metric space},
	giving it more geometric structure.
	\ii A few chapters later,
	we'll identify $V \cong V^\vee$ in a way that wasn't possible before,
	and as a corollary deduce the nice result that
	symmetric matrices with real entries
	always have real eigenvalues.
\end{itemize}

Throughout this chapter, \emph{all vector spaces are over $\CC$ or $\RR$},
unless otherwise specified.
We'll generally prefer working over $\CC$ instead of $\RR$ since,
like we saw in the chapter on spectral theory, $\CC$ is algebraically closed
in a nice way that $\RR$ is not.
Every real matrix can be thought of as a matrix with complex entries anyways.

\section{Defining the norm}
\prototype{Dot product in $\RR^n$.}

First, let's define the inner form for real spaces.
Rather than the notation $v \cdot w$ it is most customary
to use $\left< v,w \right>$ for general vector spaces.
\begin{definition}
	Let $V$ be a real vector space.
	A \vocab{real inner form} (or \vocab{bilinear form})\footnote{Other
		names include ``inner product'', ``dot product'',
		``positive definite nondegenerate symmetric bilinear form'',
		\dots}
	is a function
	\[ \left< \bullet, \bullet \right> : V \times V \to \RR \]
	which satisfies the following properties:
	\begin{itemize}
		\ii The form is \vocab{symmetric}: for any $v,w \in V$ we have
		\[ \left< v,w \right> = \left< w,v\right>. \]
		Of course, one would expect this property from a product.

		\ii The form is \textbf{linear in the first argument}, meaning that
		$\left< cx, y\right> = c \left< x,y \right>$ and
		$\left< x+y, z\right> = \left< x,z\right> + \left< y,z\right>$.
		This is often summarized by the single equation
		\[ \left< cx+y, z \right> = c \left< x,z \right> + \left< y,z \right>. \]
		Of course by symmetry it is linear in the second argument too.
		This should be thought of as ``distributive over addition''.

		\ii The form is \vocab{positive definite}, meaning $\left<v,v\right> \ge 0$
		is a nonnegative real number, and equality takes place only if $v = 0_V$.
	\end{itemize}
\end{definition}

\begin{example}
	[$\RR^n$]
	As we already know, one can define the inner form on $\RR^n$ as follows.
	Let $e_1 = (1, 0, \dots, 0)$, $e_2 = (0, 1, \dots, 0)$,
	\dots, $e_n = (0, \dots, 0, 1)$ be the usual basis.
	Then we let
	\[ 
		\left< a_1 e_1 + \dots + a_n e_n, b_1 e_1 + \dots + b_n e_n \right>
		\defeq a_1 b_1 + \dots + a_n b_n.
	\]
	It's easy to see this is bilinear
	(symmetric and linear in both arguments).
	To see it is positive definite,
	note that if $a_i = b_i$
	then the dot product is $a_1^2 + \dots + a_n^2$,
	which is zero exactly when all $a_i$ are zero.
\end{example}

The definition for a complex product space is similar, but has a minor difference:
rather than symmetry we instead have \emph{conjugate symmetry}
meaning $\left< v, w \right> = \ol{ \left< w,v \right> }$.
Thus, while we still have linearity in the first argument,
we actually have a different linearity for the second argument.
To be explicit:
\begin{definition}
	Let $V$ be a complex vector space.
	A \vocab{complex inner form} $\left< \bullet, \bullet \right> : V \times V \to \CC$
	(or \vocab{sesquilinear form})
	is a function which satisfies the following properties:
	\begin{itemize}
		\ii The form has \vocab{conjugate symmetry}, which means that
		for any $v,w \in V$ we have
		\[ \left< v,w \right> = \ol{\left< w,v\right>}. \]

		\ii The form is \textbf{linear in the first argument}, so again we have
		\[ \left< cx+y, z \right> = c \left< x,z \right> + \left< y,z \right>. \]
		However it no longer follows that it is linear in the second argument, see below!

		\ii The form is \vocab{positive definite},
		meaning $\left<v,v\right>$ is a nonnegative real number,
		and equals zero exactly when $v = 0_V$.
	\end{itemize}
\end{definition}
\begin{ques}
	Show that in a complex inner form, we instead have
	\[ \left< x, cy + z \right> = \ol c \left< x, y \right> + \left< x, z\right>.  \]
\end{ques}

\begin{example}
	[$\CC^n$]
	The dot product in $\CC^n$ is defined as follows:
	let $\ee_1$, $\ee_2$, \dots, $\ee_n$ be the standard basis.
	For complex numbers $w_i$, $z_i$ we set
	\[ 
		\left< w_1 \ee_1 + \dots + w_n \ee_n, z_1 \ee_1 + \dots + z_n \ee_n \right>
		\defeq w_1\ol{z_1} + \dots + w_n \ol{z_n}.
	\]
\end{example}
\begin{ques}
	Check that the above is in fact a complex inner form.
\end{ques}

\begin{definition}
	An \vocab{inner product} space either a real vector space
	equipped with a real inner form,
	or a complex vector space equipped with a complex inner form.
\end{definition}
\begin{remark}
	\label{rem:only_complex_forms}
	Note that we can apply the axioms for a complex inner form
	to those for a real inner form verbatim
	; in $\RR$, conjugation is the identity.
	Thus for real vector spaces, for example,
	``conjugate symmetry'' is just ``symmetry''.
	Consequently, when proving results about inner product spaces it usually
	suffices to do just the complex case.
\end{remark}

The above example explains why we have to require the complex inner form
to satisfy conjugate symmetry rather than just symmetry.
If we had tried to define the dot product as $\sum w_i z_i$,
then we would have lost the condition of being positive definite,
because there is no guarantee that $\left< v,v \right> = \sum z_i^2$ will even be a real number at all.
On the other hand, with conjugate symmetry we actually enforce $\left< v,v \right> = \ol{\left< v,v \right>}$,
i.e.\ $\left< v,v \right> \in \RR$ for every $v$.

Now that we have a dot product, we can talk both about the norm and orthogonality.

\section{Norms}
\prototype{$\RR^n$ becomes its usual Euclidean space with the vector norm.}

The inner form equips our vector space with a notion of distance, which we call the norm.
\begin{definition}
	Let $V$ be an inner product space.
	The \vocab{norm} of $v \in V$ is defined by 
	\[ \norm{v} = \sqrt{\left<v,v\right>}. \]
	This definition makes sense because we assumed our form to be positive definite.
\end{definition}

\begin{example}[$\RR^n$ and $\CC^n$ are normed vector spaces]
	When $V = \RR^n$ or $V = \CC^n$ with the standard dot product norm,
	then the norm of $v$ corresponds to the absolute value that we are used to.
\end{example}

Our goal now is to prove that
\begin{moral}
	With the metric $d(v,w) = \norm{v-w}$, $V$ becomes a metric space.
\end{moral}
To do this we have to establish the triangle inequality.
We use the following lemma as a stepping stone:
Let's now prove something we all know and love:
\begin{lemma}
	[Cauchy-Schwarz]
	Let $V$ be an inner product space.
	For any $v,w \in V$ we have
	\[ \left\lvert \left< v,w\right> \right\rvert
	\le \norm{v} \norm{w} \]
	with equality if and only if $v$ and $w$ are linearly dependent.
\end{lemma}
\begin{proof}
	The theorem is immediate if $\left< v,w\right> = 0$.
	It is also immediate if $\norm{v} \norm{w} = 0$,
	since then one of $v$ or $w$ is the zero vector.
	So henceforth we assume all these quantities are nonzero
	(as we need to divide by them later).

	The key to the proof is to think about the equality case:
	we'll use the inequality $\left< cv-w, cv-w\right> \ge 0$.
	Deferring the choice of $c$ until later, we compute
	\begin{align*}
		0 &\le \left< cv-w, cv-w \right> \\
		&= \left< cv, cv\right> - \left< cv, w\right> - \left< w, cv\right> + \left< w,w \right> \\
		&= |c|^2 \left< v,v \right> - c \left< v,w \right> - \ol c \left< w,v \right> + \left< w,w \right> \\
		&= |c|^2 \norm{v}^2 + \norm{w}^2 - c \left< v,w \right> - \ol{c \left< v,w\right> } \\
		2 \Re \left[ c \left< v,w \right> \right] &\le |c|^2 \norm{v}^2 + \norm{w}^2  \\
		\intertext{At this point, a good choice of $c$ is}
		c &= \frac{ \norm w}{\norm v} \cdot \frac{|\left< v,w\right>|}{\left< v,w\right>} \\
		\intertext{since then}
		c \left< v,w \right> &= \frac{\norm w}{\norm v} \left\lvert \left< v,w\right> \right\rvert \in \RR \\
		|c| &= \frac{\norm w}{\norm v} \\
		\intertext{whence the inequality becomes}
		 2\frac{\norm w}{\norm v} \left\lvert \left< v,w\right> \right\rvert &\le 2 \norm{w}^2  \\
		 \left\lvert \left< v,w\right> \right\rvert &\le \norm v \norm w. \qedhere
	\end{align*}
\end{proof}
Thus:
\begin{theorem}
	[Triangle inequality]
	We always have
	\[ \norm v + \norm w \ge \norm{v+w} \]
	with equality if and only if $v$ and $w$ are linearly dependent.
\end{theorem}
\begin{exercise}
	Prove this by squaring both sides, and applying Cauchy-Schwarz.
\end{exercise}

In this way, our vector space now has a topological structure of a metric space.

\section{Orthogonality}
\prototype{Still $\RR^n$!}
Our next goal is to give the geometric notion of ``perpendicular''.
The definition is easy enough:
\begin{definition}
	Two nonzero vectors $v$ and $w$ in an inner product space
	are \vocab{orthogonal} if $\left< v,w \right> = 0$.
\end{definition}

As we expect from our geometric intution in $\RR^n$, this implies independence:
\begin{lemma}[Orthogonal vectors are independent]
	Any set of nonzero pairwise orthogonal vectors is linearly independent.
\end{lemma}
\begin{proof}
	Consider a dependence
	\[ a_1 v_1 + \dots + a_n v_n = 0 \]
	for $a_i$ in $\RR$ or $\CC$.
	Then \[ 0_V = \left< v_1, \sum a_i v_i \right> = a_1 \norm{v_i}^2. \]
	Hence $a_1 = 0$, since we assumed $\norm{v_1} \neq 0$.
	Similarly $a_2 = \dots = a_m = 0$.
\end{proof}

In light of this, we can now consider a stronger condition on our bases:
\begin{definition}
	An \vocab{orthonormal} basis of a
	\emph{finite-dimensional} inner product space $V$
	is a basis $e_1$, \dots, $e_n$ such that
	$\norm{e_i} = 1$ for every $i$ and
	$\left< e_i, e_j \right> = 0$ for any $i \neq j$.
\end{definition}
\begin{example}[$\RR^n$ and $\CC^n$ have standard bases]
	In $\RR^n$ and $\CC^n$ equipped with the standard dot product,
	the standard basis $\ee_1$, \dots, $\ee_n$ is also orthonormal.
\end{example}
This is no loss of generality:
\begin{theorem}[Gram-Schmidt]
	Let $V$ be a finite-dimensional inner product space.
	Then it has an orthonormal basis.
\end{theorem}
\begin{proof}[Sketch of Proof]
	One constructs the orthonormal basis explicitly from any basis
	$e_1$, \dots, $e_n$ of $V$.
	Define $\opname{proj}_u(v) = \frac{\left< v,u\right>}{\left< u,u\right>} u$.
	Then recursively define
	\begin{align*}
		u_1 &= e_1 \\
		u_2 &= e_2 - \opname{proj}_{u_1}(e_2) \\
		u_3 &= e_3 - \opname{proj}_{u_1}(e_3) - \opname{proj}_{u_2}(e_3) \\
		&\vdotswithin{=} \\
		u_n &= e_n - \opname{proj}_{u_1}(e_n) - \dots - \opname{proj}_{u_{n-1}}(e_n).
	\end{align*}
	One can show the $u_i$ are pairwise orthogonal and not zero.
\end{proof}
Thus, we can generally assume our bases are orthonormal.

Worth remarking:
\begin{example}[The dot product is the ``only'' inner form]
	Let $V$ be a finite-dimensional inner product space,
	and consider \emph{any} orthonormal basis $e_1, \dots, e_n$.
	Then we have that
	\[ \left< a_1 e_1 + \dots + a_n e_n,
		b_1 e_1 + \dots + b_n e_n \right>
		= \sum_{i,j=1}^n a_i\ol{b_j} \left< e_i, e_j \right>
		= \sum_{i=1}^n a_i \ol{b_i} \]
	owing to the fact that the $\{e_i\}$ are orthonormal.
\end{example}
And now you know why the dot product expression is so ubiquitous.

\section{Hilbert spaces}
In algebra we are usually scared of infinity,
and so when we defined a basis of a vanilla vector space many chapters ago,
we only allowed finite linear combinations.
However, if we have an inner product space,
then it is a metric space and we \emph{can}
sometimes actually talk about convergence.

Here is how it goes:
\begin{definition}
	A \vocab{Hilbert space} is a inner product space $V$,
	such that the corresponding metric space is complete.
\end{definition}

In that case, it will now often make sense to take infinite linear combinations,
because we can look at the sequence of partial sums and let it converge.
Here is how we might do it.
Let's suppose we have $e_1$, $e_2$, \dots an infinite sequence
of vectors with norm $1$ and which are pairwise orthogonal.
Suppose $c_1$, $c_2$, \dots, is a sequence of real or complex numbers.
Then consider the sequence
\begin{align*}
	v_1 &= c_1 e_1 \\
	v_2 &= c_1 e_1 + c_2 e_2 \\
	v_3 &= c_1 e_1 + c_2 e_2 + c_3 e_3 \\
	&\vdotswithin=
\end{align*}

\begin{proposition}
	[Convergence criteria in a Hilbert space]
	The sequence $(v_i)$ defined above
	converges if and only if $\sum \left\lvert c_i \right\rvert^2 < \infty$.
\end{proposition}
\begin{proof}
	This will make more sense if you read\todo{link calc 101 chapter},
	so you could skip this proof if you haven't read the chapter.
	The sequence $v_i$ converges if and only if it is Cauchy,
	meaning that when $i < j$,
	\[ \norm{v_j - v_i}^2 = |c_i|^2 + \dots + |c_{j-1}|^2 \]
	tends to zero as $i$ and $j$ get large.
	This is equivalent to the sequence
	$s_n = |c_1|^2 + \dots + |c_n|^2$ being Cauchy.

	Since $\RR$ is complete, $s_n$ is Cauchy
	if and only if it converges.
	Since $s_n$ consists of nonnegative real numbers,
	that is if and only if $s_n$ is bounded,
	or $\sum_i \left\lvert c_i \right\rvert^2 < \infty$.
\end{proof}

Thus, when we have a Hilbert space, we change our definition slightly:
\begin{definition}
	A \vocab{orthonormal basis} for a Hilbert space $V$
	is a (possibly infinite) sequence $e_1$, $e_2$, \dots,
	of vectors such that
	\begin{itemize}
		\ii $\left< e_i, e_1 \right> = 1$ for all $i$,
		\ii $\left< e_i, e_j \right> = 0$ for $i \ne j$,
		i.e.\ the vectors are pairwise orthogonal
		\ii every element of $V$ can be expressed uniquely as an
		infinite linear combination
		\[ \sum_i c_i e_i \]
		where $\sum_i \left\lvert c_i \right\rvert^2 < \infty$,
		as described above.
	\end{itemize}
\end{definition}
That's the official definition, anyways.
(Note that if $\dim V < \infty$, this agrees with our usual definition,
since then there are only finitely many $e_i$.)
But for our purposes you can mostly not worry about it and instead think:
\begin{moral}
	A Hilbert space is an inner product space
	whose basis requires infinite linear combinations,
	not just finite ones.
\end{moral}
The technical condition $\sum \left\lvert c_i \right\rvert^2 < \infty$
is exactly the one which ensures the infinite sum makes sense.

\section{\problemhead}

\begin{problem}
	[Pythagorean theorem]
	Show that if $\left< v,w \right> = 0$ in an inner product space,
	then $\norm{v}^2 + \norm{w}^2 = \norm{v+w}^2$.
\end{problem}

\begin{sproblem}
	[Finite-dimensional $\implies$ Hilbert]
	Show that a finite-dimensional inner product space
	is a Hilbert space.
	\begin{hint}
		Fix an orthonormal basis $e_1$, \dots, $e_n$.
		Use the fact that $\RR^n$ is complete.
	\end{hint}
\end{sproblem}

%\begin{problem}
%	Let $V$ be a complex inner product space
%	with orthonormal basis $e_1$, \dots, $e_n$.
%	Suppose $x = a_1 e_1 + \dots + a_n e_n \in V$
%	and $y = b_1 e_1 + \dots + b_n e_n \in V$.
%	Show:
%	\begin{align*}
%		\left< x,x \right> &= \sum_\xi |a_i|^2 \\
%		a_1 &= \left< x, e_1 \right> \\
%		\left< x,y \right> &= \sum_i a_i \ol{b_i}.
%	\end{align*}
%\end{problem}

\begin{problem}[Taiwan IMO camp]
	\gim
	In a town there are $n$ people and $k$ clubs.
	Each club has an odd number of members,
	and any two clubs have an even number of common members.
	Prove that $k \le n$.
	\begin{hint}
		Dot products in $\FF_2$.
	\end{hint}
	\begin{sol}
		Interpret clubs as vectors in the vector space $\mathbb F_2^n$.
		Consider a ``dot product'' to show that all $k$ vectors are linearly independent.
		Thus $k \le \dim \mathbb F_2^n = n$.
	\end{sol}
\end{problem}

\begin{sproblem}[Inner product structure of tensors]
	\label{prob:inner_prod_tensor}
	Let $V$ and $W$ be finite-dimensional inner product spaces over $k$,
	where $k$ is either $\RR$ or $\CC$.
	\begin{enumerate}[(a)]
		\ii Find a canonical way to make $V \otimes_k W$ into an inner product space too.
		\ii Let $e_1$, \dots, $e_n$ be an orthonormal basis of $V$
		and $f_1$, \dots, $f_m$ be an orthonormal basis of $W$.
		What's an orthonormal basis of $V \otimes W$?
	\end{enumerate}
	\begin{hint}
		Define it on simple tensors then extend linearly.
	\end{hint}
	\begin{sol}
		The inner form given by
		\[ \left< v_1 \otimes w_1 , v_2 \otimes w_2 \right>_{V \otimes W}
		= \left< v_1,v_2 \right>_V \left< w_1,w_2\right>_W \]
		on pure tensors, then extending linearly.
		For (b) take $e_i \otimes f_j$ for $1 \le i \le n$, $1 \le j \le m$.
	\end{sol}
\end{sproblem}

\begin{problem}[Putnam 2014]
	\gim
	Let $n$ be a positive integer.
	What is the largest $k$ for which there exist
	$n\times n$ matrices $M_1,\dots,M_k$ and $N_1,\dots,N_k$
	with real entries such that for all $i$ and $j$,
	the matrix product $M_i N_j$ has a zero entry somewhere
	on its diagonal if and only if $i \ne j?$
	\begin{hint}
		$k = n^n$.
		Endow tensor products with an inner form.
		Note that ``zero entry somewhere on its diagonal''
		is equivalent to the product of those entries being zero.
	\end{hint}
\end{problem}

\begin{problem}
	[Sequence space]
	Consider the space $\ell^2$ of infinite sequences of real numbers
	$a = (a_1, a_2, \dots)$ satisfying $\sum_i a_i^2 < \infty$.
	We equip it with the dot product
	\[ \left< a, b\right> = \sum_i a_i b_i. \]
	Is this a Hilbert space?
	If so, identify a Hilbert basis.
\end{problem}
