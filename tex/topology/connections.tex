\chapter{Stronger notions of connectedness}
\todo{clean}
Originally, we defined

\section{Path-connected spaces}
\prototype{Walking around in $\CC$.}
A stronger and perhaps more intuitive notion
of a connected space is a \emph{path-connected} space.
The short description: ``walk around in the space''.
% We have general topological spaces, but so far they've just been sitting there.
% Let's start walking in them.

% Throughout this section, we let $I = [0,1]$ with the usual topology
% inherited from $\RR$.
%Also, recall that $S^n$ is the surface of the sphere living
%in $n+1$ dimensional space; hence $S^1$ is a unit circle, say.

\begin{definition}
	A \vocab{path} in the space $X$ is a continuous function
	\[ \gamma : [0,1] \to X. \]
	Its \vocab{endpoints} are the two points $\gamma(0)$ and $\gamma(1)$.
\end{definition}

You can think of $[0,1]$ as measuring ``time'', and so we'll often write $\gamma(t)$
for $t \in [0,1]$ (with $t$ standing for ``time'').
Here's a picture of a path.
\begin{center}
	\begin{asy}
		bigblob("$X$");
		pair A = Drawing("\gamma(0)", (-3,-1));
		pair B = Drawing("\gamma(1)", (2,1), dir(90));
		path p = A..(-2,0)..(0,2)..(1,0)..B;
		draw(p, EndArrow);
		MP("\gamma", midpoint(p), dir(90));
	\end{asy}
\end{center}
\begin{ques}
	Why does this agree with your intuitive notion of what a ``path'' is?
\end{ques}

\begin{definition}
	A space $X$ is \vocab{path-connected} if
	any two points in it are connected by some path.
\end{definition}

\todo{work out this detail}
\begin{exercise}
	Let $X = U \sqcup V$ be a disconnected space.
	Show that there is no path
	from a point of $U$ to point $V$.
	(If $\gamma : [0,1] \to X$, then we get $[0,1] = \gamma\pre(U) \sqcup \gamma\pre(V)$,
	but $[0,1]$ is connected.)
\end{exercise}

\begin{example}[Examples of path-connected spaces]
	\listhack
	\begin{itemize}
		\ii $\RR^2$ is path-connected,
		since we can ``connect'' any two points with a straight line.
		\ii The unit circle $S^1$ is path-connected, since
		we can just draw the major or minor arc to connect two points.
	\end{itemize}
\end{example}

% Easy, right?

\section{Homotopy and simply connected spaces}
\prototype{$\CC$ and $\CC \setminus \{0\}$.}
\label{sec:meteor}
Now let's motivate the idea of homotopy.
Consider the example of the complex plane $\CC$ (which you can
think of just as $\RR^2$) with two points $p$ and $q$.
There's a whole bunch of paths from $p$ to $q$ but somehow
they're not very different from one another.
If I told you ``walk from $p$ to $q$'' you wouldn't have too many questions.

\begin{center}
	\begin{asy}
		unitsize(0.8cm);
		bigbox("$\mathbb C$");
		pair A = Drawing("p", (-3,0), dir(180));
		pair B = Drawing("q", (3,0), dir(0));
		draw(A..(-2,0.5)..(0,2)..(1,1.2)..B, red, EndArrow);
		draw(A--B, mediumgreen, EndArrow);
		draw(A--(-1,-1)--(2,-1)--B, blue, EndArrow);
		// draw(A..(1.5,-2)..(-1.5,-2)..B, orange, EndArrow);
	\end{asy}
\end{center}

So we're living happily in $\CC$ until a meteor strikes the origin,
blowing it out of existence.
Then suddenly to get from $p$ to $q$, people might tell you two different things:
``go left around the meteor'' or ``go right around the meteor''.

\begin{center}
	\begin{asy}
		unitsize(0.8cm);
		bigbox("$\mathbb C \setminus \{0\}$");
		pair A = Drawing("p", (-3,0), dir(180));
		pair B = Drawing("q", (3,0), dir(0));
		draw(A..(-2,0.5)..(0,2)..(1,1.2)..B, red, EndArrow);
		draw(A--(-1,-1)--(2,-1)--B, blue, EndArrow);
		filldraw(scale(0.5)*unitcircle, grey, black);
	\end{asy}
\end{center}

So what's happening?
In the first picture, the red, green, and blue paths somehow all looked
the same: if you imagine them as pieces of elastic string pinned down
at $p$ and $q$, you can stretch each one to any other one.

But in the second picture, you can't move the red string to match with the blue string: there's a meteor in the way.
The paths are actually different.
\footnote{If you know about winding numbers, you might feel this is familiar.  We'll talk more about this in the chapter on homotopy groups.}

The formal notion we'll use to capture this is \emph{homotopy equivalence}.
We want to write a definition such that in the first picture,
the three paths are all \emph{homotopic}, but the two paths in the
second picture are somehow not homotopic.
And the idea is just continuous deformation.

\begin{definition}
	Let $\alpha$ and $\beta$ be paths in $X$ whose endpoints coincide.
	A (path) \vocab{homotopy} from $\alpha$ to $\beta$ is a continuous function
	$F : [0,1]^2 \to X$, which we'll write $F_s(t)$ for $s,t \in [0,1]$,
	such that
	\[ F_0(t) = \alpha(t) \text{ and } F_1(t) = \beta(t)
		\text{ for all $t \in [0,1]$} \]
	and moreover
	\[ \alpha(0) = \beta(0) = F_s(0)
		\text{ and }
		\alpha(1) = \beta(1) = F_s(1)
		\text{ for all $s \in [0,1]$}. \]
	If a path homotopy exists, we say $\alpha$ and $\beta$
	are path \vocab{homotopic} and write $\alpha \simeq \beta$.
\end{definition}
\begin{abuse}
	While I strictly should say ``path homotopy'' to describe this relation
	between two paths, I will shorten this to just ``homotopy'' instead.
	Similarly I will shorten ``path homotopic'' to ``homotopic''.
\end{abuse}
Picture: \url{https://commons.wikimedia.org/wiki/File:HomotopySmall.gif}.
Needless to say, $\simeq$ is an equivalence relation.

What this definition is doing is taking $\alpha$ and ``continuously deforming'' it to $\beta$, while keeping the endpoints fixed.
Note that for each particular $s$, $F_s$ is itself a function.
So $s$ represents time as we deform $\alpha$ to $\beta$:
it goes from $0$ to $1$, starting at $\alpha$ and ending at $\beta$.

\begin{center}
	\begin{asy}
		size(9cm);
		bigbox("$\mathbb C$");
		pair A = Drawing("p", (-3,0), dir(180));
		pair B = Drawing("q", (3,0), dir(0));
		draw(A..MP("F_{0} = \alpha", (0,2), dir(45))..B, heavygreen);
		draw(A..MP("F_{0.25}", (0,1), dir(45))..B, mediumgreen);
		draw(A..MP("F_{0.5}", (0,0), dir(90))..B, palecyan);
		draw(A..MP("F_{0.75}", (0,-1), dir(-45))..B, mediumcyan);
		draw(A..MP("F_{1} = \beta", (0,-2), dir(-45))..B, heavycyan);
		// draw(A..(1.5,-2)..(-1.5,-2)..B, orange, EndArrow);
	\end{asy}
\end{center}

\begin{ques}
	Convince yourself the above definition is right.
	What goes wrong when the meteor strikes?
\end{ques}

So now I can tell you what makes $\CC$ special:
\begin{definition}
	A space $X$ is \vocab{simply connected} if it's path-connected and
	for any points $p$ and $q$, all paths from $p$ to $q$ are homotopic.
\end{definition}
That's why you don't ask questions when walking from $p$ to $q$ in $\CC$:
there's really only one way to walk. Hence the term ``simply'' connected.

\begin{ques}
	Convince yourself that $\RR^n$ is simply connected for all $n$.
\end{ques}

\section{\problemhead}

\todo{rework this}
In fact, there exist connected spaces which are not path-connected
(one is called the \emph{topologist's sine curve}).
This shows that path-connected is stronger than connected.

\begin{problem}
	Let $f \colon X \to Y$ be a continuous function.
	Show that if $X$ is path-connected then so is $f\im(X)$.
\end{problem}

\begin{problem}
	A topological space $X$ is called \vocab{locally path-connected}
	if for every point $x \in X$, some neighborhood $U$ of $x$ is path-connected.
	Prove that $X$ is path-connected if and only if it is connected
	and locally path-connected.
	\label{prob:local_path_connected}
\end{problem}


