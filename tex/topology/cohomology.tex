\chapter{Singular Cohomology (Draft)}
Here's the main way to motivate this chapter.
Let $X = \CP^2$ and $Y = S^2 \wedge S^4$.
\begin{exercise}
	Show that $H_n(X) \cong H_n(Y)$ for every $n$.
\end{exercise}
However, as it turns out $X$ and $Y$ are not homotopy equivalent --- so how can we tell them apart?

In this chapter, we'll define a \emph{cohomology group} $H^n(X)$ and $H^n(Y)$.
We will still have $H^n(X) \cong H^n(Y)$ for every $X$ and $Y$:
in fact, the $H^n$'s are completely determined by the $H_n$'s
by the so-called \emph{Universal Coefficient Theorem}
However, it turns out that one can take all the cohomology groups and put
them together to form a \emph{cohomology ring} $H^\bullet$.
We will then see that $H^\bullet(X) \not\cong H^\bullet(Y)$ as rings.

\section{Cochain Complexes}
\begin{definition}
A \vocab{cochain complex} $A^\bullet$ is algebraically the same as a chain complex:
it is a sequence of abelian groups
\[ \dots \taking{\delta} A^{n-1} \taking\delta A^n \taking\delta A^{n+1} \taking\delta \dots. \]
such that $\delta^2 = 0$.
Notation-wise, we're now using subscripts, and use $\delta$ rather $\partial$.
We can still define the homology groups by
\[ H_n(A^\bullet) = \ker\left( A^n \taking\delta A^{n+1} \right)
	/ \img\left( A^{n-1} \taking\delta A^n \right). \]
\end{definition}

\textbf{The most common way to get a cochain complex
is to \emph{dualize} a chain complex.}
Specifically, pick an abelian group $G$;
note that $\Hom(-, G)$ is a contravariant functor,
and thus takes every chain complex
\[ \dots \taking\partial A_{n+1} \taking\partial
	A_n \taking\partial A_{n-1} \taking\partial \dots \]
into a cochain complex: letting $A^n = \Hom(A_n, G)$ we obtain
\[ \dots \taking\delta A^{n-1} \taking\delta
	A^n \taking\delta A^{n+1} \taking\delta \dots. \]
where $\delta(A_n \taking{f} G) = A_{n+1} \taking\partial A \taking{f} G$.
\begin{definition}
	Given a chain complex $A_\bullet$ of abelian groups and a group $G$,
	we define the \vocab{cohomology groups} $H^n(A_\bullet; G)$ to be the
	homology groups of the dual $A^\bullet$ obtained by applying $\Hom(-,G)$.
\end{definition}
This is admittedly horrible naming:
the \emph{cohomology groups of a chain complex}
are the \emph{homology groups of the cochain complex}
dual to that chain complex.

\section{Cohomology of Spaces}
\prototype{$C^0(X;G)$ all functions $X \to G$ while $H^0(X)$ are those functions $X \to G$
constant on path components.}

The case of interest is our usual geometric situation, with $C_\bullet(X)$.
\begin{definition}
	For a space $X$ and abelian group $G$,
	we define $C^\bullet(X;G)$ to be the dual complex to $C_\bullet(X)$,
	called the \vocab{singular cochain complex} of $X$;
	its elements are called \vocab{cochains}.

	Then we define the \vocab{cohomology groups}
	of the space $X$ as 
	\[ H^n(X; G) \defeq H^n(C_\bullet(X); G) = H_n(C^\bullet(X;G)). \]
\end{definition}

\begin{example}
	[$C^0(X; G)$, $C^1(X; G)$, and $H^0(X;G)$]
	Of course the archetypal example is to dualize the singular chain complex
	of a space $X$. In this situation, elements of $C^n$ are called \emph{cochains}.
	\begin{itemize}
		\ii $C_0(X)$ is the free abelian group on $X$,
		so $C^0(X) = \Hom(C_0(X), G)$ is a function that
		takes every point of $X$ to an element of $G$.
		\ii $C_1(X)$ is the free abelian group on $1$-simplices in $X$.
		So $C^1(X)$ needs to take every $1$-simplex to an element of $G$.
	\end{itemize}
	Let's now try to understand $\delta : C^1(X) \to C^0(X)$.

	Now given a cochain $\phi \in C^0(X)$, i.e.\ a homomorphism $\phi : C^0(X) \to G$,
	what is $\delta\phi : C^1(X) \to G$?
	Answer: 
	\[ \delta\phi : [v_0, v_1] \mapsto \phi([v_0]) - \phi([v_1]). \]
	Hence, elements of $\ker(C^0 \taking\partial C^1)$ are those cochains
	that are \emph{constant on path-connected components}.
\end{example}
In particular, much like $H_0(X)$, we have \[ H^0(X) \cong G^{\oplus r} \]
if $X$ has $r$ path-connected components (where $r$ is finite\footnote{%
	Something funny happens if $X$ has \emph{infinitely} many path-connected components:
	say $X = \bigsqcup_\alpha X_\alpha$ over an infinite indexing set.
	In this case we have
	$H_0(X) = \bigoplus_\alpha G$ while $H^0(X) = \prod_\alpha G$.
	For homology we get a \emph{direct sum} while
	for cohomology we get a \emph{direct product}.

	These are actually different for infinite indexing sets.
	For general modules $\bigoplus_\alpha M_\alpha$ is \emph{defined} to only allow
	to have \emph{finitely many} zero terms.
	(This was never mentioned earlier in the Napkin,
	since I only ever defined $M \oplus N$ and extended it to finite direct sums.)
	No such restriction holds for $\prod_\alpha G_\alpha$ a product of groups.
	This corresponds to the fact that $C_0(X)$ is formal linear sums of $0$-chains
	(which, like all formal sums, are finite)
	from the path-connected components of $G$.
	But a cochain of $C^0(X)$ is a \emph{function}
	from each path-connected component of $X$ to $G$,
	where there is no restriction.
}).

To the best of my knowledge, the higher cohomology groups $H^n(X; G)$
(or even the cochain groups $C^n(X; G) = \Hom(C_n(X), G)$) are harder to describe concretely.

\begin{abuse}
	In this chapter the only cochain complexes
	we will consider are dual complexes as above.
	So, any time we write a chain complex $A^\bullet$ it is implicitly given
	by applying $\Hom(-,G)$ to $A_\bullet$.
\end{abuse}

\section{Cohomology of Spaces is Functorial}
We now check that the cohomology groups still exhibit the same nice functorial behavior.
First, let's categorize the previous results we had:

\begin{ques}
	Define $\catname{CoCmplx}$
	the category of cochain complexes.
\end{ques}

\begin{exercise}
	Interpret $\Hom(-,G)$ as a contravariant functor
	from \[ \Hom(-,G) : \catname{Cmplx}\op \to \catname{CoCmplx}. \]
	This means in particular that given a chain map $f : A_\bullet \to B_\bullet$,
	we naturally obtain a dual map $f^\vee : B^\bullet \to A^\bullet$.
\end{exercise}

\begin{ques}
	Interpret $H^n : \catname{CoCmplx} \to \catname{Grp}$ as a functor.
	Compose these to get a contravariant functor
	$H^n(-;G) : \catname{Cmplx}\op \to \catname{Grp}$.
\end{ques}

Then in exact analog to our result that $H_n : \catname{hTop} \to \catname{Grp}$ we have:
\begin{theorem}[$H^n (-;G): \catname{hTop}\op \to \catname{Grp}$]
	For every $n$, $H^n(-;G)$ is a contravariant functor
	from $\catname{hTop}\op$ to $\catname{Grp}$.
\end{theorem}
\begin{proof}
	The idea is to leverage the work we already did in constructing
	the prism operator earlier.
	First, we construct the entire sequence of functors
	from $\catname{Top}\op \to \catname{Grp}$:
	\begin{diagram}
		\catname{Top}\op & \rTo^{C_\bullet} & \catname{Cmplx}\op & \rTo^{\Hom(-;G)}
		& \catname{CoCmplx} & \rTo^{H^n} & \catname{Grp} \\
		X && C_\bullet(X) && C^\bullet(X;G) && H^n(X;G) \\
		\dTo^f &\rMapsto& \dTo^{f_\sharp} &\rMapsto&
		\uTo^{f^\sharp} &\rMapsto& \uTo^{f^\ast} \\
		Y && C_\bullet(Y) && C^\bullet(Y;G) && H^n(Y;G).
	\end{diagram}
	Here $f^\sharp = (f_\sharp)^\vee$, and $f^\ast$
	is the resulting induced map on homology groups of the cochain complex.

	So as before all we have to show is that $f \simeq g$,
	then $f^\ast = g^\ast$.
	Recall now that there is a prism operator such that
	$f_\sharp - g_\sharp = P \partial + \partial P$.
	If we apply the entire functor $\Hom(-;G)$ we get that
	$f^\sharp - g^\sharp = \delta P^\vee + P^\vee \delta$
	where $P^\vee : C^{n+1}(Y;G) \to C^n(X;G)$.
	So $f^\sharp$ and $g^\sharp$ are chain homotopic thus $f^\ast = g^\ast$.
\end{proof}


\section{Universal Coefficient Theorem}
We now wish to show that the cohomology groups are determined up to isomorphism
by the homology groups: given $H_n(A_\bullet)$, we can extract $H^n(A_\bullet; G)$.
This is achieved by the \emph{Universal Coefficient Theorem}.
\begin{theorem}
	[Universal Coefficient Theorem]
	Let $A_\bullet$ be a chain complex of \emph{free} abelian groups,
	and let $G$ be another abelian group.
	Then there is a natural short exact sequence
	\[
		0 \to \Ext(H_{n-1}(A_\bullet), G) \to H^n(A_\bullet; G)
		\taking{h} \Hom(H_n(A_\bullet), G) \to 0. \]
	In addition, this exact sequence is \emph{split}
	so in particular
	\[ H^n(C_\bullet; G) \cong \Ext(H_{n-1}(A_\bullet, G)
		\oplus \Hom(H_n(A_\bullet), G). \]
\end{theorem}
Fortunately, in our case of interest, $A_\bullet$ is $C_\bullet(X)$
which is by definition free.

There are two things we need to explain, what the map $h$ is and the map $\Ext$ is.

It's not too hard to guess how \[ h : H^n(A_\bullet; G) \to \Hom(H_n(A_\bullet), G) \] is defined.
An element of $H^n(A_\bullet;G)$ is represented by a function which sends a cycle
in $A_n$ to an element of $G$.
The content of the theorem is to show that $h$ is surjective with kernel $\Ext(H_{n-1}(A_\bullet), G)$.

What about $\Ext$?
It turns out that $\Ext(-,G)$ is the so-called \vocab{Ext functor}, defined as follows.
Let $H$ be an abelian group, and consider a \vocab{free resolution} of $H$,
by which we mean an exact sequence
\[ \dots \taking{f_2} F_1 \taking{f_1} F_0 \taking{f_0} H \to 0. \]
Then we can apply $\Hom(-,G)$ to get a cochain complex
\[ \dots \xleftarrow{f_2^\vee} \Hom(F_1, G) \xleftarrow{f_1}
	\Hom(F_0, G) \xleftarrow{f_0} \Hom(H,G) \to 0. \]
but \emph{this cochain complex need not be exact}
(in categorical terms, $\Hom(-,G)$ does not preserve exactness).
We define \[ \Ext(H,G) \defeq \ker(f_2^\vee) / \img(f_1^\vee) \]
and it's a theorem that this doesn't depend on the choice of the free resolution.
There's a lot of homological algebra that goes into this,
which I won't take the time to discuss;
but the upshot of the little bit that I did include is that the $\Ext$
functor is very easy to compute in practice, since
you can pick any free resolution you want and compute the above.

%By ``natural'', we mean that if $f : A_\bullet \to B_\bullet$ is a chain map,
%then we obtain a commutative diagram
%\begin{diagram}
%	0 & \rTo & \Ext(H_{n-1}(A_\bullet), G) & \rTo
%		& H^n(A_\bullet;G) & \rTo & \Hom(H_n(A_\bullet), G) & \rTo & 0 \\
%	& & \uTo^{ \Ext(f_\ast, G) } & & \uTo^{f^\ast} & & \uTo^{\Hom(f_\ast, G)} & & \\
%	0 & \rTo & \Ext(H_{n-1}(B_\bullet), G) & \rTo
%		& H^n(A_\bullet;G) & \rTo & \Hom(H_n(B_\bullet), G) & \rTo & 0 \\
%\end{diagram}
%where $f_\ast$ is the induced arrow $H_n(A_\bullet) \to H_n(B_\bullet)$.

\begin{lemma}
	[Computing the $\Ext$ Functor]
	For any abelian groups $G$, $H$, $H'$ we have
	\begin{enumerate}[(a)]
		\ii $\Ext(H \oplus H, G) = \Ext(H, G) \oplus \Ext(H', G)$.
		\ii $\Ext(H,G) = 0$ for $H$ free, and
		\ii $\Ext(\Zc n, G) = G / nG$.
	\end{enumerate}
\end{lemma}
\begin{proof}
	For (a), note that if $\dots \to F_1 \to F_0 \to H \to 0$
	and $\dots \to F_1' \to F_0' \to F_0' \to H' \to 0$ are free resolutions,
	then so is $F_1 \oplus F_1' \to F_0 \oplus F_0' \to H \oplus H' \to 0$.

	For (b), note that $0 \to H \to H \to 0$ is a free resolution.
	
	Part (c) follows by taking the free resolution
	\[ 0 \to \ZZ \taking{\times n} \ZZ \to \Zc n \to 0 \]
	and applying $\Hom(-,G)$ to it.
	\begin{ques}
		Finish the proof of (c) from here. \qedhere
	\end{ques}
\end{proof}

\begin{ques}
	Some $\Ext$ practice: compute
	$\Ext(\ZZ^{\oplus 2015}, G)$ and $\Ext(\Zc{30}, \Zc 4)$.
\end{ques}

\section{Example Computation of Cohomology Groups}
\prototype{Possibly $H^n(S^m)$.}
Before we start doing some computations, we take the time
to define reduced cohomology groups as well.
For $X$ nonempty we define the \vocab{reduced cohomology groups} by $\wt H^n(X;G)$
by $H^n(\wt C_\bullet(X); G)$.
To be concrete, the augmented singular chain complex is
\[ \dots \taking\partial C_1(X) \taking\partial C_0(X) \taking\eps \ZZ \to 0 \]
dualizes to
\[
	\dots \xleftarrow\delta C^1(X;G) \xleftarrow\delta C^0(X;G)
	\xleftarrow{\eps^\vee} \underbrace{\Hom(\ZZ, G)}_{\cong G}
	\leftarrow 0.
\]
Since the $\ZZ$ we add is also free,
the Universal Coefficient Theorem still applies.

\begin{ques}
	For each $n \ge 1$, $\wt H^n(X;G) \cong H^n(X;G)$.
\end{ques}

As for $\wt H^0(X;G)$ we use the Universal Coefficient Theorem;
since the homology at $\ZZ$ is trivial, the $\Ext$ term vanishes and
$\wt H^0(X;G) \cong \Hom(\wt H_0(X), G)$.
Geometrically, $\wt H^0(X;G)$ consists of functions $X \to G$
which are constant on path-connected components,
modulo the functions which are constant on all of $X$.

\begin{example}
	[Cohomolgy Groups of $S^m$]
	It is straightforward to compute $H^n(S^m)$ now:
	all the $\Ext$ terms vanish since $H_n(S^m)$ is always free,
	and hence we obtain that 
	\[ H^n(S^m) \cong \Hom(H_n(S^m), G) \cong
		\begin{cases}
			G & n=m, n=0 \\
			0 & \text{otherwise}.
		\end{cases}
	\]
	By UCT for reduced groups, we also have
	\[ \wt H^n(S^m) \cong \Hom(\wt H_n(S^m), G) \cong
		\begin{cases}
			G & n=m \\
			0 & \text{otherwise}.
		\end{cases}
	\]
	since $\Hom(\ZZ, G)$.
\end{example}

\begin{example}
	[Cohomolgy Groups of Torus]
	This example has no nonzero $\Ext$ terms either,
	since this time $H^n(S^1 \times S^1)$ is always free.
	So we obtain
	\[ H^n(S^1 \times S^1) \cong \Hom(H_n(S^1 \times S^1), G)
		\cong
		\begin{cases}
			G & n = 0,2 \\
			G^{\oplus 2} & n = 1.
		\end{cases}
	\]
\end{example}

\begin{example}
	[Cohomolgy Groups of Klein Bottle]
	This example will actually have $\Ext$ term.
	Recall that if $K$ is a Klein Bottle then its homology groups are
	$\ZZ$ in dimension $n=0$ and $\ZZ \oplus \Zc 2$ in $n=1$, and $0$ elsewhere.

	For $n=0$, we again just have $H^0(K;G) \cong \Hom(\ZZ, G) \cong G$.
	For $n=1$, the $\Ext$ term is $\Ext(H_0(K), G) \cong \Ext(\ZZ, G) = 0$
	so \[ H^1(K;G) \cong \Hom(\ZZ \oplus \Zc2, G) \cong G \oplus \Hom(\Zc2, G). \]
	We have that $\Hom(\Zc2,G)$ is the subgroup
	of elements of order $2$ in $G$ (and $0 \in G$).

	But for $n=2$, we have our first interesting $\Ext$ group:
	the exact sequence is
	\[ 0 \to \Ext(\ZZ \oplus \Zc 2, G) \to H^2(X;G) \to \underbrace{H_2(X)}_{=0} \to 0. \]
	Thus, we have
	\[ H^2(X;G) \cong \left( \Ext(\ZZ,G) \oplus \Ext(\Zc2,G) \right) \oplus 0
		\cong G/2G. \]
	All the higher groups vanish.
	In summary:
	\[
		H^n(X;G) \cong
		\begin{cases}
			G & n = 0 \\
			G \oplus \Hom(\Zc2, G) & n = 1 \\
			G/2G & n = 2 \\
			0 & n \ge 3. 
		\end{cases}
	\]
\end{example}

From the previous example we can extract the following observation.
\begin{lemma}
	[$0$th and $1$st Homology Groups are just Duals]
	For $n = 0$ and $n = 1$, we have
	\begin{align*}
		H^n(X;G) &\cong \Hom(H_n(X), G) \\
		\wt H^n(X;G) &\cong \Hom(\wt H_n(X), G).
	\end{align*}
\end{lemma}
\begin{proof}
	It's already been shown for $n=0$.
	For $n=1$, notice that $H_0(X)$ and $\wt H_0(X)$ are free,
	so the $\Ext$ term vanishes.
\end{proof}

\section\problemhead
\begin{sproblem}
	[Wedge Product Cohomology]
	For any $G$ and $n$ we have
	\[
		\wt H^n(X \vee Y; G)
		\cong
		\wt H^n(X; G) \oplus \wt H^n(Y; G).
	\]	
\end{sproblem}
\begin{sproblem}[$\Zc2$-Cohomology of $\RP^n$]
	Show that
	\[
		H^m(\RP^n, \Zc2)
		\cong
		\begin{cases}
			\ZZ & \text{$m=0$, or $m$ is odd and $m=n$} \\
			\Zc2 & \text{$0 < m < n$ and $m$ is odd} \\
			0 & \text{otherwise}.
		\end{cases}
	\]
\end{sproblem}

