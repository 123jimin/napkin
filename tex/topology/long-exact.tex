\chapter{The Long Exact Sequence (Draft)}
In this chapter we introduce the key fact about chain complexes that will allow us to compute
the homology groups of any space: the so-called ``long exact sequence''.

For those that haven't read about abelian categories:
a sequence of morphisms of abelian groups
\[ \dots \to G_{n+1} \to G_n \to G_{n-1} \to \dots \]
is \vocab{exact} if the image of any arrow is equal to the kernel of the next arrow.
In particular,
\begin{itemize}
	\ii The map $0 \to A \to B$ is exact if and only if $A \to B$ is injective.
	\ii the map $A \to B \to 0$ is exact if and only if $A \to B$ is surjective.
\end{itemize}
A short exact sequence is one of the form $0 \to A \injto B \surjto C \to 0$.

\section{Short Exact Sequences and Four Examples}
\prototype{Relative sequence and Mayer-Vietoris sequence.}
Let $\AA = \catname{AbGrp}$.
Recall that we defined a morphism of chain complexes in $\AA$ already.
\begin{definition}
Suppose we have a map of chain complexes
\[ 0 \to A_\bullet \taking f B_\bullet \taking g C_\bullet \to 0 \]
It is said to be \vocab{short exact} if \emph{each row} of the diagram below is short exact.
\begin{diagram}
	&& \vdots && \vdots && \vdots && \\
	&& \dTo^{\partial_A} && \dTo^{\partial_B} && \dTo^{\partial_C} && \\
	0 & \rTo & A_{n+1} & \rInj^{f_{n+1}} & B_{n+1} & \rSurj^{g_{n+1}} & C_{n+1} & \rTo & 0 \\
	&& \dTo^{\partial_A} && \dTo^{\partial_B} && \dTo^{\partial_C} && \\
	0 & \rTo & A_n & \rInj^{f_n} & B_n & \rSurj^{g_n} & C_n & \rTo & 0 \\
	&& \dTo^{\partial_A} && \dTo^{\partial_B} && \dTo^{\partial_C} && \\
	0 & \rTo & A_{n-1} & \rInj^{f_{n-1}} & B_{n-1} & \rSurj^{g_{n-1}} & C_{n-1} & \rTo & 0 \\
	&& \dTo^{\partial_A} && \dTo^{\partial_B} && \dTo^{\partial_C} && \\
	&& \vdots && \vdots && \vdots &&
\end{diagram}
\end{definition}

Here are examples of short exact sequences of chain complexes.
\begin{example}
	[Relative Chain Short Exact Sequence]
	\label{ex:rel_short_exact}
	Since $C_n(X,A) \defeq C_n(X) / C_n(A)$, we have a short exact sequence
	\[ 0 \to C_\bullet(A) \injto C_\bullet(X) \surjto C_\bullet(X,A) \to 0 \]
	for every space $X$ and subspace $A$.
	The maps for each $n$ are the obvious ones:
	$C_n(A) \injto C_n(X)$ inclusion and $C_n(X) \surjto C_n(X,A)$ projection.
\end{example}
\begin{example}
	[Augmented Relative Chain Short Exact Sequence]
	\label{ex:aug_rel_short_exact}
	We can augment adding the following row to the bottom of the diagram:
	\begin{diagram}
		0 & \rTo & C_0(A) & \rInj & C_0(X) & \rSurj & C_0(X,A) & \rTo & 0\\
		&& \dSurj^{\eps} && \dSurj^{\eps} && \dTo && \\
		0 & \rTo & \ZZ & \rTo_\id & \ZZ & \rTo & 0 & \rTo & 0\\
	\end{diagram}
	In other words, this is the short exact sequence
	\[ 0 \to C_\bullet(A) \text{ augmented} \injto C_\bullet(X) \text{ augmented}
	\surjto C_\bullet(X,A) \to 0 \]
	where the augmentation is as described in \Cref{def:augment}.
\end{example}

\begin{example}
	[Mayer-Vietoris Short Exact Sequence and its Augmentation]
	\label{ex:mayer_short_exact}
	Let $X = U \cup V$ be an open cover.
	For each $n$ consider
	\begin{diagram}
		C_n(U \cap V) & \rInj & C_n(U) \oplus C_n(V) & \rSurj & C_n(U + V) \\
		c & \rMapsto & (c, -c) && \\
		&& (c, d) & \rMapsto & c + d
	\end{diagram}
	This generates a short exact sequence
	\[ 0 \to  C_\bullet(U \cap V) \injto C_\bullet(U) \oplus C_\bullet(V)
	\surjto C_\bullet(U + V) \to 0. \]
	Like before, we can \emph{augment} each of the chain complexes in the Mayer-Vietoris
	sequence as well, by appending the following to the bottom of the diagram.
	\begin{diagram}
		0 & \rTo & C_0(U \cap V) & \rInj & C_0(U) \oplus C_0(V) & \rSurj & C_0(U+V) & \rTo & 0\\
		&& \dSurj^{\eps} && \dSurj^{\eps \oplus \eps} && \dTo^\eps && \\
		0 & \rTo & \ZZ & \rTo & \ZZ \oplus \ZZ & \rTo & \ZZ & \rTo & 0
	\end{diagram}
\end{example}

\section{The Long Exact Sequence of Homology Groups}
Consider a short exact sequence $0 \to A_\bullet \taking f B_\bullet \taking g C_\bullet \to 0$.
Now, we know that we get induced maps of homology groups, i.e.\ we have
\begin{diagram}
	\vdots && \vdots && \vdots  \\
	H_{n+1}(A_\bullet) & \rTo^{f_\ast} & H_{n+1}(B_\bullet) & \rTo^{g_\ast} & H_{n+1}(C_\bullet) \\
	H_{n}(A_\bullet) & \rTo^{f_\ast} & H_{n}(B_\bullet) & \rTo^{g_\ast} & H_{n}(C_\bullet) \\
	H_{n}(A_\bullet) & \rTo^{f_\ast} & H_{n}(B_\bullet) & \rTo^{g_\ast} & H_{n}(C_\bullet) \\
	\vdots && \vdots && \vdots \\
\end{diagram}
But the theorem is that we can string these all together,
taking each $H_{n+1}(C_\bullet)$ to $H_n(A_\bullet)$.

\begin{theorem}[Short Exact $\implies$ Long Exact]
	\label{thm:long_exact}
	Let $0 \to A_\bullet \taking f B_\bullet \taking g C_\bullet \to 0$ 
	be \emph{any} short exact sequence of chain complexes we like.
	Then there is an \emph{exact} sequence
	\begin{diagram}
		&& \dots & \rTo & H_{n+2}(C_\bullet) \\
		H_{n+1}(A_\bullet) & \rTo_{f_\ast} &
		H_{n+1}(B_\bullet) & \ldTo(4,1)~\partial \rTo^{g_\ast} & H_{n+1}(C_\bullet) \\
		H_{n}(A_\bullet) & \rTo_{f_\ast} &
		H_{n}(B_\bullet) & \rTo^{g_\ast} \ldTo(4,1)~{\partial} & H_{n}(C_\bullet) \\
		H_{n-1}(A_\bullet) & \rTo_{f_\ast} &
		H_{n-1}(B_\bullet) & \rTo^{g_\ast} \ldTo(4,1)~{\partial} & H_{n-1}(C_\bullet) \\
		H_{n-2}(A_\bullet) & \rTo & \dots & \ldTo(4,1)~\partial & 
	\end{diagram}
	This is called a \vocab{long exact sequence} of homology groups.
\end{theorem}
\begin{proof}
	A very long diagram chase, valid over any abelian category.
	(Alternatively, it's actually possible to use the Snake Lemma twice.)
\end{proof}

\begin{remark}
	The map $\partial : H_n(C_\bullet) \to H_{n-1}(A_\bullet)$ can be written explicitly as follows.
	Recall that $H_n$ is ``cycles modulo boundaries'', and consider the sub-diagram
	\begin{diagram}
	&& B_n & \rSurj^{g_n} & C_n \\
	 && \dTo^{\partial_B} && \dTo_{\partial_C} \\
	A_{n-1} & \rInj^{f_{n-1}} & B_{n-1} & \rSurj_{g_{n-1}} & C_{n-1} \\
	\end{diagram}
	We need to take every cycle in $C_n$ to a cycle in $A_{n-1}$.
	(Then we need to check a ton of ``well-defined'' issues,
	but let's put that aside for now.)

	Suppose $c \in C_n$ is a cycle (so $\partial_C(c) = 0$).
	By surjectivity, there is a $b \in B_n$ with $g_n(b) = c$,
	which maps down to $\partial_B(b)$.
	Now, the image of $\partial_B(b)$ under $g_{n-1}$ is zero by commutativity of the square,
	and so we can pull back under $f_{n-1}$ to get a unique element of $A_{n-1}$
	(by exactness at $B_{n-1}$).

	In summary: we go ``\emph{left, down, left}'' to go from $c$ to $a$:
	\begin{diagram}
		&& b & \rMapsto^{g_n} & \boxed c \\
		&& \dMapsto^{\partial_B} && \dMapsto_{\partial_C} \\
		\boxed a & \rMapsto^{f_{n-1}} & \partial_B(b) & \rMapsto_{g_{n-1}} & 0
	\end{diagram}
\end{remark}
\begin{exercise}
	Check quickly that the recovered $a$ is actually a cycle,
	meaning $\partial_A(a) = 0$.
	(You'll need another row, and the fact that $\partial_B^2 = 0$.)
\end{exercise}

The final word is that:
\begin{moral}
	Short exact sequences of chain complexes give long exact sequences of homology groups.
\end{moral}
In particular, if we take our examples \Cref{ex:rel_short_exact,ex:aug_rel_short_exact,%
ex:mayer_short_exact},
we can just plug them into the theorem. The first two give:
\begin{theorem}[Long Exact Sequence for Relative Homology]
	Let $X$ be a space, and $A \subset X$ a subspace.
	There are long exact sequences
	\[ \dots \to H_n(A) \to H_n(X) \to H_n(X,A) \to H_{n-1}(A) \to \dots. \]
	and
	\[ \dots \to \wt H_n(A) \to \wt H_n(X) \to H_n(X,A) \to \wt H_{n-1}(A) \to \dots. \]
\end{theorem}
The Mayer-Vietoris ones give, for $X = U \cup V$ an open cover,
\[ \dots \to H_n(U \cap V) \to H_n(U) \oplus H_n(V) \to H_n(U+V) \to H_{n-1}(U \cap V) \to \dots. \]
and its reduced version
\[ \dots \to \wt H_n(U \cap V) \to \wt H_n(U) \oplus \wt H_n(V)
\to \wt H_n(U+V) \to \wt H_{n-1}(U \cap V) \to \dots. \]
The exactness of these sequences will give \textbf{tons of information}
about $H_n(X)$ if only we knew something about what $H_n(U+V)$
or $H_n(X,A)$ looked like.  This will be the purpose of the next section.

On the other hand, we can already make headway on $H_n(X,A)$ if we're curious what it looks like.
For example:
\begin{lemma}
	[Relative Homology to Contractible Spaces]
	\label{lem:rel_contractible}
	Let $X$ be a topological space, and let $A$ be a contractible subspace of $X$
	(for example, a single point).
	Then \[ H_n(X, A) \cong \wt H_n(X) \]
	for every $n$.
\end{lemma}
\begin{proof}
	Since $A$ is contractible, we have $\wt H_n(A) = 0$ for every $n$.
	For each $n$ there's a segment of the long exact sequence given by
	\[ \dots \to \underbrace{\wt H_n(A)}_{=0} \to \wt H_n(X) \to H_n(X,A)
	\to \underbrace{\wt H_{n-1}(A)}_{=0} \to \dots. \]
	So since $0 \to \wt H_n(X) \to H_n(X,A) \to 0$ is exact,
	this means $H_n(X,A) \cong \wt H_n(X)$.
\end{proof}


\section{The Mayer-Vietoris Sequence}
\prototype{The computation of $H_n(S^m)$ by splitting $S^m$ into two hemispheres.}

There are two major geometric results.
One is the Excision Theorem, which we defer until later.
The other we present here, which will let us take advantage of the
Mayer-Vietoris sequence.
The proofs are somewhat involved and are thus omitted;
see \cite{ref:hatcher} for details.

The first theorem is that the notation $H_n(U+V)$ that we have kept until now
is redundant, and can be replaced with just $H_n(X)$:
\begin{theorem}[Open Cover Homology Theorem]
	The inclusion $\iota : C_\bullet(U+V) \injto C_\bullet(X)$ 
	induces an isomorphism
	\[ H_n(U+V) \cong H_n(X). \]
\end{theorem}
\begin{remark}
	In fact, this is true for any open cover (even uncountable),
	not just those with two covers $U \cup V$.
	But we only state the special case with two open sets,
	because this is what is needed for \Cref{ex:mayer_short_exact}.
\end{remark}
So, \Cref{ex:mayer_short_exact} together with the above theorem implies,
after replacing all the $H_n(U+V)$'s with $H_n(X)$'s:
\begin{theorem}[Long Exact Sequence for Mayer-Vietoris]
	If $X = U \cup V$ is an open cover, then we have long exact sequences
	\[ \dots \to H_n(U \cap V) \to H_n(U) \oplus H_n(V)
		\to H_n(X) \to H_{n-1}(U \cap V) \to \dots. \]
	and
	\[ \dots \to \wt H_n(U \cap V) \to \wt H_n(U) \oplus \wt H_n(V) \to
		\wt H_n(X) \to \wt H_{n-1}(U \cap V) \to \dots. \]
\end{theorem}

At long last, we can compute the homology groups of the spheres.
The theorem is that
\begin{theorem}[The Homology Groups of $S^m$]
	\label{thm:reduced_homology_sphere}
	For integers $m$ and $n$,
	\[ \wt H_n(S^m) \cong
	\begin{cases}
		\ZZ & n=m \\
		0 & \text{otherwise}.
	\end{cases}
	\]
\end{theorem}
\begin{proof}
	This one's fun, so I'll only spoil the case $m=1$, and leave the rest to you.
	Decompose the circle $S^1$ into two arcs $U$ and $V$, as shown:
	\begin{center}
		\begin{asy}
			size(4cm);
			draw(unitcircle);
			label("$S^1$", dir(45), dir(45));
			real R = 0.1;
			draw(arc(origin,1-R,-100,100), blue+1);
			label("$U$", (1-R)*dir(0), dir(180), blue);
			draw(arc(origin,1+R,80,280), red+1);
			label("$V$", (1+R)*dir(180), dir(180), red);
		\end{asy}
	\end{center}
	Each of $U$ and $V$ is contractible, so all their reduced homology groups vanish.
	Moreover, $U \cap V$ is homotopy equivalent to two points,
	hence 
	\[ \wt H_n(U \cap V) \cong
		\begin{cases}
			\ZZ & n = 0 \\
			0 & \text{otherwise}.
		\end{cases}
	\]
	Now consider again the segment of the short exact sequence
	\[ 
		\dots \to
		\underbrace{\wt H_n(U) \oplus \wt H_n(V)}_{= 0} \to
		\wt H_n(S^1) \to \wt H_{n-1}(U \cap V) \to
		\underbrace{\wt H_{n-1}(U) \oplus \wt H_{n-1}(V)}_{=0} \to \dots.
	\]
	From this we derive that $\wt H_n(S^1)$ is $\ZZ$ for $n=1$ and $0$ elsewhere.
\end{proof}

Here are some more examples.
\begin{proposition}[The Homology Groups of the Figure Eight]
	Let $X = S^1 \wedge S^1$ be the figure eight.
	Then
	\[
		\wt H_n(X) \cong
		\begin{cases}
			\ZZ^{\oplus 2} & n = 1 \\
			0 & \text{otherwise}.
		\end{cases}
	\]
\end{proposition}
\begin{proof}
	Again, for simplicity we work with reduced homology groups.
	Let $U$ be the ``left'' half of the figure eight plus a little bit of the right,
	as shown below.
	\begin{center}
		\begin{asy}
			size(4cm);
			draw(unitcircle);
			draw(CR(2*dir(180),1), blue+2);
			draw(arc(origin,1,135,225), blue+2);
			label("$U$", 2*dir(180)+dir(135), dir(135), blue);
			label("$S^1 \wedge S^1$", dir(15), dir(15));
		\end{asy}
	\end{center}
	The set $V$ is defined symmetrically.
	In this case $U \cap V$ is contractible, while each of $U$ and $V$
	is homotopic to $S^1$.

	Thus, we can read a segment of the long exact sequence as
	\[
		\dots \to
		\underbrace{\wt H_n(U \cap V)}_{=0}
		\to \wt H_n(U) \oplus \wt H_n(V) \to \wt H_n(X) \to 
		\underbrace{\wt H_{n-1}(U \cap V)}_{=0} \to \dots.
	\]
	So we get that $\wt H_n(X) \cong \wt H_n(S^1) \oplus \wt H_n(S^1)$,
\end{proof}

Up until now, we have been very fortunate that we have always been able to make
certain parts of the space contractible.
This is not always the case, and in the next example we will have to
actually understand the maps in question to complete the solution.

\begin{proposition}
	[Homology Groups of the Torus]
	Let $X = S^1 \times S^1$ be the torus.
	Then
	\[
		\wt H_n(X)
		=
		\begin{cases}
			\ZZ^{\oplus 2} & n = 1 \\
			\ZZ & n = 2 \\
			0 & \text{otherwise}.
		\end{cases}
	\]
\end{proposition}
\begin{proof}
	To make our diagram look good on 2D paper,
	we'll represent the torus as a square with its edges identified,
	though three-dimensionally the picture makes sense as well.
	Consider $U$ (shaded light orange) and $V$ (shaded green) as shown.
	(Note that $V$ is connected due to the identification of the left and right (blue) edges,
	even if it doesn't look like it in the picture).
	\begin{center}
		\begin{asy}
			pair A = (0,0);
			pair B = (1,0);
			pair C = (1,1);
			pair D = (0,1);
			draw(A--B, red+1.5, MidArrow);
			draw(B--C, blue+1.5, MidArrow);
			draw(D--C, red+1.5, MidArrow);
			draw(A--D, blue+1.5, MidArrow);
			fill(box((0.2,0),(0.8,1)), orange+opacity(0.2));
			fill(box(A,(0.3,1)), heavygreen+opacity(0.2));
			fill(box((0.7,0),C), heavygreen+opacity(0.2));
			draw( (0.3,0)--(0.3,1), heavygreen+dashed+1.2);
			draw( (0.7,0)--(0.7,1), heavygreen+dashed+1.2);
			draw( (0.2,0)--(0.2,1), orange+dashed+1.2);
			draw( (0.8,0)--(0.8,1), orange+dashed+1.2);

			label("$U$", (0.5, 0.5));
			label("$V$", (0.1, 0.8));
			label("$V$", (0.9, 0.8));
		\end{asy}
	\end{center}
	In the three dimensional picture, $U$ and $V$ are two cylinders which together give the torus.
	This time, $U$ and $V$ are each homotopic to $S^1$, and the intersection $U \cap V$
	is the disjoint union of two circles: thus $\wt H_1(U \cap V) \cong \ZZ \oplus \ZZ$,
	and $H_0(U \cap V) \cong \ZZ^{\oplus 2} \implies \wt H_0(U \cap V) \cong \ZZ$.

	For $n \ge 3$, we have
	\[ 
		\dots \to
		\underbrace{\wt H_n(U \cap V)}_{=0}
		\to \wt H_n(U) \oplus \wt H_n(V) \to \wt H_n(X) \to 
		\underbrace{\wt H_{n-1}(U \cap V)}_{=0} \to \dots.
	\]
	and so $H_n(X) \cong 0$ for $n \ge 0$.
	Also, we have $H_0(X) \cong \ZZ$ since $X$ is path-connected.
	So it remains to compute $H_2(X)$ and $H_1(X)$.

	Let's find $H_2(X)$ first.
	We first consider the segment
	\[ 
		\dots \to
		\underbrace{\wt H_2(U) \oplus \wt H_2(V)}_{=0} \to \wt H_2(X) \xhookrightarrow{\delta}
		\underbrace{\wt H_1(U \cap V)}_{\cong \ZZ \oplus \ZZ} \xrightarrow{\phi}
		\underbrace{\wt H_1(U) \oplus \wt H_1(V)}_{\cong \ZZ \oplus \ZZ} \to \dots
	\]
	Unfortunately, this time it's not immediately clear what $\wt H_2(X)$ because
	we only have one zero at the left.
	In order to do this, we have to actually figure out what the maps $\delta$ and $\phi$ look like.
	Note that, as we'll see, $\phi$ isn't an isomorphism even though the groups are isomorphic.

	The presence of the zero term has allowed us to make the connecting map $\delta$ injective.
	First, $\wt H_2(X)$ is isomorphic to the image of of $\delta$, which is
	exactly the kernel of the arrow $\phi$ inserted.
	To figure out what $\ker \phi$ is, we have to think back to how the map
	$C_\bullet(U \cap V) \to C_\bullet(U) \oplus C_\bullet(V)$ was constructed:
	it was $c \mapsto (c, -c)$.
	So the induced maps of homology groups is actually what you would guess:
	a $1$-cycle $z$ in $\wt H_1(U \cap V)$ gets sent $(z, -z)$ in $\wt H_1(U) \oplus \wt H_1(V)$.

	In particular, consider the two generators $z_1$ and $z_2$ of
	$\wt H_1(U \cap V) = \ZZ \oplus \ZZ$,
	i.e.\ one cycle in each connected component of $U \cap V$.
	(To clarify: $U \cap V$ consists of two ``wristbands'';
	$z_i$ wraps around the $i$th one once.)
	Moreover, let $\alpha_U$ denote a generator of $\wt H_1(U) \cong \ZZ$,
	and $\alpha_V$ a generator of $H_2(U) \cong \ZZ$.
	Then we have that
	\[ z_1 \mapsto (\alpha_U, -\alpha_V) \qquad\text{and}\qquad z_2 \mapsto (\alpha_U, -\alpha_V). \]
	(The signs may differ on which direction you pick for the generators;
	note that $\ZZ$ has two possible generators.)
	We can even format this as a matrix:
	\[ \phi = \begin{pmatrix} 1 & 1 \\ -1 & -1 \end{pmatrix}. \]
	And we observe $\phi(z_1 - z_2) = 0$, meaning this map has nontrivial kernel!
	That is, \[ \ker\phi = \left< z_1 - z_2 \right> \cong \ZZ. \]
	Thus, $\wt H_2(X) \cong \img \delta \cong \ker \phi \cong \ZZ$.
	We'll also note that $\img \phi$ is the set generated by $(\alpha_U, -\alpha_V)$;
	(in particular $\img\phi \cong \ZZ$ and the quotient by $\img\phi$ is $\ZZ$ too).

	The situation is similar with $\wt H_1(X)$: this time, we have
	\[ 
		\dots 
		%\to \underbrace{\wt H_1(U \cap V)}_{\cong \ZZ \oplus \ZZ}
		\xrightarrow{\phi} \underbrace{\wt H_1(U) \oplus \wt H_1(V)}_{\cong \ZZ \oplus \ZZ}
		\overset{\psi}{\to} \wt H_1(X) \overset\partial\surjto
		\underbrace{\wt H_0(U \cap V)}_{\cong \ZZ} 
		\to \underbrace{\wt H_0(U) \oplus \wt H_0(V)}_{=0} \to \dots
	\]
	and so we know that the connecting map $\partial$ is surjective,
	hence $\img \partial \cong \ZZ$.
	Now, we also have
	\begin{align*}
		\ker \partial \cong \img \psi &\cong \left( \wt H_1(U) \oplus \wt H_1(V) \right) / \ker \psi \\
		&\cong \left( \wt H_1(U) \oplus \wt H_1(V) \right) / \img \phi \\
		&\cong \ZZ
	\end{align*}
	by what we knew about $\img \phi$ already.
	To finish off we need some algebraic tricks. The first is \Cref{prob:break_exact},
	which gives us a short exact sequence
	\[
		0 \to \underbrace{\ker\partial}_{\cong \img\psi \cong \ZZ} 
		\injto \wt H_1(X)
		\surjto \underbrace{\img\partial}_{\cong \ZZ} \to 0.
	\]
	You should satisfy yourself that $\wt H_1(X) \cong \ZZ \oplus \ZZ$ is the
	only possibility, but we'll prove this rigorously with \Cref{lem:split_exact}.
\end{proof}

Note that the previous example is of a different attitude than the previous ones,
because we had to figure out what the maps in the long exact sequence actually were.
In principle you could do this for the previous two examples as well,
by making explicit what all the arrows are.
You can do this even for the connecting arrow, because the ``left, down, left'' procedure
used in the proof of \Cref{thm:long_exact} gives us a concrete description of the arrow.

Finally, to fully justify the last step, we present the following lemma.
\begin{lemma}[Splitting Lemma]
	\label{lem:split_exact}
	For a short exact sequence $0 \to A \taking f B \taking g C \to 0$
	of abelian groups, the following are equivalent:
	\begin{enumerate}[(a)]
		\ii There exists $p : B \to A$ such that $A \taking f B \taking p A$ is the identity.
		\ii There exists $s : C \to B$ such that $C \taking s B \taking g C$ is the identity.
		(Note that this is true whenever $B$ is free!)
		\ii There is an isomorphism from $B$ to $A \oplus C$ such that the diagram
		\begin{diagram}
			&&& & B & &&&  \\
			0 & \rTo & A & \ruInj(2,1)^f & \dIsom^\cong & \rdSurj(2,1)^g & C & \rTo & 0 \\
			&&& \rdInj(2,1) & A \oplus C & \ruSurj(2,1) &&& 
		\end{diagram}
		commutes. (The maps attached to $A \oplus C$ are the obvious ones.)
	\end{enumerate}
	In these cases we say the short exact sequence \vocab{splits}.
\end{lemma}
In particular, for $C = \ZZ$, or any free abelian group,
condition (b) is necessarily true.
So, once we obtained the short exact sequence $0 \to \ZZ \to \wt H_1(X) \to \ZZ \to 0$,
we were home free.

\begin{remark}
	An example of a short exact sequence which doesn't split is
	\[ 0 \to \Zc 2 \xhookrightarrow{\times 2} \Zc 4 \surjto \Zc 2 \to 0 \]
	since it is not true that $\Zc 4 \cong \Zc2 \oplus \Zc 2$.
\end{remark}

\section\problemhead

\begin{problem}
	Complete the proof of \Cref{thm:reduced_homology_sphere},
	i.e.\ compute $H_n(S^m)$ for all $m$ and $n$.
	(Try doing $m=2$ first, and you'll see how to proceed.)
	\begin{hint}
		Induction on $m$, using hemispheres.
	\end{hint}
\end{problem}

\begin{problem}
	Compute the reduced homology groups of $\RR^n$ with $p \ge 1$ points removed.
	\begin{hint}
		One strategy is induction on $p$, with base case $p=1$.
		Another strategy is to let $U$ be the desired space and let $V$
		be the union of $p$ non intersecting balls.
	\end{hint}
	\begin{sol}
		The answer is $\wt H_{n-1}(X) \cong \ZZ^{\oplus p}$,
		with all other groups vanishing.
		For $p=1$, $\RR^n - \{\ast\} \cong S^{n-1}$ so we're done.
		For all other $p$, draw a hyperplane dividing the $p$ points into two halves
		with $a$ points on one side and $b$ points on the other (so $a+b=p$).
		Set $U$ and $V$ and use induction.

		Alternatively, let $U$ be the desired space and let $V$
		be the union of $p$ disjoint balls, one around every point.
		Then $U \cup V = \RR^n$ has all reduced homology groups trivial.
		From the Mayer-Vietoris sequence we can read $\wt H_k(U \cap V) \cong \wt H_k(U) \cap \wt H_k(V)$.
		Then $U \cap V$ is $p$ punctured balls, which are each the same as $S^{n-1}$.
		One can read the conclusion from here.
	\end{sol}
\end{problem}

\begin{dproblem}[Klein Bottle]
	\gim
	Show that the reduced homology groups of the Klein bottle are given by
	\[
		\wt H_n(X) = 
		\begin{cases}
			\ZZ \oplus \Zc 2 & n = 1 \\
			0 & \text{otherwise}.
		\end{cases}
	\]
	\begin{hint}
		It's possible to use two cylinders with $U$ and $V$.
		This time the matrix is $\begin{pmatrix} 1 & 1 \\ 1 & -1 \end{pmatrix}$
		or some variant though; in particular, it's injective, so $\wt H_2(X) = 0$.
	\end{hint}
\end{dproblem}


\todo{add to artwork gallery}


