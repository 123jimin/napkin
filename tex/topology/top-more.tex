\chapter{Topological notions}
\label{ch:top_more}
Here I'm just talking about some various properties of general topological spaces.
The chapter is admittedly a bit disconnected (pun not intended, but hey, why not).

I should say a little about metric spaces vs general topological spaces here.
In most of the sections we only really talk about continuous maps or open sets,
so it doesn't make a difference whether we restrict our attention
to metric spaces or not.
But when we talk about completeness, we consider \emph{sequences} of points,
and in particular the distances between them.
This notion doesn't make sense for general spaces, so for that section one must
keep in mind we are only working in metric spaces.

The most important topological notion is missing from this chapter:
that of a \emph{compact} space.
It is so important that I have dedicated a separate chapter just for it.

Note that in contrast to the warning on open/closed sets I gave earlier,
\begin{moral}
	The adjectives in this chapter will be used to describe \emph{spaces}.
\end{moral}

%As I alluded to earlier, sequences in metric spaces are super nice,
%but sequences in general topological spaces \emph{suck} (to the point where
%I didn't bother to define convergence of general sequences).

\section{Connected spaces}
It is possible for a set to be both open and closed (or \vocab{clopen}) in a
topological space $X$;
for example $\varnothing$ and the entire space are examples of clopen sets.
In fact, the presence of a nontrivial clopen set other than these two leads
to a so-called \emph{disconnected} space.

\begin{ques}
	Show that a space $X$ has a nontrivial clopen set
	(one other than $\varnothing$ and $X$)
	if and only if $X$ can be written as a disjoint union $U \sqcup V$
	where $U$ and $V$ are both open and nonempty.
	(Use the definition that $X$ is the complement of an open set.)
\end{ques}

We say $X$ is \vocab{disconnected} if there are nontrivial clopen sets,
and \vocab{connected} otherwise.
To see why this should be a reasonable definition, it might help
to solve \Cref{prob:disconnected_better_def}.

\begin{example}[Disconnected and connected spaces]
	\listhack
	\begin{enumerate}[(a)]
		\ii The metric space 
		\[ \{ (x,y) \mid x^2+y^2\le 1 \} \cup \{ (x,y) \mid (x-4)^2+y^2\le1\} \subseteq \RR^2 \]
		is disconnected (it consists of two disks).

		\ii A discrete space on more than one point is disconnected,
		since \emph{every} set is clopen in the discrete space.

		\ii Convince yourself that the set 
		\[ \left\{ x \in \QQ : x^2 < 2014 \right\} \]
		is a clopen subset of $\QQ$. 
		Hence $\QQ$ is disconnected too -- it has \emph{gaps}.

		\ii $[0,1]$ is connected.
	\end{enumerate}
\end{example}

%\begin{remark}
%	For general topological spaces $X$, it is still true that if $S$ is closed
%	(meaning $X \setminus S$ is open), then it contains the limits of all its sequences.
%	But the converse of this statement no longer holds.
%\end{remark}

\section{Path-connected spaces}
\prototype{Walking around in $\CC$.}
A stronger and perhaps more intuitive notion of a connected space is a \emph{path-connected} space.
The short description: ``walk around in the space''.
% We have general topological spaces, but so far they've just been sitting there.
% Let's start walking in them.

% Throughout this section, we let $I = [0,1]$ with the usual topology
% inherited from $\RR$.
%Also, recall that $S^n$ is the surface of the sphere living
%in $n+1$ dimensional space; hence $S^1$ is a unit circle, say.

\begin{definition}
	A \vocab{path} in the space $X$ is a continuous function
	\[ \gamma : [0,1] \to X. \]
	Its \vocab{endpoints} are the two points $\gamma(0)$ and $\gamma(1)$.
\end{definition}

You can think of $[0,1]$ as measuring ``time'', and so we'll often write $\gamma(t)$
for $t \in [0,1]$ (with $t$ standing for ``time'').
Here's a picture of a path.
\begin{center}
	\begin{asy}
		bigblob("$X$");
		pair A = Drawing("\gamma(0)", (-3,-1));
		pair B = Drawing("\gamma(1)", (2,1), dir(90));
		path p = A..(-2,0)..(0,2)..(1,0)..B;
		draw(p, EndArrow);
		MP("\gamma", midpoint(p), dir(90));
	\end{asy}
\end{center}
\begin{ques}
	Why does this agree with your intuitive notion of what a ``path'' and ``loop'' is?
\end{ques}

\begin{definition}
	A space $X$ is \vocab{path-connected} if
	any two points in it are connected by some path.
\end{definition}

\begin{exercise}
	Let $X = U \sqcup V$ be a disconnected space.
	Show that there is no path
	from a point of $U$ to point $V$.
	(If $\gamma : [0,1] \to X$, then we get $[0,1] = \gamma\pre(U) \sqcup \gamma\pre(V)$,
	but $[0,1]$ is connected.)
\end{exercise}
In fact, there exist connected spaces which are not path-connected
(one is called the \emph{topologist's sine curve}).
This shows that path-connected is stronger than connected.

\begin{example}[Examples of path-connected spaces]
	\listhack
	\begin{itemize}
		\ii $\RR^2$ is path-connected, since we can ``connect'' any two points with a straight line.
		\ii The unit circle $S^1$ is path-connected, since
		we can just draw the major or minor arc to connect two points.
	\end{itemize}
\end{example}

% Easy, right?

\section{Homotopy and simply connected spaces}
\prototype{$\CC$ and $\CC \setminus \{0\}$.}
Now let's motivate the idea of homotopy.
Consider the example of the complex plane $\CC$ (which you can
think of just as $\RR^2$) with two points $A$ and $B$.
There's a whole bunch of paths from $A$ to $B$ but somehow they're not very different from one another.
If I told you ``walk from $A$ to $B$'' you wouldn't have too many questions.

\begin{center}
	\begin{asy}
		unitsize(0.8cm);
		bigbox("$\mathbb C$");
		pair A = Drawing("A", (-3,0), dir(180));
		pair B = Drawing("B", (3,0), dir(0));
		draw(A..(-2,0.5)..(0,2)..(1,1.2)..B, red, EndArrow);
		draw(A--B, mediumgreen, EndArrow);
		draw(A--(-1,-1)--(2,-1)--B, blue, EndArrow);
		// draw(A..(1.5,-2)..(-1.5,-2)..B, orange, EndArrow);
	\end{asy}
\end{center}

So we're living happily in $\CC$ until a meteor strikes the origin,
blowing it out of existence.
Then suddenly to get from $A$ to $B$, people might tell you two different things: ``go left around the meteor'' or ``go right around the meteor''.

\begin{center}
	\begin{asy}
		unitsize(0.8cm);
		bigbox("$\mathbb C \setminus \{0\}$");
		pair A = Drawing("A", (-3,0), dir(180));
		pair B = Drawing("B", (3,0), dir(0));
		draw(A..(-2,0.5)..(0,2)..(1,1.2)..B, red, EndArrow);
		draw(A--(-1,-1)--(2,-1)--B, blue, EndArrow);
		filldraw(scale(0.5)*unitcircle, grey, black);
	\end{asy}
\end{center}

So what's happening?
In the first picture, the red, green, and blue paths somehow all looked
the same: if you imagine them as pieces of elastic string pinned down at $A$ and $B$, you can stretch each one to any other one.

But in the second picture, you can't move the red string to match with the blue string: there's a meteor in the way.
The paths are actually different.
\footnote{If you know about winding numbers, you might feel this is familiar.  We'll talk more about this in the chapter on homotopy groups.}

The formal notion we'll use to capture this is \emph{homotopy equivalence}.
We want to write a definition such that in the first picture,
the three paths are all \emph{homotopic}, but the two paths in the
second picture are somehow not homotopic.
And the idea is just continuous deformation.

\begin{definition}
	Let $\alpha$ and $\beta$ be paths in $X$ whose endpoints coincide.
	A (path) \vocab{homotopy} from $\alpha$ to $\beta$ is a continuous function
	$F : [0,1]^2 \to X$, which we'll write $F_s(t)$ for $s,t \in [0,1]$,
	such that
	\[ F_0(t) = \alpha(t) \text{ and } F_1(t) = \beta(t)
		\text{ for all $t \in [0,1]$} \]
	and moreover
	\[ \alpha(0) = \beta(0) = F_s(0)
		\text{ and }
		\alpha(1) = \beta(1) = F_s(1)
		\text{ for all $s \in [0,1]$}. \]
	If a path homotopy exists, we say $\alpha$ and $\beta$
	are path \vocab{homotopic} and write $\alpha \simeq \beta$.
\end{definition}
\begin{abuse}
	While I strictly should say ``path homotopy'' to describe this relation
	between two paths, I will shorten this to just ``homotopy'' instead.
	Similarly I will shorten ``path homotopic'' to ``homotopic''.
\end{abuse}
Picture: \url{https://commons.wikimedia.org/wiki/File:HomotopySmall.gif}.
And needless to say, $\simeq$ is an equivalence relation.

What this definition is doing is taking $\alpha$ and ``continuously deforming'' it to $\beta$, while keeping the endpoints fixed.
Note that for each particular $s$, $F_s$ is itself a function.
So $s$ represents time as we deform $\alpha$ to $\beta$:
it goes from $0$ to $1$, starting at $\alpha$ and ending at $\beta$.

\begin{center}
	\begin{asy}
		size(9cm);
		bigbox("$\mathbb C$");
		pair A = Drawing("A", (-3,0), dir(180));
		pair B = Drawing("B", (3,0), dir(0));
		draw(A..MP("F_{0} = \alpha", (0,2), dir(45))..B, heavygreen);
		draw(A..MP("F_{0.25}", (0,1), dir(45))..B, mediumgreen);
		draw(A..MP("F_{0.5}", (0,0), dir(90))..B, palecyan);
		draw(A..MP("F_{0.75}", (0,-1), dir(-45))..B, mediumcyan);
		draw(A..MP("F_{1} = \beta", (0,-2), dir(-45))..B, heavycyan);
		// draw(A..(1.5,-2)..(-1.5,-2)..B, orange, EndArrow);
	\end{asy}
\end{center}

\begin{ques}
	Convince yourself the above definition is right.
	What goes wrong when the meteor strikes?
\end{ques}

So now I can tell you what makes $\CC$ special:
\begin{definition}
	A space $X$ is \vocab{simply connected} if it's path-connected and
	for any paths $A$ and $B$, all paths from $A$ to $B$ are homotopic.
\end{definition}
That's why you don't ask questions when walking from $A$ to $B$ in $\CC$:
there's really only one way to walk. Hence the term ``simply'' connected.

\begin{ques}
	Convince yourself that $\RR^n$ is simply connected for all $n$.
\end{ques}

\section{Bases of spaces}
\prototype{$\RR$ has a basis of open intervals, and $\RR^2$ has a basis of open disks.}

You might have noticed that the open sets of $\RR$ are a little annoying to describe:
the prototypical example of an open set is $(0,1)$,
but there are other open sets like
\[
	(0,1)
	\cup \left( 1, \frac32 \right)
	\cup \left( 2, \frac 73 \right)
	\cup (2014, 2015). \]
\begin{ques}
	Check this is an open set.
\end{ques}

But okay, this isn't \emph{that} different.
All I've done is taken a bunch of my prototypes and threw a bunch of $\cup$ signs at it.
And that's the idea behind a basis.

\begin{definition}
	A \vocab{basis} for a topological space $X$
	is a subset $\mathcal B$ of the open sets
	such that every open set in $X$
	is a union of some (possibly infinite) number of elements in
	$\mathcal B$.
\end{definition}

And all we're doing is saying
\begin{example}[Basis of $\RR$]
	The open intervals form a basis of $\RR$.
\end{example}
In fact, more generally we have:
\begin{theorem}[Basis of metric spaces]
	The $r$-neighborhoods form a basis of any metric space $M$.
\end{theorem}
\begin{proof}
	Kind of silly -- given an open set $U$
	draw a $r_p$-neighborhood $U_p$ contained entirely inside $U$.
	Then $\bigcup_p U_p$ is contained in $U$ and covers
	every point inside it.
\end{proof}

Hence, an open set in $\RR^2$ is nothing more than a union
of a bunch of open disks, and so on.
The point is that in a metric space, the only open sets you really
ever have to worry too much about are the $r$-neighborhoods.


\section{Completeness}
\prototype{$\RR$ is complete, but $\QQ$ and $(0,1)$ are not.}
Completeness is a property of \emph{metric} spaces.

So far we can only talk about sequences converging if they have a limit.
But consider the sequence $x_1 = 1$, $x_2 = 1.4$, $x_3 = 1.41$, $x_4 = 1.414$, \dots.
It converges to $\sqrt 2$ in $\RR$, of course.

But it fails to converge in $\QQ$.
And so somehow, if we didn't know about the existence of $\RR$, we would
have \emph{no idea} that the sequence $(x_n)$ is ``approaching'' something.

That seems to be a shame.
Let's set up a new definition to describe these sequences whose terms
get close to each other,
even if they don't approach any particular point in the space.
Thus, we only want to mention the given points in the definition.
This is hard to do in a general topological space,
but such a notion does exist for metric spaces.

\begin{definition}
	Let $x_1, x_2, \dots$ be a sequence which lives in a metric space $M = (M,d_M)$.
	We say the sequence is \vocab{Cauchy} if for any $\eps > 0$, we have
	\[ d_M(x_m, x_n) < \eps \]
	for all sufficiently large $m$ and $n$.
\end{definition}

\begin{ques}
	Show that a sequence which converges is automatically Cauchy.
	(Draw a picture.)
\end{ques}

Now we can define:
\begin{definition}
	A metric space $M$ is \vocab{complete} if every
	Cauchy sequence converges.
\end{definition}

\begin{example}
	[Examples of complete spaces]
	\listhack
	\begin{enumerate}[(a)]
		\ii $\RR$ is complete. (Depending on your definition of $\RR$, this either follows
		by definition, or requires some work. We won't go through this here.)
		\ii The discrete space is complete, as the only Cauchy sequences are eventually constant.
		\ii The closed interval $[0,1]$ is complete.
		\ii $\RR^n$ is complete as well. (You're welcome to prove this by induction on $n$.)
	\end{enumerate}
\end{example}
\begin{example}
	[Non-Examples of complete spaces]
	\listhack
	\begin{enumerate}[(a)]
		\ii The rationals $\QQ$ are not complete.
		\ii The open interval $(0,1)$ is not complete, as the sequence $x_n = \frac 1n$
		is Cauchy but does not converge.
	\end{enumerate}
\end{example}

So, metric spaces need not be complete, like $\QQ$.
But it turns out that every metric space can be \emph{completed}
by ``filling in the gaps'', resulting in a space
called the \vocab{completion} of the metric space.
For example, the completion of $\QQ$ is $\RR$ (in fact, this is often taken as the definition of $\RR$).

%It's a theorem that $\RR$ is complete.
%To prove this I'd have to define $\RR$ rigorously, which I won't do here.
%In fact, there are some competing definitions of $\RR$.
%It is sometimes defined as the completion of the space $\QQ$.
%Other times it is defined using something called \emph{Dedekind cuts}.
%For now, let's just accept that $\RR$ behaves as we expect and is complete.

%And thus, I can now tell you exactly what $\RR$ is.
%You might notice that it's actually not that easy to define:
%we can define $\QQ = \left\{ a/b : a,b \in \NN \right\}$, but
%what do we say for $\RR$?
%Here's the answer:
%\begin{definition}
%	$\RR$ is the completion of the metric space $\QQ$.
%\end{definition}

\section{Subspacing}
Earlier in this chapter,
I declared that all the adjectives we defined were used to describe spaces.
However, I now want to be more flexible and describe
how they can be used to define subsets of spaces as well.

We've talked about some properties that a space can have;
say, a space is \emph{path-connected}, or \emph{simply connected}, or \emph{complete}.
If you think about it, it would make sense to use the same adjectives on sets.
You'd say a set $S$ is path-connected if there's a path between
every pair of points in $S$ through points in $S$. And so on.

The reason you can do this is that for a metric space $M$,
\begin{moral}
	Every subset $S \subseteq M$ is a metric space in its own right .
\end{moral}
To see this, we just use the same distance function;
this is called the \vocab{subspace topology}.
For example, you can think of a circle $S^1$ as a connected set
by viewing it as a subspace of $\RR^2$.
Thus,
\begin{abuse}
	Any adjective used to describe a space can equally be used for a subset of a space,
	and I'll thus be letting these mix pretty promiscuously.
	So in what follows, I might refer to subsets being complete,
	even when I only defined completeness for spaces.
	This is what I mean.
\end{abuse}
So to be perfectly clear:
\begin{itemize}
	\ii The statement ``$[0,1]$ is complete'' makes sense (and is true);
	it says that $[0,1]$ is a complete metric space.
	\ii The statement ``$[0,1]$ is a complete subset of $\RR$'' is valid;
	it says that the subspace $[0,1]$ of $\RR$ is a complete metric space,
	which is of course true.
	\ii The statement ``$[0,1]$ is a closed subset of $\RR$'' makes sense;
	it says that the set of points $[0,1]$ form a closed subset of a parent space $\RR$.
	\ii The statement ``$[0,1]$ is closed'' does \emph{not} make sense.
	Closed sets are only defined relative to parent spaces.
\end{itemize}

To make sure you understand this:
\begin{ques}
	Let $M$ be a complete metric space and let $S \subseteq M$.
	Prove that $S$ is complete if and only if it is closed in $M$.
	(This is obvious once you figure out what the question is asking.)
	In particular, $[0,1]$ is complete.
\end{ques}

One can also define the subspace topology in a general topological space $X$.
Given a subset $S \subseteq X$, the open sets of the subspace $S$ are those of the form $S \cap U$,
where $U$ is an open set of $X$.
So for example, if we view $S^1$ as a subspace of $\RR^2$,
then any open arc is an open set, because you can view it as the intersection of an open disk with $S^1$.
\begin{center}
	\begin{asy}
		size(3cm);
		draw(unitcircle, black+1);
		MP("S^1", dir(60), dir(60));
		MP("\mathbb R^2", dir(-45)*1.2, dir(-45));
		pair A = dir(-30);
		pair B = dir(50);
		draw(CP(dir(10), A), dotted);
		draw(arc(origin,A,B), blue+2);
		dotfactor *= 2;
		opendot(A, blue);
		opendot(B, blue);
	\end{asy}
\end{center}

Needless to say, for metric spaces it doesn't matter which of these definitions I choose.
(Proving this turns out to be surprisingly annoying, so I won't do so.)

\section{Hausdorff spaces}
\prototype{Every space that's not the Zariski topology (defined much later).}
In analysis, almost all topological spaces we encounter satisfy
some additional hypotheses about their points.
There is a whole hierarchy of such axioms, labelled $T_n$ for
integers $n$ (with $n=0$ being the weakest and $n=6$ the strongest).
These axioms are called \vocab{separation axioms}.

By far the most common hypothesis is the $T_2$ axiom,
which bears a special name.
\begin{definition}
	A topological space $X$ is \vocab{Hausdorff} if
	for any two disjoint points $p$ and $q$ in $X$,
	there exists a neighborhood $U$ of $p$
	and a neighborhood $V$ of $q$ such that
	\[ U \cap V = \varnothing. \]
\end{definition}
In other words, around any two distinct points we should be
able to draw disjoint neighborhoods.
\begin{ques}
	Important special case: all metric spaces are Hausdorff.
\end{ques}

I just want to define this here so that I can use this word later.
In any case, basically any space we will encounter other than
the Zariski topology is Hausdorff.

\section\problemhead

\begin{dproblem}
	Let $X$ be a topological space.
	Show that there exists a nonconstant continuous function $X \to \{0,1\}$ if and
	only if $X$ is disconnected (here $\{0,1\}$ is given the discrete topology).
	\label{prob:disconnected_better_def}
\end{dproblem}

\begin{sproblem}
	Let $f : X \to Y$ be a continuous function.
	\begin{enumerate}[(a)]
		\ii Show that if $X$ is connected then so is $f``(X)$.
		\ii Show that if $X$ is path-connected then so is $f``(X)$.
	\end{enumerate}
\end{sproblem}

%\begin{problem}
%	A topological space $X$ is called \vocab{locally path-connected}
%	if for every point $x \in X$, some neighborhood $U$ of $x$ is path-connected.
%	Prove that $X$ is path-connected if and only if it is connected
%	and locally path-connected.
%	\label{prob:local_path_connected}
%\end{problem}

\begin{dproblem}[Banach fixed point theorem]
	Let $M = (M,d)$ be a complete metric space.
	Suppose $T : M \to M$ is a continuous map such that for any $p,q \in M$,
	\[ d\left( T(p), T(q) \right) < 0.999 d(p,q). \]
	(We call $T$ a \vocab{contraction}.)
	Show that $T$ has a unique fixed point.
\end{dproblem}

\begin{problem}
	\gim
	We know that any open set $U \subseteq \RR$
	is a union of open intervals (allowing $\pm\infty$ as endpoints).
	One can show that it's actually possible to write $U$ as the
	union of \emph{pairwise disjoint} open intervals.\footnote{You are invited to try
	and prove this, but I personally found the proof quite boring.}
	Prove that there exists such a disjoint union with at most \emph{countably many}
	intervals in it.
	\begin{hint}
		Appeal to $\QQ$.
	\end{hint}
	\begin{sol}
		You can pick a rational number in each interval and
		there are only countably many rational numbers. Done!
	\end{sol}
\end{problem}

\begin{problem}
	\yod
	From the plane $\RR^2$ we delete two distinct points $p$ and $q$.
	Is it possible to partition the remaining points into
	non-overlapping circumferences of nonzero radius?
	\begin{hint}
		Color a circle magenta if it contains $p$ but not $q$ and color a circle cyan if it contains $q$ but not $p$. Color $p$ itself magenta and $q$ itself cyan as well. Finally, color a circle neon yellow if it contains both $p$ and $q$. 
	\end{hint}
	\begin{sol}
		Color a circle magenta if it contains $p$ but not $q$ and color a circle cyan if it contains $q$ but not $p$. Color $p$ itself magenta and $q$ itself cyan as well. Finally, color a circle neon yellow if it contains both $p$ and $q$. By repeating the argument in (1) there are no circles enclosing neither $p$ nor $q$. Hence every point is either magenta, cyan, or neon yellow.
		Now note that given any magenta circle, its interior is completely magenta. Actually, the magenta circles can be totally ordered by inclusion (since they can't intersect). So we consider two cases: 
		\begin{itemize}
		 \ii If there is a maximal magenta circle (i.e. a magenta circle not contained in any other magenta circle) then the set of all magenta points is just a closed disk.
		 \ii If there is no maximal magenta circle, then the set of magenta points can also be expressed as the union over all magenta circles of their interiors. This is a union of open sets, so it is itself open.
		 \end{itemize}

		We conclude the set of magenta points is either a closed disk or an open set. Similarly for the set of cyan points. Moreover, the set of such points is convex.

		To finish the problem:
		\begin{itemize}
		\ii Suppose there are no neon yellow points. If the magenta points form a closed disk, then the cyan points are $\mathbb R^2$ minus a disk which is not convex. Contradiction. So the magenta points must be open. Similarly the cyan points must be open. But $\mathbb R^2$ is connected, so it can't be written as the union of two open sets.
		\ii Now suppose there are neon yellow points. We claim there is a neon yellow circle minimal by inclusion. If not, then repeat the argument of (1) to get a contradiction, since any neon yellow circle must have diameter the distance from $p$ to $q$. So we can find a neon yellow circle $\mathscr C$ whose interior is all magenta and cyan. Now repeat the argument of the previous part, replacing $\mathbb R^2$ by the interior of $\mathscr C$.
		 \end{itemize}
	\end{sol}
\end{problem}

\begin{dproblem}[Completion]
	\yod
	Let $M$ be a metric space.
	Show that we can add some points to $M$ to obtain a metric space $\ol M$
	which is complete, in such a way that every open set of $\ol M$ contains
	a point of $M$ (meaning $M$ is \vocab{dense} in $\ol M$).
\end{dproblem}
