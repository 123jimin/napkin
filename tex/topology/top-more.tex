\chapter{Topological spaces}
\label{ch:top_more}

In \Cref{ch:metric_space} we introduced the notion
of space by describing metrics on them.
This gives you a lot examples, and nice intuition,
and tells you how you should draw pictures of open and closed sets.

However, moving forward, it will be useful to begin
thinking about topological spaces in terms of just their \emph{open sets}.
(One motivation is that our fishy \Cref{ex:fishy} shows that
in some ways the notion of homeomorphism really wants
to be phrased in terms of open sets, not in terms of the metric.)
As we are going to see, the open sets manage to actually retain
nearly all the information we need, but are simpler.\footnote{The
	reason I adamantly introduce metric spaces first
	is because I think otherwise the examples make much less sense.}
This will be done in just a few sections,
and after that we will start describing more adjectives
that we can apply to topological (and hence metric) spaces.

The most important topological notion is missing from this chapter:
that of a \emph{compact} space.
It is so important that I have dedicated a separate chapter just for it.

Quick note for those who care:
the adjectives ``Hausdorff'', ``connected'',
and later ``compact'' are all absolute adjectives.

%Note that in contrast to the warning on open/closed sets I gave earlier,
%\begin{moral}
%	The adjectives in this chapter will be used to describe \emph{spaces}.
%\end{moral}

%As I alluded to earlier, sequences in metric spaces are super nice,
%but sequences in general topological spaces \emph{suck} (to the point where
%I didn't bother to define convergence of general sequences).

\section{Forgetting the metric}
Recall \Cref{thm:open_set}:
\begin{quote}
	A function $f \colon M \to N$ of metric spaces is continuous
	if and only if the pre-image of every open set in $N$ is open in $M$.
\end{quote}
Despite us having defined this in the context of metric spaces,
this nicely doesn't refer to the metric at all,
only the open sets.
As alluded to at the start of this chapter,
this is a great motivation for how we can forgot
about the fact that we had a metric to begin with,
and rather \emph{start} with the open sets instead.

\begin{definition}
	A \vocab{topological space} is a pair $(X, \mathcal T)$,
	where $X$ is a set of points,
	and $\mathcal T$ is the \vocab{topology},
	which consists of several subsets of $X$, called the \vocab{open sets} of $X$.
	The topology must obey the following axioms.
	\begin{itemize}
		\ii $\varnothing$ and $X$ are both in $\mathcal T$.
		\ii Finite intersections of open sets are also in $\mathcal T$.
		\ii Arbitrary unions (possibly infinite) of open sets are also in $\mathcal T$.
	\end{itemize}
\end{definition}
So this time, the open sets are \emph{given}.
Rather than defining a metric and getting open sets from the metric,
we instead start from just the open sets.
\begin{abuse}
	We abbreviate $(X, \mathcal T)$ by just $X$,
	leaving the topology $\mathcal T$ implicit.
	(Do you see a pattern here?)
\end{abuse}

\begin{example}[Examples of topologies]
	\listhack
	\begin{enumerate}[(a)]
		\ii Given a metric space $M$, we can let $\mathcal T$ be
		the open sets in the metric sense.
		The point is that the axioms are satisfied.
		\ii In particular, \vocab{discrete space}
		is a topological space in which every set is open. (Why?)
		\ii Given $X$, we can let $\mathcal T = \left\{ \varnothing, X \right\}$,
		the opposite extreme of the discrete space.
	\end{enumerate}
\end{example}

Now we can port over our metric definitions.
\begin{definition}
	An \vocab{open neighborhood}\footnote{In literature,
		a ``neighborhood'' refers to a set
		which contains some open set around $x$.
		We will not use this term,
		and exclusively refer to ``open neighborhoods''.}
	of a point $x \in X$ is an
	open set $U$ which contains $x$ (see figure).
\end{definition}
\begin{center}
	\begin{asy}
		size(4cm);
		bigblob("$X$");
		pair p = Drawing("x", (0.3,0.1), dir(-90));
		real r = 1.55;
		draw(shift(p) * scale(1.6,1.2)*unitcircle, dashed);
		label("$U$", p+r*dir(45), dir(45));
	\end{asy}
\end{center}

\begin{abuse}
	Just to be perfectly clear:
	by an ``open neighborhood'' I mean \emph{any} open set containing $x$.
	But by an ``$r$-neighborhood'' I always mean the
	points with distance less than $r$ from $x$,
	and so I can only use this term if my space is a metric space.
\end{abuse}

\section{Re-definitions}
Now that we've defined a topological space,
for nearly all of our metric notions we can write down
as the definition the one that required only open sets
(which will of course agree with our old definitions
when we have a metric space).

\subsection{Continuity}
Here was our motivating example, continuity:
\begin{definition}
	We say function $f \colon X \to Y$ of topological spaces
	is \vocab{continuous} at a point $p \in X$ if the pre-image of any
	open neighborhood of $f(p)$ is an open neighborhood of $p$.
	The function is continuous if it is continuous at every point.
\end{definition}

Thus homeomorphisms carries over:
a bijection which is continuous in both directions.
\begin{definition}
	A \vocab{homeomorphism} of topological spaces
	$(X, \tau_X)$ and $(Y, \tau_Y)$
	is a bijection $f \colon X \to Y$
	which induces a bijection from $\tau_X$ to $\tau_Y$:
	i.e.\ the bijection preserves open sets.
\end{definition}
\begin{ques}
	Show that this is equivalent to $f$ and its inverse
	both being continuous.
\end{ques}
Therefore, any property defined only in
terms of open sets is preserved by homeomorphism.
Such a property is called a \vocab{topological property}.
However, the later adjectives we define 
(``connected'', ``Hausdorff'', ``compact'') will all be defined
only in terms of the open sets, so they will be.



%\begin{definition}
%	A sequence $(x_n)$ of points in a topological space $X$ is said to \vocab{converge to} $x \in X$ if for every open neighborhood of $x$,
%	eventually all terms of the sequence lie in that open neighborhood.
%\end{definition}
%\begin{remark}
%	Unfortunately, for general topological spaces we no longer have the nice property
%	that any function which preserves sequential limits is automatically continuous.
%\end{remark}
%There's one other property of open sets that we have in a metric space that isn't implied by the above: for any two points of $X$, we can find an open set containing one but not the other.
%A space which also has this property is called a \vocab{Kolmogorov space}.
%This property is a good property to have, because if $x,y \in X$ are in the same open sets, the topology can't tell them apart.

\subsection{Closed sets}
We saw last time there were two equivalent definitions
for closed sets, but one of them relies only on open sets, and we use it:
\begin{definition}
	\label{def:closure}
	In a general topological space $X$, we say that $S \subseteq X$ is
	\vocab{closed} in $X$ if the complement $X \setminus S$ is open in $X$.

	If $S \subseteq X$ is any set, the \vocab{closure} of $S$,
	denoted $\ol S$,
	is defined as the smallest closed set containing $S$.
\end{definition}
Thus for general topological spaces,
open and closed sets carry the same information,
and it is entirely a matter of taste whether we define everything in terms
of open sets or closed sets.
In particular, you can translate axioms and properties of open sets to closed ones:
\begin{ques}
	Show that the (possibly infinite) intersection of closed sets is closed
	while the union of finitely many closed sets is closed.
	(Look at complements.)
\end{ques}
\begin{exercise}
	Show that a function is continuous if and only if the pre-image
	of every closed set is closed.
\end{exercise}
Mathematicians seem to have agreed that they like open sets better.


\subsection{Properties that don't carry over}
Not everything works:

\begin{remark}
	[Complete and (totally) bounded are metric properties]
	The two metric properties we have seen,
	``complete'' and ``(totally) bounded'',
	are not topological properties.
	They rely on a metric,
	so as written we cannot apply them to topological spaces.
	One might hope that maybe,
	there is some alternate definition (like we saw for ``continuous function'')
	that is just open-set based.
	But \Cref{ex:fishy} showing $(0,1) \cong \RR$
	tells us that it is hopeless.
\end{remark}

\begin{remark}
	[Sequences don't work well]
	You could also try to port over the notion
	of sequences and convergent sequences.
	However, this turns out to break a lot of desirable properties.
	Therefore I won't bother to do so,
	and thus if we are discussing sequences you should
	assume that we are working with a metric space.
\end{remark}

\section{Hausdorff spaces}
\prototype{Every space that's not the Zariski topology (defined much later).}

As you might have guessed,
there exist topological spaces which cannot be realized
as metric spaces (in other words, are not \vocab{metrizable}).
One example is just to take $X = \{a,b,c\}$ and the topology
$\tau_X = \left\{ \varnothing, \{a,b,c\} \right\}$.
This topology is fairly ``stupid'':
it can't tell apart any of the points $a$, $b$, $c$!
But any metric space can tell its points apart (because $d(x,y) > 0$ when $x \neq y$).

We'll see less trivial examples later,
but for now we want to add a little more sanity condition onto our spaces.
There is a whole hierarchy of such axioms, labelled $T_n$ for
integers $n$ (with $n=0$ being the weakest and $n=6$ the strongest);
these axioms are called \vocab{separation axioms}.

By far the most common hypothesis is the $T_2$ axiom,
which bears a special name.
\begin{definition}
	A topological space $X$ is \vocab{Hausdorff} if
	for any two distinct points $p$ and $q$ in $X$,
	there exists an open neighborhood $U$ of $p$
	and an open neighborhood $V$ of $q$ such that
	\[ U \cap V = \varnothing. \]
\end{definition}

In other words, around any two distinct points we should be
able to draw disjoint open neighborhoods.
Here's a picture to go with above,
but not much going on.
\begin{center}
\begin{asy}
	size(5cm);
	pair p = (-1.5,0);
	pair q = ( 1.5,0);
	filldraw(CR(p,1), opacity(0.1)+lightcyan, blue+dashed);
	filldraw(CR(q,1), opacity(0.1)+lightred, red+dashed);
	dot("$p$", p, dir(-90), blue);
	dot("$q$", q, dir(-90), red);
\end{asy}
\end{center}

\begin{ques}
	Show that all metric spaces are Hausdorff.
\end{ques}

I just want to define this here so that I can use this word later.
In any case, basically any space we will encounter other than
the Zariski topology is Hausdorff.


\section{Subspaces}
\prototype{$S^1$ is a subspace of $\RR^2$.}

One can also take subspaces of general topological spaces.
\begin{definition}
	Given a topological space $X$,
	and a subset $S \subseteq X$,
	we can make $S$ into a topological space
	by declaring that the open subsets of $S$ are $U \cap S$
	for open $U \subseteq X$.
	This is called the \vocab{subspace topology}.
\end{definition}
So for example, if we view $S^1$ as a subspace of $\RR^2$,
then any open arc is an open set,
because you can view it as the intersection of an open disk with $S^1$.
\begin{center}
	\begin{asy}
		size(3cm);
		draw(unitcircle, black+1);
		MP("S^1", dir(60), dir(60));
		MP("\mathbb R^2", dir(-45)*1.2, dir(-45));
		pair A = dir(-30);
		pair B = dir(50);
		draw(CP(dir(10), A), dotted);
		draw(arc(origin,A,B), blue+2);
		dotfactor *= 2;
		opendot(A, blue);
		opendot(B, blue);
	\end{asy}
\end{center}

Needless to say, for metric spaces it doesn't matter
which of these definitions I choose.
(Proving this turns out to be surprisingly annoying,
so I won't do so.)

\section{Connected spaces}
\prototype{$[0,1] \cup [2,3]$ is disconnected.}
Even in metric spaces, it is possible for a set
to be both open and closed.
\begin{definition}
	A subset $S$ of a topological space $X$
	is \vocab{clopen} if it is both closed and open in $X$.
	(Equivalently, both $S$ and its complement are open.)
\end{definition}
For example $\varnothing$ and the entire space are examples of clopen sets.
In fact, the presence of a nontrivial clopen set other
than these two leads to a so-called \emph{disconnected} space.

\begin{ques}
	Show that a space $X$ has a nontrivial clopen set
	(one other than $\varnothing$ and $X$)
	if and only if $X$ can be written as a disjoint union
	of two nonempty open sets.
\end{ques}

We say $X$ is \vocab{disconnected}
if there are nontrivial clopen sets,
and \vocab{connected} otherwise.
To see why this should be a reasonable definition,
it might help to solve \Cref{prob:disconnected_better_def}.

\begin{example}[Disconnected and connected spaces]
	\listhack
	\begin{enumerate}[(a)]
		\ii The metric space 
		\[ \{ (x,y) \mid x^2+y^2\le 1 \}
			\cup \{ (x,y) \mid (x-4)^2+y^2\le1\} \subseteq \RR^2 \]
		is disconnected (it consists of two disks).

		\ii The space $[0,1] \cup [2,3]$ is disconnected:
		it consists of two segments,
		each of which is a clopen set.

		\ii A discrete space on more than one point is disconnected,
		since \emph{every} set is clopen in the discrete space.

		\ii Convince yourself that the set
		\[ \left\{ x \in \QQ : x^2 < 2014 \right\} \]
		is a clopen subset of $\QQ$.
		Hence $\QQ$ is disconnected too -- it has \emph{gaps}.

		\ii $[0,1]$ is connected.
	\end{enumerate}
\end{example}

%\begin{remark}
%	For general topological spaces $X$, it is still true that if $S$ is closed
%	(meaning $X \setminus S$ is open), then it contains the limits of all its sequences.
%	But the converse of this statement no longer holds.
%\end{remark}


\section{Path-connected spaces}
\prototype{Walking around in $\CC$.}
A stronger and perhaps more intuitive notion
of a connected space is a \emph{path-connected} space.
The short description: ``walk around in the space''.
% We have general topological spaces, but so far they've just been sitting there.
% Let's start walking in them.

% Throughout this section, we let $I = [0,1]$ with the usual topology
% inherited from $\RR$.
%Also, recall that $S^n$ is the surface of the sphere living
%in $n+1$ dimensional space; hence $S^1$ is a unit circle, say.

\begin{definition}
	A \vocab{path} in the space $X$ is a continuous function
	\[ \gamma : [0,1] \to X. \]
	Its \vocab{endpoints} are the two points $\gamma(0)$ and $\gamma(1)$.
\end{definition}

You can think of $[0,1]$ as measuring ``time'', and so we'll often write $\gamma(t)$
for $t \in [0,1]$ (with $t$ standing for ``time'').
Here's a picture of a path.
\begin{center}
	\begin{asy}
		bigblob("$X$");
		pair A = Drawing("\gamma(0)", (-3,-1));
		pair B = Drawing("\gamma(1)", (2,1), dir(90));
		path p = A..(-2,0)..(0,2)..(1,0)..B;
		draw(p, EndArrow);
		MP("\gamma", midpoint(p), dir(90));
	\end{asy}
\end{center}
\begin{ques}
	Why does this agree with your intuitive notion of what a ``path'' is?
\end{ques}

\begin{definition}
	A space $X$ is \vocab{path-connected} if
	any two points in it are connected by some path.
\end{definition}

\begin{exercise}
	[Path-connected implies connected]
	Let $X = U \sqcup V$ be a disconnected space.
	Show that there is no path
	from a point of $U$ to point $V$.
	(If $\gamma \colon [0,1] \to X$, then we get
	$[0,1] = \gamma\pre(U) \sqcup \gamma\pre(V)$,
	but $[0,1]$ is connected.)
\end{exercise}

\begin{example}[Examples of path-connected spaces]
	\listhack
	\begin{itemize}
		\ii $\RR^2$ is path-connected,
		since we can ``connect'' any two points with a straight line.
		\ii The unit circle $S^1$ is path-connected, since
		we can just draw the major or minor arc to connect two points.
	\end{itemize}
\end{example}

% Easy, right?

\section{Homotopy and simply connected spaces}
\prototype{$\CC$ and $\CC \setminus \{0\}$.}
\label{sec:meteor}

Now let's motivate the idea of homotopy.
Consider the example of the complex plane $\CC$ (which you can
think of just as $\RR^2$) with two points $p$ and $q$.
There's a whole bunch of paths from $p$ to $q$ but somehow
they're not very different from one another.
If I told you ``walk from $p$ to $q$'' you wouldn't have too many questions.

\begin{center}
	\begin{asy}
		unitsize(0.8cm);
		bigbox("$\mathbb C$");
		pair A = Drawing("p", (-3,0), dir(180));
		pair B = Drawing("q", (3,0), dir(0));
		draw(A..(-2,0.5)..(0,2)..(1,1.2)..B, red, EndArrow(TeXHead));
		draw(A--B, mediumgreen, EndArrow(TeXHead));
		draw(A--(-1,-1)--(2,-1)--B, blue, EndArrow(TeXHead));
		// draw(A..(1.5,-2)..(-1.5,-2)..B, orange, EndArrow(TeXHead));
	\end{asy}
\end{center}

So we're living happily in $\CC$ until a meteor strikes the origin,
blowing it out of existence.
Then suddenly to get from $p$ to $q$, people might tell you two different things:
``go left around the meteor'' or ``go right around the meteor''.

\begin{center}
	\begin{asy}
		unitsize(0.8cm);
		bigbox("$\mathbb C \setminus \{0\}$");
		pair A = Drawing("p", (-3,0), dir(180));
		pair B = Drawing("q", (3,0), dir(0));
		draw(A..(-2,0.5)..(0,2)..(1,1.2)..B, red, EndArrow(TeXHead));
		draw(A--(-1,-1)--(2,-1)--B, blue, EndArrow(TeXHead));
		filldraw(scale(0.5)*unitcircle, grey, black);
	\end{asy}
\end{center}

So what's happening?
In the first picture, the red, green, and blue paths somehow all looked
the same: if you imagine them as pieces of elastic string pinned down
at $p$ and $q$, you can stretch each one to any other one.

But in the second picture,
you can't move the red string to match with the blue string:
there's a meteor in the way.
The paths are actually different.\footnote{If you know about winding numbers,
	you might feel this is familiar.
	We'll talk more about this in the chapter on the fundamental group.}

The formal notion we'll use to capture this is \emph{homotopy equivalence}.
We want to write a definition such that in the first picture,
the three paths are all \emph{homotopic}, but the two paths in the
second picture are somehow not homotopic.
And the idea is just continuous deformation.

\begin{definition}
	Let $\alpha$ and $\beta$ be paths in $X$ whose endpoints coincide.
	A (path) \vocab{homotopy} from $\alpha$ to $\beta$ is a continuous function
	$F : [0,1]^2 \to X$, which we'll write $F_s(t)$ for $s,t \in [0,1]$,
	such that
	\[ F_0(t) = \alpha(t) \text{ and } F_1(t) = \beta(t)
		\text{ for all $t \in [0,1]$} \]
	and moreover
	\[ \alpha(0) = \beta(0) = F_s(0)
		\text{ and }
		\alpha(1) = \beta(1) = F_s(1)
		\text{ for all $s \in [0,1]$}. \]
	If a path homotopy exists, we say $\alpha$ and $\beta$
	are path \vocab{homotopic} and write $\alpha \simeq \beta$.
\end{definition}
\begin{abuse}
	While I strictly should say ``path homotopy'' to describe this relation
	between two paths, I will shorten this to just ``homotopy'' instead.
	Similarly I will shorten ``path homotopic'' to ``homotopic''.
\end{abuse}
Animated picture: \url{https://commons.wikimedia.org/wiki/File:HomotopySmall.gif}.
Needless to say, $\simeq$ is an equivalence relation.

What this definition is doing is taking $\alpha$ and ``continuously deforming'' it to $\beta$, while keeping the endpoints fixed.
Note that for each particular $s$, $F_s$ is itself a function.
So $s$ represents time as we deform $\alpha$ to $\beta$:
it goes from $0$ to $1$, starting at $\alpha$ and ending at $\beta$.

\begin{center}
	\begin{asy}
		size(9cm);
		bigbox("$\mathbb C$");
		pair A = Drawing("p", (-3,0), dir(180));
		pair B = Drawing("q", (3,0), dir(0));
		draw(A..MP("F_{0} = \alpha", (0,2), dir(45))..B, heavygreen);
		draw(A..MP("F_{0.25}", (0,1), dir(45))..B, mediumgreen);
		draw(A..MP("F_{0.5}", (0,0), dir(90))..B, palecyan);
		draw(A..MP("F_{0.75}", (0,-1), dir(-45))..B, mediumcyan);
		draw(A..MP("F_{1} = \beta", (0,-2), dir(-45))..B, heavycyan);
		// draw(A..(1.5,-2)..(-1.5,-2)..B, orange, EndArrow);
	\end{asy}
\end{center}

\begin{ques}
	Convince yourself the above definition is right.
	What goes wrong when the meteor strikes?
\end{ques}

So now I can tell you what makes $\CC$ special:
\begin{definition}
	A space $X$ is \vocab{simply connected} if it's path-connected and
	for any points $p$ and $q$, all paths from $p$ to $q$ are homotopic.
\end{definition}
That's why you don't ask questions when walking from $p$ to $q$ in $\CC$:
there's really only one way to walk. Hence the term ``simply'' connected.

\begin{ques}
	Convince yourself that $\RR^n$ is simply connected for all $n$.
\end{ques}

\section{Bases of spaces}
\prototype{$\RR$ has a basis of open intervals, and $\RR^2$ has a basis of open disks.}

You might have noticed that the open sets of $\RR$ are a little annoying to describe:
the prototypical example of an open set is $(0,1)$,
but there are other open sets like
\[
	(0,1)
	\cup \left( 1, \frac32 \right)
	\cup \left( 2, \frac 73 \right)
	\cup (2014, 2015). \]
\begin{ques}
	Check this is an open set.
\end{ques}

But okay, this isn't \emph{that} different.
All I've done is taken a bunch of my prototypes and threw a bunch of $\cup$ signs at it.
And that's the idea behind a basis.

\begin{definition}
	A \vocab{basis} for a topological space $X$
	is a subset $\mathcal B$ of the open sets
	such that every open set in $X$
	is a union of some (possibly infinite) number of elements in
	$\mathcal B$.
\end{definition}

And all we're doing is saying
\begin{example}[Basis of $\RR$]
	The open intervals form a basis of $\RR$.
\end{example}
In fact, more generally we have:
\begin{theorem}[Basis of metric spaces]
	The $r$-neighborhoods form a basis of any metric space $M$.
\end{theorem}
\begin{proof}
	Kind of silly -- given an open set $U$
	draw an $r_p$-neighborhood $U_p$ contained entirely inside $U$.
	Then $\bigcup_p U_p$ is contained in $U$ and covers
	every point inside it.
\end{proof}

Hence, an open set in $\RR^2$ is nothing more than a union
of a bunch of open disks, and so on.
The point is that in a metric space, the only open sets you really
ever have to worry too much about are the $r$-neighborhoods.


\section{\problemhead}

\begin{dproblem}
	Let $X$ be a topological space.
	Show that there exists a nonconstant continuous function $X \to \{0,1\}$ if and
	only if $X$ is disconnected (here $\{0,1\}$ is given the discrete topology).
	\label{prob:disconnected_better_def}
\end{dproblem}

\begin{sproblem}
	Let $X$ and $Y$ be topological spaces
	and let $f \colon X \to Y$ be a continuous function.
	\begin{enumerate}[(a)]
		\ii Show that if $X$ is connected then so is $f\im(X)$.
		\ii Show that if $X$ is path-connected then so is $f\im(X)$.
	\end{enumerate}
\end{sproblem}

\begin{problem}
	[Hausdorff implies $T_1$ axiom]
	Let $X$ be a Hausdorff topological space.
	Prove that for any point $p \in X$
	the set $\{p\}$ is closed.
\end{problem}

\begin{problem}
	[\cite{ref:pugh}, Exercise 2.56]
	Let $M$ be a metric space with more than one point
	but at most countably infinitely many points.
	Show that $M$ is disconnected.
	\begin{hint}
		Let $p$ be any point.
		If there is a real number $r$
		such that $d(p,q) \ne r$ for any $q \in M$,
		then the $r$-neighborhood of $p$ is clopen.
	\end{hint}
\end{problem}

\begin{problem}[Furstenberg]
	We declare a subset of $\ZZ$ to be open
	if it's the union (possibly empty or infinite)
	of arithmetic sequences $\left\{ a + nd \mid n \in \ZZ \right\}$,
	where $a$ and $d$ are positive integers.
	\begin{enumerate}[(a)]
		\ii Verify this forms a topology on $\ZZ$,
		called the \vocab{evenly spaced integer topology}.
		\ii Prove there are infinitely many primes by considering $\bigcup_p p\ZZ$
		for primes $p$.
	\end{enumerate}
	\begin{hint}
		Note that $p\ZZ$ is closed for each $p$.
		If there were finitely many primes, then
		$\bigcup p\ZZ = \ZZ \setminus \{-1,1\}$ would have to be closed;
		i.e.\ $\{-1,1\}$ would be open, but all open sets here are infinite.
	\end{hint}
\end{problem}

\begin{problem}
	\gim
	Prove that the evenly spaced integer topology on $\ZZ$ is metrizable.
	In other words, show that one can impose a metric $d : \ZZ^2 \to \RR$
	which makes $\ZZ$ into a metric space whose open sets are those described above.
	% https://teratologicmuseum.wordpress.com/2009/05/05/a-metric-for-the-evenly-spaced-integer-topology/
	\begin{hint}
		The balls at $0$ should be of the form $n! \cdot \ZZ$.
	\end{hint}
	\begin{sol}
		Let $d(x,y) = 2017^{-n}$,
		where $n$ is the largest integer
		such that $n!$ divides $\left\lvert x-y \right\rvert$.
	\end{sol}
\end{problem}

%\begin{problem}
%	A topological space $X$ is called \vocab{locally path-connected}
%	if for every point $x \in X$ and open neighborhood $U$ of $x$,
%	some open neighborhood $V$ of $x$ contained in $U$ is path-connected.
%	Prove that $X$ is path-connected if and only if it is connected
%	and locally path-connected.
%	\label{prob:local_path_connected}
%\end{problem}

\begin{problem}
	\gim
	We know that any open set $U \subseteq \RR$
	is a union of open intervals (allowing $\pm\infty$ as endpoints).
	One can show that it's actually possible to write $U$ as the
	union of \emph{pairwise disjoint} open intervals.\footnote{You are invited to try
	and prove this, but I personally found the proof quite boring.}
	Prove that there exists such a disjoint union with at most \emph{countably many}
	intervals in it.
	\begin{hint}
		Appeal to $\QQ$.
	\end{hint}
	\begin{sol}
		You can pick a rational number in each interval and
		there are only countably many rational numbers. Done!
	\end{sol}
\end{problem}
