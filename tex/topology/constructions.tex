\chapter{Some Topological Constructions (In Progress)}
In this short chapter we briefly describe some common spaces and constructions
in topology that we haven't yet discussed.

\section{Spheres}
Recall that
\[ S^n = \left\{ (x_0, \dots, x_n)
	\mid x_0^2 + \dots + x_n^2 = 1 \right\} \subset \RR^{n+1} \]
is the surface of an $n$-sphere while
\[ D^{n+1} = \left\{ (x_0, \dots, x_n)
	\mid x_0^2 + \dots + x_n^2 \le 1 \right\} \subset \RR^{n+1} \]
is the corresponding \emph{closed disk}
(So for example, $D^2$ is a disk in a plane while $S^1$ is the unit circle.)

In particular, $S^0$ consists of two points,
while $D^1$ can be thought of as the interval $[-1,1]$.

\begin{center}
	\begin{asy}
		size(8cm);
		draw(dir(0)--dir(180), blue);
		dot(dir(0), red+4);
		dot(dir(180), red+4);
		label("$S^0$", dir(0), dir(90), red);
		label("$D^1$", dir(0)--dir(180), blue);
		add(shift(-4,0)*CC());
		unitsize(2cm);
		filldraw(unitcircle, lightblue+opacity(0.2), red);
		label("$D^2$", origin, blue);
		label("$S^1$", dir(45), dir(45), red);
	\end{asy}
\end{center}


\section{Quotient Topology}
\prototype{$D^n / S^{n-1} = S^n$, or the torus.}

Given a space $X$, we can \emph{identify} some of the points together
by any equivalence relation $\sim$;
for an $x \in X$ we denote its equivalence class by $[x]$.
Geometrically, this is the space achieved by welding together points
equivalent under $\sim$.

Formally,
\begin{definition}
	Let $X$ be a topological space, and $\sim$ an equivalence relation
	and the points of $X$.
	Then $X / {\sim}$ is the space whose
	\begin{itemize}
		\ii Points are equivalence classes of $X$, and
		\ii $U \subseteq X / {\sim}$ is open if and only if
		$\left\{ x \in X \text{ such that } [x] \in U  \right\}$
		is open in $X$.
	\end{itemize}
\end{definition}
As far as I can tell, this definition is mostly useless for intuition,
so here are some examples.

\begin{example}[Interval Modulo Endpoints]
	Suppose we take $D^1 = [-1, 1]$
	and quotient by the equivalence relation which identifies
	the endpoints $-1$ and $1$.
	(Formally, $x \sim y \iff (x=y) \text{ or } \{x,y\} = \{-1,1\}$.)
	In that case, we simply recover $S^1$:
	\begin{center}
		\begin{asy}
			size(8cm);
			draw(dir(0)--dir(180), blue);
			dot("$-1$", dir(0), dir(90), red+4);
			dot("$-1$", dir(180), dir(90), red+4);
			label("$D^1$", dir(0)--dir(180), blue);
			add(shift(-4,0)*CC());
			unitsize(2cm);
			draw(unitcircle, blue);
			label("$S^1 \approx D^1 / {\sim}$", dir(45), dir(45), blue);
			dot("$-1 \sim 1$", dir(90), dir(90), red);
		\end{asy}
	\end{center}
	Observe that a small neighborhood around $-1 \sim 1$ in the quotient space
	corresponds to two half-intervals at $-1$ and $1$ in the original space $D^1$.
	This should convince you the definition we gave is the right one.
\end{example}

\begin{example}[More Quotient Spaces]
	Convince yourself each of the following is correct:
	\begin{itemize}
		\ii Generalizing the previous example, $D^n$ modulo its boundary $S^{n-1}$ is $S^n$.
		\ii Given a square $ABCD$, suppose we identify segments $AB$ and $DC$ together.
		Then we get a cylinder. (Think elementary school, when you would tape
		up pieces of paper together to get cylinders.)
		\ii In the previous example, if we also identify $BC$ and $DA$ together,
		then we get a torus. (Imagine taking our cylinder and putting the two
		circles at the end together.)
		\ii Let $X = \RR$, and let $x \sim y$ if $y -x \in \ZZ$.
		Then $X / {\sim}$ is $S^1$ as well.
	\end{itemize}
\end{example}

One special case that we did above:
\begin{definition}
	Let $A \subseteq X$.
	Consider the equivalence relation which identifies
	all the points of $A$ with each other
	while leaving all remaining points inequivalent.
	(In other words, $x \sim y$ if $x=y$ or $x,y \in A$.)
	Then the resulting quotient space is denoted $X/A$.
\end{definition}

So in this notation, \[ D^n / S^{n-1} = S^n. \]

\begin{abuse}
	Note that I'm deliberately being sloppy, and saying
	``$D^n / S^{n-1} = S^n$'' or ``$D^n / S^{n-1}$ \emph{is} $S^n$'',
	when I really ought to say ``$D^n / S^{n-1}$ is homeomorphic to $S^n$''.
	This is a general theme in mathematics:
	objects which are homoeomorphic/isomorphic/etc.\ are generally
	not carefully distinguished from each other.
\end{abuse}

\section{Product Topology}
\prototype{$\RR \times \RR$ is $\RR^2$, $S^1 \times S^1$ is the torus.}

\begin{definition}
	Given topological spaces $X$ and $Y$,
	the product space $X \times Y$ is the space whose
	\begin{itemize}
		\ii Points are pairs $(x,y)$ with $x \in X$, $y \in Y$, and
		\ii Topology is given as follows: the \emph{basis} of
		the topology for $X \times Y$ is $U \times V$,
		for $U \subseteq X$ open and $V \subseteq Y$ open.
	\end{itemize}
\end{definition}
\begin{remark}
	It is not hard to show that, in fact,
	one need only consider basis elements for $U$ and $V$.
	That is to say,
	\[ \left\{ U \times V \mid
		U,V \text{ basis elements for } X,Y \right\} \]
	is also a basis for $X \times Y$.
\end{remark}

This does exactly what you think it would.
\begin{example}[The Unit Square]
	Let $X = [0,1]$ and consider $X \times X$.
	We of course expect this to be the unit square.
	Pictured below is an open set of $X \times X$ in the basis.
	\begin{center}
		\begin{asy}
		size(6cm);
		filldraw(unitsquare, opacity(0.2)+lightblue, black);

		pair B = (0,1);
		pair A = (1,0);
		fill(box(0.3*A+0.2*B,0.6*A+0.7*B), lightred+opacity(0.5));
		label("$U \times V$", (0.45,0.45), brown);

		draw(0.3*A--(0.3*A+B), heavygreen+dashed+1);
		draw(0.6*A--(0.6*A+B), heavygreen+dashed+1);
		draw(0.2*B--(0.2*B+A), heavycyan+dashed+1);
		draw(0.7*B--(0.7*B+A), heavycyan+dashed+1);

		draw( 0.3*A--0.6*A, heavygreen+2 );
		opendot( 0.3*A,  heavygreen+2);
		opendot( 0.6*A, heavygreen+2);
		label("$U$", 0.45*A, dir(-90), heavygreen);
		draw( 0.2*B--0.7*B, heavycyan+2 );
		opendot( 0.2*B, heavycyan+2);
		opendot( 0.7*B, heavycyan+2);
		label("$V$", 0.45*B, dir(180), heavycyan);
		\end{asy}
	\end{center}
\end{example}
\begin{exercise}
	Convince yourself this basis gives the same topology
	as the product metric on $X \times X$.
	So this is the ``right'' definition.
\end{exercise}

\begin{example}[More Product Spaces]
	\listhack
	\begin{enumerate}[(a)]
		\ii $\RR \times \RR$ is the Euclidean plane.
		\ii $S^1 \times [0,1]$ is a cylinder.
		\ii $S^1 \times S^1$ is a torus! (Why?)
	\end{enumerate}
\end{example}

\section{Disjoint Union Wedge Product}
\prototype{$S^1 \wedge S^1$ is the figure eight.}

Figure 8

warning about how not Euclidean

\section{CW Complexes}

\section{The Torus, Klein Bottle, and $\RP^n$}


