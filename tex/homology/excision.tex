\chapter{Excision and Relative Homology}
We have already seen how to use the Mayer-Vietoris sequence:
we started with a sequence
\[ \dots \to H_n(U \cap V) \to H_n(U) \oplus H_n(V) \to H_n(U+V) \to H_{n-1}(U \cap V) \to \dots \]
and its reduced version,
then appealed to the geometric fact that $H_n(U+V) \cong H_n(X)$.
This allowed us to algebraically make computations on $H_n(X)$.

In this chapter, we turn our attention to the long exact
sequence associated to the chain complex
\[ 0 \to C_n(A) \injto C_n(X) \surjto C_n(X,A) \to 0. \]
The setup will look a lot like the previous two chapters,
except in addition to $H_n : \catname{hTop} \to \catname{Grp}$
we will have a functor $H_n : \catname{hPairTop} \to \catname{Grp}$
which takes a pair $(X,A)$ to $H_n(X,A)$.
Then, we state (again without proof) the key geometric result,
and use this to make deductions.

\section{The Long Exact Sequences}
Recall \Cref{thm:long_exact_rel}, which says that the sequences
\[ \dots \to H_n(A) \to H_n(X) \to H_n(X,A) \to H_{n-1}(A) \to \dots. \]
and
\[ \dots \to \wt H_n(A) \to \wt H_n(X) \to H_n(X,A) \to \wt H_{n-1}(A) \to \dots \]
are long exact.
By \Cref{prob:triple_long_exact} we even have a long exact sequence
\[
	\dots
	\to H_n(B,A)
	\to H_n(X,A)
	\to H_n(X,B)
	\to H_{n-1}(B,A)
	\to \dots.
\]
for $A \subset B \subset X$.
An application of the second long exact sequence above gives:
\begin{lemma}
	[Homology Relative to Contractible Spaces]
	\label{lem:rel_contractible}
	Let $X$ be a topological space,
	and let $A \subset X$ be contractible.
	For all $n$, \[ H_n(X, A) \cong \wt H_n(X). \]
\end{lemma}
\begin{proof}
	Since $A$ is contractible, we have $\wt H_n(A) = 0$ for every $n$.
	For each $n$ there's a segment of the long exact sequence given by
	\[ \dots \to \underbrace{\wt H_n(A)}_{=0} \to \wt H_n(X) \to H_n(X,A)
	\to \underbrace{\wt H_{n-1}(A)}_{=0} \to \dots. \]
	So since $0 \to \wt H_n(X) \to H_n(X,A) \to 0$ is exact,
	this means $H_n(X,A) \cong \wt H_n(X)$.
\end{proof}

In particular, the theorem applies if $A$ is a single point.
The case $A = \varnothing$ is also worth noting.
We compile these results in the following lemma.
\begin{lemma}
	[Relative Homology Generalizes Absolute Homology]
	Let $X$ be any space, and $\ast \in X$ a point. Then for all $n$,
	\[
		H_n(X, \{\ast\}) \cong \wt H_n(X)
		\qquad\text{and}\qquad
		H_n(X, \varnothing) = H_n(X).
	\]
\end{lemma}

\section{The Category of Pairs}
Since we now have an $H_n(X,A)$ instead of just $H_n(X)$,
a natural next step is to create a suitable category of \emph{pairs}
and give ourselves the same functorial setup as before.

\begin{definition}
	Let $\varnothing \neq A \subset X$ and $\varnothing \neq B \subset X$
	be subspaces, and consider a map $f : X \to Y$.
	If the image of $A$ is contained within $B$, we write
	\[ f : (X,A) \to (Y,B). \]
	We say $f$ is a \vocab{map of pairs},
	between the pairs $(X,A)$ and $(Y,B)$.
\end{definition}
\begin{definition}
	We say that $f,g : (X,A) \to (Y,B)$ are \vocab{pair-homotopic} if they
	are ``homotopic through maps of pairs''.

	More formally, a \vocab{pair-homotopy}
	$f, g : (X,A) \to (Y,B)$ is a map $F : [0,1] \times X \to Y$,
	which we'll write as $F_t(X)$, such that
	$F$ is a homotopy of the maps $f,g : X \to Y$
	and each $F_t$ is itself a map of pairs.
\end{definition}
Thus, we naturally arrive at two categories:
\begin{itemize}
	\ii $\catname{PairTop}$, the category of \emph{pairs} of
	toplogical spaces, and
	\ii $\catname{hPairTop}$, the same category except
	with maps only equivalent up to homotopy.
\end{itemize}
\begin{definition}
	As before, we say pairs $(X,A)$ and $(Y,B)$ are
	\vocab{pair-homotopy equivalent}
	if they are isomorphic in $\catname{hPairTop}$.
	An isomorphism of $\catname{hPairTop}$ is a
	\vocab{pair-homotopy equivalence}.
\end{definition}

We can do the same song and dance as before
with the prism operator to obtain the following result.
\begin{lemma}[Induced Maps of Relative Homology]
	We have a functor 
	\[ H_n : \catname{hPairTop} \to \catname{Grp}. \]
\end{lemma}
That is, if $f : (X,A) \to (Y,B)$ then we obtain an induced map
\[ f_\ast : H_n(X,A) \to H_n(Y,B). \]
and if two such $f$ and $g$ are pair-homotopic
then $f_\ast = g_\ast$.

Now, we want an analog of contractible spaces for our pairs:
i.e.\ pairs of spaces $(X,A)$ such that $H_n(X,A) = 0$.
The correct definition is:
\begin{definition}
	Let $A \subset X$.
	We say that $A$ is a \vocab{deformation retract} of $X$
	if there is a map of pairs $r : (X, A) \to (A, A)$
	which is a pair homotopy equivalence.
\end{definition}
\begin{example}
	[Examples of Deformation Retracts]
	\listhack
	\begin{enumerate}[(a)]
		\ii If a single point $p$ is a deformation retract of a space $X$,
		then $X$ is contractible, since the retraction $r : X \to \{\ast\}$
		(when viewed as a map $X \to X$)
		is homotopic to the identity map $\id_X : X \to X$.
		\ii The punctured disk $D^2 \setminus \{0\}$
		deformation retracts onto its boundary $S^1$.
		\ii More generally, $D^{n} \setminus \{0\}$
		deformation retracts onto its boundary $S^{n-1}$.
		\ii Similarly, $\RR^n \setminus \{0\}$
		deformation retracts onto a sphere $S^{n-1}$.
	\end{enumerate}
\end{example}
Of course in this situation we have that
\[ H_n(X,A) \cong H_n(X,X) = 0. \]

\begin{exercise}
	Show that if $A \subset V \subset X$,
	and $A$ is a deformation retract of $V$,
	then $H_n(X,A) \cong H_n(X,V)$ for all $n$.
	(Use \Cref{prob:triple_long_exact}. Solution in next section.)
\end{exercise}

\section{Excision}
Now for the key geometric result, which is the analog of
\Cref{thm:open_cover_homology} for our relative homology groups.
\begin{theorem}
	[Excision]
	Let $Z \subset A \subset X$ be subspaces such that
	the closure of $Z$ is contained in the interior of $A$.
	Then the inclusion $\iota (X \setminus Z, A \setminus Z) \injto (X,A)$
	(viewed as a map of pairs) induces an isomorphism of
	relative homology groups
	\[ H_n(X \setminus Z, A \setminus Z) \cong H_n(X,A). \]
\end{theorem}
This means we can \emph{excise} (delete) a subset $Z$ of $A$ in computing
the relative homology groups $H_n(X,A)$.
This should intuitively make sense:
since we are ``modding out by points in $A$'',
the internals of the point $A$ should not matter so much.

The main application of Excision is to decide
when $H_n(X,A) \cong \wt H_n(X/A)$.
Answer:

\begin{theorem}
	[Relative Homology $\implies$ Quotient Space]
	\label{thm:good_pair}
	Let $X$ be a space and $A$ be a subspace such that
	$A$ is a deformation retract of some neighborhood $V \subset X$.
	Then the quotient map $q : X \to X/A$ induces an isomorphism
	\[ H_n(X,A) \cong H_n(X/A, A/A) \cong \wt H_n(X/A). \]
\end{theorem}
\begin{proof}
	By hypothesis, we can consider the following maps of pairs:
	\begin{align*}
		r & : (V,A) \to (A,A)  \\
		q & : (X,A) \to (X/A, A/A) \\
		\widehat q &: (X-A, V-A) \to (X/A-A/A, V/A-A/A).
	\end{align*}
	Moreover, $r$ is a pair-homotopy equivalence.
	Considering the long exact sequence of of a triple
	(which was \Cref{prob:triple_long_exact})
	we have a diagram
	\begin{diagram}
		H_n(V,A) & \rTo & H_n(X,A) & \rTo^f & H_n(X, V) & \rTo & H_{n-1}(V,A) \\
		\dTo_r^\cong && && && \dTo_r^\cong \\
		\underbrace{H_n(A,A)}_{=0} && && && \underbrace{H_{n-1}(A,A)}_{=0}
	\end{diagram}
	where the isomorphisms arise since $r$ is a pair-homotopy equivalence.
	So $f$ is an isomorphism.
	Similarly the map
	\[ g : H_n(X/A, A/A) \to H_n(X/A, V/A) \]
	is an isomorphism.

	Now, consider the commutative diagram
	\begin{diagram}
		H_n(X,A) & \rTo^f & H_n(X,V) & \lTo^{\text{Excise}} & H_n(X-A, V-A) \\
		\dTo^{q_\ast} && && \dTo^{\widehat{q}_\ast}_{\cong} \\
		H_n(X/A,A/A) & \rTo^g & H_n(X/A,V/A)
		& \lTo^{\text{Excise}} & H_n(X/A-A/A, V/A-A/A) \\
	\end{diagram}
	and observe that the rightmost arrow $\widehat{q}_\ast$ is an isomorphism,
	because outside of $A$ the map $\widehat q$ is the identity.
	We know $f$ and $g$ are isomorphisms,
	as are the two arrows marked with ``Excise'' (by Excision).
	From this we conclude that $q_\ast$ is an isomorphism.
	Of course we already know that homology relative to a point
	is just the relative homology groups
	(this is the important case of \Cref{lem:rel_contractible}).
\end{proof}

\section{Some Applications}
One nice application of Excision is to compute $\wt H_n(X \vee Y)$.
\begin{theorem}[Homology of Wedge Sums]
	Let $X$ and $Y$ be spaces with basepoints $x_0 \in X$ and $y_0 \in Y$,
	and assuming each point is a deformation retract of some neighborhood.
	Then for every $n$ we have
	\[
		\wt H_n(X \vee Y)
		= \wt H_n(X) \oplus \wt H_n(Y).
	\]
\end{theorem}
\begin{proof}
	Apply \Cref{thm:good_pair} with the subset $\{x_0, y_0\}$ of $X \amalg Y$,
	\begin{align*}
		\wt H_n (X \vee Y)
		\cong \wt H_n( (X \amalg Y) / \{x_0, y_0\} )
		&\cong H_n(X \amalg Y, \{x_0,y_0\}) \\
		&\cong H_n(X, \{x_0\}) \oplus H_n(Y, \{y_0\}) \\
		&\cong\wt H_n(X) \oplus \wt H_n(Y). \qedhere
	\end{align*}
\end{proof}

Another application is to give a second method
of computing $H_n(S^m)$.
To do this, we will prove that
\[ \wt H_n(S^m) \cong \wt H_{n-1}(S^{m-1}) \]
for any $n,m > 1$.
However, 
\begin{itemize}
	\ii $\wt H_0(S^n)$ is $\ZZ$ for $n=0$ and $0$ otherwise.
	\ii $\wt H_n(S^0)$ is $\ZZ$ for $m=0$ and $0$ otherwise.
\end{itemize}
So by induction on $\min \{m,n\}$ we directly obtain that
\[
	\wt H_n(S^m) \cong
	\begin{cases}
		\ZZ & m=n \\
		0 & \text{otherwise}
	\end{cases}
\]
which is what we wanted.

To prove the claim, let's consider the exact sequence
formed by the pair $X = D^2$ and $A = S^1$.
\begin{example}[The Long Exact Sequence for $(X,A) = (D^2, S^1)$]
	Consider $D^2$ (which is contractible) with boundary $S^1$.
	Clearly $S^1$ is a deformation retraction of $D^2 \setminus \{0\}$,
	and if we fuse all points on the boundary together we get $D^2 / S^1 \cong S^2$.
	So we have a long exact sequence
	\begin{diagram}
		\wt H_2(S^1) & \rTo & \underbrace{\wt H_2(D^2)}_{=0} & \rTo & \wt H_2(S^2) \\
		&&& \ldTo(4,2) & \\
		\wt H_1(S^1) & \rTo & \underbrace{\wt H_1(D^2)}_{=0} & \rTo & \wt H_1(S^2) \\
		&&& \ldTo(4,2) & \\
		\wt H_0(S^1) & \rTo & \underbrace{\wt H_0(D^2)}_{=0} & \rTo & \underbrace{\wt H_0(S^2)}_{=0}
	\end{diagram}
	From this diagram we read that
	\[
		\dots, \quad
		\wt H_3(S^2) = \wt H_2(S^1), \quad
		\wt H_2(S^2) = \wt H_1(S^1), \quad
		\wt H_1(S^2) = \wt H_0(S^1).
	\]
\end{example}
More generally, the exact sequence for the pair $(X,A) = (D^m, S^{m-1})$
shows that $\wt H_n(S^m) \cong \wt H_{n-1}(S^{m-1})$,
which is the desired conclusion.

\section{Invariance of Dimension}
Here is one last example of an application of Excision.
\begin{definition}
	Let $X$ be a space and $p \in X$ a point.
	The $k$t \vocab{local homology group} of $p$ at $X$ is defined as
	\[ H_k(X, X \setminus \{p\}). \]
\end{definition}
Note that for any neighborhood $U$ of $p$, we have by Excision that
\[ H_k(X, X \setminus \{p\}) \cong H_k(U, U \setminus \{p\}). \]
Thus this local homology group only depends on the space near $p$.

\begin{theorem}
	[Invaniance of Dimension, Brouwer 1910]
	Let $U \subseteq \RR^n$ and $V \subseteq \RR^m$ be nonempty open sets.
	If $U$ and $V$ are homeomorphic, then $m = n$.
\end{theorem}
\begin{proof}
	Consider a point $x \in U$ and its local homology groups. By Excision,
	\[ H_k(\RR^n, \RR^n \setminus \{x\}) \cong
		H_k(U, U \setminus \{x\}). \]
	But since $\RR^n \setminus \{x\}$ is homotopic to $S^{n-1}$,
	the long exact sequence of \Cref{thm:long_exact_rel} tells us
	that 
	\[
		H_k(\RR^n, \RR^n \setminus \{x\}) 
		\cong
		\begin{cases}
			\ZZ & k = n \\
			0 & \text{otherwise}.
		\end{cases}
	\]
	Analogously, given $y \in V$ we have
	\[ H_k(\RR^m, \RR^m \setminus\{y\}) \cong H_k(V, V\setminus\{y\}). \]
	If $U \cong V$, we thus
	deduce that
	\[ H_k(\RR^n, \RR^n\setminus\{x\}) \cong H_k(\RR^m, \RR^m\setminus\{y\}) \]
	for all $k$.  This of course can only happen if $m=n$.
\end{proof}

\section\problemhead
\begin{problem}
	Let $X = S^1 \times S^1$ and $Y = S^1 \vee S^1 \vee S^2$.
	Show that \[ H_n(X) \cong H_n(Y) \] for every integer $n$.
\end{problem}

\begin{problem}[Hatcher \S2.1 Exercise 18]
	Consider $\QQ \subset \RR$.
	Compute $\wt H_1(\RR, \QQ)$.
	\begin{hint}
		Use \Cref{thm:long_exact_rel}.
	\end{hint}
	\begin{sol}
		We have an exact sequence
		\[
			\underbrace{\wt H_1(\RR)}_{=0}
			\to \wt H_1(\RR, \QQ) \to \wt H_0(\QQ) \to 
			\underbrace{\wt H_0(\RR)}_{=0}.
		\]
		Now, since $\QQ$ is path-disconnected
		(i.e.\ no two of its points are path-connected)
		it follows that $\wt H_0(\QQ)$ consists of
		countably infinitely many copies of $\ZZ$.
	\end{sol}
\end{problem}

\begin{sproblem}
	What are the local homology groups of a topological $n$-manifold?
\end{sproblem}

\begin{problem}
	Let \[ X = \{(x,y) \mid x \ge 0\} \subseteq \RR^2 \]
	denote the half-plane.
	What are the local homology groups of points in $X$?
	% http://math.stackexchange.com/questions/350667/local-homology-group-a-homeomorphism-takes-the-boundary-to-the-boundary
\end{problem}
