\chapter{The Cup Product (In Progress)}
I won't do much more in this chapter other than
introduce the cup product and say a few examples.
For an actual treatise, see \cite{ref:hatcher} or \cite{ref:maxim752}.

\section{Graded Rings}
\prototype{Polynomial rings are graded rings.}
In this chapter, the ring $R$ is commutative with identity;
but the remaining rings need not be commutative.
However, the word ``ring'' refers to a ring with identity,
or else we will use \vocab{pseudo-ring} to denote a ring that need not possess $1$.

\todo{define graded ring}

\todo{graded commutative}

\section{Cup Products}
Suppose now we replace our abelian group $G$ by a commutative ring $R$ with identity.
Then this gives us a way to multiply two cochains, as follows.
Suppose $\phi \in C^k(X;R)$ and $\psi \in C^\ell(X;R)$.
Then we can define their \vocab{cup product}
$\phi\smile\psi \in C^{k+\ell}(X;R)$ to be
\[
	(\phi\smile\psi)([v_0, \dots, v_{k+\ell}])
	= \phi\left( [v_0, \dots, v_k] \right)
	\cdot \psi\left( [v_{k+1}, \dots, v_{k+\ell}] \right)
\]
where the multiplication is in $R$.

\begin{ques}
	Which $0$-cochain is the identity for $\smile$?
\end{ques}

This behaves well with respect to $\delta$:
\begin{lemma}[$\delta$ with Cup Products]
	We have
	$\delta(\phi\smile\psi) = \delta\phi\smile\psi + (-1)^k\phi\smile\delta\psi$.
\end{lemma}
\begin{proof}
	Direct $\sum$ computations.
\end{proof}
From this, it is not hard to see that we get an induced map
\[ H^k(X;R) \times H^\ell(X;R) \taking\smile H^{k+\ell}(X;R).  \]
We then define the \vocab{cohomology ring} $H^\bullet(X;R)$
whose elements are finite sums in 
\[ \bigoplus_{k \ge 0} H^k(X;R) \]
and with multiplication given by $\smile$.

Observe that $H^\bullet(X;R)$ is a \emph{graded ring}.
We say that an element $\alpha \in H^k(X;R)$ has dimension $k$,
and write $|\alpha|=k$.
It turns out (though we won't prove it; see \cite{ref:hatcher})
that the cup product is in fact graded commutative.

\begin{example}[Cohomology Ring of $S^n$]
	Consider $S^n$ for $n \ge 1$.
	The nontrivial cohomology groups are given by
	$H^0(S^n; \ZZ) \cong H^n(S^n, \ZZ) \cong \ZZ$.
	So as an abelian group
	\[ H^\bullet(S^n, \ZZ) \cong \ZZ \oplus \alpha \ZZ \]
	where $\alpha$ is the generator of $H^n(S^n, \ZZ)$.
	
	Now, observe that $|\alpha\smile\alpha| = 2n$, but
	since $H^{2n}(S^n, \ZZ) = 0$ we must have $\alpha\smile\alpha=0$.
	So even more succinctly,
	\[ H^\bullet(S^n, \ZZ) \cong \ZZ[\alpha]/(\alpha^2). \]
\end{example}

\begin{example}[Real and Complex Projective Space]
	Though we won't prove it, it turns out that
	\begin{align*}
		H^\bullet(\RP^n, \Zc2) \cong \Zc2[\alpha]/(\alpha^{n+1}) \\
		H^\bullet(\CP^n, \ZZ) \cong \ZZ[\beta]/(\beta^{n+1})
	\end{align*}
	where $|\alpha|=1$ is a generator of $H^1(\RP^n; \Zc2)$
	and $|\beta|=2$ is a generator of $H^2(\CP^n; \ZZ)$.
\end{example}

\section{Cup Product Structure}
In general, it's not so easy to figure out what the cup product looks like
for an arbitrary space.


\section{Kunneth Formula}
\section{The Borsuk-Ulam Theorem}

\section\problemhead

