\chapter{Application of cohomology}
In this final chapter on topology, I'll state (mostly without proof)
some nice properties of cohomology groups, and in particular
introduce the so-called cup product.
For an actual treatise on the cup product,
see \cite{ref:hatcher} or \cite{ref:maxim752}.

\section{Poincar\'e duality}
First cool result:
you may have noticed symmetry in the (co)homology groups of
``nice'' spaces like the torus or $S^n$.
In fact this is predicted by:
\begin{theorem}
	[Poincar\'e duality]
	If $M$ is a smooth oriented compact $n$-manifold,
	then we have a natural isomorphism
	\[ H^k(M; \ZZ) \cong H_{n-k}(M) \]
	for every $k$.
	In particular, $H^k(M) = 0$ for $k > n$.
\end{theorem}
So for smooth oriented compact manifolds,
cohomology and homology groups are not so different.

From this follows the symmetry that we mentioned
when we first defined the Betti numbers:
\begin{corollary}
	[Symmetry of Betti numbers]
	Let $M$ be a smooth oriented compact $n$-manifold,
	and let $b_k$ denote its Betti number.
	Then \[ b_k = b_{n-k}. \]
\end{corollary}
\begin{proof}
	\Cref{prob:betti}.
\end{proof}


\section{de Rham cohomology}
We now reveal the connection between
differential forms and singular cohomology.

Let $M$ be a smooth manifold.
We are interested in the homology and cohomology groups of $M$.
We specialize to the case $G = \RR$, the additive group of real numbers.
\begin{ques}
	Check that $\Ext(H, \RR) = 0$ for any finitely generated abelian group $H$.
\end{ques}
Thus, with real coefficients the universal coefficient theorem says that
\[ H^k(M; \RR) \cong \Hom(H_k(M), \RR) = \left( H_k(M) \right)^\vee \]
where we view $H_k(X)$ as a real vector space.
So, we'd like to get a handle on either $H_k(M$) or $H^k(M; \RR)$.

Consider the cochain complex
\[
	0 \to \Omega^0(M)
	\taking d \Omega^1(M)
	\taking d \Omega^2(M)
	\taking d \Omega^3(M)
	\taking d \dots
\]
and let $\HdR^k(M)$ denote its cohomology groups.
Thus the de Rham cohomology is the closed forms modulo the exact forms.
\[
	\text{Cochain} : \text{Cocycle} : \text{Coboundary}
	= \text{$k$-form} : \text{Closed form} : \text{Exact form}. 
\]

The whole punch line is:
\begin{theorem}
	[de Rham's theorem]
	For any smooth manifold $M$, we have a natural isomorphism
	\[ H^k(M; \RR) \cong \HdR^k(M). \]
\end{theorem}
So the theorem is that the real cohomology groups of manifolds $M$
are actually just given by the behavior of differential forms.
Thus, 
\begin{moral}
	One can metaphorically think of elements of cohomology groups
	as $G$-valued differential forms on the space.
\end{moral}

Why does this happen?
In fact, we observed already behavior of differential
forms which reflects holes in the space.
For example, let $M = S^1$ be a circle
and consider the \textbf{angle form} $\alpha$
(see \Cref{ex:angle_form}).
The from $\alpha$ is closed, but not exact,
because it is possible to run a full circle around $S^1$.
So the failure of $\alpha$ to be exact is signaling
that $H_1(S^1) \cong \ZZ$.

\section{Graded rings}
\prototype{Polynomial rings are commutative graded rings,
while $\Lambda^\bullet(V)$ is anticommutative.}
In the de Rham cohomology, the differential forms can interact in another way:
given a $k$-form $\alpha$ and an $\ell$-form $\beta$, we can consider
a $(k+\ell)$-form
\[ \alpha \wedge \beta. \]
So we can equip the set of forms with a ``product'', satisfying
$\beta \wedge \alpha = (-1)^{k\ell} \alpha \wedge \beta$
This is a special case of a more general structure:

\begin{definition}
	A \vocab{graded pseudo-ring} $R$ is an abelian group
	\[ R = \bigoplus_{d \ge 0} R^d \]
	where $R^0$, $R^1$, \dots, are abelian groups,
	with an additional associative binary operation $\times : R \to R$.
	We require that if $r \in R^d$ and $s \in R^e$, we have $rs \in R^{d+e}$.
	Elements of an $R^d$ are called \vocab{homogeneous elements};
	if $r \in R^d$ and $r \neq 0$, we write $|r| = d$.
\end{definition}
Note that we do \emph{not} assume commutativity.
In fact, these ``rings'' may not even have an identity $1$.
We use other words if there are additional properties:
\begin{definition}
	A \vocab{graded ring} is a graded pseudo-ring with $1$.
	If it is commutative we say it is a \vocab{commutative graded ring}.
\end{definition}
\begin{definition}
	A graded (pseudo-)ring $R$ is \vocab{anticommutative} if
	for any homogeneous $r$ and $s$ we have
	\[ rs = (-1)^{|r| |s|} sr. \]
\end{definition}

To summarize:
\begin{center}
	\small
	\begin{tabular}[h]{|c|cc|}
		\hline
		\textbf{Flavors of graded rings} &
		Need not have $1$ & Must have a $1$ \\ \hline
		No Assumption & graded pseudo-ring & graded ring \\
		Anticommutative & anticommutative pseudo-ring & anticommutative ring \\
		Commutative &  & commutative graded ring \\ \hline
	\end{tabular}
\end{center}

\begin{example}[Examples of graded rings]
	\listhack
	\begin{enumerate}[(a)]
		\ii The ring $R = \ZZ[x]$ is a \textbf{commutative graded ring},
		with the $d$th component being the multiples of $x^d$.
		\ii The ring $R = \ZZ[x,y,z]$ is a \textbf{commutative graded ring},
		with the $d$th component being the abelian group
		of homogeneous degree $d$ polynomials (and $0$).
		\ii Let $V$ be a vector space, and consider
		the abelian group
		\[ \Lambda^\bullet(V) = \bigoplus_{d \ge 0} \Lambda^d(V). \]
		For example, $e_1 + (e_2 \wedge e_3) \in \Lambda^\bullet(V)$, say.
		We endow $\Lambda^\bullet(V)$ with the product $\wedge$,
		which makes it into an \textbf{anticommutative ring}.
		\ii Consider the set of differential forms of a manifold $M$,
		say \[ \Omega^\bullet(M) = \bigoplus_{d \ge 0} \Omega^d(M) \]
		endowed with the product $\wedge$.
		This is an \textbf{anticommutative ring}.
	\end{enumerate}
	All four examples have a multiplicative identity.
\end{example}

Let's return to the situation of $\Omega^\bullet(M)$.
Consider again the de Rham cohomology groups $\HdR^k(M)$,
whose elements are closed forms modulo exact forms.
We claim that:
\begin{lemma}
	[Wedge product respects de Rham cohomology]
	The wedge product induces a map
	\[ \wedge : \HdR^k(M) \times \HdR^\ell(M) \to \HdR^{k+\ell}(M). \]
\end{lemma}
\begin{proof}
	First, we recall that the operator $d$ satisfies
	\[
		d(\alpha \wedge \beta)
		= (d\alpha) \wedge \beta + \alpha \wedge (d\beta).
	\]
	Now suppose $\alpha$ and $\beta$ are closed forms.
	Then from the above, $\alpha \wedge \beta$ is clearly closed.
	Also if $\alpha$ is closed and $\beta = d\omega$ is exact,
	then $\alpha \wedge \beta$ is exact, from the identity
	\[ d(\alpha \wedge \omega)
		= d\alpha \wedge\omega + \alpha \wedge d\omega = \alpha \wedge \beta. \]
	Similarly if $\alpha$ is exact and $\beta$ is closed
	then $\alpha \wedge \beta$ is exact.
	Thus it makes sense to take the product modulo exact forms,
	giving the theorem above.
\end{proof}

Therefore, we can obtain a \emph{anticommutative ring}
\[ \HdR^\bullet(M) = \bigoplus_{k \ge 0} \HdR^k(M) \]
with $\wedge$ as a product, and $1 \in \Lambda^0(\RR) = \RR$ as the identity

\section{Cup products}
Inspired by this, we want to see if we can construct a similar product
on $\bigoplus_{k \ge 0} H^k(X; R)$ for any topological space $X$ and ring $R$
(where $R$ is commutative with $1$ as always).
The way to do this is via the \emph{cup product}.

Then this gives us a way to multiply two cochains, as follows.
\begin{definition}
	Suppose $\phi \in C^k(X;R)$ and $\psi \in C^\ell(X;R)$.
	Then we can define their \vocab{cup product}
	$\phi\smile\psi \in C^{k+\ell}(X;R)$ to be
	\[
		(\phi\smile\psi)([v_0, \dots, v_{k+\ell}])
		= 
		\phi\left( [v_0, \dots, v_k] \right)
		\cdot
		\psi\left( [v_k, \dots, v_{k+\ell}] \right)
	\]
	where the multiplication is in $R$.
\end{definition}

\begin{ques}
	Assuming $R$ has a $1$, which $0$-cochain is the identity for $\smile$?
\end{ques}

First, we prove an analogous result as before:
\begin{lemma}[$\delta$ with cup products]
	We have
	$\delta(\phi\smile\psi) = \delta\phi\smile\psi
	+ (-1)^k\phi\smile\delta\psi$.
\end{lemma}
\begin{proof}
	Direct $\sum$ computations.
\end{proof}
Thus, by the same routine we used for de Rham cohomology, we get
an induced map
\[ \smile : H^k(X;R) \times H^\ell(X;R) \to H^{k+\ell}(X;R).  \]
We then define the \vocab{singular cohomology ring}
whose elements are finite sums in 
\[ H^\bullet(X;R) = \bigoplus_{k \ge 0} H^k(X;R) \]
and with multiplication given by $\smile$.
Thus it is a graded ring (with $1_R \in R$ the identity)
and is in fact anticommutative:
\begin{proposition}[Cohomology is anticommutative]
	$H^\bullet(X; R)$ is an anticommutative ring,
	meaning $\phi \smile \psi = (-1)^{k\ell} \psi \smile \phi$.
\end{proposition}
For a proof, see \cite[Theorem 3.11, pages 210-212]{ref:hatcher}.
Moreover, we have the de Rham isomorphism
\begin{theorem}
	[de Rham extends to ring isomorphism]
	For any smooth manifold $M$, the isomorphism
	of de Rham cohomology groups to singular cohomology
	groups in facts gives an isomorphism
	\[ H^\bullet(M; \RR) \cong \HdR^\bullet(M) \]
	of anticommutative rings.
\end{theorem}

Therefore, if ``differential forms'' are the way to visualize
the elements of a cohomology group, the wedge product is the
correct way to visualize the cup product.

We now present (mostly without proof)
the cohomology rings of some common spaces.

\begin{example}
	[Cohomology of torus]
	The cohomology ring $H^\bullet(S^1 \times S^1; \ZZ)$
	of the torus is generated by elements $|\alpha| = |\beta| = 1$
	which satisfy the relations
	$\alpha \smile \alpha = \beta \smile \beta = 0$,
	and $\alpha \smile \beta = -\beta \smile \alpha$.
	(It also includes an identity $1$.)
	Thus as a $\ZZ$-module it is
	\[ H^\bullet(S^1 \times S^1; \ZZ)
		\cong \ZZ \oplus \left[ \alpha \ZZ \oplus \beta \ZZ \right]
		\oplus (\alpha \smile \beta) \ZZ. \]
	This gives the expected dimensions $1+2+1=4$.
	It is anti-commutative.
\end{example}

\begin{example}[Cohomology ring of $S^n$]
	Consider $S^n$ for $n \ge 1$.
	The nontrivial cohomology groups are given by
	$H^0(S^n; \ZZ) \cong H^n(S^n; \ZZ) \cong \ZZ$.
	So as an abelian group
	\[ H^\bullet(S^n; \ZZ) \cong \ZZ \oplus \alpha \ZZ \]
	where $\alpha$ is the generator of $H^n(S^n, \ZZ)$.
	
	Now, observe that $|\alpha\smile\alpha| = 2n$, but
	since $H^{2n}(S^n; \ZZ) = 0$ we must have $\alpha\smile\alpha=0$.
	So even more succinctly,
	\[ H^\bullet(S^n; \ZZ) \cong \ZZ[\alpha]/(\alpha^2). \]
	Confusingly enough, this graded ring is both
	commutative \emph{and} anti-commutative.
	The reason is that $\alpha \smile \alpha = 0 = -(\alpha \smile \alpha)$.
\end{example}

\begin{example}[Cohmology ring of real and complex projective space]
	It turns out that
	\begin{align*}
		H^\bullet(\RP^n; \Zc2) &\cong \Zc2[\alpha]/(\alpha^{n+1}) \\
		H^\bullet(\CP^n; \ZZ) &\cong \ZZ[\beta]/(\beta^{n+1})
	\end{align*}
	where $|\alpha| = 1$ is a generator of $H^1(\RP^n; \Zc2)$
	and $|\beta| = 2$ is a generator of $H^2(\CP^n; \ZZ)$.

	Confusingly enough, both graded rings are commutative \emph{and} anti-commutative.
	In the first case it is because we work in $\Zc 2$, for which $1 = -1$,
	so anticommutative is actually equivalent to commutative.
	In the second case, all nonzero homogeneous elements have degree $2$.
\end{example}


\section{Relative cohomology pseudo-rings}
For $A \subseteq X$, one can also define a relative cup product
\[ H^k(X,A;R) \times H^\ell(X,A;R) \to H^{k+\ell}(X,A;R). \]
After all, if either cochain vanishes on chains in $A$,
then so does their cup product.
This lets us define \vocab{relative cohomology pseudo-ring}
and \vocab{reduced cohomology pseudo-ring} (by $A = \{\ast\}$), say
\begin{align*}
H^\bullet(X,A;R) &= \bigoplus_{k \ge 0} H^k(X,A; R) \\
\wt H^\bullet(X;R) &= \bigoplus_{k \ge 0} \wt H^k(X;R).
\end{align*}
These are both \textbf{anticommutative pseudo-rings}.
Indeed, often we have $\wt H^0(X;R) = 0$ and thus there is no identity at all.

Once again we have functoriality:
\begin{theorem}
	[Cohomology (pseudo-)rings are functorial]
	Fix a ring $R$ (commutative with $1$).
	Then we have functors
	\begin{align*}
		H^\bullet(-; R) &: \catname{hTop}\op \to \catname{GradedRings} \\
		H^\bullet(-,-; R) &: \catname{hPairTop}\op \to \catname{GradedPseudoRings}.
	\end{align*}
\end{theorem}

Unfortunately, unlike with (co)homology groups,
it is a nontrivial task to determine the cup product
for even nice spaces like CW complexes.
So we will not do much in the way of computation.
However, there is a little progress we can make.

\section{Wedge sums}
Our goal is to now compute $\wt H^\bullet(X \wedge Y)$.
To do this, we need to define the product of two graded pseudo-rings:
\begin{definition}
	Let $R$ and $S$ be two graded pseudo-rings.
	The \vocab{product pseudo-ring} $R \times S$ is the graded pseudo-ring
	defined by taking the underlying abelian group as 
	\[ R \oplus S = \bigoplus_{d \ge 0} (R^d \oplus S^d). \]
	Multiplication comes from $R$ and $S$, followed by
	declaring $r \cdot s = 0$ for $r \in R$, $s \in S$.
\end{definition}
Note that this is just graded version of the product ring
defined in \Cref{ex:product_ring}.
\begin{exercise}
	Show that if $R$ and $S$ are graded rings (meaning they have $1_R$ and $1_S$),
	then so is $R \times S$.
\end{exercise}

Now, the theorem is that:
\begin{theorem}
	[Cohomology pseudo-rings of wedge sums]
	We have
	\[
		\wt H^\bullet(X \wedge Y; R)
		\cong \wt H^\bullet(X;R)
		\times \wt H^\bullet(Y;R)
	\]
	as graded pseudo-rings.
\end{theorem}

This allows us to resolve the first question posed at the beginning.
Let $X = \CP^2$ and $Y = S^2 \vee S^4$.
We have that
\[ H^\bullet(\CP^2; \ZZ) \cong \ZZ[\alpha] / (\alpha^3). \]
Hence this is a graded ring generated by there elements:
\begin{itemize}
	\ii $1$, in dimension $0$.
	\ii $\alpha$, in dimension $2$.
	\ii $\alpha^2$, in dimension $4$.
\end{itemize}
Next, consider the reduced cohomology pseudo-ring
\[ \wt H^\bullet(S^2 \vee S^4; \ZZ) \cong
	\wt H^\bullet(S^2; \ZZ)
	\oplus \wt H^\bullet(S^4 ; \ZZ).
\]
Thus the absolute cohomology ring $H^\bullet(S^2 \vee S^4 \; \ZZ)$
is a graded ring also generated by three elements.
\begin{itemize}
	\ii $1$, in dimension $0$ (once we add back in the $0$th dimension).
	\ii $a_2$, in dimension $2$ (from $H^\bullet(S^2 ; \ZZ)$).
	\ii $a_4$, in dimension $4$ (from $H^\bullet(S^4 ; \ZZ)$).
\end{itemize}
Each graded component is isomorphic, like we expected.
However, in the former, the product of two degree $2$ generators is
\[ \alpha \cdot \alpha = \alpha^2. \]
In the latter, the product of two degree $2$ generators is
\[ a_2 \cdot a_2 = a_2^2 = 0 \]
since $a_2 \smile a_2 = 0 \in H^\bullet(S^2; \ZZ)$.

Thus $S^2 \vee S^4$ and $\CP^2$ are not homotopy equivalent.

\section{K\"unneth formula}
We now wish to tell apart the spaces $S^2 \times S^4$ and $\CP^3$.
In order to do this, we will need a formula
for $H^n(X \times Y; R)$ in terms of $H^n(X;R)$ and $H^n(Y;R)$.
Thus formulas are called \vocab{K\"unneth formulas}.
In this section we will only use a very special case,
which involves the tensor product of two graded rings.

\begin{definition}
	Let $A$ and $B$ be two graded rings which are also $R$-modules
	(where $R$ is a commutative ring with $1$).
	We define the \vocab{tensor product} $A \otimes_R B$ as follows.
	As an abelian group, it is 
	\[ A \otimes_R B = \bigoplus_{d \ge 0}
		\left( \bigoplus_{k=0}^{d} A^k \otimes_R B^{d-k}  \right). \]
	The multiplication is given on basis elements by
	\[ \left( a_1 \otimes b_1 \right)\left( a_2 \otimes b_2 \right)
		= (a_1a_2) \otimes (b_1b_2).
	\]
	Of course the multiplicative identity is $1_A \otimes 1_B$.
\end{definition}

Now let $X$ and $Y$ be topological spaces, and take the product:
we have a diagram
\begin{diagram}
	&& X \times Y && \\
	X &\ldTo(2,1)^{\pi_X} && \rdTo(2,1)^{\pi_Y} & Y \\
\end{diagram}
where $\pi_X$ and $\pi_Y$ are projections.
As $H^k(-; R)$ is functorial, this gives induced maps
\begin{align*}
	\pi_X^\ast &: H^k(X \times Y; R) \to H^k(X; R) \\
	\pi_Y^\ast &: H^k(X \times Y; R) \to H^k(Y; R)
\end{align*}
for every $k$.

By using this, we can define a so-called cross product.
\begin{definition}
	Let $R$ be a ring, and $X$ and $Y$ spaces.
	Let $\pi_X$ and $\pi_Y$ be the projections of $X \times Y$
	onto $X$ and $Y$.
	Then the \vocab{cross product} is the map
	\[
		H^\bullet(X; R) \otimes_R H^\bullet(Y;R)
		\taking{\times} H^\bullet(X \times Y; R)
	\]
	acting on cocycles as follows:
	$\phi \times \psi = \pi_X^\ast(\phi) \smile \pi_Y^\ast(\phi)$.
\end{definition}

This is just the most natural way to take a $k$-cycle
on $X$ and an $\ell$-cycle on $Y$, and create a $(k+\ell)$-cycle
on the product space $X \times Y$.


\begin{theorem}
	[K\"unneth formula]
	Let $X$ and $Y$ be CW complexes such that $H^k(Y;R)$
	is a finitely generated free $R$-module for every $k$.
	Then the cross product is an isomorphism of anticommutative rings
	\[
		H^\bullet(X;R) \otimes_R H^\bullet(Y;R)
		\to H^\bullet(X \times Y; R). 
	\]
\end{theorem}

In any case, this finally lets us resolve the question
set out at the beginning.
We saw that $H_n(\CP^3) \cong H_n(S^2 \times S^4)$ for every $n$,
and thus it follows that $H^n(\CP^3; \ZZ) \cong H^n(S^2 \times S^4; \ZZ)$ too.

But now let us look at the cohomology rings. First, we have
\[ H^\bullet(\CP^3; \RR) \cong \ZZ[\alpha] / (\alpha^3)
	\cong \ZZ \oplus \alpha\ZZ \oplus \alpha^2\ZZ \oplus \alpha^3\ZZ
\] 
where $|\alpha| = 2$; hence this is a graded ring generated by
\begin{itemize}
	\ii $1$, in degree $0$.
	\ii $\alpha$, in degree $2$.
	\ii $\alpha^2$, in degree $4$.
	\ii $\alpha^3$, in degree $6$.
\end{itemize}

Now let's analyze
\[ H^\bullet(S^2 \times S^4; \RR) \cong 
	\ZZ[\beta] / (\beta^2)
	\otimes
	\ZZ[\gamma] / (\gamma^2).
\]
It is thus generated thus by the following elements:
\begin{itemize}
	\ii $1 \otimes 1$, in degree $0$.
	\ii $\beta \otimes 1$, in degree $2$.
	\ii $1 \otimes \gamma$, in degree $4$.
	\ii $\beta \otimes \gamma$, in degree $6$.
\end{itemize}
Again in each dimension we have the same abelian group.
But notice that if we square $\beta \otimes 1$ we get
\[ (\beta \otimes 1)(\beta \otimes 1) = \beta^2 \otimes 1 = 0. \]
Yet the degree $2$ generator of $H^\bullet(\CP^3; \ZZ)$
does not have this property.
Hence these two graded rings are not isomorphic.

So it follows that $\CP^3$ and $S^2 \times S^4$ are not homotopy equivalent.

	
% Borsuk Ulam

\section\problemhead

\begin{dproblem}
	[Symmetry of Betti numbers by Poincar\'e duality]
	\label{prob:betti}
	Let $M$ be a smooth oriented compact $n$-manifold,
	and let $b_k$ denote its Betti number.
	Prove that $b_k = b_{n-k}$.
	\begin{hint}
		Write $H^k(M; \ZZ)$ in terms of $H_k(M)$
		using the UCT, and analyze the ranks.
	\end{hint}
\end{dproblem}

\begin{problem}
	Show that $\RP^n$ is not orientable for even $n$.
	\begin{hint}
		Use the previous result on Betti numbers.
	\end{hint}
\end{problem}

\begin{problem}
	Show that $\RP^3$ is not homotopy equivalent to $\RP^2 \wedge S^3$.
	\begin{hint}
		Use the $\Zc2$ cohomologies, and find the cup product.
	\end{hint}
\end{problem}

\begin{problem}
	\gim
	Show that $S^m \wedge S^n$ is not a deformation retract
	of $S^m \times S^n$ for any $m,n \ge 1$.
	\begin{hint}
		Assume that $r : S^m \times S^n \to S^m \wedge S^n$ is such a map.
		Show that the induced map
		$H^\bullet(S^m \wedge S^n; \ZZ) \to H^\bullet(S^m \times S^n; \ZZ)$
		between their cohomology rings is monic
		(since there exists an inverse map $i$).
	\end{hint}
	\begin{sol}
		See \cite[Example 3.3.14, pages 68-69]{ref:maxim752}.
	\end{sol}
\end{problem}
