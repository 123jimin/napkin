\chapter{Cardinals}
An ordinal measures a total ordering.
However, it does not do a fantastic job at measuring size.
For example, there is a bijection between the elements of $\omega$ and $\omega+1$:
\[
	\begin{array}{rccccccc}
		\omega+1 = & \{ & \omega & 0 & 1 & 2 & \dots & \} \\
		\omega = & \{ & 0 & 1 & 2 & 3 & \dots & \}.
	\end{array}
\]
In fact, as you likely already know,
there is even a bijection between $\omega$ and $\omega^2$:
\[
	\begin{array}{l|cccccc}
		+ & 0 & 1 & 2 & 3 & 4 & \dots \\ \hline
		0 & 0 & 1 & 3 & 6 & 10 & \dots \\
		\omega & 2 & 4 & 7 & 11 & \dots & \\
		\omega \cdot 2 & 5 & 8 & 12 & \dots & & \\
		\omega \cdot 3 & 9 & 13 & \dots & & & \\
		\omega \cdot 4 & 14 & \dots & & & &
	\end{array}
\]
So ordinals do not do a good job of keeping track of size.
For this, we turn to the notion of a cardinal number.

\section{Equinumerous Sets and Cardinals}
\begin{definition}
	Two sets $A$ and $B$ are \vocab{equinumerous}, written $A \approx B$,
	if there is a bijection between them.
\end{definition}

\begin{definition}
	A \vocab{cardinal} is an ordinal $\kappa$ such that
	for no $\alpha < \kappa$ do we have $\alpha \approx \kappa$.
\end{definition}
\begin{example}[Examples of Cardinals]
	Every finite number is a cardinal.
	Moreover, $\omega$ is a cardinal.
	However, $\omega+1$, $\omega^2$, $\omega^{2015}$ are not,
	because they are countable.
\end{example}
\begin{example}[$\omega^\omega$ is Countable]
	Even $\omega^\omega$ is not a cardinal,
	since it is a countable union
	\[ \omega^\omega = \bigcup_n \omega^n \]
	and each $\omega^n$ is countable.
\end{example}
\begin{ques}
	Why must an infinite cardinal be a limit ordinal?
\end{ques}

\begin{remark}
	There is something fishy about the definition of a cardinal:
	it relies on an \emph{external} function $f$.
	That is, to verify $\kappa$ is a cardinal I can't just look at $\kappa$ itself;
	I need to examine the entire universe $V$ to make sure
	there does not exist a bijection $f : \kappa \to \alpha$ for $\alpha < \kappa$.
	For now this is no issue, but later in model theory
	this will lead to some highly counterintuitive behavior.
\end{remark}

\section{Cardinalities}
Now that we have defined a cardinal, we can discuss the size
of a set by linking it to a cardinal.

\begin{definition}
	The \vocab{cardinality} of a set $X$
	is the \emph{least} ordinal $\kappa$ such that $X \approx \kappa$.
	We denote it by $\left\lvert X \right\rvert$.
\end{definition}
\begin{ques}
	Why must $\left\lvert X \right\rvert$ be a cardinal?
\end{ques}
\begin{remark}
	One needs the Well-Ordering Theorem (equivalently, Choice)
	in order to establish that such an ordinal $\kappa$ actually exists.
\end{remark}
Since cardinals are ordinals, it makes sense to ask whether $\kappa_1 \le \kappa_2$,
and so on.
Our usual intuition works well here.
\begin{proposition}[Restatement of Cardinality Properties]
	Let $X$ and $Y$ be sets.
	\begin{enumerate}[(i)]
		\ii $X \approx Y$ if and only $\left\lvert X \right\rvert = \left\lvert Y \right\rvert$,
		if and only if there is a bijection between $X$ and $Y$.
		\ii $\left\lvert X \right\rvert \le \left\lvert Y \right\rvert$
		if and only if there is an injective map $X \injto Y$.
	\end{enumerate}
\end{proposition}
\begin{ques}
	Prove this.
\end{ques}

\section{Aleph Numbers}
\prototype{$\aleph_0$ is $\omega$, and $\aleph_1$ is the first uncountable}
First, let us check that cardinals can get arbitrarily large:
\begin{proposition}
	We have $\left\lvert X \right\rvert < \left\lvert \PP(X) \right\rvert$ for every set $X$.
\end{proposition}
\begin{proof}
	There is an injective map $X \injto \PP(X)$
	but there is no injective map $\PP(X) \injto X$ by \Cref{lem:cantor_diag}.
\end{proof}

Thus we can define:
\begin{definition}
	For a cardinal $\kappa$, we define $\kappa^+$ to be the least cardinal above $\kappa$,
	called the \vocab{successor cardinal}.
\end{definition}
This $\kappa^+$ exists and has $\kappa^+ \le \left\lvert \PP(\kappa) \right\rvert$.

Next, we claim that:
\begin{exercise}
	Show that if $A$ is a set of cardinals, then $\cup A$ is a cardinal.
\end{exercise}

Thus by transfinite induction we obtain that:
\begin{definition}
	For any $\alpha \in \On$, we define the \vocab{aleph numbers} as
	\begin{align*}
		\aleph_0 &= \omega \\
		\aleph_{\alpha+1} &= \left( \aleph_\alpha \right)^+ \\
		\aleph_{\lambda} &= \bigcup_{\alpha < \lambda} \aleph_\alpha.
	\end{align*}
\end{definition}

Thus we have the following sequence of cardinals:
\[
	0 < 1 < 2 < \dots < \aleph_0 < \aleph_1 < \dots < \aleph_\omega < \aleph_{\omega+1} < \dots
\]
By definition, $\aleph_0$ is the cardinality of the natural numbers,
$\aleph_1$ is the first uncountable ordinal, \dots.

We claim the aleph numbers constitute all the cardinals:
\begin{lemma}[Aleph Numbers Constitute All Infinite Cardinals]
	If $\kappa$ is a cardinal then
	either $\kappa$ is finite (i.e.\ $\kappa \in \omega$) or
	$\kappa = \aleph_\alpha$ for some $\alpha \in \On$.
\end{lemma}
\begin{proof}
	Assume $\kappa$ is infinite, and take $\alpha$ minimal with $\aleph_\alpha \ge \kappa$.
	Suppose for contradiction that we have $\aleph_\alpha > \kappa$.
	We may assume $\alpha > 0$, since the case $\alpha = 0$ is trivial.

	If $\alpha = \ol\alpha + 1$ is a successor, then
	\[ \aleph_{\ol\alpha} < \kappa < \aleph_{\alpha}
		= (\aleph_{\ol\alpha})^+ \]
	which contradicts the fact the definition of the successor cardinal.
	
	If $\alpha = \lambda$ is a limit ordinal, then $\aleph_\lambda$ is the
	supremum $\bigcup_{\gamma < \lambda} \aleph_\gamma$.
	So there must be some $\gamma < \lambda$ has $\aleph_\gamma > \kappa$,
	which contradicts the minimality of $\alpha$.
\end{proof}

\begin{definition}
	An infinite cardinal which is not a successor cardinal
	is called a \vocab{limit cardinal}.
	It is exactly those cardinals of the form $\aleph_\lambda$,
	for $\lambda$ a limit ordinal, plus $\aleph_0$.
\end{definition}


\section{Cardinal Arithmetic}
\prototype{$\aleph_0 \cdot \aleph_0 = \aleph_0 + \aleph_0 = \aleph_0$}
Recall the way we set up ordinal arithmetic.
Note that in particular, $\omega + \omega > \omega$ and $\omega^2 > \omega$.
Since cardinals count size, this property is undesirable, and
we want to have
\begin{align*}
	\aleph_0 + \aleph_0 &= \aleph_0 \\
	\aleph_0 \cdot \aleph_0 &= \aleph_0
\end{align*}
because $\omega + \omega$ and $\omega \cdot \omega$ are countable.
In the case of cardinals, we simply ``ignore order''.

The definition of cardinal arithmetic is as expected:
\begin{definition}[Cardinal Arithmetic]
	Given cardinals $\kappa$ and $\mu$, define
	\[ \kappa + \mu
		\defeq
		\left\lvert 
		\left( \left\{ 0 \right\} \times \kappa \right)
		\cup
		\left( \left\{ 1 \right\} \times \mu \right)
		\right\rvert
	\]
	and
	\[
		k \cdot \mu
		\defeq
		\left\lvert \mu \times \kappa \right\rvert
		.
	\]
\end{definition}


\begin{ques}
	Check this agrees with what you learned in pre-school
	for finite cardinals.
\end{ques}

\begin{abuse}
	This is a slight abuse of notation since we are using
	the same symbols as for ordinal arithmetic,
	even though the results are different ($\omega \cdot \omega = \omega^2$
	but $\aleph_0 \cdot \aleph_0 = \aleph_0$).
	In general, I'll make it abundantly clear whether I am talking
	about cardinal arithmetic or ordinal arithmetic.
\end{abuse}
To help combat this confusion, we use separate symbols for ordinals and cardinals.
Specifically, $\omega$ will always refer to $\{0,1,\dots\}$ viewed as an ordinal;
$\aleph_0$ will always refer to the same set viewed as a cardinal.
More generally,
\begin{definition}
	Let $\omega_\alpha = \aleph_\alpha$ viewed as an ordinal.
\end{definition}

However, as we've seen already we have that $\aleph_0 \cdot \aleph_0 = \aleph_0$.
In fact, this holds even more generally:

\begin{theorem}[Infinite Cardinals Squared]
	Let $\kappa$ be an infinite cardinal.
	Then $\kappa \cdot \kappa = \kappa$.
\end{theorem}
\begin{proof}
	Obviously $\kappa \cdot \kappa \ge \kappa$,
	so we want to show $\kappa \cdot \kappa \le \kappa$.

	The idea is to repeat the same proof that we had for $\aleph_0 \cdot \aleph_0 = \aleph_0$,
	so we re-iterate it here. We took the ``square'' of elements of $\aleph_0$, and then
	\emph{re-ordered} it according to the diagonal:
	\[
	\begin{array}{l|cccccc}
		  & 0 & 1 & 2 & 3 & 4 & \dots \\ \hline
		0 & 0 & 1 & 3 & 6 & 10 & \dots \\
		1 & 2 & 4 & 7 & 11 & \dots & \\
		2 & 5 & 8 & 12 & \dots & & \\
		3 & 9 & 13 & \dots & & & \\
		4 & 14 & \dots & & & &
	\end{array}
	\]
	Let's copy this idea for a general $\kappa$.

	We proceed by transfinite induction on $\kappa$.
	The base case is $\kappa = \aleph_0$, done above.
	For the inductive step, first we put the ``diagonal'' ordering $<_{\text{diag}}$
	on $\kappa \times \kappa$ as follows:
	for $(\alpha_1, \beta_1)$ and $(\alpha_1, \beta_2)$ in $\kappa \times \kappa$
	we declare $(\alpha_1, \beta_1) <_{\text{diag}} (\alpha_2, \beta_2)$ if
	\begin{itemize}
		\ii $\max \left\{ \alpha_1, \beta_1 \right\} < \max \left\{ \alpha_2, \beta_2 \right\}$
		(they are on different diagonals), or
		\ii $\max \left\{ \alpha_1, \beta_1 \right\} = \max \left\{ \alpha_2, \beta_2 \right\}$
		and $\alpha_1 < \alpha_2$ (same diagonal).
	\end{itemize}
	
	Then $<_{\text{diag}}$ is a well-ordering of $\kappa \times \kappa$,
	so we know it is in order-preserving bijection with some ordinal $\gamma$.
	Our goal is to show that $\left\lvert \gamma \right\rvert \le \kappa$.
	To do so, it suffices to prove that for any $\ol\gamma \in \gamma$,
	we have $\left\lvert \ol\gamma \right\rvert < \kappa$.

	Suppose $\ol\gamma$ corresponds to the point $(\alpha, \beta) \in \kappa$
	under this bijection.
	If $\alpha$ and $\beta$ are both finite, then certainly $\ol\gamma$ is finite too.
	Otherwise, let $\ol\kappa = \max \{\alpha, \beta\} < \kappa$;
	then the number of points below $\ol\gamma$ is at most
	\[ 
		\left\lvert \alpha \right\rvert \cdot \left\lvert \beta \right\rvert
		\le \ol\kappa \cdot \ol\kappa
		= \ol\kappa
	\]
	by the inductive hypothesis.
	So $\left\lvert \ol\gamma \right\rvert \le \ol\kappa < \kappa$ as desired.
\end{proof}

From this it follows that cardinal addition and multiplication is really boring:
\begin{theorem}[Infinite Cardinal Arithmetic is Trivial]
	Given cardinals $\kappa$ and $\mu$, one of which is infinite,
	we have \[ \kappa \cdot \mu = \kappa + \mu = \max\left( \kappa, \mu \right).\]
\end{theorem}
\begin{proof}
	The point is that both of these are less than the square of the maximum.
	Writing out the details:
	\begin{align*}
		\max \left( \kappa, \mu \right)
		&\le \kappa + \mu \\ 
		&\le \kappa \cdot \mu \\ 
		&\le \max \left( \kappa, \mu \right) \cdot \max \left( \kappa, \mu  \right) \\
		&= \max\left( \kappa, \mu \right). \qedhere
	\end{align*}
\end{proof}




\section{Cardinal Exponentiation}
\prototype{$2^\kappa = \left\lvert \PP(\kappa) \right\rvert$.}
\begin{definition}
	Suppose $\kappa$ and $\lambda$ are cardinals.
	Then
	\[ \kappa^\lambda
		\defeq \left\lvert \mathscr F(\lambda, \kappa) \right\rvert.
	\]
	Here $\mathscr F(A,B)$ is the set of functions from $A$ to $B$.
\end{definition}

\begin{abuse}
	As before, we are using the same notation for
	both cardinal and ordinal arithmetic. Sorry!
\end{abuse}

In particular, $2^\kappa = \left\lvert \PP(\kappa) \right\rvert > \kappa$,
and so from now on we can use the notation $2^\kappa$ freely.
(Note that this is totally different from ordinal arithmetic;
there we had $2^\omega = \bigcup_{n\in\omega} 2^n = \omega$.
In cardinal arithmetic $2^{\aleph_0} > \aleph_0$.)

I have unfortunately not told you what $2^{\aleph_0}$ equals.
A natural conjecture is that $2^{\aleph_0} = \aleph_1$; this is called the
\vocab{Continuum Hypothesis}.
It turns out to that this is \emph{undecidable} -- it is not possible
to prove or disprove this from the $\ZFC$ axioms.

\section{Cofinality}
\prototype{$\aleph_0$, $\aleph_1$, \dots\ are all regular, but $\aleph_\omega$ has cofinality $\omega$.}

\begin{definition}
	Let $\lambda$ be a limit ordinal, and $\alpha$ another ordinal.
	A map $f : \alpha \to \lambda$ of ordinals is called \vocab{cofinal}
	if for every $\ol\lambda < \lambda$, there is some $\ol\alpha \in \alpha$
	such that $f(\ol\alpha) \ge \ol\lambda$.
	In other words, the map reaches arbitrarily high into $\lambda$.
\end{definition}
\begin{example}
	[Example of a Cofinal Map]
	\listhack
	\begin{enumerate}[(a)]
		\ii The map $\omega \to \omega^\omega$ by $n \mapsto \omega^n$ is cofinal.
		\ii For any ordinal $\alpha$, the identity map $\alpha \to \alpha$ is cofinal.
	\end{enumerate}
\end{example}

\begin{definition}
	Let $\lambda$ be a limit ordinal.
	The \vocab{cofinality} of $\lambda$, denoted $\cof(\lambda)$,
	is the smallest ordinal $\alpha$ such that there is a cofinal map
	$\alpha \to \lambda$.
\end{definition}
\begin{ques}
	Why must $\alpha$ be an infinite cardinal?
\end{ques}

Usually, we are interested in taking the cofinality of a cardinal $\kappa$.

Pictorially, you can imagine standing at the bottom of the universe and looking
up the chain of ordinals to $\kappa$.
You have a machine gun and are firing bullets upwards, and you want to get arbitrarily
high but less than $\kappa$.
The cofinality is then the number of bullets you need to do this.

We now observe that ``most'' of the time, the cofinality of a cardinal is itself.
Such a cardinal is called \vocab{regular}.
\begin{example}[$\aleph_0$ is Regular]
	$\cof(\aleph_0) = \aleph_0$, because no finite subset of $\omega$ can reach arbitrarily high.
\end{example}
\begin{example}[$\aleph_1$ is Regular]
	$\cof(\aleph_1) = \aleph_1$.
	Indeed, assume for contradiction that some countable
	set of ordinals $A = \{ \alpha_0, \alpha_1, \dots \} \subseteq \omega_1$
	reaches arbitrarily high inside $\omega_1$.
	Then $\Lambda = \cup A$ is a \emph{countable} ordinal,
	because it is a countable union of countable ordinals.
	In other words $\Lambda \in \omega_1$.
	But $\Lambda$ is an upper bound for $A$, contradiction.
\end{example}
On the other hand, there \emph{are} cardinals which are not regular;
since these are the ``rare'' cases we call them \vocab{singular}.
\begin{example}[$\aleph_\omega$ is Not Regular]
	Notice that $\aleph_0 < \aleph_1 < \aleph_2 < \dots$ reaches
	arbitrarily high in $\aleph_\omega$, despite only having $\aleph_0$ terms.
	It follows that $\cof(\aleph_\omega) = \aleph_0$.
\end{example}

We now confirm a suspicion you may have:
\begin{theorem}
	[Successor Cardinals Are Regular]
	If $\kappa = \ol\kappa^+$ is a successor cardinal,
	then it is regular.
\end{theorem}
\begin{proof}
	We copy the proof that $\aleph_1$ was regular.

	Assume for contradiction that for some $\mu \le \ol\kappa$,
	there are $\mu$ sets reaching arbitrarily high in $\kappa$ as a cardinal.
	Observe that each of these sets must have cardinality at most $\ol\kappa$.
	We take the union of all $\mu$ sets, which gives an ordinal $\Lambda$
	serving as an upper bound.

	The number of elements in the union is at most
	\[ \#\text{sets} \cdot \#\text{elms}
		\le \mu \cdot \ol\kappa = \ol\kappa \]
	and hence $\left\lvert \Lambda \right\rvert \le \ol\kappa < \kappa$.
\end{proof}

\section{Inaccessible Cardinals}
So, what about limit cardinals?
It seems to be that most of them are singular: if $\aleph_\lambda \ne \aleph_0$ is a limit ordinal,
then the sequence $\{\aleph_\alpha\}_{\alpha \in \lambda}$ (of length $\lambda$) is certainly cofinal.

\begin{example}[Beth Fixed Point]
	Consider the monstrous cardinal
	\[ \kappa = \aleph_{\aleph_{\aleph_{\ddots}}}. \]
	This might look frighteningly huge, as $\kappa = \aleph_\kappa$,
	but its cofinality is $\omega$ as it is the limit of the sequence
	\[ \aleph_0, \aleph_{\aleph_0}, \aleph_{\aleph_{\aleph_0}}, \dots \]
\end{example}

More generally, one can in fact prove that
\[ \cof(\aleph_\lambda) = \cof(\lambda). \]
But it is actually conceivable that $\lambda$ is so large
that $\left\lvert \lambda \right\rvert = \left\lvert \aleph_\lambda \right\rvert$.

A regular limit cardinal other than $\aleph_0$ has a special name: it is \vocab{weakly inaccessible}.
Such cardinals are so large that it is impossible to prove or disprove their existence in $\ZFC$.
It is the first of many so-called ``large cardinals''.

An infinite cardinal $\kappa$ is a strong limit cardinal if
\[ \forall \ol\kappa < \kappa \quad 2^{\ol\kappa} < \kappa \]
for any cardinal $\ol\kappa$.  For example, $\aleph_0$ is a strong limit cardinal.
\begin{ques}
	Why must strong limit cardinals actually be limit cardinals?
	(This is offensively easy.)
\end{ques}
A regular strong limit cardinal other than $\aleph_0$
is called \vocab{strongly inaccessible}.

\section\problemhead
\begin{problem}
	Compute $\left\lvert V_\omega \right\rvert$.
	\begin{hint}
		$\sup_{k \in \omega} \left\lvert V_k \right\rvert$.
	\end{hint}
\end{problem}

\begin{problem}
	Prove that for any limit ordinal $\alpha$, $\cof(\alpha)$ is a \emph{regular} cardinal.
	\begin{hint}
		Rearrange the cofinal maps to be nondecreasing.
	\end{hint}
\end{problem}

\begin{sproblem}
	[Strongly Inaccessible Cardinals]
	\label{prob:strongly_inaccessible}
	Show that for any strongly inaccessible $\kappa$,
	we have $\left\lvert V_\kappa \right\rvert = \kappa$.
\end{sproblem}

\begin{problem}
	[Konig's Theorem]
	Show that \[ \kappa^{\cof(\kappa)} > \kappa \] for every infinite cardinal $\kappa$.
\end{problem}
