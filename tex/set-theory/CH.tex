\chapter{The Failure of the Continuum Hypothesis}
We now use the technique of forcing to break the Contiuum Hypothesis by choosing a good poset $\Po$.

\section{Forcing $V \neq L$ is really easy}
As a small aside, to check we're on the right track we show the following result.

\begin{theorem}[$V \ne L$]
	Let $M$ be a countable transitive model of $\ZFC$.
	Let $\Po \in M$ be \emph{any} splitting poset,
	and let $G \subseteq \Po$ be $M$-generic.
	Then $M[G] \vDash (V \neq L)$.
\end{theorem}
\begin{proof}
	Since $L$ has a $\Sigma_1$ definition,
	we have \[ L^{M[G]} = L^M \subseteq M \subsetneq M[G] \]
	where the last part follows from $G \notin M[G]$.
\end{proof}

Thus $M[G] \vDash \ZFC + (V \ne L)$ for any splitting poset $\Po$,
and we are one step closer to breaking $\CH$.

\section{Adding in Reals}
Starting with a \emph{countable} transitive model $M$.

We want to choose $\Po \in M$ such that $(\aleph_2)^M$ many real numbers appear,
and then worry about cardinal collapse later.

Recall the earlier situation where we set $\Po$ to be the infinite complete binary tree; its nodes can be thought of as partial functions $n \to 2$ where $n < \omega$.
Then $G$ itself is a path down this tree; i.e.\ it can be encoded as a total function $G : \omega \to 2$,
and corresponds to a real number.

\begin{center}
	\begin{asy}
		size(8cm);
		pair P = Drawing("\varnothing", red, (0,4), red, dir(90));
		pair P0 = Drawing("0", red, (-5,2), red, 1.5*dir(90));
		pair P1 = Drawing("1", (5,2),  1.5*dir(90));
		pair P00 = Drawing("00", (-7,0), 1.4*dir(120));
		pair P01 = Drawing("01", red, (-3,0), red, 1.4*dir(60));
		pair P10 = Drawing("10", (3,0),  1.4*dir(120));
		pair P11 = Drawing("11", (7,0),  1.4*dir(60));

		pair P000 = Drawing("000", (-8,-3));
		pair P001 = Drawing("001", (-6,-3));
		pair P010 = Drawing("010", red, (-4,-3), red);
		pair P011 = Drawing("011", (-2,-3));

		pair P100 = Drawing("100", (2,-3));
		pair P101 = Drawing("101", (4,-3));
		pair P110 = Drawing("110", (6,-3));
		pair P111 = Drawing("111", (8,-3));

		draw(P01--P0--P00);
		draw(P11--P1--P10);
		draw(P0--P--P1);
		draw(P000--P00--P001);
		draw(P100--P10--P101);
		draw(P010--P01--P011);
		draw(P110--P11--P111);

		draw(P--P0--P01--P010--(P010+2*dir(-90)), red+1.4);
		MP("G", P010+2*dir(-90), dir(-90), red);
	\end{asy}
\end{center}

We want to do something similar, but with $\omega_2$ many real numbers instead of just one.
In light of this, consider in $M$ the following poset:
\[
	\Po = 
	\opname{Add} \left( \omega_2, \omega \right)
	\defeq
	\left( 
	\left\{ p : \omega_2 \times \omega \to 2,
		\dom(p) < \omega
	\right\},
	\supseteq
	\right).
\]
These elements (conditions) are ``partial functions'':
we take some finite subset of $\omega \times \omega_2$ and map it into $2=\{0,1\}$.
Moreover, we say $p \le q$ if $\dom(p) \supseteq \dom(q)$ and the two functions agree over $\dom(q)$.
\begin{ques}
	What is $1_\Po$ here?
\end{ques}

\begin{exercise}
	Show that a generic $G$ can be encoded as a function $\omega_2 \times \omega \to 2$.
\end{exercise}

%Let $G \subseteq \opname{Add}(\omega_2, \omega)$ be an $M$-generic.
%We claim that, like in the binary case, $G$ can be encoded as a function $\omega_2 \times \omega \to 2$.
%To see this, consider $\alpha \in \omega_2$ and $n \in \omega$; we have the dense set
%\[ D_{\alpha, n}
%	= \left\{ p \in \opname{Add}(\omega_2, \omega)
%	\mid (\alpha, n) \in \dom(p) \right\}
%\]
%(this is obviously dense, given any $p$ add in $(\alpha, n)$ if it's not in there already).
%So $G$ hits this dense set, meaning that for every $(\alpha, n)$ there's a function in $G$ which defines it.
%Using the fact that $G$ is upwards closed and a filter, we may as before we may interpret $G$ as a function $\omega_2 \times \omega \to 2$.

\begin{lemma}[$G$ encodes distinct real numbers]
	For $\alpha \in \omega_2$ define
	\[ G_\alpha = \left\{ n \mid G\left( \alpha,n \right) = 0 \right\} \in \PP(\NN). \]
	Then $G_\alpha \neq G_\beta$ for any $\alpha \neq \beta$.
\end{lemma}
\begin{proof}
	We claim that the set
	\[ D = \left\{ q \mid \exists n \in \omega :
		q\left( \alpha, n \right) \neq q\left( \beta, n \right)
		\text{ are both defined}
	\right\} \]
	is dense.
	\begin{ques}
		Check this.
		(Use the fact that the domains are all finite.)
	\end{ques}
%	This is pretty easy to see.
%	Consider $p \in \opname{Add}(\omega_2, \omega)$.
%	Then you can find an $n$ such that
%	neither $(\alpha, n)$ nor $(\beta, n)$ is defined,
%	just because $\dom(p)$ is finite.
%	Then you make $p'$ as $p$ plus $p'( (\alpha, n) ) = 1$
%	and $p'( (\beta, n) ) = 0$.
%	Hence the set is dense.

	Since $G$ is an $M$-generic it hits this dense set $D$.
	Hence $G_\alpha \neq G_\beta$.
\end{proof}

Since $G \in M[G]$ and $M[G] \vDash \ZFC$,
it follows that each $G_\alpha$ is in $M[G]$.
So there are at least $\aleph_2^M$ real numbers in $M$.
We are done once we can show there is no cardinal collapse.

\section{The Countable Chain Condition}
It remains to show that with $\Po = \opname{Add}(\omega, \omega_2)$, we have that
\[ \aleph_2^{M[G]} = \aleph_2^M. \]
In that case, since $M[G]$ will have $\aleph_2^M = \aleph_2^{M[G]}$ many reals, we will be done.

To do this, we'll rely on the following combinatorial property of $\Po$:

\begin{definition}
	We say that $A \subset \mathcal P$ is a \vocab{strong antichain}
	if for any distinct $p$ and $q$ in $A$, we have $p \perp q$.
\end{definition}
\begin{example}[Example of an Antichain]
	In the infinite binary tree, 
	the set $A = \{00, 01, 10, 11\}$ is a strong antichain
	(in fact maximal by inclusion).
\end{example}
This is stronger than the notion of ``antichain'' than you might be used to!\footnote{%
	In the context of forcing, some authors use ``antichain'' to refer to ``strong antichain''.
	I think this is lame.}
We don't merely require that every two elements are incomparable,
but that they are in fact \emph{incompatible}.
\begin{ques}
	Draw a finite poset and an antichain of it which is not strong.
\end{ques}

\begin{definition}
	A poset $\Po$ has the \vocab{$\kappa$-chain condition}
	(where $\kappa$ is a cardinal) if all strong antichains
	in $\Po$ have size less than $\kappa$.
	The special case $\kappa = \aleph_1$ is called the \vocab{countable chain condition},
	because it implies that every strong antichain is countable.
\end{definition}

We are going to show that if the poset has the $\kappa$-chain condition
then it preserves all cardinals greater than $\kappa$.
% or was it > \kappa?
In particular, the countable chain condition will show that $\Po$ preserves all the cardinals.
Then, we'll show that $\opname{Add}(\omega, \omega_2)$ does indeed have this property.
This will complete the proof.

We isolate the following lemma.
\begin{lemma}[Possible Values Argument]
	Suppose $M$ is a transitive model of $\ZFC$ and $\Po$ is a partial order
	such that $\Po$ has the $\kappa$-chain condition in $M$.
	Let $X,Y \in M$ and let $f: X \to Y$
	be some function in $M[G]$, but $f \notin M$.

	Then there exists a function $F \in M$, with $F: X \to \PP(Y)$ and such that
	for any $x \in X$,
	\[ f(x) \in F(x) \quad\text{and}\quad \left\lvert F(x) \right\rvert^M < \kappa. \]
\end{lemma}
What this is saying is that if $f$ is some new function that's generated,
$M$ is still able to pin down the values of $f$ to at most $\kappa$ many values.

\begin{proof}
	The idea behind the proof is easy: any possible value of $f$ gives us some condition in
	the poset $\Po$ which forces it.
	Since distinct values must have incompatible conditions,
	the $\kappa$-chain condition guarantees
	there are at most $\kappa$ such values.

	Here are the details.
	Let $\dot f$, $\check X$, $\check Y$ be names for $f$, $X$, $Y$.
	Start with a condition $p$ such that $p$ forces the sentence
	\[ \text{``$\dot f$ is a function from $\check X$ to $\check Y$''}. \]
	We'll work just below here.

	For each $x \in X$, we can consider (using the Axiom of Choice) a maximal antichain $A(x)$
	of incompatible conditions $q \le p$ which forces $f(x)$ to equal some value $y \in Y$.
	Then, we let $F(x)$ collect all the resulting $y$-values.
	These are all possible values, and there are less than $\kappa$ of them.
\end{proof}

\section{Preserving Cardinals}
TODO: actually write everything past this point
\todo{everything below here sucks}

\begin{definition}
	For $M$ a transitive model of ZFC and $\Po \in M$ a poset,
	we say $\Po$ \vocab{preserves cardinals} if
	$\forall G \subseteq \Po$ an $M$-generic,
	the model $M$ and $M[G]$ agree on the sentence ``$\kappa$ is a cardinal'' for every $\kappa$.

	In the same way we will talk about $\Po$ preserving cofinalities, et cetera.
\end{definition}

\todo{write out this exercise}
\begin{exercise}
	Let $M$ be a transitive model of ZFC.
	Let $\Po \in M$ be a poset.
	Show that the following are equivalent for each $\lambda$:
	\begin{enumerate}[(1)]
		\ii $\Po$ preserves cofinalities less than or equal to $\lambda$.
		\ii $\Po$ preserves regular cardinals less than or equal to $\lambda$.
	\end{enumerate}
	Moreover the same holds if we replace ``less than or equal to''
	by ``greater than or equal to''.
\end{exercise}
Thus, to show that $\Po$ preserves cardinality and cofinalities it suffices to show that $\Po$ preserves regularity.

\begin{theorem}
	Suppose $M$ is a transitive model of ZFC, and $\Po \in M$ is a poset.
	Suppose $M$ satisfies the sentence ``$\Po$ has the $\kappa$ chain condition and $\kappa$ is regular''.
	Then $\Po$ preserves cardinals and cofinalities greater than or equal to $\kappa$.
\end{theorem}
\begin{proof}
	It suffices to show that $\Po$ preserves regularity greater than or equal to $\kappa$.
	Consider $\lambda > \kappa$ which is regular in $M$,
	and suppose for contradiction that $\lambda$ is not regular in $M[G]$.
	That's the same as saying that there is a function $f \in M[G]$,
	$f : \ol \lambda \to \lambda$ cofinal, with $\ol \lambda < \lambda$.
	Then by the Possible Values Argument,
	there exists a function $F \in M$ from $\ol \lambda \to \PP(\lambda)$
	such that $f(\alpha) \in F(\alpha)$ and $\left\lvert F(\alpha) \right\rvert^M < \kappa$
	for every $\alpha$.

	Now we work in $M$ again.
	Note for each $\alpha \in \ol\lambda$,
	$F(\alpha)$ is bounded in $\lambda$ since $\lambda$ is regular in $M$ and
	greater than $\left\lvert F(\alpha) \right\rvert$.
	Now look at the function $\ol \lambda \to \lambda$ in $M$ by just
	\[ \alpha \mapsto \cup F(\alpha) < \lambda. \]
	This is cofinal in $M$, contradiction.
\end{proof}

\section{Infinite Combinatorics}
In particular, if $\Po$ has the countable chain condition then $\Po$ preserves all the cardinals (and cofinalities).
Therefore, it remains to show that $\opname{Add}(\omega, \omega_2)$ satisfies the countable chain condition.
And this is going to be infinite combinatorics.

\begin{definition}
	Suppose $C$ is an uncountable collection of finite sets.
	$C$ is a \textbf{$\Delta$-system} if there exists a \textbf{root} $R$
	with the condition that for any distinct $X$ and $Y$
	in $C$, we have $X \cap Y = R$.
\end{definition}

\begin{lemma}
	[$\Delta$-System Lemma] Suppose $C$ is an uncountable collection of finite sets.
	Then $\exists \ol C \subseteq C$ such that
	\begin{enumerate}[(1)]
		\ii $\ol C$ is uncountable.
		\ii $\ol C$ is a $\Delta$-system.
	\end{enumerate}
\end{lemma}
\begin{proof}
	There exists an integer $n$ such that $C$ has uncountably many guys of length $n$.
	So we can throw away all the other sets, and just assume that all sets in $C$ have size $n$.

	We now proceed by induction on $n$.
	The base case $n=1$ is trivial, since we can just take $R = \varnothing$.
	For the inductive step we consider two cases.

	First, assume there exists an $a \in C$ contained in uncountably many $F \in C$.
	Throw away all the other guys.
	Then we can just delete $a$, and apply the inductive hypothesis.

	Now assume that for every $a$, only countably many members of $C$ have $a$ in them.
	We claim we can even get a $\ol C$ with $R = \varnothing$.
	First, pick $F_0 \in C$.
	It's straightforward to construct an $F_1$ such that $F_1 \cap F_0 = \varnothing$.
	And we can just construct $F_2, F_3, \dots$
\end{proof}

\begin{lemma}
	For all $\kappa$, $\opname{Add}(\omega, \kappa)$ satisfies the countable chain condition.
\end{lemma}
\begin{proof}
	Assume not. Let
	\[ \left\{ p_\alpha : \alpha < \omega_1 \right\} \]
	be an antichain.  Let
	\[ C = \left\{ \dom(p_\alpha) : \alpha < \omega_1 \right\}. \]
	Let $\ol C \subseteq C$ be such that $\ol C$ is uncountable, and $\ol C$ is a $\Delta$-system which root $R$.
	Then let
	\[ B = \left\{ p_\alpha : \dom(p_\alpha) \in R \right\}. \]
	Each $p_\alpha \in B$ is a function $p_\alpha : R \to \{0,1\}$,
	so there are two that are the same.
\end{proof}
\todo{blah blah blah}

