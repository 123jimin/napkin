\chapter{Forcing}
We are now going to introduce Paul Cohen's technique of \vocab{forcing},
which we then use to break the Continuum Hypothesis.

Here is how it works.
Given a transitive model $M$ and a poset $\Po$ inside it,
we can consider a ``generic'' subset $G \subseteq \Po$, where $G$ is not in $M$.
Then, we are going to construct a bigger universe $M[G]$ which contains both $M$ and $G$.
(This notation is deliberately the same as $\ZZ[\sqrt2]$, for example -- in the algebra case,
we are taking $\ZZ$ and adding in a new element $\sqrt 2$, plus everything that can be generated from it.)
By choosing $\Po$ well, we can cause $M[G]$ to have desirable properties.

Picture:

\begin{center}
	\begin{asy}
		size(14cm);
		pair A = (12,30);
		pair B = -conj(A);
		pair M = midpoint(A--B);
		pair O = origin;
		MP("V", A, dir(10));
		draw(A--O--B);

		fill(A--O--B--cycle, opacity(0.3)+palecyan);

		MP("V_0 = \varnothing", origin, dir(-20));
		MP("V_1 = \{\varnothing\}", 0.05*A, dir(0));
		MP("V_2 = \{\varnothing, \{\varnothing\} \}", 0.10*A, dir(0));

		pair A1 = 0.4*A;
		pair B1 = 0.4*B;
		draw(MP("V_\omega", A1, dir(0))--B1);
		draw(MP("V_{\omega+1} = \mathcal P(V_n)", 0.45*A, dir(0))--0.45*B);
		Drawing("\omega", 0.45*M, dir(45));

		filldraw(O--A1--(A1+0.30*M)..(0.85*M)..(B1+0.30*M)--B1--cycle,
			opacity(0.3)+lightgreen, heavygreen+1);
		draw(O--0.85*M, heavygreen+1);
		filldraw(O--(1.3*A1)..(0.85*M)..(1.3*B1)--cycle,
			opacity(0.1)+lightred, heavyred+1);

		Drawing("\aleph_1^V", 0.95*M, dir(0));

		pair F = 0.55*B+0.10*A;
		Drawing("f", F, dir(90));
		draw(F--0.55*M, dotted, EndArrow, Margins);
		draw(F--0.45*M, dotted, EndArrow, Margins);

		Drawing("\aleph_1^M", 0.55*M, dir(45));
		Drawing("\aleph_1^{M[G]}", 0.65*M, dir(45));

		draw(0.85*M--M);
		MP("\mathrm{On}^V", M, dir(90));
		MP("\mathrm{On}^M = \mathrm{On}^{M[G]}", 0.85*M, dir(135));

		MP("M \subseteq M[G]", 0.85*M, 3*dir(0)+dir(45));

		Drawing("\mathbb P", 0.3*M+0.3*A, dir(135));
		Drawing("G", 0.55*A+0.1*B, dir(45));
	\end{asy}
\end{center}

The model $M$ is drawn in green, and its extension $M[G]$ is drawn in red.

The models $M$ and $M[G]$ will share the same ordinals, which is represented here
as $M$ being no taller than $M[G]$.
But one issue with this is that forcing may introduce some new bijections between cardinals of $M$
that were not there originally; this leads to the phenomenon called \vocab{cardinal collapse}:
quite literally, cardinals in $M$ will no longer be cardinals in $M[G]$, and instead just an ordinal.
This is because in the process of adjoining $G$, we may accidentally pick up some bijections which were not in the earlier universe.
In the diagram drawn, this is the function $f$ mapping $\omega$ to $\aleph_1^M$.
Essentially, the difficulty is that ``$\kappa$ is a cardinal'' is a $\Pi_1$ statement.

In the case of the Continuum Hypothesis, we'll introduce a $\Po$ such that
any generic subset $G$ will ``encode'' $\aleph_2^M$ real numbers.
We'll then show cardinal collapse does not occur, meaning $\aleph_2^{M[G]} = \aleph_2^M$.
Thus $M[G]$ will have $\aleph_2^{M[G]}$ real numbers, as desired.

\section{Setting up posets}
\prototype{Infinite Binary Tree}
Let $M$ be a transitive model of $\ZFC$.
Let $\Po = (\Po, \le) \in M$ be a poset with a maximal element $1_\Po$
which lives inside a model $M$.
The elements of $\Po$ are called \vocab{conditions};
because they will force things to be true in $M[G]$.

\begin{definition}
	A subset $D \subseteq \Po$ is \vocab{dense} if for all $p \in \Po$,
	there exists a $q  \in D$ such that $q \le p$.
\end{definition}
Examples of dense subsets include the entire $\Po$ as well
as any downwards ``slice''.

\begin{definition}
	For $p,q \in \Po$ we write $p \parallel q$,
	saying ``$p$ is \vocab{compatible} with $q$'',
	if there exists $r \in \Po$ with $r \le p$ and $r \le q$.
	Otherwise, we say $p$ and $q$ are \vocab{incompatible}
	and write $p \perp q$.
\end{definition}
\begin{example}[Infinite binary tree]
	Let $\Po = 2^{<\omega}$ be the \vocab{infinite binary tree} shown below,
	extended to infinity in the obvious way:
	\begin{center}
		\begin{asy}
			size(8cm);
			pair P = Drawing("\varnothing", (0,4), dir(90));
			pair P0 = Drawing("0", (-5,2), 1.5*dir(90));
			pair P1 = Drawing("1", (5,2),  1.5*dir(90));
			pair P00 = Drawing("00", (-7,0), 1.4*dir(120));
			pair P01 = Drawing("01", (-3,0), 1.4*dir(60));
			pair P10 = Drawing("10", (3,0),  1.4*dir(120));
			pair P11 = Drawing("11", (7,0),  1.4*dir(60));

			pair P000 = Drawing("000", (-8,-3));
			pair P001 = Drawing("001", (-6,-3));
			pair P010 = Drawing("010", (-4,-3));
			pair P011 = Drawing("011", (-2,-3));

			pair P100 = Drawing("100", (2,-3));
			pair P101 = Drawing("101", (4,-3));
			pair P110 = Drawing("110", (6,-3));
			pair P111 = Drawing("111", (8,-3));

			label("$\vdots$", (-7,-3), dir(-90));
			label("$\vdots$", (-3,-3), dir(-90));
			label("$\vdots$", (3,-3), dir(-90));
			label("$\vdots$", (7,-3), dir(-90));

			draw(P01--P0--P00);
			draw(P11--P1--P10);
			draw(P0--P--P1);
			draw(P000--P00--P001);
			draw(P100--P10--P101);
			draw(P010--P01--P011);
			draw(P110--P11--P111);
		\end{asy}
	\end{center}

	\begin{enumerate}[(a)]
		\ii The maximal element $1_\Po$ is the empty string $\varnothing$.
		\ii $D = \{\text{all strings ending in $001$}\}$ is an example of a dense set.
		\ii No two elements of $\Po$ are compatible unless they are comparable.
	\end{enumerate}
\end{example}


Now, I can specify what it means to be ``generic''.
\begin{definition}
	A nonempty set $G \subseteq \Po$ is a \vocab{filter} if
	\begin{enumerate}[(a)]
		\ii The set $G$ is upwards-closed:
		$\forall p \in G (\forall q \ge p) (q \in G)$.
		\ii Any pair of elements in $G$ is compatible.
	\end{enumerate}
	We say $G$ is \vocab{$M$-generic} if for all $D$ which are \emph{in the model $M$},
	if $D$ is dense then $G \cap D \neq \varnothing$.
\end{definition}
\begin{ques}
	Show that if $G$ is a filter then $1_\Po \in G$.
\end{ques}
\begin{example}[Generic filters on the infinite binary tree]
	Let $\Po = 2^{<\omega}$.
	The generic filters on $\Po$ are sets of the form
	\[ \left\{ 0,\; b_1,\; b_1b_2,\; b_1b_2b_3,\; \dots \right\}. \]
	So every generic filter on $\Po$ correspond to a binary number $b = 0.b_1b_2b_3\dots$.

	It is harder to describe which reals correspond to generic filters,
	but they should really ``look random''.
	For example, the set of strings ending in $011$ is dense,
	so one should expect ``$011$'' to appear inside $b$,
	and more generally that $b$ should contain every binary string.
	So one would expect the binary expansion of $\pi-3$ might correspond to a generic,
	but not something like $0.010101\dots$.
	That's why we call them ``generic''.
\end{example}

\begin{center}
	\begin{asy}
		size(8cm);
		pair P = Drawing("\varnothing", red, (0,4), red, dir(90));
		pair P0 = Drawing("0", red, (-5,2), red, 1.5*dir(90));
		pair P1 = Drawing("1", (5,2),  1.5*dir(90));
		pair P00 = Drawing("00", (-7,0), 1.4*dir(120));
		pair P01 = Drawing("01", red, (-3,0), red, 1.4*dir(60));
		pair P10 = Drawing("10", (3,0),  1.4*dir(120));
		pair P11 = Drawing("11", (7,0),  1.4*dir(60));

		pair P000 = Drawing("000", (-8,-3));
		pair P001 = Drawing("001", (-6,-3));
		pair P010 = Drawing("010", red, (-4,-3), red);
		pair P011 = Drawing("011", (-2,-3));

		pair P100 = Drawing("100", (2,-3));
		pair P101 = Drawing("101", (4,-3));
		pair P110 = Drawing("110", (6,-3));
		pair P111 = Drawing("111", (8,-3));

		draw(P01--P0--P00);
		draw(P11--P1--P10);
		draw(P0--P--P1);
		draw(P000--P00--P001);
		draw(P100--P10--P101);
		draw(P010--P01--P011);
		draw(P110--P11--P111);

		draw(P--P0--P01--P010--(P010+2*dir(-90)), red+1.4);
		MP("G", P010+2*dir(-90), dir(-90), red);
	\end{asy}
\end{center}

\begin{exercise}
	Verify that these are every generic filter $2^{<\omega}$ has the form above.
	Show that conversely, a binary number gives a filter, but it need not be generic.
\end{exercise}

Notice that if $p \ge q$, then the sentence $q \in G$ tells us more information than the sentence $p \in G$.
In that sense $q$ is a \emph{stronger} condition.
In another sense $1_\Po$ is the weakest possible condition,
because it tells us nothing about $G$; we always have $1_\Po \in G$
since $G$ is upwards closed.

\section{More properties of posets}
We had better make sure that generic filters exist.
In fact this is kind of tricky, but for countable models it works:
\begin{lemma}[Rasiowa-Sikorski lemma]
	Suppose $M$ is a \emph{countable} transitive model of $\ZFC$
	and $\Po$ is a partial order.
	Then there exists an $M$-generic filter $G$.
\end{lemma}
\begin{proof}
	Essentially, hit them one by one.
	\Cref{prob:rslemma}.
\end{proof}

Fortunately, for breaking $\CH$ we would want $M$ to be countable anyways.
% This is really just the proof of the Baire category theorem.

The other thing we want to do to make sure we're on the right track is guarantee
that a generic set $G$ is not actually in $M$.
(Analogy: $\ZZ[3]$ is a really stupid extension.)
The condition that guarantees this is:

\begin{definition}
	A partial order $\Po$ is \vocab{splitting} if
	for all $p \in \Po$, there exists $q,r \le p$
	such that $q \perp r$.
\end{definition}
\begin{example}[Infinite binary tree is (very) splitting]
	The infinite binary tree is about as splitting as you can get.
	Given $p \in 2^{<\omega}$, just consider the two elements right under it.
\end{example}

\begin{lemma}[Splitting posets omit generic sets]
	Suppose $\Po$ is splitting.  Then if $F \subseteq \Po$ is a filter
	such that $F \in M$, then $\Po \setminus F$ is dense.
	In particular, if $G \subseteq \Po$ is generic, then $G \notin M$.
\end{lemma}
\begin{proof}
	Consider $p \notin \Po \setminus F \iff p \in F$.
	Then there exists $q, r \le p$ which are not compatible.
	Since $F$ is a filter it cannot contain both;
	we must have one of them outside $F$, say $q$.
	Hence every element of $p \in \Po \setminus (\Po \setminus F)$
	has an element $q \le p$ in $\Po \setminus F$.
	That's enough to prove $\Po \setminus F$ is dense.
	\begin{ques}
		Deduce the last assertion of the lemma about generic $G$. \qedhere
	\end{ques}
\end{proof}

\section{Names, and the generic extension}
We now define the \emph{names} associated to a poset $\Po$.

\begin{definition}
	Suppose $M$ is a transitive model of $\ZFC$, $\Po = (\Po, \le) \in M$ is a partial order.
	We define the hierarchy of \vocab{$\Po$-names} recursively by
	\begin{align*}
		\Name_0 &= \varnothing \\
		\Name_{\alpha+1} &= \PP(\Name_\alpha \times \Po) \\
		\Name_{\lambda} &= \bigcup_{\alpha < \lambda} \Name_\alpha.
	\end{align*}
	Finally, $\Name = \bigcup_\alpha \Name_\alpha$ denote the class of all $\Po$-names.
	% For $\tau \in \Name$, let $\nrank(\tau)$ be the least $\alpha$ such that $\tau \in \Name_\alpha$.
\end{definition}
(These $\Name_\alpha$'s are the analog of the $V_\alpha$'s:
each $\Name_\alpha$ is just the set of all names with rank $\le \alpha$.)

\begin{definition}
	For a filter $G$, we define the \vocab{interpretation} of $\tau$ by $G$,
	denote $\tau^G$, using the transfinite recursion
	\[ \tau^G
		= \left\{ \sigma^G
		\mid \left<\sigma, p\right> \in \tau
		\text{ and } p \in G\right\}. \]
	We then define the model
	\[ M[G] = \left\{ \tau^G \mid \tau \in \Name^M \right\}. \]
	In words, $M[G]$ is the interpretation of all the possible $\Po$-names
	(as computed by $M$).
\end{definition}

\textbf{You should think of a $\Po$-name as a ``fuzzy set''.}
Here's the idea.
Ordinary sets are collections of ordinary sets,
so fuzzy sets should be collections of fuzzy sets.
These fuzzy sets can be thought of like the Ghosts of Christmases yet to come:
they represent things that might be, rather than things that are certain.
In other words, they represent the possible futures of $M[G]$ for various choices of $G$.

Every fuzzy set has an element $p \in \Po$ pinned to it.
When it comes time to pass judgment,
we pick a generic $G$ and filter through the universe of $\Po$-names.
The fuzzy sets with an element of $G$ attached to it materialize into the real world,
while the fuzzy sets with elements outside of $G$ fade from existence.
The result is $M[G]$.

\begin{example}[First few levels of the name hierarchy]
	Let us compute
	\begin{align*}
		\Name_0 &= \varnothing \\
		\Name_1 &= \PP(\varnothing \times \Po) \\
		&= \{\varnothing\} \\
		\Name_2 &= \PP(\{\varnothing\} \times \Po) \\
		&= \PP\left( \left\{ 
			\left<\varnothing, p\right>
			\mid p \in \Po
		\right\} \right).
	\end{align*}
\end{example}
Compare the corresponding von Neuman universe.
\[ V_0 = \varnothing, \; V_1 = \{\varnothing\}, \;
V_2 = \left\{ \varnothing, \left\{ \varnothing \right\} \right\}. \]

\begin{example}[Example of an interpretation]
	As we said earlier, $\Name_1 = \{\varnothing\}$.
	Now suppose
	\[ \tau =
		\left\{
			\left<\varnothing, p_1\right>,
			\left<\varnothing, p_2\right>,
			\dots
			\left<\varnothing, p_n\right>
		\right\} 
		\in \Name_2. \]
	Then 
	\[
		\tau^G
		= \left\{ \varnothing \mid
		\left<\varnothing, p\right> \in \tau \text{ and } p \in G\right\}
		=
		\begin{cases}
			\{\varnothing\} & \text{if some } p_i \in G \\
			\varnothing & \text{otherwise}.
		\end{cases}
	\]
	In particular, remembering that $G$ is nonempty we see that
	\[ \left\{ \tau^G \mid \tau \in \Name_2 \right\} = V_2^M. \]
	In fact, this holds for any natural number $n$, not just $2$.
\end{example}
So, $M[G]$ and $M$ agree on finite sets.

Now, we want to make sure $M[G]$ contains the elements of $M$.
To do this, we take advantage of the fact that $1_\Po$ must be in $G$, and define
for every $x \in M$ the set
\[ \check x = \left\{ \left<\check y, 1_\Po\right> \mid y \in x \right\} \]
by transfinite recursion.
Basically, $\check x$ is just a copy of $x$ where we add check marks and tag every element with $1_\Po$.

\begin{example}
	Compute $\check 0 = 0$ and $\check 1 = \left\{ \left<\check 0, 1_\Po\right> \right\}$.
	Thus \[ (\check 0)^G = 0 \quad\text{and}\quad (\check 1)^G = 1. \]
\end{example}
\begin{ques}
	Show that in general, $(\check x)^G = x$.
	(Rank induction.)
\end{ques}

However, we'd also like to cause $G$ to be in $M[G]$.
In fact, we can write down the name exactly:
\[ \dot G = \left\{ \left<\check p, p\right> \mid p \in \Po \right\}. \]
\begin{ques}
	Show that $(\dot G)^G = G$.
\end{ques}
\begin{ques}
	Verify that $M[G]$ is transitive:
	that is, if $\sigma^G \in \tau^G \in M[G]$, show that $\sigma^G \in M[G]$.
	(This is offensively easy.)
\end{ques}

In summary,
\begin{moral}
	$M[G]$ is a transitive model extending $M$ (it contains $G$).
\end{moral}

Moreover, it is reasonably well-behaved even if $G$ is just a filter.
Let's see what we can get off the bat.
\begin{lemma}[Properties obtained from filters]
	Let $M$ be a transitive model of $\ZFC$.
	If $G$ is a filter, then $M[G]$ is transitive
	and satisfies $\Extensionality$, $\Foundation$, 
	$\EmptySet$, $\Infinity$, $\Pairing$, and $\Union$.
\end{lemma}

This leaves $\PowerSet$, $\Replacement$, and Choice.
\begin{proof}
	Hence, we get $\Extensionality$ and $\Foundation$ for free.
	Then $\Infinity$ and $\EmptySet$ follows from $M \subseteq M[G]$.

	For $\Pairing$, suppose $\sigma_1^G, \sigma_2^G \in M[G]$.
	Then
	\[ \sigma = 
		\left\{ \left<\sigma_1, 1_\Po\right>, \left<\sigma_2, 1_\Po\right> \right\}
	\]
	works satisfies $\sigma^G = \{\sigma_1, \sigma_2\}$.
	(Note that we used $M \vDash \Pairing$.)
	$\Union$ is left as a problem, which you are encouraged to try now.
\end{proof}
Up to here, we don't need to know anything about when a sentence is true in $M[G]$;
all we had to do was contrive some names like $\check x$ or
$\left\{ \left<\sigma_1, 1_\Po\right>, \left<\sigma_2, 1_\Po\right> \right\}$
to get the facts we wanted.
But for the remaining axioms, we \emph{are} going to need this extra power
are true in $M[G]$.
For this, we have to introduce the fundamental theorem of forcing.

\section{Fundamental theorem of forcing}
The model $M$ unfortunately has no idea what $G$ might be,
only that it is some generic filter.\footnote{%
	You might say this is a good thing, here's why.
	We're trying to show that $\neg \CH$ is consistent with $\ZFC$,
	and we've started with a model $M$ of the real universe $V$.
	But for all we know $\CH$ might be true in $V$ (what if $V=L$?),
	in which case it would also be true of $M$.

	Nonetheless, we boldly construct $M[G]$ an extension of the model $M$.
	In order for it to behave differently from $M$, it has to be out of reach of $M$.
	Conversely, if $M$ could compute everything about $M[G]$,
	then $M[G]$ would have to conform to $M$'s beliefs.

	That's why we worked so hard to make sure $G \in M[G]$ but $G \notin M$.
}
Nonetheless, we are going to define a relation $\Vdash$, called the \emph{forcing} relation.
Roughly, we are going to write
\[ p \Vdash \varphi(\sigma_1, \dots, \sigma_n) \]
where $p \in \Po$, $\sigma_1, \dots, \sigma_n \in M[G]$, if and only if:
\begin{quote}
	For \emph{any} generic $G$,
	if $p \in G$,
	then $M[G] \vDash \varphi(\sigma_1^G, \dots, \sigma_n^G)$.
\end{quote}
Note that $\Vdash$ is defined without reference to $G$:
it is something that $M$ can see.
We say $p$ \vocab{forces} the sentences $\varphi(\sigma_1, \dots, \sigma_n)$.
And miraculously, we can define this relation in such a way that the converse is true:
\emph{a sentence holds if and only if some $p$ forces it}.


\begin{theorem}
	[Fundamental theorem of forcing]
	Suppose $M$ is a transitive model of ZF.
	Let $\Po \in M$ be a poset, and $G \subseteq \PP$ is an $M$-generic filter.
	Then,
	\begin{enumerate}[(1)]
		\ii Consider $\sigma_1, \dots, \sigma_n \in \Name^M$,
		Then
		\[ M[G] \vDash \varphi[\sigma_1^G, \dots, \sigma_n^G] \]
		if and only if there exists a condition $p \in G$
		such that $p$ \emph{forces} the sentence $\varphi(\sigma_1, \dots, \sigma_n)$.
		We denote this by $p \Vdash \varphi(\sigma_1, \dots, \sigma_n)$.
		\ii This forcing relation is (uniformly) definable in $M$.
	\end{enumerate}
\end{theorem}

I'll tell you how the definition works in the next section.

\section{(Optional) Defining the relation}
Here's how we're going to go.
We'll define the most generous condition possible such that
the forcing works in one direction ($p \vDash \varphi(\sigma_1, \dots, \sigma_n)$ means
$M[G] \vDash \varphi[\sigma_1^G, \dots, \sigma_n^G]$).
We will then cross our fingers that the converse also works.

We proceed by induction on the formula complexity.
It turns out in this case that the atomic formula (base cases)
are hardest and themselves require induction on ranks.

For some motivation, let's consider how we should define $p \Vdash \tau_1 \in \tau_2$ given that we've already defined $p \vDash \tau_1 = \tau_2$.
We need to ensure this holds iff
\[ \forall \text{$M$-generic $G$ with $p \in G$}:
	\ M[G] \vDash \tau_1^G \in \tau_2^G. \]
So it suffices to ensure that any generic $G \ni p$ hits a condition $q$ which forces $\tau_1^G$ to \emph{equal} a member $\tau^G$ of $\tau_2^G$.
In other words, we want to choose the definition of $p \Vdash \tau_1 \in \tau_2$ to hold if and only if
\[
	\left\{ q \in \Po
	\mid \exists \left<\tau, r\right> \in \tau_2 
	\left( q \le r \land q \Vdash(\tau=\tau_1) \right)
	\right\}
\]
is dense below in $p$.
In other words, if the set is dense, then the generic must hit $q$, so it must hit $r$, meaning that $\left<\tau_r\right> \in \tau_2$ will get interpreted such that $\tau^G \in \tau_2^G$, and moreover the $q \in G$ will force $\tau_1 = \tau$.

Now let's write down the definition\dots
In what follows, the $\Vdash$ omits the $M$ and $\Po$.
\begin{definition}
	Let $M$ be a countable transitive model of ZFC.
	Let $\Po \in M$ be a partial order.
	For $p \in \Po$ and $\varphi(\sigma_1, \dots, \sigma_n)$ a formula in LST, we write $\tau \Vdash \varphi(\sigma_1, \dots, \sigma_n)$ to mean the following, defined by induction on formula complexity plus rank.
	\begin{enumerate}[(1)]
		\ii $p \Vdash \tau_1 = \tau_2$ means
		\begin{enumerate}[(i)]
			\ii For all $\left<\sigma_1, q_1\right> \in \tau_1$ the set
			\[ D_{\sigma_1, q_1}
				\defeq
				\left\{ r \mid
				r \le q_1 \lthen \exists \left<\sigma_2, q_2\right> \in \tau_2 \left( r \le q_2 \land r \Vdash (\sigma_1 = \sigma_2) \right)\right\}.
			\]
			is dense in $p$.
			(This encodes ``$\tau_1 \subseteq \tau_2$''.)
			\ii For all $\left<\sigma_2, q_2\right> \in \tau_2$,
			the set $D_{\sigma_2, q_2}$ defined similarly is dense below $p$.
		\end{enumerate}
		\ii $p \Vdash \tau_1 \in \tau_2$ means
		\[
		\left\{ q \in \Po
		\mid \exists \left<\tau, r\right> \in \tau_2 
		\left( q \le r \land q \Vdash(\tau=\tau_1) \right)
		\right\} \]
		is dense below $p$.
		\ii $p \Vdash \varphi \land \psi$ means $p \Vdash \varphi$ and $p \Vdash \psi$.
		\ii $p \Vdash \neg \varphi$ means $\forall q \le p$, $q \not\Vdash \varphi$.
		\ii $p \Vdash \exists x \varphi(x, \sigma_1, \dots, \sigma_n)$ means that the set
		\[
			\left\{ q \mid \exists \tau \left( q \Vdash
				\varphi(\tau, \sigma_1, \dots, \sigma_n \right) \right)
			\right\}
		\]
		is dense below $p$.
	\end{enumerate}
\end{definition}
This is definable in $M$!
All we've referred to is $\Po$ and names, which are in $M$.
(Note that being dense is definable.)
Actually, in parts (3) through (5) of the definition above,
we use induction on formula complexity.
But in the atomic cases (1) and (2) we are doing induction on the ranks of the names.

So, the construction above gives us one direction (I've omitted tons of details, but\dots).

Now, how do we get the converse: that a sentence is true if and only if something forces it?
Well, by induction, we can actually show:
\begin{lemma}[Consistency and Persistence]
	We have
	\begin{enumerate}[(1)]
		\ii (Consistency) If $p \Vdash \varphi$ and $q \le p$ then $q \Vdash \varphi$.
		\ii (Persistence) If $\left\{ q \mid q \Vdash \varphi \right\}$
		is dense below $p$ then $p \Vdash \varphi$.
	\end{enumerate}
\end{lemma}
You can prove both of these by induction on formula complexity.
From this we get:
\begin{corollary}[Completeness]
	The set $\left\{ p \mid p \Vdash \varphi \text{ or } p \Vdash \neg\varphi \right\}$
	is dense.
\end{corollary}
\begin{proof}
	We claim that whenever $p \not\Vdash \varphi$ then
	for some $\ol p \le p$ we have $\ol p \Vdash \neg\varphi$;
	this will establish the corollary.

	By the contrapositive of the previous lemma,
	$\{q \mid q \Vdash \varphi\}$ is not dense below $p$,
	meaning for some $\ol p \le p$, every $q \le \ol p$ gives $q \not\Vdash \varphi$.
	By the definition of $p \vDash \neg\varphi$,
	we have $\ol p \vDash \neg\varphi$.
\end{proof}
And this gives the converse: the $M$-generic $G$ has to hit some condition
that passes judgment, one way or the other.
This completes the proof of the fundamental theorem.

\section{The remaining axioms}
\begin{theorem}[The generic extension satisfies $\ZFC$]
	Suppose $M$ is a transitive model of $\ZFC$.
	Let $\Po \in M$ be a poset, and $G \subseteq \PP$ is an $M$-generic filter.
	Then \[ M[G] \vDash \ZFC. \]
\end{theorem}
\begin{proof}
	We'll just do $\Comprehension$, as the other remaining axioms are similar.
	
	Suppose $\sigma^G, \sigma_1^G, \dots, \sigma_n^G \in M[G]$
	are a set and parameters, and
	$\varphi(x,x_1, \dots, x_n)$ is an LST formula.
	We want to show that the set
	\[ A = \left\{ 
		x \in \sigma^G \mid M[G] \vDash \varphi[x, \sigma_1^G, \dots, \sigma_n^G]
	\right\} \]
	is in $M[G]$; i.e.\ it is the interpretation of some name.

	Note that every element of $\sigma^G$ is of the form $\rho^G$
	for some $\rho \in \dom(\sigma)$ (a bit of abuse here,
	$\sigma$ is a bunch of pairs of names and $p$'s,
	and the domain is just the set of names).
	So by the fundamental theorem of forcing, we may write
	\[ A = 
		\left\{ \rho^G \mid \rho \in \dom(\sigma)
			\text{ and }
			\exists p \in G
			\left( p \Vdash \rho \in \sigma
			\land \varphi(\rho, \sigma_1, \dots, \sigma_n)
			\right)
		\right\}.
	\]
	To show $A \in M[G]$ we have to write down a $\tau$
	such that the name $\tau^G$ coincides with $A$.
	We claim that
	\[
		\tau
		=
		\left\{ \left<\rho, p\right>
			\in \dom(\sigma) \times \Po \mid
			p \Vdash 
			\land \varphi(\rho, \sigma_1, \dots, \sigma_n)
		\right\}
	\]
	is the correct choice.
	It's actually clear that $\tau^G = A$ by construction;
	the ``content'' is showing that $\tau$ is in actually a name of $M$,
	which follows from $M \vDash \Comprehension$.

	So really, the point of the fundamental theorem of forcing
	is just to let us write down this $\tau$;
	it lets us show that $\tau$ is in $\Name^M$
	without actually referencing $G$.
\end{proof}


\section\problemhead
\begin{problem}
	For a filter $G$ and $M$ a transitive model of $\ZFC$,
	show that $M[G] \vDash \Union$.
\end{problem}

\begin{problem}[Rasiowa-Sikorski lemma]
	\label{prob:rslemma}
	Show that in a countable transitive model $M$ of $\ZFC$,
	one can find an $M$-generic filter on any partial order.
	\begin{hint}
		Let $D_1$, $D_2$, \dots be the dense sets (there are countably many of them).
	\end{hint}
	\begin{sol}
	Since $M$ is countable, there are only countably many dense sets (they live in $M$!),
	say \[ D_1, D_2, \ldots, D_n, \ldots \in M. \]
	Using Choice,
	let $p_1 \in D_1$, and then let $p_2 \le p_1$ such that $p_2 \in D_2$
	(this is possible since $D_2$ is dense), and so on.
	In this way we can inductively exhibit a chain
	\[ p_1 \ge p_2 \ge p_3 \ge \dots \]
	with $p_i \in D_i$ for every $i$.

	Hence, we want to generate a filter from the $\{p_i\}$.
	Just take the upwards closure -- let $G$ be the set of $q \in \Po$ such that $q \ge p_n$ for some $n$.
	By construction, $G$ is a filter (this is actually trivial).
	Moreover, $G$ intersects all the dense sets by construction.
	\end{sol}
\end{problem}

%\begin{exercise}
%	Show that $\rank \sigma^G \le \nrank(\sigma)$ for any $\sigma \in \Name^M$.
%\end{exercise}

%\begin{exercise}
%	Check that
%	\begin{enumerate}[(1)]
%		\ii $(\check x)^G = x$.
%		\ii $(\dot G)^G = G$.
%	\end{enumerate}
%\end{exercise}
