\chapter{Flavors of rings}

We continue our exploration of rings by considering
some nice-ness properties that rings or ideals can satisfy,
which will be valuable later on.
As before, number theory is interlaced as motivation.
I guess I can tell you at the outset what the completed table
is going to look like, so you know what to expect.

\begin{center}
	\begin{tabular}[h]{lll}
		Ring noun & Ideal adjective & Relation \\ \hline
		PID & principal &
			$R$ is a PID $\iff$ $R$ is an integral domain, \\
			&& \qquad and every $I$ is principal \\
		Noetherian ring & finitely generated &
			$R$ is Noetherian $\iff$ every $I$ is fin.\ gen. \\
		field & maximal & $R/I$ is a field $\iff$ $I$ is maximal \\
		integral domain & prime & $R/I$ is an integral domain
			$\iff$ $I$ is prime \\
	\end{tabular}
\end{center}

\section{Fields}
\prototype{$\QQ$ is a field, but $\ZZ$ is not.}

As you might already know, if the multiplication is invertible,
then we call the ring a field.
To be explicit, let me write the relevant definitions.

\begin{definition}
	\label{def:unit}
	A \vocab{unit} of a ring $R$
	is an element $u \in R$ which is invertible:
	for some $x \in R$ we have $ux = 1_R$.
\end{definition}
\begin{example}
	[Examples of units]
	\listhack
	\begin{enumerate}[(a)]
	\ii The units of $\ZZ$ are $\pm 1$,
	because these are the only things which ``divide $1$''
	(which is the reason for the name ``unit'').
	\ii On the other hand, in $\QQ$ everything is a unit (except $0$).
	For example, $\frac 35$ is a unit since
	$\frac 35 \cdot \frac 53 = 1$.
	\ii The Gaussian integers $\ZZ[i]$ have four units:
	$\pm 1$ and $\pm i$.
	\end{enumerate}
\end{example}

\begin{definition}
	A nontrivial (commutative) ring is a \vocab{field}
	when all its nonzero elements are units.
\end{definition}

Colloquially, we say that
\begin{moral}
	A field is a structure where you can add, subtract, multiply, and divide.
\end{moral}

\begin{example}
	[First examples of fields]
	\listhack
	\begin{enumerate}[(a)]
		\ii $\QQ$, $\RR$, $\CC$ are fields,
		since the notion $\frac 1c$ makes sense in them.
		\ii If $p$ is a prime, then $\Zc p$ is a field,
		which we denote will usually denote by $\FF_p$.
	\end{enumerate}
	The trivial ring $0$ is \emph{not} considered a field,
	since we require fields to be nontrivial..
\end{example}

\begin{exercise}
	If $k$ is a field, show that there are exactly two ideals.
	What are they?
\end{exercise}


%\begin{remark}
%	You might say at this point that ``fields are nicer than rings'',
%	but as you'll see in this chapter, the conditions for
%	being a field are somehow ``too strong''.
%	To give an example of what I mean:
%	if you try to think about the concept of ``divisibility''
%	in $\ZZ$, you've stepped into the vast and bizarre realm of
%	number theory.  Try to do the same thing in $\QQ$ and you get nothing:
%	any nonzero $a$ ``divides'' any nonzero $b$
%	because $b = a \cdot \frac ba$.
%
%	I know at least one person who instead
%	thinks of this as an argument for why people
%	shouldn't care about number theory
%	(studying chaos rather than order).
%\end{remark}

\section{Integral domains}
\prototype{$\ZZ$ is an integral domain.}
Now it would be nice if we could still conclude the zero product property:
if $ab = 0$ then either $a = 0$ or $b = 0$.
If our ring is a field, this is true: if $b \neq 0$,
then we can multiply by $b\inv$ to get $a = 0$.
But many other rings we consider like $\ZZ$ and $\ZZ[x]$ also have this property,
despite not being full-fledged fields.

Not for all rings though: in $\Zc{15}$,
\[ 3 \cdot 5 \equiv 0 \pmod{15}. \]
If $a, b \neq 0$ but $ab=0$ then we say $a$ and $b$ are \vocab{zero divisors}
of the ring $R$.
So we give a name to such rings.
\begin{definition}
	A nontrivial ring with no zero divisors
	is called an \vocab{integral domain}.\footnote{Some
		authors abbreviate this to ``domain'', notably Artin.}
\end{definition}
\begin{ques}
	Show that a field is an integral domain.
\end{ques}

\begin{example}
	[Examples of integral domains]
	Every field is an integral domain,
	so all the previous examples apply.
	In addition:
	\begin{enumerate}[(a)]
		\ii $\ZZ$ is an integral domain, but it is not a field.
		\ii $\RR[x]$ is not a field,
		since there is no polynomial $P(x)$ with $xP(x) = 1$.
		However, $\RR[x]$ is an integral domain,
		because if $P(x) Q(x) \equiv 0$ then one of $P$ or $Q$ is zero.
		\ii $\ZZ[x]$ is also an example of an integral domain.
		In fact, $R[x]$ is an integral domain for any integral domain $R$ (why?).
		\ii $\Zc n$ is a field (hence integral domain)
		exactly when $n$ is prime.
		When $n$ is not prime, it is a ring but not an integral domain.
	\end{enumerate}
	The trivial ring $0$ is \emph{not} considered an integral domain.
\end{example}

At this point, we go ahead and say:
\begin{definition}
	An integral domain where all ideals are principal
	is called a \vocab{principal ideal domain (PID)}.
\end{definition}
The ring $\ZZ/6\ZZ$ is an example of a ring
which is a principal ideal ring, but not an integral domain.
As we alluded to earlier, we will never really use ``principal ideal ring''
in any real way: we typically will want to strengthen it to PID.

\section{Prime ideals}
\prototype{$(5)$ is a prime ideal of $\ZZ$.}

We know that every integer can be factored (up to sign)
as a unique product of primes; for example $15 = 3 \cdot 5$
and $-10 = -2 \cdot 5$.
You might remember the proof involves the so-called B\'ezout's lemma,
which essentially says that $(a,b) = (\gcd(a,b))$;
in other words we've carefully used the fact that $\ZZ$ is a PID.

It turns out that for general rings, the situation is not as nice
as factoring elements because most rings are not PID's.
The classic example of something going wrong is
\[ 6 = 2 \cdot 3 = \left( 1-\sqrt{-5} \right)\left( 1+\sqrt{-5} \right) \]
in $\ZZ[\sqrt{-5}]$.
Nonetheless, we can sidestep the issue
and talk about factoring \emph{ideals}:
somehow the example $10 = 2 \cdot 5$ should be $(10) = (2) \cdot (5)$,
which says ``every multiple of $10$ is the product of a
multiple of $2$ and a multiple of $5$''.
I'd have to tell you then how to multiply two ideals, which I do
in the chapter on unique factorization.

Let's at least figure out what primes are.
In $\ZZ$, we have that $p \neq 1$ is prime if whenever $p \mid xy$,
either $p \mid x$ or $p \mid y$.
We port over this definition to our world of ideals.
\begin{definition}
	\label{def:prime_ideal}
	A \emph{proper} ideal $I \subsetneq R$ is a \vocab{prime ideal}
	if whenever $xy \in I$, either $x \in I$ or $y \in I$.
\end{definition}
The condition that $I$ is proper is analogous to the
fact that we don't consider $1$ to be a prime number.

\begin{example}[Examples and non-examples of prime ideals]
	\listhack
	\begin{enumerate}[(a)]
		\ii The ideal $(7)$ of $\ZZ$ is prime.
		\ii The ideal $(8)$ of $\ZZ$ is not prime,
		since $2 \cdot 4 = 8$.

		\ii The ideal $(x)$ of $\ZZ[x]$ is prime.
		\ii The ideal $(x^2)$ of $\ZZ[x]$ is not prime,
		since $x \cdot x = x^2$.

		\ii The ideal $(3,x)$ of $\ZZ[x]$ is prime.
		This is actually easiest to see
		using \Cref{thm:prime_ideal_quotient} below.

		\ii The ideal $(5) = 5\ZZ + 5i\ZZ$ of $\ZZ[i]$
		is not prime, since the elements
		$3+i$ and $3-i$ have product $10 \in (5)$,
		yet neither is itself in $(5)$.
	\end{enumerate}
\end{example}
\begin{remark}
	\label{rem:unit_sign_issue}
	Ideals have the nice property that they get rid of ``sign issues''.
	For example, in $\ZZ$, do we consider $-3$ to be a prime?
	When phrased with ideals, this annoyance goes away: $(-3) = (3)$.
	More generally, for a ring $R$, talking about ideals
	lets us ignore multiplication by a unit.
	(Note that $-1$ is a unit in $\ZZ$.)
\end{remark}

\begin{exercise}
	What do you call a ring $R$ for which the zero ideal $(0)$ is prime?
\end{exercise}

We also have:
\begin{theorem}[Prime ideal $\iff$ quotient is integral domain]
	\label{thm:prime_ideal_quotient}
	An ideal $I$ is prime if and only if $R/I$ is an integral domain.
\end{theorem}
\begin{exercise}
	[Mandatory]
	Convince yourself the theorem is true;
	it is just definition chasing.
	(A possible start is to consider $R = \ZZ$ and $I = (15)$.)
\end{exercise}

I now must regrettably inform you that unique factorization is still
not true even with the notion of a ``prime'' ideal
(though again I haven't told you how to multiply two ideals yet).
But it will become true with some additional assumptions
that will arise in algebraic number theory
(relevant buzzword: Dedekind domain).

\section{Maximal ideals}
\prototype{The ideal $(x,5)$ is maximal in $\ZZ[x]$, by quotient-ing.}

Here's another flavor of an ideal.
\begin{definition}
	A proper ideal $I$ of a ring $R$ is \vocab{maximal} if
	it is not contained in any other proper ideal.
\end{definition}
\begin{example}
	[Examples of maximal ideals]
	\listhack
	\begin{enumerate}[(a)]
		\ii The ideal $I = (7)$ of $\ZZ$ is maximal, because
		if an ideal $J$ contains $7$
		and an element $n$ not in $I$
		it must contain $\gcd(7,n) = 1$, and hence $J = \ZZ$.
		\ii The ideal $(x)$ is \emph{not} maximal in $\ZZ[x]$,
		because it's contained in $(x,5)$ (among others).
		\ii On the other hand, $(x,5)$ is indeed maximal in $\ZZ[x]$.
		This is actually easiest to verify using
		\Cref{thm:max_ideal_quotient} below.
		\ii Also, $(x)$ is maximal in $\CC[x]$,
		again appealing to \Cref{thm:max_ideal_quotient} below.
	\end{enumerate}
\end{example}

\begin{exercise}
	What do you call a ring $R$ for which the zero ideal $(0)$ is maximal?
\end{exercise}

There's an analogous theorem to the one for prime ideals.
\begin{theorem}
	[$I$ maximal $\iff$ $R/I$ field]
	\label{thm:max_ideal_quotient}
	An ideal $I$ is maximal if and only if $R/I$ is a field.
\end{theorem}
\begin{proof}
	A ring is a field if and only if $(0)$ is the only maximal ideal.
	So this follows by \Cref{prob:inclusion_reversing}.
\end{proof}

\begin{corollary}
	[Maximal ideals are prime]
	If $I$ is a maximal ideal of a ring $R$, then $I$ is prime.
\end{corollary}
\begin{proof}
	If $I$ is maximal, then $R/I$ is a field,
	hence an integral domain, so $I$ is prime.
\end{proof}

In practice, because modding out by generated ideals is pretty convenient,
this is a very efficient way to check whether an ideal is maximal.
\begin{example}
	[Modding out in $\ZZ{[x]}$]
	\listhack
	\begin{enumerate}[(a)]
		\ii This instantly implies that $(x,5)$ is a maximal ideal
		in $\ZZ[x]$, because if we mod out by $x$ and $5$ in $\ZZ[x]$,
		we just get $\FF_5$, which is a field.
		\ii On the other hand, modding out by just $x$ gives $\ZZ$,
		which is an integral domain but not a field; that's why $(x)$ is
		prime but not maximal.
	\end{enumerate}
\end{example}

As we saw, any maximal ideal is prime.
But now note that $\ZZ$ has the special property that
all of its nonzero prime ideals are also maximal.
It's with this condition and a few other minor conditions
that you get a so-called \emph{Dedekind domain}
where prime factorization of ideals \emph{does} work.
More on that later.

\section{Field of fractions}
\prototype{The field of fractions of $\ZZ$ is $\QQ$.}
As long as we are here, we take the time to introduce a useful
construction that turns any integral domain into a field.

\begin{definition}
	Given an integral domain $R$,
	we define its \vocab{field of fractions} $K$ as follows:
	it consists of elements $a / b$, where $a,b \in R$ and $b \neq 0$.
	We set $a / b \sim c / d$ if and only if $bc = ad$.
	Addition and multiplication is defined by
	\begin{align*}
		\frac ab + \frac cd &= \frac{ad+bc}{bd} \\
		\frac ab \cdot \frac cd &= \frac{ab}{cd}.
	\end{align*}
\end{definition}
In fact everything you know about $\QQ$ basically carries over by analogy.
You can prove if you want that this indeed a field, but 
considering how comfortable we are that $\QQ$ is well-defined,
I wouldn't worry about it\dots

\begin{definition}
	Let $k$ be a field.
	The field of fractions of $k[x]$ is denoted $k(x)$
	(read ``$k$ of $x$''),
	and called the \vocab{field of rational functions}.
\end{definition}

\begin{example}
	[Examples of fraction fields]
	\listhack
	\begin{enumerate}[(a)]
		\ii The field of fractions of $\ZZ$ is $\QQ$, \emph{by definition}.
		\ii The field $\RR(x)$ consists of rational functions in $x$:
		\[ \RR(x) = \left\{ \frac{f(x)}{g(x)} \mid f,g \in \RR[x] \right\}. \]
		For example, $\frac{2x}{x^2-3}$ might be a typical element.
	\end{enumerate}
\end{example}

\begin{example}
	[Gaussian rationals]
	\label{ex:gaussian_rationals}
	Just like we defined $\ZZ[i]$ by abusing notation,
	we can also write $\QQ(i)$.
	Officially, it should consist of
	\[ \QQ(i) = \left\{ \frac{f(i)}{g(i)} \mid g(i) \ne 0 \right\} \]
	for polynomials $f$ and $g$ with rational coefficients.
	But since $i^2=-1$ this just leads to
	\[ \QQ(i) = \left\{ \frac{a+bi}{c+di} \mid a,b,c,d \in \QQ,
		(c,d) \ne (0,0) \right\}. \]
	And since $\frac{1}{c+di} = \frac{c-di}{c^2+d^2}$ we end up with
	\[ \QQ(i) = \left\{ a+bi \mid a,b \in \QQ \right\}. \]
\end{example}


\section{\problemhead}
Not olympiad problems, but again the spirit is very close
to what you might see in an olympiad.

\begin{problem}
	Is $\QQ[\sqrt2] = \left\{ a + b\sqrt2 \mid a,b \in \QQ \right\}$ a field?
	\begin{hint}
		Yes.
	\end{hint}
\end{problem}

\begin{sproblem}
	Prove that a integral domain with finitely many elements is a field.
	\label{prob:finite_domain_field}
\end{sproblem}

\begin{sproblem}
	[Krull's theorem]
	\label{prob:krull_max_ideal}
	Let $R$ be a ring and $J$ a proper ideal.
	\begin{enumerate}[(a)]
		\ii Prove that if $R$ is Noetherian,
		then $J$ is contained in a maximal ideal $I$.
		\ii Use Zorn's lemma (\Cref{ch:zorn})
		to prove the result even if $R$ isn't Noetherian.
	\end{enumerate}
	\begin{hint}
		Just keep on adding in elements to get an ascending chain.
	\end{hint}
	\begin{sol}
		For part (b), look at the poset of \emph{proper} ideals.
		Apply Zorn's lemma (again using a union trick to verify the condition;
		be sure to verify that the union is proper!).
		In part (a) we are given no ascending infinite chains,
		so no need to use Zorn's lemma.
	\end{sol}
\end{sproblem}

\begin{problem}
	[{$\Spec k[x]$}]
	Describe the prime ideals of $\CC[x]$ and $\RR[x]$.
	\begin{hint}
		Use the fact that both are PID's.
	\end{hint}
	\begin{sol}
		The ideal $(0)$ is of course prime in both.
		Also, both rings are PID's.

		For $\CC[x]$ we get a prime ideal $(x-z)$ for each $z \in \CC$.

		For $\RR[x]$ a prime ideal $(x-a)$ for each $a \in \RR$
		and a prime ideal $(x^2 - ax + b)$ for each quadratic
		with two conjugate real roots.
	\end{sol}
\end{problem}

\begin{problem}
	Prove that any nonzero prime ideal of $\ZZ[\sqrt 2]$ is also a maximal ideal.
	%(This is starred since the main idea of the solution will get used extensively in
	%the algebraic NT chapters, even though the problem itself won't.)
	\label{prob:dedekind_sample}
	\begin{hint}
		Show that the quotient $\ZZ[\sqrt2]/I$ has finitely many elements
		for any nonzero prime ideal $I$.
		Therefore, the quotient is an integral domain, it is also a field,
		and thus $I$ was a maximal ideal.
	\end{hint}
\end{problem}

