\chapter{Rings and Ideals}
This is a typical chapter on commutative algebra.

\section{Number Theory Motivation}
Commutative algebra is closely tied to algebraic geometry:
lots of the ideas in commutative algebra have nice ``geometric'' interpretations that motivate the definitions.
Unfortunately, algebraic geometry also has commutative algebra as a prerequisite, leading to a
chicken-and-egg problem: to appreciate commutative algebra fully you need to know algebraic geometry,
but to learn algebraic geometry you need to know commutative algebra.

To try and patch this as best I could, I'll try to flavor and motivate the examples with olympiad number theory.
This way, your intuition from all those shortlist N6's
you had to do can hopefully carry over to make sense of the examples here.
Basically, we'll try to generalize properties of the ring $\ZZ$ to
any abelian structure in which we can also multiply.
That's why, for example, you can talk about ``irreducible polynomials in $\QQ[x]$'' in the same
way you can talk about ``primes in $\ZZ$'', or about ``factoring polynomials modulo $p$''
in the same way we can talk ``unique factorization in $\ZZ$''.
Even if you only care about $\ZZ$ (say, you're a number theorist), this has a lot of value:
I assure you that trying to solve $x^n+y^n = z^n$ (for $n > 2$) requires going into a ring other than $\ZZ$!

For all the sections that follow, keep $\ZZ$ in mind as your prototype.

\section{Definition and Examples of Rings}
\prototype{$\ZZ$ all the way! Also $R[x]$ and various fields.}

Well, I guess I'll define a ring\footnote{%
	Or, according to some authors, a ``commutative ring with unit''.
	You see, some authors don't require rings to have commutative multiplication,
	and other authors don't even require there to be multiplicative units.
	I've seen some reasonable noncommutative rings, but I've never seen a ring
	without identity that I care about. So, just be warned that some authors
	have different definitions of ring.
	For us, ``ring'' always means ``commutative ring with $1$''.}.

\begin{definition}
	A \vocab{ring} is a triple $(R, +, \times)$, the two operations usually called addition and multiplication, such that
	\begin{enumerate}[(i)]
		\ii $(R,+)$ is an abelian group, with identity $0_R$, or just $0$.
		\ii $\times$ is an associative, commutative binary operation on $R$ with some
		identity, written $1_R$ or just $1$.
		\ii Multiplication distributes over addition.
	\end{enumerate}
\end{definition}
\begin{abuse}
	As usual, we will abbreviate $(R, +, \times)$ to just $R$.
\end{abuse}

These definitions are just here for completeness.
The examples are much more important.
\begin{example}[Typical Rings and Fields]
	\listhack
	\begin{enumerate}[(a)]
		\ii The sets $\ZZ$, $\QQ$, $\RR$ and $\CC$ are all rings
		with the usual addition and multiplication.
		\ii The integers modulo $n$ are also a ring
		with the usual addition and multiplication.
		We denote it by $\ZZ / n\ZZ$ for reasons that we'll explain momentarily.
		\ii The \vocab{trivial ring} is the ring $R$ with one element $0_R = 1_R$.
	\end{enumerate}
\end{example}

Since I've defined this structure, I may as well state the obligatory facts about it.
\begin{fact}
	For any ring $R$ and $r \in R$, $r \cdot 0_R = 0_R$.
	Moreover, $r \cdot (-1_R) = -r$.
\end{fact}
\begin{proof}
	Boring.
\end{proof}

Here are some more examples of rings.
\begin{example}[Polynomial Ring]
	Given any ring $R$, the ring $R[x]$ is defined as the set of polynomials
	with coefficients in $R$:
	\[ R[x] = \left\{ a_nx^n+a_{n-1}x^{n-1}+\dots+a_0 \mid a_0, \dots, a_n \in R \right\}. \]
	Addition and multiplication are done exactly in the way you would expect.
\end{example}
\begin{remark}
	[Digression] Happily, polynomial division also does what we expect: if $p(x) \in R[x]$
	and $p(a) = 0$, then $(x-a)q(x) = p(x)$ for some polynomial $q$.
	Proof: just do polynomial long division.
	With that, note the caveat that 
	\[ x^2-1 \equiv (x-1)(x+1) \pmod 8 \] has \emph{four} roots $1$, $3$, $5$, $7$ in
	$\ZZ/8\ZZ$!

	The problem is that $2 \cdot 4 = 0$ even though $2$ and $4$ are not zero;
	we call $2$ and $4$ \emph{zero divisors} for that reason.
	In an \emph{integral domain} (a ring without zero divisors), this pathology goes away,
	and just about everything you know about polynomials carries over.
	(I'll say this all again next section.)
\end{remark}
\begin{example}
	[Multi-variable Polynomial Ring]
	We can consider polynomials in $n$ variables with coefficients in $R$,
	denoted \[ R[x_1, \dots, x_n]. \]
	(We can even adjoin infinitely many $x$'s if we like, but alas\dots)
\end{example}
\begin{example}
	[Gaussian Integers are a Ring]
	The \vocab{Gaussian integers} are the set of complex numbers
	with integer real and imaginary parts, that is
	\[ \ZZ[i] = \left\{ a+bi \mid a,b \in \ZZ \right\}. \]
\end{example}

\section{Integral Domains and Fields}
\prototype{$\ZZ$ is an integral domain; $\QQ$ is a field.}

As you already know, if the multiplication is invertible,
then we call the ring a field.
To be explicit, let me write the relevant definitions.

\begin{definition}
	A \vocab{unit} of a ring $R$
	is an element $u \in R$ which is invertible:
	for some $x \in R$ we have $ux = 1_R$.
\end{definition}
\begin{example}
	[Examples of Units]
	\listhack
	\begin{enumerate}[(a)]
	\ii The units of $\ZZ$ are $\pm 1$,
	because these are the only things which ``divide $1$''
	(which is the reason for the name ``unit'').
	\ii On the other hand, in $\QQ$ everything is a unit.
	For example, $\frac 35$ is a unit since $\frac 35 \cdot \frac 53 = 1$.
	\ii The Gaussian integers $\ZZ[i]$ have four units: $\pm 1$ and $\pm i$.
	\end{enumerate}
\end{example}

\begin{definition}
	A nontrivial ring is a \vocab{field}
	when all its nonzero elements are units.
\end{definition}

Colloquially, we say that
\begin{moral}
A field is a structure where you can add, subtract, multiply, and divide.
\end{moral}

\begin{remark}
	You might say at this point that ``fields are nicer than rings'',
	but as you'll see in this chapter, the conditions for
	being a field are somehow ``too strong''.
	To give an example of what I mean:
	if you try to think about the concept of ``divisibility''
	in $\ZZ$, you've stepped into the vast and bizarre realm of
	number theory.  Try to do the same thing in $\QQ$ and you get nothing:
	any nonzero $a$ ``divides'' any nonzero $b$
	because $b = a \cdot \frac ba$.

	I know at least one person who instead
	thinks of this as an argument for why people
	shouldn't care about number theory
	(studying chaos rather than order).
\end{remark}

Now it would be nice if we could still conclude the zero product property:
if $ab = 0$ then either $a = 0$ or $b = 0$.
If our ring is a field, this is true: if $b \neq 0$,
then we can multiply by $b\inv$ to get $a = 0$.
But many other rings we consider like $\ZZ$ and $\ZZ[x]$ also have this property,
despite not being full-fledged fields.

Not for all rings though: in $\ZZ/15\ZZ$, $3 \cdot 5 \equiv 0 \pmod{15}$.
If $a, b \neq 0$ but $ab=0$ then we say $a$ and $b$ are \vocab{zero divisors}
of the ring $R$.
A nontrivial ring with no zero divisors is called an \vocab{integral domain}.

\begin{example}
	[Integral Domains and Fields]
	\listhack
	\begin{enumerate}[(a)]
		\ii $\QQ$, $\RR$, $\CC$ are fields, since the notion $\frac 1c$ makes sense in them.
		\ii $\RR[x]$ is not a field, since there is no polynomial $P(x)$ with $xP(x) = 1$.
		However, $\RR[x]$ is an integral domain, because if $P(x) Q(x) \equiv 0$ then one
		of $P$ or $Q$ is zero.
		\ii $\ZZ[x]$ is also an example of an integral domain.
		In fact, $R[x]$ is an integral domain for any integral domain $R$ (why?).
		\ii $\ZZ/n\ZZ$ is a field exactly when $n$ is prime.
		When $n$ is not prime, it is a ring but not an integral domain.
	\end{enumerate}
\end{example}

\section{Ideals}
\prototype{$5\ZZ$ is an ideal of $\ZZ$.}
This section is going to go briskly --
it's the obvious generalization of all the stuff we did with quotient groups.\footnote{%
	I once found an abstract algebra textbook which teaches rings
	before groups.  At the time I didn't understand why,
	but now I think I get it -- modding out by things in
	commutative rings is far more natural, and you can start talking
	about all the various flavors of rings and fields.
	You also have (in my opinion) more vivid first examples
	for rings than for groups.
	I actually sympathize a lot with this approach -- maybe I'll convert
	Napkin to follow it one day.}
First, we define a homomorphism and isomorphism.

\begin{definition}
	Let $R = (R, +_R, \times_R)$ and $S = (S, +_S, \times_S)$ be rings.
	A \vocab{ring homomorphism} is a map $\phi : R \to S$
	such that 
	\begin{enumerate}[(i)]
		\ii $\phi(x +_R y) = \phi(x) +_S \phi(y)$ for each $x,y \in R$.
		\ii $\phi(x \times_R y) = \phi(x) \times_S \phi(y)$ for each $x,y \in R$.
		\ii $\phi(1_R) = 1_S$.
	\end{enumerate}
	If $\phi$ is a bijection then $\phi$ is an \vocab{isomorphism}
	and we say that rings $R$ and $S$ are \vocab{isomorphic}.
\end{definition}
Just what you would expect.
The only surprise is that we also demand $\phi(1_R)$ to go to $1_S$.
This condition is not extraneous:
consider the map $\ZZ \to \ZZ$ called ``multiply by zero''.

Now, just like we were able to mod out by groups, we'd also like to define quotient rings.
So once again,
\begin{definition}
	The \vocab{kernel} of a ring homomorphism $\phi : R \to S$,
	denoted $\ker \phi$, is the set of $r \in R$ such that $\phi(r) = 0$.
\end{definition}

In group theory, we were able to characterize the ``normal'' subgroups by a few
obviously necessary conditions (namely, $gHg\inv = H$).
We can do the same thing for rings, and it's in fact easier because our operations are commutative.

First, note two obvious facts:
\begin{itemize}
	\ii If $\phi(x) = \phi(y) = 0$, then $\phi(x+y) = 0$ as well.
	So $\ker \phi$ should be closed under addition.
	\ii If $\phi(x) = 0$, then for any $r \in R$ we have
	$\phi(rx) = \phi(r)\phi(x) = 0$ too.
	So for $x \in \ker \phi$ and \emph{any} $r \in R$,
	we have $rx \in \ker\phi$.
\end{itemize}

A subset $I \subseteq R$ is called an \vocab{ideal} if it satisfies these properties.
Note that in the second condition, $r$ need not be in $I$!
So this is stronger than just saying $R$ is closed under multiplication.

\begin{figure}[ht]
	\centering
	\includegraphics[height=10cm]{/home/evan/Pictures/TopologicalGG/ideals.jpg}
	\caption{Ideals absorb multiplication.}
\end{figure}

\begin{example}
	[Prototypical Example of an Ideal]
	Consider the set $I = 5\ZZ = \{\dots,-10,-5,0,5,10,\dots\}$ as an ideal in $\ZZ$.
	We indeed see $I$ is the kernel of the ``take mod $5$'' homomorphism:
	\[ \ZZ \surjto \ZZ/5\ZZ. \]
	It's clearly closed under addition,
	but it absorbs multiplication from \emph{all} elements of $\ZZ$:
	given $15 \in I$, $999 \in \ZZ$, we get $15 \cdot 999 \in I$.
\end{example}

\begin{ques}
	Let $R$ be a ring and $I$ an ideal.
	Convince yourself that if $I$ contains any units, then $I = R$.
	(As an example, consider any ideal in the Gaussian integers
	which contains $i$.)
\end{ques}
\begin{ques}
	Verify that in $\RR$ (or actually any nontrivial field),
	the only two ideals are $(0) = \{0\}$ and $(1) = \RR$.
\end{ques}

Now we claim that these conditions are sufficient.
More explicitly,
\begin{theorem}
	[Ring Analog of Normal Subgroups]
	Let $R$ be a ring and $I$ a subring.
	Then $I$ is the kernel of some homomorphism if and only if it's an ideal.
\end{theorem}
\begin{proof}
	It's quite similar to the proof for the normal subgroup thing,
	and you might try it yourself as an exercise.
	
	Obviously the conditions are necessary.
	To see they're sufficient, we \emph{define} a ring by ``cosets''
	\[ S = \left\{ r + I \mid r \in R \right\}. \]
	These are the equivalence where we say $r_1 \sim r_2$ if $r_1 - r_2 \in I$
	(think of this as taking ``mod $I$'').
	To see that these form a ring, we just have to check that the addition
	and multiplication we put on them is well-defined.
	Specifically, we want to check that if $r_1 \sim s_1$ and $r_2 \sim s_2$,
	then $r_1 + r_2 \sim s_1 + s_2$ and $r_1r_2 \sim s_1s_2$.
	We actually already did the first part -- just think of $R$ and $S$ as abelian
	groups, forgetting for the moment that we can multiply.
	The multiplication is more interesting.
	\begin{exercise}
		[Recommended]
		Show that if $r_1 \sim s_1$ and $r_2 \sim s_2$, then $r_1r_2 \sim s_1s_2$.
		You will need to use the fact that $I$ absorbs multiplication
		from \emph{any} elements of $R$, not just those in $I$.
	\end{exercise}
	Anyways, since this addition and multiplication is well-defined there
	is now a surjective homomorphism $R \to S$ with kernel exactly $I$.
\end{proof}

\begin{definition}
	Given an ideal $I$, we define as above the \vocab{quotient ring}
	\[ R/I \defeq \left\{ r+I \mid r \in R \right\}. \]
	It's the ring of these equivalence classes.
\end{definition}
\begin{example}
	The integers modulo $5$ formed by ``modding out additively by $5$''
	are for this reason denoted $\ZZ / 5\ZZ$.
\end{example}
But here's an important point:
just as we don't actually think of $\ZZ/5\ZZ$ as consisting of
$k + 5\ZZ$ for $k=0,\dots,4$,
we also don't really want to think about $R/I$ as elements $r+I$.
The better way to think about it is
\begin{moral}
	$R/I$ is the result when we declare that elements of $I$ are all zero;
	that is, we ``mod out by elements of $I$''.
\end{moral}
For example, modding out by $5\ZZ$ means that we consider
all elements in $\ZZ$ divisible by $5$ to be zero.
This gives you the usual modular arithmetic!


\section{Generating Ideals}
Let's give you some practice with ideals.

\begin{exercise}
	Show that the only ideals of $\ZZ$ are precisely those
	sets of the form $n\ZZ$, where $n$ is an integer.
\end{exercise}

Thus, while ideals of fields are not terribly interesting,
ideals of $\ZZ$ look eerily like elements of $\ZZ$.
Let's make this more precise.
\begin{definition}
	Let $R$ be a ring.
	The \vocab{ideal generated} by a set of elements $x_1, \dots, x_n \in R$
	is denoted by $I = (x_1, x_2, \dots, x_n)$
	and given by
	\[
		I = \left\{ r_1x_1 + \dots + r_nx_n \mid r_i \in R \right\}.
	\]
	One can think of this as ``the smallest ideal containing all the $x_i$''.
\end{definition}
%The analogy of putting the $\{x_i\}$ in a sealed box and shaking vigorously
%kind of works here too.
The ``additive structure'' of ideals is emphasized if I say
\begin{moral}
	An ideal is an $R$-module.
	The ideal $(x_1, \dots, x_n)$ is the submodule
	spanned by $x_1, \dots, x_n$.
\end{moral}
In particular, if $I = (x)$ then $I$ consists of exactly the
``multiples of $x$'', \emph{id est}
numbers of the form $rx$ for $r \in R$.
\begin{remark}
	We can also apply this definition to infinite generating sets, as long as
	only finitely many of the $r_i$ are not zero (since infinite sums
	don't make sense in general).
\end{remark}

\begin{example}[Examples of Generated Ideals]
	\listhack
	\begin{enumerate}[(a)]
		\ii As $(n) = n\ZZ$ for all $ \in \ZZ$,
		every ideal in $\ZZ$ is of the form $(n)$.
		\ii In $\ZZ[i]$, we have
		$(5) = \left\{ 5a + 5b i \mid a,b \in \ZZ \right\}$.
		\ii In $\ZZ[x]$, the ideal $(x)$ consists of polynomials
		with zero constant terms.
		\ii In $\ZZ[x,y]$, the ideal $(x,y)$ again consists
		of polynomials with zero constant terms.
		\ii In $\ZZ[x]$, the ideal $(x,5)$ consists of polynomials
		whose constant term is divisible by $5$.
	\end{enumerate}
\end{example}
\begin{ques}
	Please check that the set 
	$I = \left\{ r_1x_1 + \dots + r_nx_n \mid r_i \in R \right\}$
	is indeed always an ideal (closed under addition,
	and absorbs multiplication).
\end{ques}

Now suppose $I = (x_1, \dots, x_n)$.
What does $R/I$ look like?
According to what I said at the end of the last section,
it's what happens when we ``mod out'' by each of the elements $x_i$.
For example\dots
\begin{example}
	[Modding Out by Generated Ideals]
	\listhack
	\begin{enumerate}[(a)]
		\ii Let $R = \ZZ$ and $I = (5)$. Then $R/I$ is literally
		$\ZZ/5\ZZ$, or the ``integers modulo $5$'':
		it is the result of declaring $5 = 0$.
		\ii Let $R = \ZZ[x]$ and $I = (x)$.
		Then $R/I$ means we send $x$ to zero; hence $R/I \cong \ZZ$
		as given any polynomial $p(x) \in R$,
		we simply get its constant term.
		\ii Let $R = \ZZ[x]$ again and now let $I = (x-3)$.
		Then $R/I$ should be thought of as the quotient when $x-3 \equiv 0$,
		that is, $x \equiv 3$.
		So given a polynomial $p(x)$ its image after
		we mod out should be thought of as $p(3)$.
		Again $R/I \cong \ZZ$, but in a different way.
		\ii Finally, let $I = (x-3,5)$.
		Then $R/I$ not only sends $x$ to three, but also $5$ to zero.
		So given $p \in R$, we get $p(3) \pmod 5$.
		Then $R/I \cong \ZZ/5\ZZ$.
	\end{enumerate}
\end{example}
By the way, given an ideal $I$ of a ring $R$, it's totally legit to write
\[ x \equiv y \pmod I \]
to mean that $x-y \in I$.
Everything you learned about modular arithmetic carries over.

\section{Principal Ideal Domains and Noetherian Rings}
\prototype{$\ZZ$ is a PID, $\ZZ[x]$ is not but is at least Noetherian. $\ZZ[x_1, x_2, \dots]$ is not Noetherian.}

What happens if we put multiple generators in an ideal, like $(10,15) \subseteq \ZZ$?
Well, we have by definition that $(10,15)$ is given as a set by
\[ 
	(10,15) \defeq \left\{ 10x + 15y \mid x,y \in \ZZ \right\}.
\]
If you're good at Number Theory you'll instantly recognize that this as just $5\ZZ = (5)$.
Surprise! In $\ZZ$, the ideal $(a,b)$ is exactly $\gcd(a,b) \ZZ$.
And that's exactly the reason you often see the GCD of two numbers denoted $(a,b)$.

We call such an ideal (one generated by a single element) a \vocab{principal ideal}.
So, in $\ZZ$, every ideal is principal.
But the same is not true in more general rings.
\begin{example}
	[A Non-Principal Ideal]
	In $\ZZ[x]$, $I = (x,2015)$ is \emph{not} a principal ideal.
	For if $I = (p)$ for some polynomial $p \in I$
	then $p$ divides $x$ and $2015$.
	This can only occur if $p = \pm 1$,
	but then $I$ contains a unit.
\end{example}
A ring with the property that all its ideals are principal is called a \vocab{principal ideal domain},
which is abbreviated PID.
We like PID's because they effectively let us take the ``greatest common factor''
in a similar way as the GCD in $\ZZ$.

If it's too much to ask that an ideal is generated by \emph{one} element,
perhaps we can at least ask that our ideals are generated by \emph{finitely many} elements.
Unfortunately, in certain weird rings this is also not the case.
\begin{example}
	[Non-Noetherian Ring]
	Consider the ring $R = \ZZ[x_1, x_2, x_3, \dots]$
	which has \emph{infinitely} many free variables.
	Then the ideal $I = (x_1, x_2, \dots) \subseteq R$
	cannot be written with a finite generating set.
\end{example}
Nonetheless, most ``sane'' rings we work in \emph{do} have the property that their ideals are finitely generated.
We now name such rings and give two equivalent definitions:
\begin{proposition}[The Equvialent Definitions of a Noetherian Ring]
	For a ring $R$, the following are equivalent.
	\begin{enumerate}[(a)]
		\ii Every ideal $I$ of $R$ is finitely generated (\emph{id est}, can be written with a finite generating set).
		\ii There does \emph{not} exist an infinite ascending chain of ideals
		\[ I_1 \subsetneq I_2 \subsetneq I_3 \subsetneq \dots. \]
		The absence of such chains is often called the \vocab{ascending chain condition}.
	\end{enumerate}
	Such rings are called \vocab{Noetherian}.
\end{proposition}
\begin{example}
	[Non-Noetherian Ring Breaks ACC]
	In the ring $R = \ZZ[x_1, x_2, x_3, \dots]$ we have
	an infinite ascending chain
	\[ (x_1) \subsetneq (x_1, x_2) \subsetneq (x_1,x_2,x_3) \subsetneq \dots. \]
\end{example}
From the example, you can kind of see why the proposition is true:
from an infinitely generated ideal you can extract an ascending chain
by throwing elements in one at a time.
I'll leave the proof to you if you want to do it.\footnote{On the other hand, every undergraduate class in this topic I've seen makes you do it as homework. Admittedly I haven't gone to that many such classes.}

\begin{ques}
	Why are fields Noetherian?
	Why are PID's (such as $\ZZ$) Noetherian?
\end{ques}
This leaves the question:
is our prototypical non-example of a PID,
$\ZZ[x]$, a Noetherian ring?
The answer is a glorious yes,
according to the celebrated Hilbert Basis Theorem.
\begin{theorem}[Hilbert Basis Theorem]
	Given a Noetherian ring $R$,
	the ring $R[x]$ is also Noetherian.
	Thus by induction, $R[x_1, x_2, \dots, x_n]$ is Noetherian
	for any integer $n$.
	\label{thm:hilbert_basis}
\end{theorem}
The proof of this theorem is really olympiad flavored, so I couldn't possibly spoil it -- I've
left it as a problem at the end of this chapter.

Noetherian rings really shine in algebraic geometry,
and it's a bit hard for me to motivate them right now,
other than to just say ``almost all rings you'll ever care about are Noetherian''.
Please bear with me!

\section{Prime Ideals}
We know that every integer can be factored (up to sign)
as a unique product of primes; for example $15 = 3 \cdot 5$
and $-10 = -2 \cdot 5$.
You might remember the proof involves the so-called Bezout's Lemma,
which essentially says that $(a,b) = (\gcd(a,b))$;
in other words we've carefully used the fact that $\ZZ$ is a PID.

It turns out that for general rings, the situation is not as nice
as factoring elements because most rings are not PID's.
The classic example of something going wrong is
\[ 6 = 2 \cdot 3 = \left( 1-\sqrt5 \right)\left( 1+\sqrt5 \right) \]
in $\ZZ[\sqrt{-5}]$.
Nonetheless, we can sidestep the issue
and talk about factoring \emph{ideals}:
somehow the example $10 = 2 \cdot 5$ should be $(10) = (2) \cdot (5)$,
which says ``every multiple of $10$ is the product of a
multiple of $2$ and a multiple of $5$''.
I'd have to tell you then how to multiply two ideals, which I do
in the chapter on unique factorization.

Let's at least figure out what primes are.
In $\ZZ$, we have that $p \neq 1$ is prime if whenever $p \mid xy$,
either $p \mid x$ or $p \mid y$.
We port over this definition to our world of ideals.
\begin{definition}
	An ideal $I$ is \vocab{proper} if $I \neq R$,
	or equivalently, $I$ contains no units.
\end{definition}
\begin{definition}
	A non-proper ideal $I$ is a \vocab{prime ideal}
	if whenever $xy \in I$, either $x \in I$ or $y \in I$.
\end{definition}
The condition that $I$ is proper is analogous to the
fact that we don't consider $1$ to be a prime number.

\begin{example}[Examples of Prime Ideals]
	\listhack
	\begin{enumerate}[(a)]
		\ii The ideal $(7)$ of $\ZZ$ is prime.
		\ii The ideal $(x)$ of $\ZZ[x]$ is prime.
		\ii The ideal $(5) = 5\ZZ + 5i\ZZ$ of $\ZZ[i]$ is \emph{not} prime, since
		the elements $3+i$ and $3-i$ have product $10 \in (5)$,
		yet neither is itself in $(5)$.
	\end{enumerate}
\end{example}
\begin{remark}
	Ideals have the nice property that they get rid of ``sign issues''.
	For example, in $\ZZ$, do we consider $-3$ to be a prime?
	When phrased with ideals, this annoyance goes away: $(-3) = (3)$.
\end{remark}

\begin{exercise}
	What do you call a ring $R$ for which the zero ideal $(0)$ is prime?
\end{exercise}

We also have the following result.
\begin{theorem}[Prime Ideal $\iff$ Quotient is Integral Domain]
	An ideal $I$ is prime if and only if $R/I$ is an integral domain.
\end{theorem}
\begin{exercise}
	(Mandatory) Convince yourself the theorem is true.
	(A possible start is to consider $R = \ZZ$ and $I = (15)$.)
\end{exercise}

I now must regrettably inform you that unique factorization is still
not true even with the notion of a ``prime'' ideal
(though again I haven't told you how to multiply two ideals yet).
But it will become true with some additional assumptions,
namely that we have a Dedekind domain.

\section{Maximal Ideals}
Here's another flavor of an ideal.
\begin{definition}
	A proper ideal $I$ of a ring $R$ is \vocab{maximal} if
	no proper ideal $J$ contains it.
\end{definition}
\begin{example}
	[Examples of Maximal Ideals]
	\listhack
	\begin{enumerate}[(a)]
		\ii The ideal $I = (7)$ of $\ZZ$ is maximal, because
		if an ideal $J$ contains $7$
		and an element $n$ not in $I$
		it must contain $\gcd(7,n) = 1$, and hence $J = \ZZ$.
		\ii The ideal $(x)$ is \emph{not} maximal in $\ZZ[x]$,
		because it's contained in $(x,5)$ (among others).
		\ii On the other hand, $(x,5)$ is indeed maximal in $\ZZ[x]$,
		as we will see in a moment.
	\end{enumerate}
\end{example}

\begin{exercise}
	What do you call a ring $R$ for which the zero ideal $(0)$ is maximal?
\end{exercise}

There's an analogous theorem to the one for prime ideals.
\begin{theorem}
	An ideal $I$ is maximal if and only if $R/I$ is a field.
\end{theorem}
\begin{proof}
	See if you can convince yourself of the following:
	there is a one-to-one correspondence between
	\begin{enumerate}[(i)]
		\ii Ideals $J$ with $I \subseteq J \subseteq R$, and
		\ii Ideals of $R/I$.
	\end{enumerate}
	You may want to start by taking $R = \ZZ$ and $I = (15)$.
	In any case, the theorem follows from this.
\end{proof}

\begin{ques}
	Show that maximal ideals are prime.
	(The proof is very short.)
\end{ques}

In practice, because modding out by generated ideals is pretty convenient,
this is a very efficient way to check whether an ideal is maximal.
\begin{example}
	[Modding Out In $\ZZ{[x]}$]
	\listhack
	\begin{enumerate}[(a)]
		\ii This instantly implies that $(x,5)$ is a maximal ideal
		in $\ZZ[x]$, because if we mod out by $x$ and $5$ in $\ZZ[x]$,
		we just get $\ZZ/5\ZZ$, which is a field.
		\ii On the other hand, modding out by just $x$ gives $\ZZ$,
		which is an integral domain but not a field; that's why $(x)$ is
		prime but not maximal.
	\end{enumerate}
\end{example}

As we say, any maximal ideal is prime.
But now note that $\ZZ$ has the special property that
all of its prime ideals are also maximal.
It's with this condition and a few other minor conditions
that you get a so-called \emph{Dedekind domain}
where prime factorization of ideals \emph{does} work.
More on that later.

\section{Problems}
Not olympiad problems, but again the spirit is very close
to what you might see in an olympiad.

\begin{sproblem}
	Prove that a finite integral domain is in fact a field!
	\label{prob:finite_domain_field}
\end{sproblem}
\begin{problem}
	Prove the Hilbert Basis Theorem.
\end{problem}
\begin{sproblem}
	Prove that any nonzero prime ideal of $\ZZ[\sqrt 2]$ is maximal.
	(This is starred since the main idea of the solution will get used extensively in
	the algebraic NT chapters, even though the problem itself won't.)
	\label{prob:dedekind_sample}
\end{sproblem}

\begin{problem}
	[Krull's Theorem]
	Let $R$ be a ring and $J$ an ideal.
	Then $J$ is contained in some maximal ideal $I$.
	\begin{hint}
		Zorn.
	\end{hint}
	\begin{sol}
		Look at the poset of \emph{proper} ideals.
		Apply Zorn's Lemma (again using a union trick to verify the condition;
		be sure to verify that the union is proper!).
	\end{sol}
\end{problem}

\begin{problem}[BC] % From Brian Chen
	\gim
	Let $A \subseteq B \subseteq C$ be rings.
	Suppose $C$ is a finitely generated $A$-module.
	Does it follow that $B$ is a finitely generated $A$-module?
	% Assume $A$ is Noetherian. Show that $B$ is finitely generated as an $A$-module.
	% Find a counterexample where $A$ is not Noetherian.
	\begin{hint}
		I think the result is true if you add the assumption $A$ is Noetherian,
		so look for trouble by picking $A$ not Noetherian.
	\end{hint}
	\begin{sol}
		Nope! Pick
		\begin{align*}
			A &= \ZZ[x_1, x_2, \dots] \\
			B &= \ZZ[x_1, x_2, \dots, \eps x_1, \eps x_2, \dots] \\
			C &= \ZZ[x_1, x_2, \dots, \eps].
		\end{align*}
		where $\eps \neq 0$ but $\eps^2 = 0$.
		I think the result is true if you add the assumption $A$ is Noetherian.
	\end{sol}
\end{problem}

%\bigskip\noindent
%Bonus problem if you've read the appendix on Zorn's Lemma: prove the following.

