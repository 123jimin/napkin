\chapter{Representations of algebras}
In the 19th century, a group was literally defined
as a subset of $\GL(n)$ or $S_n$: the word ``group'' hadn't been invented yet.
Only much later did the abstract definition of a group was given,
an abstract set $G$ which was an object in its own right.

While this abstraction is good for some reasons,
it is often also useful to work with concrete representations.
This is the subject of representation theory.
Linear algebra is easier than abstract algebra,
so if we can take a group $G$ and represent it concretely
as a set of matrices in $\opname{GL}(n)$,
this makes them easier to study.
This is the \emph{representation theory of groups}:
how can we take a group and represent its elements as matrices?

\section{Algebras}
\prototype{$k[x_1, \dots, x_n]$ and $k[G]$.}
Rather than working directly with groups from the beginning,
it will be more convenient to deal with so-called $k$-algebras.
This setting is more natural and general than that of groups,
so once we develop the theory of algebras well enough,
it will be fairly painless to specialize to the case of groups.

Colloquially,
\begin{moral}
	An associative $k$-algebra is
	a possibly noncommutative ring with a copy of $k$ inside it.
	It is thus a $k$-vector space.
\end{moral}
% In particular this makes such an algebra into a $k$-vector space.
I'll present examples before the definition:
\begin{example}
	[Examples of $k$-Algebras]
	Let $k$ be any field.
	\begin{enumerate}[(a)]
		\ii The field $k$ itself is a $k$-algebra.
		\ii The polynomial ring $k[x_1, \dots, x_n]$.
		\ii The set of $n \times n$ matrices with entries in $k$,
		which we denote by $\Mat_n(k)$.
		Note the multiplication here is not commutative.
		\ii The set $\Mat(V)$ of linear operators $T : V \to V$,
		with multiplication given by the composition of operators.
		(Here $V$ is some vector space over $k$.)
		This is really the same as the previous example.
	\end{enumerate}
\end{example}
\begin{definition}
	Let $k$ be a field.
	A \vocab{$k$-algebra} $A$ is a \emph{possibly noncommutative} ring,
	equipped with a ring homomorphism $k \to A$
	(whose image is the ``copy of $k$'').
	In particular, $1_k \mapsto 1_A$.

	Thus we can consider $k$ as a subset of $A$, and
	we then additionally require $\lambda \cdot a = a \cdot \lambda$
	for each $\lambda \in k$ and $a \in A$.

	If the multiplication operation is also commutative,
	then we say $A$ is a \vocab{commutative algebra}.
\end{definition}
\begin{definition}
	Equivalently, a \vocab{$k$-algebra} $A$ is a
	$k$-\emph{vector space} which also has an associative,
	bilinear multiplication operation (with an identity $1_A$).
	The ``copy of $k$'' is obtained by considering elements
	$\lambda 1_A$ for each $\lambda \in k$
	(i.e.\ scaling the identity by the elements of $k$,
	taking advantage of the vector space structure).
\end{definition}

\begin{abuse}
	Some authors don't require $A$ to be associative or
	an identity, so to them what we have just defined is an
	``associative algebra with $1$''.
	However, this is needlessly wordy for our purposes.
\end{abuse}

\begin{example}
	[Group algebra]
	The \vocab{group algebra} $k[G]$ is the $k$-vector space
	whose \emph{basis elements} are the elements of a group $G$,
	and where the product of two basis elements is the group multiplication.
	For example, suppose $G = \Zc 2 = \{1_G, x\}$.
	Then
	\[ k[G] = \left\{ a1_G + bx \mid a,b \in k \right\} \]
	with multiplication given by
	\[ (a1_G + bx)(c1_G+dx) = (ac+bd)1_G + (bc+ad)x. \]
\end{example}
\begin{ques}
	When is $k[G]$ commutative?
\end{ques}
The example $k[G]$ is very important,
because (as we will soon see) a representation of the algebra $k[G]$
amounts to a representation of the group $G$ itself.

It is worth mentioning at this point that:
\begin{definition}
	A \vocab{homomorphism} of $k$-algebras $A$, $B$ is a
	linear map $T : A \to B$ which respects multiplication
	(i.e.\ $T(xy) = T(x)T(y)$) and which sends $1_A$ to $1_B$.
	In other words, $T$ is both a homomorphism as a ring and as a vector space.
\end{definition}
\begin{definition}
	Given $k$-algebras $A$ and $B$, the \vocab{direct sum} $A \oplus B$
	is defined as pairs $a + b$, where addition is done in the obvious way,
	but we declare $ab = 0$ for any $a \in A$ and $b \in B$.
\end{definition}
\begin{ques}
	Show that $1_A + 1_B$ is the multiplicative identity of $A \oplus B$.
\end{ques}

\section{Representations}
\prototype{$k[S_3]$ acting on $k^{\oplus 3}$ is my favorite.}

\begin{definition}
	A \vocab{representation} of a $k$-algebra $A$
	(also called a \vocab{left $A$-module}) is:
	\begin{enumerate}[(i)]
		\ii A $k$-vector space $V$, and
		\ii An \emph{action} $\cdot$ of $A$ on $V$: thus, for every $a \in A$
		we can take $v \in V$ and act on it to get $a \cdot v$.
		This satisfies the usual axioms:
		\begin{itemize}
			\ii $(a+b) \cdot v = a \cdot v + b \cdot v$,
			and $(ab) \cdot v = a(b \cdot v)$.
			\ii $\lambda \cdot v = \lambda v$ for $\lambda \in k$.
			In particular, $1_A \cdot v = v$.
		\end{itemize}
	\end{enumerate}
\end{definition}

\begin{definition}
	The action of $A$ can be more succinctly described as saying
	that there is a \emph{$k$-algebra homomorphism} $\rho : A \to \Mat(V)$.
	(So $a \cdot v = \rho(a)(v)$.)
	Thus we can also define a \vocab{representation} of $A$ as a pair
	\[ \left( V, \rho : A \to \Mat(V) \right). \]
\end{definition}
\begin{remark}
	This is completely analogous to how a group action $G$ on a set $X$
	with $n$ elements just amounts to a group homomorphism $G \to S_n$.
\end{remark}
From this perspective, what we are really trying to do is:
\begin{moral}
	If $A$ is an algebra,
	we are trying to \emph{represent}
	the elements of $A$ as matrices.
\end{moral}

\begin{abuse}
	While a representation is a pair $(V, \rho)$
	of both the vector space $V$ and the action $\rho$,
	we frequently will just abbreviate it to ``$V$''.
	This is probably one of the worst abuses I will commit,
	but everyone does it so it's all good!
\end{abuse}


\begin{example}
	[Representations of $\Mat(V)$]
	\listhack
	\begin{enumerate}[(a)]
		\ii Let $A = \Mat_2(\RR)$.
		Then there is a representation $(\RR^{\oplus 2}, \rho)$
		where a matrix $a \in A$ just acts by $a \cdot v = \rho(a)(v) = a(v)$.

		\ii More generally, given a vector space $V$ over any field $k$,
		there is an obvious representation of $A = \Mat(V)$
		by $a \cdot v = \rho(a)(v) = a(v)$ (since $a \in \Mat(V)$).

		From the matrix perspective: if $A = \Mat(V)$,
		then we can just represent $A$ as matrices over $V$.

		\ii There are other representations of $A = \Mat_2(\RR)$.
		A silly example is the representation $(\RR^{\oplus 4}, \rho)$ given by
		\[
			\rho : 
			\begin{bmatrix} a & b \\ c & d \end{bmatrix} 
			\mapsto
			\begin{bmatrix} a & b & 0 & 0 \\ c & d & 0 & 0 \\
				0 & 0 & a & b \\ 0 & 0 & c & d \end{bmatrix} .
		\]
		More abstractly, viewing $\RR^{\oplus 4}$ as
		$(\RR^{\oplus 2}) \oplus (\RR^{\oplus 2})$,
		this is $a \cdot (v_1,v_2) = (a \cdot v_1, a \cdot v_2)$.
	\end{enumerate}
\end{example}

\begin{example}
	[Representations of polynomial algebras]
	\listhack
	\begin{enumerate}[(a)]
		\ii Let $A = k$.
		Then a representation of $k$ is just any $k$-vector space $V$.

		\ii If $A = k[x]$,
		then a representation $(V, \rho)$ of $A$
		amounts to a vector space $V$ plus the choice of
		a linear operator $T \in \Mat(V)$ (by $T = \rho(x)$).

		\ii If $A = k[x] / (x^2)$
		then a representation $(V, \rho)$ of $A$
		amounts to a vector space $V$ plus the choice of
		a linear operator $T \in \Mat(V)$ satisfying $T^2 = 0$.

		\ii We can create arbitrary ``functional equations'' with this pattern.
		For example, if $A = k[x,y] / (x^2 - x+y, y^4)$
		then representing $A$ by $V$ amounts to finding operators
		$S, T \in \Mat(V)$ satisfying $S^2 = S-T$ and $T^4 = 0$.
	\end{enumerate}
\end{example}

\begin{example}
	[Representations of groups]
	\listhack
	\begin{enumerate}[(a)]
		\ii Let $A = \RR[S_3]$.
		Then let 
		\[ V = \RR^{\oplus 3} = \{ (x,y,z) \mid x,y,z \in \RR \}. \]
		We can let $A$ act on $V$ as follows:
		given a permutation $\pi \in S_3$, we permute the corresponding
		coordinates in $V$.
		So for example, if 
		\[ \text{If } \pi = (1 \; 2)
		\text{ then } \pi \cdot (x,y,z) = (y,x,z). \]
		This extends linearly to let $A$ act on $V$,
		by permuting the coordinates.

		From the matrix perspective, what we are doing
		is representing the permutations in $S_3$
		as permutation matrices on $k^{\oplus 3}$, like
		\[ (1 \; 2)
		\mapsto \begin{bmatrix} 0&1&0 \\ 1&0&0 \\ 0&0&1 \end{bmatrix}. \]
		
		\ii More generally, let $A = k[G]$.
		Then a representation $(V, \rho)$ of $A$
		amounts to a group homomorphism $\psi : G \to \GL(V)$.
		(In particular, $\rho(1_G) = \id_V$.)
		We call this a \vocab{group representation} of $G$.
	\end{enumerate}
\end{example}
\begin{example}[Regular representation]
	Any $k$-algebra $A$ is a representation $(A, \rho)$ over itself,
	with $a \cdot b = \rho(a)(b) = ab$ (i.e.\ multiplication given by $A$).
	This is called the \vocab{regular representation}, denoted $\Reg(A)$.
\end{example}

\section{Direct sums}
\prototype{The example with $\RR[S_3]$ seems best.}
\begin{definition}
	Let $A$ be $k$-algebra and let $V = (V, \rho_V)$ and $W = (W, \rho_W)$
	be two representations of $A$.
	Then $V \oplus W$ is a representation, with action $\rho$ given by
	\[ a \cdot (v,w) = (a \cdot v, a \cdot w). \]
	This representation is called the \vocab{direct sum} of $V$ and $W$.
\end{definition}
\begin{example}
	Earlier we let $\Mat_2(\RR)$ act on $\RR^{\oplus 4}$ by
	\[
		\rho : 
		\begin{bmatrix} a & b \\ c & d \end{bmatrix} 
		\mapsto
		\begin{bmatrix} a & b & 0 & 0 \\ c & d & 0 & 0 \\
			0 & 0 & a & b \\ 0 & 0 & c & d \end{bmatrix} .
	\]
	So this is just a direct sum of two two-dimensional representations.
\end{example}
More generally, given representations $(V, \rho_V)$ and $(W, \rho_W)$
the representation $\rho$ of $V \oplus W$ looks like
\[ \rho(a) =
	\begin{bmatrix}
		\rho_V(a) & 0 \\ 0 & \rho_W(a)
	\end{bmatrix}.
\]
\begin{example}[Representation of $S_n$ decomposes]
	Let $A = \RR[S_3]$ again,
	acting via permutation of coordinates on
	\[ V = \RR^{\oplus 3} = \{ (x,y,z) \mid x,y,z \in \RR \}. \]
	Consider the two subspaces
	\begin{align*}
		W_1 &= \left\{ (t,t,t) \mid t \in \RR \right\} \\
		W_2 &= \left\{ (x,y,z) \mid x+y+z = 0 \right\}.
	\end{align*}
	Note $V = W_1 \oplus W_2$ as vector spaces.
	But each of $W_1$ and $W_2$ is a subrepresentation
	(since the action of $A$ keeps each $W_i$ in place),
	so $V = W_1 \oplus W_2$ as representations too.
\end{example}

Direct sums also come up when we play with algebras.
\begin{proposition}[Representations of $A \oplus B$ are $V_A \oplus V_B$]
	\label{prop:rep_direct_sum}
	Let $A$ and $B$ be $k$-algebras.
	Then every representation of $A \oplus B$ is of the form 
	\[ V_A \oplus V_B \]
	where $V_A$ and $V_B$ are representations of $A$ and $B$, respectively.
\end{proposition}
\begin{proof}[Sketch of Proof]
	Let $(V, \rho)$ be a representation of $A \oplus B$.
	For any $v \in V$, $\rho(1_A+1_B)v = \rho(1_A)v + \rho(1_B)v$.
	One can then set $V_A = \{ \rho(1_A)v \mid v \in V \}$
	and $V_B = \{ \rho(1_B)v \mid v \in V \}$.
	These are disjoint, since if $\rho(1_A) v = \rho(1_B) v'$,
	we have $\rho(1_A)v = \rho(1_A1_A)v = \rho(1_A1_B) v' = 0_V$,
	and similarly for the other side.
\end{proof}

\section{Irreducible and indecomposable representations}
\prototype{$k[S_3]$ decomposes as the sum of two spaces.}

One of the goals of representation theory will be to classify
all possible representations of an algebra $A$.
If we want to have a hope of doing this,
then we want to discard ``silly'' representations such as
\[
	\rho : 
	\begin{bmatrix} a & b \\ c & d \end{bmatrix} 
	\mapsto
	\begin{bmatrix} a & b & 0 & 0 \\ c & d & 0 & 0 \\
		0 & 0 & a & b \\ 0 & 0 & c & d \end{bmatrix}
\]
and focus our attention instead on ``irreducible'' representations.
This motivates:
\begin{definition}
	Let $V$ be a representation of $A$.
	A \vocab{subrepresentation} $W \subseteq V$ is a subspace $W$
	with the property that for any $a \in A$ and $w \in W$,
	$a \cdot w = w$.
	In other words, this subspace is invariant under actions by $A$.
\end{definition}
Thus if $V \neq W_1 \oplus W_2$ for representations $W_1$, $W_2$
then $W_1$ and $W_2$ are subrepresentations.

\begin{definition}
	If $V$ has no proper subrepresentations then it is \vocab{irreducible}.
	If $V \neq W_1 \oplus W_2$ for proper subrepresentations $W_1$, $W_2$,
	then we say it is \vocab{indecomposable}.
\end{definition}
\begin{definition}
	For brevity, an \vocab{irrep} of an algebra/group is a
	\emph{finite-dimensional} irreducible representation.
\end{definition}

\begin{example}[Representation of $S_n$ decomposes]
	Let $A = \RR[S_3]$ again, acting via permutation of coordinates on
	\[ V = \RR^{\oplus 3} = \{ (x,y,z) \mid x,y,z \in \RR \}. \]
	Consider again the two subspaces
	\begin{align*}
		W_1 &= \left\{ (t,t,t) \mid t \in \RR \right\} \\
		W_2 &= \left\{ (x,y,z) \mid x+y+z = 0 \right\}.
	\end{align*}
	As we've seen, $V = W_1 \oplus W_2$, and thus $V$ is not irreducible.
	But one can show that $W_1$ and $W_2$ are irreducible 
	(and hence indecomposable) as follows.
	\begin{itemize}
		\ii For $W_1$ it's obvious, since $W_1$ is one-dimensional.
		\ii For $W_2$, consider any vector $w = (a,b,c)$
		with $a+b+c=0$ and not all zero.  Then WLOG we can assume $a \neq b$ 
		(since not all three coordinates are equal).
		In that case, $(1 \; 2)$ sends $w$ to $w' = (b,a,c)$.
		Then $w$ and $w'$ span $W_2$.
	\end{itemize}
	Thus $V$ breaks down completely into irreps.
\end{example}

Unfortunately, if $W$ is a subrepresentation of $V$,
then it is not necessarily the case that we can find a
supplementary vector space $W'$ such that $V = W \oplus W'$.
Put another way, if $V$ is reducible, we know that it has a subrepresentation,
but a decomposition requires \emph{two} subrepresentations.
Here is a standard counterexample:
\begin{exercise}
	\label{exer:irred_not_indecomp}
	Let $A = \RR[x]$, and $V = \RR^{\oplus 2}$ the representation with action
	\[ \rho(x) = \begin{bmatrix} 1 & 1 \\ 0 & 1 \end{bmatrix}. \]
	Show that the only subrepresentation is $W = \{ (t,0) \mid t \in \RR \}$.
	So $V$ is not irreducible, but it is indecomposable.
\end{exercise}

Here is a slightly more optimistic example,
and the ``prototypical example'' that you should keep in mind.

\begin{exercise}
	Let $A = \Mat_d(k)$ and consider the obvious representation $k^{\oplus d}$
	of $A$ that we described earlier. Show that it is irreducible.
	(This is obvious if you understand the definitions well enough.)
\end{exercise}

\section{Morphisms of representations}
We now proceed to define the morphisms between representations.

\begin{definition}
	Let $(V, \rho_1)$ and $(W, \rho_2)$ be representations of $A$.
	An \vocab{intertwining operator}, or \vocab{morphism}, is a
	linear map $T : V \to W$ such that
	\[ T(a \cdot v) = a \cdot T(v) \]
	for any $a \in A$, $v \in V$.
	(Note the first $\cdot$ is the action $\rho_1$
	and the second $\cdot$ is the action of $\rho_2$.)
	This is exactly what you expect if you think that $V$ and $W$
	are ``left $A$-modules''.
	If $T$ is invertible, then it is an \vocab{isomorphism} of representations
	and we say $V \cong W$.
\end{definition}
\begin{remark}
	[For commutative diagram lovers]
	The condition $T(a \cdot v) = a \cdot T(v)$ can be read as saying that
	\begin{diagram}
		V & \rTo^{\rho_1(a)} & V \\
		\dTo^T & & \dTo_T \\
		W & \rTo_{\rho_2(a)} & W
	\end{diagram}
	commutes for any $a \in A$.
\end{remark}

\begin{remark}
	[For category lovers]
	A representation is just a ``bilinear'' functor from an
	abelian one-object category $\{\ast\}$ (so $\Hom(\ast, \ast) \cong A$)
	to the abelian category $\catname{Vect}_k$.
	Then an intertwining operator is just a \emph{natural transformation}.
\end{remark}

Here are some examples of intertwining operators.
\begin{example}[Intertwining operators]
	\listhack
	\begin{enumerate}[(a)]
		\ii For any $\lambda \in k$, the scalar map $T(v) = \lambda v$
		is intertwining.
		\ii If $W \subseteq V$ is a subrepresentation,
		then the inclusion $W \injto V$ is an intertwining operator.
		\ii The projection map $V_1 \oplus V_2 \surjto V_1$
		is an intertwining operator.
		\ii Let $V  = \RR^{\oplus 2}$ 
		and represent $A = k[x]$ by $(V, \rho)$ where
		\[ \rho(x) = \begin{bmatrix} 0 & 1 \\ -1 & 0 \end{bmatrix}. \]
		Thus $\rho(x)$ is rotation by $90\dg$ around the origin.
		Let $T$ be rotation by $30\dg$.
		Then $T : V \to V$ is intertwining (the rotations commute).
	\end{enumerate}
\end{example}

\begin{exercise}[Kernel and Image Are Subrepresentations]
	Let $T : V \to W$ be an intertwining operator.
	\begin{enumerate}[(a)]
		\ii Show that $\ker T \subseteq V$ is a subrepresentation of $V$.
		\ii Show that $\img T \subseteq W$ is a subrepresentation of $W$.
	\end{enumerate}
\end{exercise}

The previous lemma gives us the famous Schur's lemma.
\begin{theorem}
	[Schur's lemma]
	Let $V$ and $W$ be representations of a $k$-algebra $A$.
	Let $T : V \to W$ be a \emph{nonzero} intertwining operator.
	Then
	\begin{enumerate}[(a)]
		\ii If $V$ is irreducible, then $T$ is injective.
		\ii If $W$ is irreducible, then $T$ is surjective.
	\end{enumerate}
	In particular if both $V$ and $W$ are irreducible then $T$
	is either zero or an isomorphism.
\end{theorem}
An important special case is if $k$ is algebraically closed:
then the only intertwining operators $V \to V$
are the trivial constant ones.
\begin{theorem}
	[Schur's lemma for algebraically closed fields]
	Let $k$ be an algebraically closed field.
	Let $V$ be an irrep of a $k$-algebra $A$.
	Then any intertwining operator $T : V \to V$ is multiplication by a scalar.
	\label{thm:schur_algclosed}
\end{theorem}
\begin{exercise}
	Use the fact that $T$ has an eigenvalue $\lambda$ to
	deduce this from Schur's lemma.
	(Consider $T - \lambda \cdot \id_V$, and use Schur to deduce it's zero.)
\end{exercise}
We have already seen the counterexample of rotation by $90\dg$ for $k = \RR$;
this was the same counterexample we gave to the assertion that all linear maps
have eigenvalues.

\section{The representations of $\Mat_d(k)$}
To give an example of the kind of progress already possible, we prove:
\begin{theorem}
	[Representations of $\Mat_d(k)$]
	\label{thm:rep_1mat}
	Let $k$ be any field, $d$ be a positive integer and
	let $W = k^{\oplus d}$ be the obvious representation of $A = \Mat_d(k)$.
	Then the only finite-dimensional representations
	of $\Mat_d(k)$ are $W^{\oplus n}$
	for some positive integer $n$ (up to isomorphism).
	In particular, it is irreducible if and only if $n=1$.
\end{theorem}
For concreteness, I'll just sketch the case $d=2$,
since the same proof applies verbatim to other situations.
This shows that the examples of representations of $\Mat_2(\RR)$
we gave earlier are the only ones.

As we've said this is essentially a functional equation.
The algebra $A = \Mat_2(k)$ has basis given by four matrices
\[
	E_1 = \begin{bmatrix} 1 & 0 \\ 0 & 0 \end{bmatrix},
	\qquad
	E_2 = \begin{bmatrix} 0 & 0 \\ 0 & 1 \end{bmatrix},
	\qquad
	E_3 = \begin{bmatrix} 0 & 1 \\ 0 & 0 \end{bmatrix},
	\qquad
	E_4 = \begin{bmatrix} 0 & 0 \\ 1 & 0 \end{bmatrix}
\]
satisfying relations like $E_1 + E_2 = \id$, $E_i^2 = E_i$, $E_1E_2 = 0$, etc.
So let $V$ be a representation of $A$, and let $M_i = \rho(E_i)$ for each $i$;
we want to classify the possible matrices $M_i$ on $V$
satisfying the same functional equations.
This is because, for example,
\[ \id_V = \rho(\id_A) = \rho(E_1+E_2) = M_1 + M_2. \]
By the same token $M_1M_3 = M_3$.
Proceeding in a similar way, we can obtain the following multiplication table:
\[
	\begin{array}{r|llll}
		\times & M_1 & M_2 & M_3 & M_4 \\ \hline
		M_1 & M_1 & 0 & M_3 & 0 \\
		M_2 & 0 & M_2 & 0 & M_4 \\
		M_3 & 0 & M_3 & 0 & M_1 \\
		M_4 & M_4 & 0 & M_2 & 0
	\end{array}
	\qquad \text{and} \qquad
	M_1 + M_2 = \id_V
\]
Note that each $M_i$ is a linear operator $V \to V$;
for all we know, it could have hundreds of entries.
Nonetheless, given the multiplication table of the basis $E_i$
we get the corresponding table for the $M_i$.

So, in short, the problem is as follows:
\begin{moral}
	Find all vector spaces $V$ and quadruples of matrices $M_i$
	satisfying the multiplication table above.
\end{moral}

Let $W_1 = M_1``(V)$ and $W_2 = M_2``(V)$ be the images of $M_1$ and $M_2$.
\begin{claim}
	$V = W_1 \oplus W_2$.
\end{claim}
\begin{proof}
	First, note that for any $v \in V$ we have
	\[ v = \rho(\id)(v) = (M_1+M_2)v = M_1v + M_2v. \]
	Moreover, we have that $W_1 \cap W_2 = \{0\}$, because if
	$M_1v_1 = M_2v_2$ then $M_1v_1 = M_1(M_1v_1) = M_1(M_2v_2) = 0$.
\end{proof}
\begin{claim}
	$W_1 \cong W_2$.
\end{claim}
\begin{proof}
	Check that the maps 
	\[ W_1 \taking{\times M_4} W_2
		\quad\text{and}\quad
		W_2 \taking{\times M_3} W_1 \]
	are well-defined and mutually inverse.
\end{proof}
Now, let $e_1, \dots, e_n$ be basis elements of $W_1$;
thus $M_4e_1$, \dots, $M_4e_n$ are basis elements of $W_2$.
However, each $\{e_j, M_4e_j\}$ forms a basis of a subrepresentation
isomorphic to $W = k^{\oplus 2}$ (what's the isomorphism?).

This finally implies that all representations of $A$
are of the form $W^{\oplus n}$.
In particular, $W$ is irreducible because there are no representations
of smaller dimension at all!

\section\problemhead
\begin{dproblem}
	\label{prob:one_dim}
	Suppose we have \emph{one-dimensional} representations
	$V_1 = (V_1, \rho_1)$ and $V_2 = (V_2, \rho_2)$ of $A$.
	Show that $V_1 \cong V_2$ if and only if
	$\rho_1(a)$ and $\rho_2(a)$ are multiplication
	by the same constant for every $a \in A$.
\end{dproblem}

\begin{dproblem}
	[Schur's lemma for commutative algebras]
	Let $A$ be a \emph{commutative} algebra
	over an algebraically closed field $k$.
	Prove that any irrep of $A$ is one-dimensional.
	\begin{hint}
		For any $a \in A$, the map $v \mapsto a \cdot v$ is intertwining.
	\end{hint}
\end{dproblem}

\begin{sproblem}
	\label{prob:reg_mat}
	Let $(V, \rho)$ be a representation of $A$.
	Then $\Mat(V)$ is a representation of $A$
	with action given by 
	\[ a \cdot T = \rho(a) \circ T \]
	for $T \in \Mat(V)$.
	\begin{enumerate}[(a)]
		\ii Show that $\rho : \Reg(A) \to \Mat(V)$ is an intertwining operator.
		\ii If $V$ is $d$-dimensional, show that $\Mat(V) \cong V^{\oplus d}$
		as representations of $A$.
	\end{enumerate}
	\begin{hint}
		For part (b), pick a basis and do $T \mapsto (T(e_1), \dots, T(e_n))$.
	\end{hint}
\end{sproblem}

\begin{sproblem}
	\label{prob:regA_intertwine}
	Fix an algebra $A$.
	Find all intertwining operators 
	\[ T : \Reg(A) \to \Reg(A). \]
	\begin{hint}
		Right multiplication.
	\end{hint}
	\begin{sol}
		The operators are those of the form $T(a) = ab$
		for some fixed $b \in A$.
		One can check these work, since for $c \in A$
		we have $T(c \cdot a) = cab = c \cdot T(a)$.
		To see they are the only ones, note that
		$T(a) = T(a \cdot 1_A) = a \cdot T(1_A)$ for any $a \in A$.
	\end{sol}
\end{sproblem}

\begin{problem}
	\gim
	Let $(V, \rho)$ be an \emph{indecomposable}
	(not irreducible) representation of an algebra $A$.
	Prove that any intertwining operator $T : V \to V$
	is either nilpotent or an isomorphism.

	(Note that \Cref{thm:schur_algclosed} doesn't apply,
	since the field $k$ may not be algebraically closed.)
	\begin{hint}
		Apply \Cref{prob:endomorphism_eventual_lemma}.
	\end{hint}
\end{problem}
