\chapter{Representation Theory: Maschke and Schur}
\todo{adaptation}

Historically, group theory came up when we were studying matrices.
We literally had ``groups'' of linear maps that satisfied the group axioms,
but were not commutative.
Representation theory brings us back in history: given a group $G$,
we try to express it as a collection of a bunch of matrices.

\section{Definition and Examples}
Fix a ground field $k$ (for all vector spaces).  
Let $G$ be a group.

\begin{definition}
	A \textbf{representation} of $G$
	consists of a pair $\rho = (V, \cdot_\rho)$
	where $V$ is a vector space over $k$a
	and $\cdot_\rho$ is a group action of $G$ on $V$ which is linear in $V$.
	If $V$ is finite-dimensional then the \textbf{dimension} of $\rho$ is just the dimension of $V$.
\end{definition}
Explicitly the conditions on $\cdot_\rho$ are that
\[
	\begin{aligned}
		1 \cdot_\rho v &= v \\
		g_1 \cdot_\rho (g_2 \cdot_\rho v) &= (g_1g_2) \cdot_\rho v \\
		g_1 \cdot_\rho (v_1 + v_2) &= g \cdot_\rho v_1 + g \cdot_\rho v_2 \\
		g_1 \cdot_\rho (cv) &= c(g \cdot_\rho v).
	\end{aligned}
\]

\begin{ques}
	If you've read the chapter on category theory:
	show that a representation is the same as a functor from $G$
	to $\mathbf{Vect}_k$.
\end{ques}

By abuse of notation, we occasionally refer to $\rho$ by just its underlying vector space $V$ in the case that $\cdot_\rho$ is clear from context.
We may also abbreviate $g \cdot_\rho v$ as just $g \cdot v$.

A simple example of a nontrivial representation is the following.
\begin{example}
	If $V = k^{n}$ and $G = S_n$,
	then an example of an action is $\rho = (V, \cdot_\rho)$
	is simply \[ \sigma \cdot_\rho \left<x_1, \dots, x_n\right> =
	\left<x_{\sigma(1)}, \dots, x_{\sigma(n)}\right> \]
	meaning we permute the basis elements of $V$.
	We denote this representation by $\refl_n$.
\end{example}
Let us give another useful example.
\begin{definition}
	Let $X$ be a set acted on by $G$.
	We define the vector space
	\[ \Fun(X) \defeq \left\{ \text{maps } X \to k \right\} \]
	with the standard addition of functions.
\end{definition}
\begin{example}
	We define a representation on $\Fun(X)$ by the following action: every $f \in \Fun(X)$ gets sent to a $g \cdot f \in \Fun(X)$ by
	\[ (g \cdot_{\Fun(X)} f)(x) = f\left( g\inv \cdot_X x \right). \]
	By abuse of notation we will let $\Fun(X)$ refer both to the vector space and the corresponding representation.
\end{example}
Now that we have two nontrivial examples, we also give a trivial example.
\begin{example}
	Let $G$ be a group.
	We define the \textbf{trivial representation} $\triv_G$, or just $\triv$, as the representation $\triv_G = (k, \cdot_\triv)$, where
	\[ g \cdot_\triv a = a \]
	for every $a \in k$.
	In other words, $G$ acts trivially on $k$.
\end{example}

\section{Homomorphisms of Representations}
Just like with every other structure we've considered, we
can consider the structure-preserving maps between representations.

\begin{definition}
	Let $\rho_1 = (V_1, \cdot_{\rho_1})$ and $\rho_2 = (V_2, \cdot_{\rho_2})$ be representations of the same group $G$.
	A \textbf{homomorphism of $G$-representations} is a linear map $T : V_1 \to V_2$ which respects the $G$-action: for any $g \in G$ and $v \in V$,
	\[ g \cdot_{\rho_2} T(v) = T\left( g \cdot_{\rho_1} v \right). \]
	The set of all these homomorphisms is written $\Hom_G(\rho_1, \rho_2)$,
	which is itself a vector space over $k$.
\end{definition}

\begin{ques}
	Again, for those of you that have read the category theory:
	show that this is just a natural transformation.
\end{ques}

To see an example of this definition in action, we give the following as an exercise.
\begin{proposition}
	Let $\rho = \left( V, \cdot_\rho \right)$.
	We define the \textbf{$G$-invariant} space $\rho^G \subseteq V$ to be
	\[ \rho^G \defeq \left\{ v \in V \mid g \cdot_\rho v = v \; \forall g \in G  \right\}. \]
	Then there is a natural bijection of vector spaces $\Hom_G(\triv_G, \rho) \simeq \rho^G$.
\end{proposition}
\begin{proof}
	Let $\rho = (V, \cdot_\rho)$.
	The set $\Hom_G(\triv_G, \rho)$ consists of maps $T : k \to V$ with
	\[ g \cdot_\rho T(a) = T(g \cdot_{\triv} a) = T(a) \]
	for every $a \in k$.
	Since $T : k \to V$ is linear, it is uniquely defined by $T(1)$
	(since $T(a) = a T(1)$ in general).
	So $g \cdot_\rho T(1) = T(1)$, i.e.\ $T(1) \in \rho^G$,
	is necessary and sufficient.
	Thus the bijection is just $T \mapsto T(1)$.
\end{proof}
This proposition will come up again at the end of Part 4.

\section{Subrepresentations, Irreducibles, and Maschke's Theorem}
Now suppose I've got a representation $\rho = (V, \cdot_\rho)$.
\begin{definition}
	Suppose we have a subspace $W \subseteq V$ which is $\rho$-invariant, meaning that $g \cdot_\rho w \in W$ for every $w \in W$ and $g \in G$.
	Then we can construct a representation of $G$ on $W$ by restricting the action to $W$:
	\[ \rho' = \left( W, \cdot_\rho|_W \right). \]
	In that case the resulting $\rho$ is called a \textbf{subrepresentation} of $V$.
\end{definition}
Every $\rho$ has an obvious subrepresentation, namely $\rho$ itself,
as well as a stupid subrepresenation on the zero-dimensional vector space $\{0\}$.
But it's the case that some representations have \emph{interesting} subrepresentations.
\begin{example}
	Consider the representation $\refl = (k^n, \cdot)$ of $S_n$ on $k^n$ defined in the first section.
	For all $n \ge 2$, $\refl$ is not irreducible.
\end{example}
\begin{proof}
	Consider the subspace $W \subset k^n$ given by
	\[ W = \left\{ (x_1, x_2, \dots, x_n) \mid x_1 + \dots + x_n = 0. \right\} \]
	then $W$ is invariant under $\refl$, so we have a subrepresentation
	of $\refl$, which we'll denote $\refl_0$.
\end{proof}

This motivates the ideas of irreducibles.
\begin{definition}
	A representation $\rho$ is \textbf{irreducible} if it has no nontrivial subrepresentations.
\end{definition}

Of course the first thing we ask is whether any representation decomposes as a product of irreducible representations.
But what does it mean to compose two representations, anyways? It's just the ``natural'' definition with the \href{http://en.wikipedia.org/wiki/Direct_sum}{direct sum}.
\begin{definition}
	Let $\rho_1 = (W_1, \cdot_{\rho_1})$ and $\rho_2 = (W_2, \cdot_{\rho_2})$ be representations and suppose we have $V = W_1 \oplus W_2$.
	Then we define the representation $\rho = \rho_1 \oplus \rho_2$
	by $\rho = (V, \cdot_\rho)$ where
	\[
		g \cdot_\rho (w_1 + w_2)
		= (g \cdot_{\rho_1} w_1)
		+ (g \cdot_{\rho_2} w_2).
	\]
\end{definition}
Just like every integer decomposes into prime factors, we hope that every representation decomposes into irreducibles.
But this is too much to hope for.
\begin{example}
	Let $G = S_2$, let $k = \mathbb F_2$ be the finite field of order $2$ (aka $\ZZ/2\ZZ$), and consider $\refl = (k^2, \cdot)$, which is not irreducible.
	However, we claim that we cannot write $\refl = \rho_1 \oplus \rho_2$
	for \emph{any} nontrivial $\rho_1$ and $\rho_2$.
\end{example}
\begin{proof}
	This is a good concrete exercise.

	Assume not, and let $V_1$ and $V_2$ be the underlying vector spaces of $\rho_1$ and $\rho_2$.
	By nontriviality, $\dim V_1 = \dim V_2 = 1$,
	and in particular we have that as sets, $\left\lvert V_1 \right\rvert = \left\lvert V_2 \right\rvert = 2$.
	Take the only nonzero elements $(a,b) \in V_1$ and $(c,d) \in V_2$.
	Since $V_1$ is invariant under $\refl$, $(b,a) \in V_1$, so $(a,b) = (b,a) \implies (a,b) = (1,1) \in V_1$.
	Similarly, $(1,1) \in V_2$, which is impossible.
\end{proof}
So we hoped for perhaps too much.
However, with seemingly trivial modifications we can make the above example work.
\begin{example}
	In the same example as above, suppose we replace $k$ with any field which does not have characteristic $2$. Then $\rho$ does decompose.
\end{example}
\begin{proof}
	Consider the following two subspaces of $V = k^2$:
	\[
		\begin{aligned}
			W_1 &= \left\{ \left<a, a\right> \mid a \in k \right\} \\
			W_2 &= \left\{ \left<a, -a\right> \mid a \in k \right\}.
		\end{aligned}
	\]
	It's easy to see that both $W_1$ and $W_2$ are both invariant under $\rho$.
	Moreover, if $\opname{char} k \neq 2$ then we in fact have
	\[ V = W_1 \oplus W_2 \]
	because $\left<x,y\right> = \half \left<x+y, x+y\right> + \half \left<x-y, y-x\right>$ for any $x,y \in k$.
	So if we let $\rho_1 = (W_1, \cdot_{\rho_1})$ be the subrepresentation corresponding to $W_1$, and define $\rho_2$ on $W_2$ similarly,
	then we have $\rho = \rho_1 \oplus \rho_2$.
\end{proof}

Thus the only thing in the way of the counterexample was the fact that $\opname{char} k = 2$. And it turns out in general this is the only obstacle, a result called Maschke's Theorem.
\begin{theorem}
	[Maschke's Theorem]
	Suppose that $G$ is a finite group, and $\opname{char} k$ does not divide $\left\lvert G \right\rvert$.
	Then every finite-dimensional representation decomposes as a direct sum of irreducibles.
\end{theorem}
Before proceeding to the proof, I'll draw an analogy between the proof that every positive integer $m$ decomposes as the product of primes. We use by strong induction on $m$; if $m$ is prime we are done, and if $m$ is composite there is a nontrivial divisor $d \mid m$, so we apply the inductive hypothesis to $d$ and $m/d$ and combine these factorizations. We want to mimic the proof above in our proof of Maschke's Theorem, but we have a new obstacle: we have to show that somehow, we can ``divide''.

So why is it that we can divide in certain situations?
The idea is that we want to be able to look at an ``average'' of the form
\[ \frac{1}{\left\lvert G \right\rvert} \sum_{g \in G} g \cdot v \]
because this average has the nice property of being $G$-invariant.
We'll use this to obtain our proof of Maschke's Theorem.

\begin{proof}
	We proceed by induction on the dimension of the representation $\rho$.
	Let $\rho = (V, \cdot_\rho)$ be a representation and assume its not irreducible, so it has a nontrivial subspace $W$ which is $\rho$-invariant.
	It suffices to prove that there exists a subspace $W' \subset V$ such that $W'$ is also $\rho$-invariant and $V = W \oplus W'$, because then we can apply the inductive hypothesis to the subrepresentations induced by $W$ and $W'$.

	Let $\pi : V \to W$ be any projection of $V$ onto $W$.
	We consider the \emph{averaging} map $T : V \to V$ by
	\[ 
		T(v) = \frac{1}{\left\lvert G \right\rvert}
		\sum_{g \in G} g\inv \cdot_\rho \pi(g \cdot_\rho v).
	\]
	We'll use the following properties of the map.
	\begin{exercise}
		Show that the map $T$ has the following three properties.
		\begin{itemize}
			\ii For any $w \in W$, $T(w) = w$.
			\ii For any $v \in V$, $T(w) \in W$.
			\ii $T \in \Hom_G(\rho, \rho)$.
		\end{itemize}
	\end{exercise}
	As with any projection map $T$, we must have $V = \ker T \oplus \img T$.
	But $\img T = W$.
	Moreover, because the map $T$ is $G$-invariant, it follows that $\ker T$ is $\rho$-invariant.
	Hence taking $W' = \ker T$ completes the proof.
\end{proof}

This completes our proof of Maschke's Theorem, telling us how all irreducibles decompose.
Said another way, Maschke's Theorem tells us that any finite-dimensional representation $\rho$ can be decomposed as
\[
	\bigoplus_{\rho_\alpha \in \Irrep(G)} \rho_{\alpha}^{\oplus n_\alpha}
\]
where $n_\alpha$ is some nonnegative integer, and $\Irrep(G)$ is the set of all (isomorphism classes of) irreducibles representations.

This begs the question: what is $n_\alpha$?
Is it even uniquely determined by $\rho$?

\section{Schur's Lemma}
To answer this I first need to compute $\Hom_G(\rho, \pi)$ for any two distinct irreducible representations $\rho$ and $\pi$.
One case is easy.
\begin{lemma}
	Let $\rho$ and $\pi$ be non-isomorphic irreducible representations (not necessarily finite dimensional).
	Then there are no nontrivial homomorphisms $\phi : \rho \to \pi$.
	In other words, $\Hom_G(\rho, \pi) = \{0\}$.
\end{lemma}
I haven't actually told you what it means for representations to be isomorphic, but you can guess -- it just means
that there's a homomorphism of $G$-representations between them which is also a bijection of the underlying vector spaces.

\begin{proof}
	Let $\phi : \rho_1 \to \rho_2$ be a nonzero homomorphism.
	We can actually prove the following stronger results.
	\begin{itemize}
		\ii If $\rho_2$ is irreducible then $\phi$ is surjective.
		\ii If $\rho_1$ is irreducible then $\phi$ is injective.
	\end{itemize}
	\begin{ques}
		Prove the above two results.
		(Hint: show that $\img \phi$ and $\ker \phi$ give rise to subrepresentations.)
	\end{ques}
	Combining these two results gives the lemma because $\phi$ is now a bijection,
	and hence an isomorphism.
\end{proof}

Thus we only have to consider the case $\rho \simeq \pi$.
The result which relates these is called Schur's Lemma, but is important enough that we refer to it as a theorem.
\begin{theorem}
	[Schur's Lemma]
	Assume $k$ is algebraically closed.
	Let $\rho$ be a finite dimensional irreducible representation.
	Then $\Hom_{G} (\rho, \rho)$ consists precisely of maps of the form $v \mapsto \lambda v$, where $\lambda \in k$; the only possible maps are multiplication by a scalar.
	In other words, \[ \Hom_{G} (\rho, \rho) \simeq k \]
	and $\dim \Hom_G(\rho, \rho) = 1$.
\end{theorem}

\begin{proof}[Proof]
	One direction is trivial:
	\begin{ques}
		Check that any map $v \mapsto \lambda v$ respects the $G$-action.
	\end{ques}

	Let's prove the other direction now.
	Suppose $T : V \to V$ respects the $G$-action.
	The trick is the following result, which I'll just quote without proof.

	\begin{quote}
		In an algebraically closed field, there exists a nontrivial subspace $V_\lambda$ such that $T$ is just multiplication by $\lambda$,
		called an \emph{eigenvalue}.
	\end{quote}
	Thus $V_\lambda$ is a nontrivial subspace over which $T$
	is multiplication by a constant.
	But then $V_\lambda$ is a $G$-invariant subspace of $V$! Since $\rho$ is irreducible, this can only happen if $V = V_\lambda$.
	That means $T$ is multiplication by $\lambda$ for the entire space $V$.
\end{proof}

This is NOT in general true without the algebraically closed condition, as the following example shows.
\begin{example}
	Let $k = \RR$, let $V = \RR^2$,
	and let \[ G = \ZZ_3 = \left<g \mid g^3=1\right> \]
	act on $V$ by rotating every $\vec x \in \RR^2$
	by $120\dg$ around the origin, giving a representation $\rho$.
	This representation is irreducible because for any $v \neq 0$,
	the vectors $v$ and $g \cdot v$ are linearly independent.

	We claim $\rho$ is a counterexample to Schur's Lemma.
	We can regard now $\rho$ as a map in $\CC$ which is multiplication by $e^{\frac{2\pi i}{3}}$.
	Then for any other complex number $\xi$, the map ``multiplication by $\xi$'' commutes with the map ``multiplication by $e^{\frac{2\pi i}{3}}$''.
	So in fact
	\[ \Hom_G(\rho, \rho) \simeq \CC \]
	which has dimension $2$.
\end{example}

\section{Computing dimensions of homomorphisms}
Since we can now compute the dimension of the $\Hom_G$ of any two irreducible representations,
we can compute the dimension of the $\Hom_G$ for any composition of irreducibles, as follows.
\begin{corollary}
	We have
	\[
		\dim \Hom_G
		\left( \bigoplus_\alpha \rho_\alpha^{\oplus n_\alpha},
		\bigoplus_\beta \rho_\beta^{\oplus m_\beta} \right)
		= \sum_{\alpha} n_\alpha m_\alpha
	\]
	where the direct sums run over the isomorphism classes of irreducibles.
\end{corollary}
\begin{proof}
	The $\Hom$ just decomposes over each of the components as
	\[
		\begin{aligned}
		\Hom_G
		\left( \bigoplus_\alpha \rho_\alpha^{\oplus n_\alpha},
		\bigoplus_\beta \rho_\beta^{\oplus m_\beta} \right)
		&\simeq
		\bigoplus_{\alpha, \beta}
		\Hom_G(\rho_\alpha^{\oplus n_\alpha}, \rho_\beta^{\oplus m_\beta}) \\
		&\simeq
		\bigoplus_{\alpha, \beta}
		\Hom_G(\rho_\alpha, \rho_\beta)^{\oplus n_\alpha m_\alpha}.
		\end{aligned}
	\]
	Here we're using the fact that $\Hom_G(\rho_1 \oplus \rho_2, \rho) = \Hom_G(\rho_1, \rho) \oplus \Hom_G(\rho_2, \rho)$ (obvious) and its analog.
	The claim follows from our lemmas now.
\end{proof}
As a special case of this, we can quickly derive the following.
\begin{corollary}
	Suppose $\rho = \bigoplus_\alpha \rho_\alpha^{n_\alpha}$ as above.
	Then for any particular $\beta$,
	\[ n_\beta = \dim \Hom_G(\rho, \rho_\beta). \]
\end{corollary}

This settles the ``unique decomposition'' in the affirmative. Hurrah!

It might be worth noting that we didn't actually need Schur's Lemma if we were
solely interested in uniqueness, since without it we would have obtained \[ n_\beta = \frac{\dim \Hom_G(\rho, \rho_\beta)}{\dim \Hom_G(\rho_\beta, \rho_\beta)}. \]
However, the denominator in that expression is rather unsatisfying, don't you think?

\section{Problems}
\todo{write them}
