\chapter{Finite fields}
In this short chapter, we classify all fields with finitely many elements
and compute the Galois groups.
Nothing in here is very hard, and so most of the proofs are just sketches;
if you like, you should check the details yourself.

The whole point of this chapter is to prove the following results:
\begin{itemize}
	\ii A finite field $F$ must have order $p^n$, with $p$ prime and $n$ an integer.
	\ii In this case, $F$ has characteristic $p$.
	\ii The extension $F/\FF_p$ is Galois, and $\Gal(F/\FF_p)$ is a cyclic group of order $n$.
	The generator is the automorphism \[ \sigma : F \to F \quad\text{by}\quad x \mapsto x^p. \]
\end{itemize}
If you're in a hurry you can just remember these results and skip to the next chapter.

\section{Example of a finite field}
Before diving in, we give some examples.

Recall that the \emph{characteristic} of a field $F$ is the smallest positive integer
$n$ such that \[ \underbrace{1_F + \dots + 1_F}_{\text{$n$ times}} = 0 \]
or $0$ if no such integer exists.

\begin{example}[Base field]
	Let $\FF_p$ denote the field of integers modulo $p$.
	This is a field with $p$ elements, with characteristic $p$.
\end{example}

\begin{example}[The finite field of nine elements]
	Let
	\[ F \cong \FF_3[X]/(X^2+1) \cong \ZZ[i] / (3). \]
	We can think of its elements as \[ \left\{ a + bi \mid 0 \le a,b \le 2 \right\}. \]
	Since $(3)$ is prime in $\ZZ[i]$, the ring of integers of $\QQ(i)$,
	we see $F$ is a field with $3^2 = 9$ elements inside it.
	Note that, although this field has $3$ elements, every element $x$ has the property that
	\[ 3x = \underbrace{x + \dots + x}_{\text{$3$ times}} = 0. \]
	In particular, $F$ has characteristic $3$.
\end{example}

\section{Finite fields have prime power order}
\begin{lemma}
	If the characteristic of $F$ isn't zero, it must be a prime number.
\end{lemma}
\begin{proof}
	Assume not, so $n = ab$ for $a,b < n$.
	Then let
	\[ A = \underbrace{1_F + \dots + 1_F}_{\text{$a$ times}} \neq 0 \]
	and
	\[ B = \underbrace{1_F + \dots + 1_F}_{\text{$b$ times}} \neq 0. \]
	Then $AB = 0$, contradicting the fact that $F$ is a field.
\end{proof}

We like fields of characteristic zero, but unfortunately for finite fields
we are doomed to have nonzero characteristic.

\begin{lemma}
	[Finite fields have prime power orders]
	Let $F$ be a finite field of characteristic $p$.
	Then
	\begin{enumerate}[(a)]
		\ii $p$ is not zero, and hence prime.
		\ii $F$ is a finite extension of $\FF_p$,
		and in particular it is an $\FF_p$ vector space.
		\ii $\left\lvert F \right\rvert = p^n$ for some prime $p$, integer $n$.
	\end{enumerate}
\end{lemma}
\begin{proof}
	Very briefly, since this is easy:
	\begin{enumerate}[(a)]
		\ii Apply Lagrange's theorem (or pigeonhole principle, they're equivalent here)
		to $(F, +)$ to get the character isn't zero.
		\ii The additive subgroup of $(F,+)$ generated by $1_F$ is an isomorphic copy of $\FF_p$.
		\ii Since it's a field extension, $F$ is a finite-dimensional vector space over $\FF_p$,
		with some basis $e_1, \dots, e_n$.
		It follows that there are $p^n$ elements of $F$. \qedhere
	\end{enumerate}
\end{proof}
\begin{remark}
	An amusing alternate proof of (c) by contradiction:
	if a prime $q \neq p$ divides $\left\lvert F \right\rvert$, then
	by Cauchy's theorem (\Cref{thm:cauchy_group}) on $(F, +)$
	there's a (nonzero) element $x$ of order $q$.
	Evidently \[ x \cdot ( \underbrace{1 + \dots + 1}_{\text{$q$ times}} ) = 0 \] 
	then, but $x \neq 0$, and hence the characteristic of $F$ also divides $q$, which is impossible.
\end{remark}

An important point in the above proof is that
\begin{lemma}[Finite fields are field extensions of $\FF_p$]
	If $\left\lvert F \right\rvert = p^n$ is a finite field,
	then there is an isomorphic copy of $\FF_p$ sitting inside $F$.
	Thus $F$ is a field extension of $\FF_p$.
\end{lemma}

We want to refer a lot to this copy of $\FF_p$, so in what follows:
\begin{abuse}
	Every integer $n$ can be identified as an element of $F$, namely
	\[ n \defeq \underbrace{1 + \dots + 1}_{\text{$n$ times}}. \]
	Note that (as expected) this depends only on $n \pmod p$.
\end{abuse}

Convince yourself that this lets us do something like the following:
\begin{theorem}
	[Freshman's dream]
	For any $a,b \in F$ we have
	\[ (a+b)^p = a^p + b^p. \]
\end{theorem}
\begin{proof}
	Use the Binomial theorem, and the fact that $\binom pi$ is divisible by $p$ for $0 < i < p$.
\end{proof}
\begin{exercise}
	Convince yourself that this proof works.
\end{exercise}

\section{All finite fields are isomorphic}
We next proceed to prove ``Fermat's little theorem'':
\begin{theorem}
	[Fermat's little theorem in finite fields]
	Let $F$ be a finite field of order $p^n$.
	Then every element $x \in F$ satisfies
	\[ x^{p^n} - x = 0. \]
\end{theorem}
\begin{proof}
	If $x = 0$ it's true; otherwise, use Lagrange's theorem
	on the abelian group $(F, \times)$ to get $x^{p^n-1} = 1_F$.
\end{proof}

We can now prove:
\begin{theorem}[Complete classification of finite fields]
	A field $F$ is a finite field with $p^n$ elements if and only if
	it is a splitting field of $x^{p^n}-x$ over $\FF_p$.
\end{theorem}
\begin{proof}
	By ``Fermat's little theorem'', all the elements of $F$ satisfy this polynomial.
	So we just have to show that the roots of this polynomial are distinct (i.e. that it is separable)
	Then we'll be done.

	To do this, we do the derivative trick again: the derivative of this polynomial is
	\[ p^n \cdot x^{p^n-1} - 1  = -1 \]
	which has no roots at all, so the polynomial cannot have any double roots. \qedhere
\end{proof}

Note that the polynomial $x^{p^n}-x \pmod p$ is far from irreducible, but
the computation above shows that it's separable.
\begin{example}[The finite field of order nine again]
	The polynomial $x^9-x$ is separable modulo $3$ and has factorization
	\[ x(x+1)(x+2)(x^2+1)(x^2+x+2)(x^2+2x+2) \pmod 3. \]

	So if $F$ has order $9$, then we intuitively expect it to be the field
	generated by adjoining all the roots: $0$, $1$, $2$, as well as
	$\pm i$, $1 \pm i$, $2 \pm i$.
	Indeed, that's the example we had at the beginning of this chapter.

	This is a little imprecise since the complex number $i$ doesn't really make sense
	(just what does ``$i \pmod 3$'' mean?), but it carries the right idea.
\end{example}


\section{The Galois theory of finite fields}
Let $F$ be a finite field of $p^n$ elements.
By the above theorem, it's the splitting field of a separable polynomial,
hence we know that $F/\FF_p$ is a Galois extension.
We would like to find the Galois group.

In fact, we are very lucky: it is cyclic.
First, we exhibit one such element $\sigma_p \in \Gal(F/\FF_p)$:

\begin{theorem}[The $p$th power automorphism]
	The map $\sigma_p : F \to F$ defined by
	\[ \sigma_p(x) : x \mapsto x^p \]
	is an automorphism, and moreover fixes $\FF_p$.
\end{theorem}
\begin{proof}
	It's a homomorphism since it fixes $1_F$, respects multiplication,
	and respects addition.
	\begin{ques}
		Why does it respect addition?
	\end{ques}
	Next, we claim that it is injective. To see this, note that
	\[ x^p = y^p
		\iff x^p - y^p = 0
		\iff (x-y)^p = 0
		\iff x=y.
	\]
	Here we have again used the Freshman's Dream.
	Since $F$ is finite, this injective map is automatically bijective.
	The fact that it fixes $\FF_p$ is Fermat's little theorem.
\end{proof}

Now we're done:
\begin{theorem}
	[Galois group of the extension $F/\FF_p$]
	Let $F$ have size $p^n$. Then $\Gal(F/\FF_p) \cong \Zc n$
	with generator $\sigma_p$.
\end{theorem}
\begin{proof}
	Since $[F:\FF_p] = n$, the Galois group $G$ has order $n$.
	So we just need to show $\sigma_p \in G$ has order $n$.

	Note that $\sigma_p$ applied $k$ times gives $x \mapsto x^{p^k}$.
	Hence, $\sigma_p$ applied $n$ times is the identity, as all elements of $F$ satisfy $x^{p^n}=x$.
	But if $k < n$, then $\sigma_p$ applied $k$ times
	cannot be the identity or $x^{p^k}-x$ would have too many roots.
\end{proof}

We can see an example of this again with the finite field of order $9$.
\begin{example}
	[Galois group of finite field of order $9$]
	Let $F$ be the finite field of order $9$, represented as $K = \ZZ[i]/(3)$.
	Let $\sigma_3 : F \to F$ be $x \mapsto x^3$. We can witness the fate of all nine elements:
	\begin{diagram}
		0 & 1 & 2 & i & 1+i & 2+i \\
		&&& \dIsom^\sigma & \dIsom^\sigma & \dIsom^\sigma \\
		&&& -i & 1-i & 2-i \\
	\end{diagram}
	(As claimed, $0$, $1$, $2$ are the fixed points, so I haven't drawn arrows for them.)
	As predicted, the Galois group has order two:
	\[ \Gal(K/\FF_3) = \left\{ \id, \sigma_3 \right\} \cong \Zc 2. \]
\end{example}

This concludes the proof of all results stated at the beginning of this chapter.

\section\problemhead
\todo{borrow some from Victor Wang?}
