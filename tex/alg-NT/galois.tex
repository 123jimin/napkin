\chapter{Things Galois}
%This chapter is mostly optional.
%Read the first two sections and then decide
%whether you want to read the rest of this chapter.

\section{Motivation}
There is a second way to interpret the ``conjugates'' I talked about before.
Let $K = \QQ(i)$, for example.
Then we can speak of \emph{embeddings} $\sigma : K \injto \CC$ 
these are maps with the following two properties.
\begin{itemize}
	\ii They are field homomorphisms, so $\sigma(x+y)=\sigma(x)+\sigma(y)$ and $\sigma(xy)=\sigma(x)\sigma(y)$.
	\ii The map $\sigma$ is the identity on $\QQ$.
\end{itemize}
For $K$, there are two such embeddings:
\begin{itemize}
	\ii $\sigma_1 : K \injto \CC$ the identity map: send $z$ to $z$.
	\ii $\sigma_2 : K \injto \CC$ by complex conjugation: send $z$ to $\ol z$.
\end{itemize}
Note that \emph{each embedding sends $\alpha \in K$ to one of its Galois conjugates}.
This is true in general: if we write down the minimal polynomial
\[ 0 = c_n \alpha^n + c_{n-1} \alpha^{n-1} + \dots + c_1\alpha + c_0 \]
where the $c_i$ are rational, then by applying any embedding $\sigma$ to both sides gives
\begin{align*}
	0 &= \sigma(c_n \alpha^n + c_{n-1} \alpha^{n-1} + \dots + c_1\alpha + c_0) \\
	% &= \sigma(c_n \alpha^n) + \sigma(c_{n-1} \alpha^{n-1})
	% + \dots + \sigma(c_1\alpha) + \sigma(c_0) \\
	&= \sigma(c_n) \sigma(\alpha)^n + \sigma(c_{n-1}) \sigma(\alpha)^{n-1}
	+ \dots + \sigma(c_1)\sigma(\alpha) + \sigma(c_0) \\
	&= c_n \sigma(\alpha)^n + c_{n-1} \sigma(\alpha)^{n-1} + \dots + c_1\sigma(\alpha) + c_0
\end{align*}
where in the last step we have used the fact that $c_i \in \QQ$, so they are fixed by $\sigma$.

Galois Theory aims to develop these ideas in full generality, replacing $\QQ$ with any base field $F$,
and $K$ with any ``extension'' of $F$.
If you substitute $F = \QQ$ everywhere in this chapter,
you'll see how this connects with the content of algebraic number theory.
This also explains the notation $\Tr_{K/\QQ}(\alpha)$;
with the necessary tools we can define $\Tr_{K/F}(\alpha)$ for any $\alpha \in K$,
and $F$ an extension of $K$.

\section{Outline of This Chapter}
To do this total generalization, we do the following.
\begin{itemize}
	\ii First, we generalize $K/\QQ$ to any $K/F$, by just saying $K$ is an $F$-vector space.
	\ii We then algebraic closure for $K$, the analog of $\CC$ for $\RR$.
	\ii We try to prove that irreducible polynomials can't have double roots,
	but find it fails for stupid reasons, leading us to need a new adjective, ``separable''.
	\ii With this adjective, we examine embeddings into algebraic closures, like $K \injto \CC$.
	\ii We take a detour into Galois theory by looking at embeddings $K \injto K$.
\end{itemize}
The main results of this chapter are that
\begin{itemize}
	\ii There are $n$ embeddings from $K$ into $\CC$, where $K$ has degree $n$.
	\ii An extension $K/\QQ$ is \emph{Galois} if the images of all these embeddings lie in $K$.
	For example, $\QQ(i)$ is Galois, but $K = \QQ(\sqrt[3]{2})$ is not, since there's an embedding
	$\sigma : K \to \CC$ which sends $\sqrt[3]{2}$ to $\omega \sqrt[3]{2}$, where $\omega = e^{\frac23 \pi i}$.
	\ii Fundamental Theorem of Galois Theory.
\end{itemize}

%If you're willing to accept the boxed result above on faith
%(which IMO is totally reasonable),
%then you can skip the rest of this chapter for a while;
%the Galois material won't be used again until we start trying to
%study ramification theory.
%Even then, you'll really only need to know the definition of a Galois group
%and how to detect whether an extension is Galois (which I've already explained),
%so it might suffice to just skim the relevant definitions.


\section{Field Extensions}
\prototype{$\QQ(\sqrt[3]{2})/\QQ$ or something similar.}
First, we define a notion of one field sitting inside another.
\begin{definition}
	Let $K$ and $F$ be fields.
	If $F \subseteq K$, we write $K/F$ and say $K$ is a
	\vocab{field extension} of $F$.
	
	Thus $K$ is automatically an $F$-vector space
	(just like $\QQ(\sqrt 2)$ is automatically a $\QQ$-vector space).
	The \vocab{degree} is the dimension of this space, denoted $[K:F]$.
	If $[K:F]$ is finite, we say $K/F$ is a \vocab{finite (field) extension}.
\end{definition}
That's really all. There's nothing tricky at all.

\begin{ques}
	What do you call a finite extension of $\QQ$?
\end{ques}

Degrees of finite extensions are multiplicative.
\begin{theorem}[Field Extensions Have Multiplicative Degree]
	Let $F \subseteq K \subseteq L$ be fields with $L/K$, $K/F$ finite. Then
	\[ [L:K][K:F] = [L:F]. \]
\end{theorem}
\begin{proof}
	Basis bash: you can find a basis of $L$ over $K$, and then expand that into a basis $K$ over $F$.
	Details are left to a very diligent reader.
\end{proof}

\section{Algebraic Closures}
\prototype{$\ol \RR = \CC$.}
At this point, I may as well also tell you about algebraic closures.
\begin{theorem}[Algebraic Closures]
	Let $F$ be a field.
	Then there exists a field extension $\ol F$ containing $F$, the \vocab{algebraic closure},
	such that all polynomials in $\ol F[x]$ factor completely.
\end{theorem}
\begin{example}
	[Some Algebraic Closures]
	\listhack
	\begin{enumerate}[(a)]
	\ii The canonical example, of course, is that the algebraic closure of $\RR$ is $\CC$.
	% The theorem says that if every \emph{real} polynomial factors in $\CC$, then it factors in $\RR$.
	\ii The algebraic closure of $\QQ$ is the algebraic numbers by definition,
	which explains the notation $\ol\QQ$ we used earlier.
	\end{enumerate}
\end{example}

I may as well define a splitting field now.
\begin{definition}
	Let $F$ be a field and $p \in F$ a polynomial of degree $n$.
	Then $p$ has roots $\alpha_1, \dots, \alpha_n$ in the algebraic closure of $F$.
	The \vocab{splitting field} of $F$ is defined as $F(\alpha_1, \dots, \alpha_n)$.
\end{definition}
In other words, the splitting field is the smallest field in which $p$ splits.
\begin{example}[Examples of Splitting Fields]
	\listhack
	\begin{enumerate}[(a)]
		\ii The splitting field of $x^2 - 5$ over $\QQ$ is $\QQ(\sqrt 5)$.
		This is a degree $2$ extension.
		\ii The splitting field of $x^2+x+1$ over $\QQ$ is $\QQ(\omega)$,
		where $\omega$ is a cube root of unity.
		This is a degree $3$ extension.
		% In particular, the splitting field of $x^3-2$ is a degree \emph{six} extension.
		\ii The splitting field of $x^2+3x+2 = (x+1)(x+2)$ is just $\QQ$!
		There's nothing to do.
	\end{enumerate}
\end{example}
\begin{example}
	[Splitting Fields: A Cautionary Tale]
	The splitting field of $x^3 - 2$ over $\QQ$ is in fact
	\[ \QQ\left( \sqrt[3]{2}, \omega \right) \]
	and not just $\QQ(\sqrt[3]{2})$!
	One must really adjoin \emph{all} the roots, and it's not necessarily the case that
	these roots will generate each other.

	To be clear:
	\begin{itemize}
	\ii For $x^2-5$, we adjoin $\sqrt 5$ and this will automatically include $-\sqrt 5$.
	\ii For $x^2+x+1$, we adjoin $\omega$ and get the other root $\omega^2$ for free.
	\ii But if we adjoin $\sqrt[3]{2}$, we do NOT get $\omega\sqrt[3]{2}$ and $\omega^2\sqrt[3]{2}$ for free:
	indeed, $\QQ(\sqrt[3]{2}) \subset \RR$!
	\end{itemize}
	Note that in particular, the splitting field of $x^3-2$ over $\QQ$ is \emph{degree six}.
\end{example}

In general,
\textbf{the splitting field of a polynomial can be an extension of degree up to $n!$}.
The reason is that if $p(x)$ has $n$ roots and none of them are ``related'' to each other,
then any permutation of the roots will work.

\section{Everyone Hates Characteristic 2: Separable vs Irreducible}
\prototype{$\QQ$ has characteristic zero, hence irreducible polynomials are separable.}
We wish to prove that
\begin{quote}
	\itshape Irreducible polynomials have no double roots.
\end{quote}
% That's important, since our embeddings are sending elements to their conjugates.
Let's call a polynomial with no double roots \vocab{separable};
thus we want irreducible $\implies$ separable.
%For example, $x^2+1 \in \QQ[x]$ and $x^2-5x+6 \in \QQ[x]$
%are separable but $x^2-2x+1 \in \QQ[x]$ are not.
%Note that irreducible polynomials in $\QQ$ are separable.
We did this for $\QQ$ in the last chapter (\Cref{lem:irred_complex}) by taking derivatives.
Should work for any field, right?

Nope.
Suppose we took the derivative of some polynomial like $2x^3 + 24x + 9$,
namely $6x^2 + 24$.
In $\CC$ is it's obvious that the derivative of a nonconstant polynomial $f'$ isn't zero.
But suppose we considered the above as a polynomial in $\FF_3$, i.e.\ modulo three.
Then the derivative is zero.
Oh, no!

We impose a condition that prevents something like this from happening.
\begin{definition}
	For a field $F$, let $p$ be the smallest positive integer such that
	\[ \underbrace{1+\dots+1}_{\text{$p$ times}} = 0 \]
	or zero if no such integer $p$ exists.
	Then we say $F$ has \vocab{characteristic} $p$.
\end{definition}
\begin{example}[Field Characteristics]
	Old friends $\RR$, $\QQ$, $\CC$ all have characteristic zero.
	But $\FF_p$, the integers modulo $p$, is a field of characteristic $p$.
\end{example}
\begin{exercise}
	Let $F$ be a field of characteristic $p$.
	Show that if $p > 0$ then $p$ is a prime number.
	(A proof is given next chapter.)
\end{exercise}
With the assumption of characteristic zero, our earlier proof works.
\begin{lemma}[Separability in Characteristic Zero]
	Any irreducible polynomial in a characteristic zero field is separable.
\end{lemma}
Unfortunately, this lemma is false if the ``characteristic zero'' condition is dropped.
So in what follows, I'll have the word ``separable'' attached to a lot of things.

The reason it's called \emph{separable} is (I think) is this picture:
I have a polynomial and I want to break it into irreducible parts.
Normally, if I have a double root in a polynomial, that means it's not irreducible.
But in characteristic $p > 0$ this fails.
So inseparable polynomials are strange when you think about them: somehow
you have double roots that can't be separated from each other.


\section{Embeddings Into Algebraic Closures}
Now that I've defined all these ingredients, I can give the following theorem.
\begin{theorem}[The $n$ embeddings of a number field]
	\label{thm:n_embeddings}
	Let $K$ be a number field.
	Then there are exactly $n$ field homomorphisms $K \injto \CC$,
	say $\sigma_1, \dots, \sigma_n$ which fix $\QQ$.
\end{theorem}
\begin{remark}
	Note that a nontrivial homomorphism of fields is necessarily injective
	(the kernel is an ideal).
	This justifies the use of ``$\injto$'', and we call each $\sigma_i$ an
	\vocab{embedding} of $\QQ$ into $\CC$.
\end{remark}
\begin{proof}
	This is kind of fun.
	Recall that any irreducible polynomial over $\QQ$ has distinct roots.
	We'll adjoin elements $\alpha_1, \alpha_2, \dots, \alpha_n$ one at a time to $\QQ$,
	until we eventually get all of $K$, that is, 
	\[ K = \QQ(\alpha_1, \dots, \alpha_n). \]
	Diagrammatically, this is
	\begin{diagram}
		\QQ & \rInj & \QQ(\alpha_1) & \rInj & \QQ(\alpha_1, \alpha_2) & \rInj & \dots & \rInj & K \\
		\dInj^\id && \dInj^{\tau_1} && \dInj^{\tau_2} && \dots && \dInj_{\tau_n = \sigma} \\
		\CC & \rTo & \CC & \rTo & \CC & \rTo & \dots & \rTo & \CC \\
	\end{diagram}

	First, we claim there are exactly \[ [\QQ(\alpha_1) : \QQ] \] ways to pick $\tau_1$.
	Observe that $\tau_1$ is determined by where it sends $\alpha_1$ (since it has to fix $\QQ$).
	Letting $p_1$ be the minimal polynomial of $\alpha_1$, we see that there are $\deg p_1$ choices for $\tau_1$,
	one for each (distinct) root of $p_1$. That proves the claim.

	Similarly, given a choice of $\tau_1$, there are
	\[ [\QQ(\alpha_1, \alpha_2) : \QQ(\alpha_1)] \]
	ways to pick $\tau_2$.
	(It's a little different: $\tau_1$ need not be the identity.
	But it's still true that $\tau_2$ is determined by where it sends $\alpha_2$,
	and as before there are $[\QQ(\alpha_1, \alpha_2) : \QQ(\alpha_1)]$ possible ways.)

	Multiplying these all together gives the desired $[K:\QQ]$.
\end{proof}
\begin{remark}
	A shorter proof can be given using the Primitive Element Theorem.
	Do you see how?
\end{remark}

It's common to see expressions like ``let $K$ be a number field of degree $n$,
and $\sigma_1, \dots, \sigma_n$ its $n$ embeddings'' without further explanation.
The relation between these embeddings and the Galois conjugates is given as follows.
\begin{theorem}[Embeddings Are Evenly Distributed Over Conjugates]
	Let $K$ be a number field of degree $n$ with $n$ embeddings $\sigma_1$, \dots, $\sigma_n$,
	and let $\alpha \in K$ have $m$ Galois conjugates over $\QQ$. 

	Then $\sigma_j(\alpha)$ is ``evenly distributed'' over each of these $m$ conjugates:
	for every $i$, exactly $\frac nm$ of the embeddings send $\alpha$ to $\beta_i$.
\end{theorem}
\begin{proof}
	In the previous proof, adjoin $\alpha_1 = \alpha$ first.
\end{proof}

We can get this to work for any field extension in which separability is not an issue.
\begin{definition}
	A \vocab{separable extension} $K/F$ is one in which every irreducible
	polynomial in $F$ is separable (for example, if $F$ has characteristic zero).
\end{definition}
\begin{theorem}[The $n$ Embeddings of Any Separable Extension]
	Let $K/F$ be a separable extension of degree $n$ and let $\ol F$ be an algebraic closure of $F$.
	Then there are exactly $n$ field homomorphisms $K \injto \ol F$,
	say $\sigma_1, \dots, \sigma_n$ which fix $F$.
\end{theorem}

This lets us define the trace for \emph{any} separable normal extension.
\begin{definition}
Let $K/F$ be a separable extension of degree $n$, and let $\sigma_1$, \dots, $\sigma_n$
be the $n$ embeddings into an algebraic closure of $F$. Then we define
\[
	\Tr_{K/F}(\alpha) = \sum_{i=1}^n \sigma_i(\alpha)
	\quad\text{and}\quad
	\Norm_{K/F}(\alpha) = \prod_{i=1}^n \sigma_i(\alpha).
\]
When $F = \QQ$ and the algebraic closure is $\CC$, this coincides with our earlier definition!
\end{definition}

%%fakesection Norms and Trace (deleted)
%\section{Norms and Trace Revisited}
%\prototype{Look at $\Norm_{\QQ(i) / \QQ}(2)$.}
%We now define the trace and norm for any $K/F$, where $K$ and $F$ are both \emph{number fields}.
%Thus we have a tower of fields
%\[ \QQ \subseteq F \subseteq K. \]
%
%\begin{definition}
%	Let $K/F$ be an extension of number fields $K$ and $F$, and let $\alpha \in K$.
%	We set \[ \mu_\alpha : K \to K \quad\text{by}\quad x \mapsto \alpha x \]
%	and view it as a linear map of $K$-vector spaces.
%	Then $\Tr_{K/F}(\alpha)$ and $\Norm_{K/F}(\alpha)$ are defined as the trace
%	and norm of this map, respectively.
%\end{definition}
%Our work earlier is the special case $F = \QQ$.
%
%%\begin{example}
%%	[Running a determinant in $\QQ(i)$]
%%	Note that $\{1, i\}$ is a basis of $K = \QQ(i)$ thought of as a $\QQ$-vector space.
%%	Let $\alpha = a+bi$.
%%	Since $(a+bi)(x+yi) = (ax-by) + (bx+ay)i$, we find that $\mu_\alpha$ can be written as a matrix
%%	\[
%%		\mu_\alpha = 
%%		\left(
%%		\begin{array}{cc}
%%			a & b \\
%%			-b & a
%%		\end{array}
%%		\right)
%%	\]
%%	and so $\Norm_{K/\QQ} (\alpha) = a^2+b^2$, which is what we expected.
%%\end{example}
%
%
%\begin{example}
%	[Norms in $\QQ(i)$ as product of Galois Conjugates]
%	Let $K = \QQ(i)$, $F = \QQ$, so $[K:F] = 2$.
%	Then the two $\sigma$'s are simply the identity and complex conjugation.
%	So
%	\[ \Norm_{K/\QQ}(\alpha) = \alpha \cdot \ol\alpha. \]
%	For most complex numbers $a+bi$, this gives $(a+bi)(a-bi) = a^2+b^2$.
%	For real numbers like $2$, we have $2 \cdot 2 = 4$.
%	The feeling is that $K/\QQ$ has degree $2$, while the number $2$ itself has degree $1$
%	(and hence no Galois conjugates);
%	thus we ``compensate'' by squaring: $2^2 = 4$.
%	In general, if $K / \QQ$ has degree $n$ and $\alpha$ has degree $m$,
%	we compensate by raising to the $\frac nm$th power.
%\end{example}
%\begin{proof}
%	The proof is exactly the same as before, but with a small difference.
%	Suppose $\alpha$ has minimal polynomial with degree $n$, say
%	\[ x^n + c_{n-1}x^{n-1} + \dots + c_0. \]
%	It's no longer the case that $\{1, \alpha, \dots, \alpha^{n-1}\}$ is a basis of $K$.
%	Certainly, though, we have 
%	\[ \QQ \subseteq \QQ(\alpha) \subseteq K. \]
%
%	Let $k = [K : \QQ(\alpha)]$, so $nk = [K : \QQ]$.
%	Then you can take a basis of the form 
%	\[ \{\alpha^i \beta_j \mid 0 \le i < n, 1 \le j \le k \} \]
%	under which the matrix $\mu_\alpha$ looks like
%	\[
%		\mu_\alpha
%		=
%		\underbrace{
%		\left(
%		\begin{array}{cccc}
%			B & & & \\
%			& B & & \\
%			& & \ddots & \\
%			& & & B
%		\end{array}
%		\right)
%		}_{\text{$k$ copies}}
%		\quad
%		\text{where}
%		\quad
%		B = 
%		\left(
%		\begin{array}{cccccc}
%			0 & 0 & 0 & \dots & 0 & -c_0 \\
%			1 & 0 & 0 & \dots & 0 & -c_1 \\
%			0 & 1 & 0 & \dots & 0 & -c_2 \\
%			\vdots & \vdots & \vdots & \ddots & 0 & -c_{n-2} \\
%			0 & 0 & 0 & \dots & 1 & -c_{n-1}
%		\end{array}
%		\right).
%	\]
%	Note that $B$ is our matrix from before!
%	So 
%	\[ \Tr_{K/\QQ} \mu_\alpha = k \cdot \Tr_{\QQ(\alpha)/\QQ}(\alpha)
%		\quad\text{and}\quad
%		\det_{K/\QQ} \mu_\alpha = \Norm_{\QQ(\alpha)/\QQ}(\alpha)^k. \]
%	Use now our original result with $\Tr_{\QQ(\alpha)/\QQ}$ and $\Norm_{\QQ(\alpha)/\QQ}$
%	as the sum and product of the Galois conjugates:
%	by adapting the proof of the previous theorem with $\alpha_1 = \alpha$,
%	we get that the $\sigma$'s are evenly distributed over each of $\alpha$'s Galois conjugates,
%	giving the compensation of $k$.
%\end{proof}
%From the proof of the above we can isolate a corollary which really
%emphasizes how the norm and trace are the product and sum of Galois conjugates.
%\begin{corollary}[Norm/Trace in terms of Galois Conjugates]
%	Let $K/\QQ$ be an extension of number fields, pick $\alpha \in K$,
%	and let $k = [K : \QQ(\alpha)]$. Then
%	\[ \Norm_{K/\QQ}(\alpha) = \left( \prod \text{Galois conj} \right)^k
%		\quad\text{and}\quad
%		\Tr_{K/\QQ}(\alpha) = k \cdot \sum \text{Galois conj} \]
%	and in particular they are both integers.
%\end{corollary}

\section{Automorphism Groups}
\prototype{There are two ways of embedding $\QQ(i)$ into itself.}
Now, rather than embedding into $\QQ(i) \to \CC$ we will look at isomorphisms $\QQ(i) \to \QQ(i)$.
As before, there are only two, $i \mapsto \pm i$.
We can interpret this as saying
\begin{moral}
As far as $\QQ$ can tell, the two roots of $x^2+1$ are the same.
\end{moral}
Let's now do this in greater generality.

\begin{definition}
	Suppose $K/F$ is an finite extension.
	Then $\Aut(K/F)$ is the set of \emph{field isomorphisms} $\sigma : K \to K$ which fix $F$.
	In symbols
	\[ \Aut(K/F) =
		\left\{
		\sigma : K \to K \mid
		\text{$\sigma$ is identity on $F$}
	  \right\}.
	\]
	It is a group under function composition!
\end{definition}

Our previous work shows, \[ \Aut(\QQ(i) / \QQ) \cong \Zc 2 \]
as we described in the beginning of the chapter.
Here's a second example.
\begin{example}[Galois Group of $\QQ(\sqrt2,\sqrt3)$]
	Let $K = \QQ(\sqrt2, \sqrt3)$.
	It turns out that $\Gal(K/\QQ) = \{1, \sigma, \tau, \sigma\tau\}$, where
	\[
		\sigma :
		\begin{cases}
			\sqrt2 &\mapsto -\sqrt2 \\
			\sqrt3 &\mapsto \sqrt3
		\end{cases}
		\quad\text{and}\quad
		\tau :
		\begin{cases}
			\sqrt2 &\mapsto \sqrt2 \\
			\sqrt3 &\mapsto -\sqrt3.
		\end{cases}
	\]
	In other words, $\Aut(K/\QQ)$ is the Klein Four Group.
\end{example}

Notice that elements are getting sent to one of their Galois conjugates.
This is true in general, exactly as before:
\begin{lemma}
	[Root Shuffling in $\Aut(K/F)$]
	Let $f \in F[x]$, suppose $K/F$ a finite extension, and assume $\alpha \in K$ is a root of $f$.
	Then for any $\sigma \in \Aut(K/F)$, $\sigma(\alpha)$ is also a root of $f$.
	\label{lem:root_shuffle}
\end{lemma}
\begin{proof}
	Let $f(x) = c_nx^n + c_{n-1}x^{n-1} + \dots + c_0$, where $c_i \in F$.
	Thus,
	\[ 0 = \sigma(f(\alpha)) = \sigma\left( c_n\alpha^n + \dots + c_0 \right)
	= c_n\sigma(\alpha)^n + \dots + c_0 = f(\sigma(\alpha)). \qedhere \]
\end{proof}
In particular, taking $f$ to be the minimal polynomial of $\alpha$ we deduce
\begin{moral}
	An embedding $\sigma \in \Aut(K/F)$ sends an $\alpha \in F$
	one of its various Galois conjugates (over $F$).
\end{moral}


\section{Galois Extensions}
\prototype{Splitting fields like $\QQ(i)$ are Galois. The typical non-example is $\QQ(\sqrt[3]{2})$.}
Let's observe something else.
In both the earlier examples we see that $\Aut(K/\QQ)$ has $[K : \QQ]$ elements.
Indeed, this makes sense: if $K$ is a number field and there are $n$ embeddings of it into $\CC$,
shouldn't we expect that $\Aut(K/\QQ)$ has $n$ automorphisms?
Alas, this is not the case, and again the counterexample is our classic $\QQ(\cbrt{2})$.
\begin{example}
	[Galois Group of $\QQ(\cbrt{2})$]
	Let $K = \QQ(\cbrt{2})$.
	An embedding $\sigma : K \to K$ is determined by where it sends $\cbrt{2}$,
	and $\sigma(\cbrt{2})$ must be a root of $x^3-2$.
	But $x^3-2$ only has one root in $K$,
	the other roots are complex!

	Thus the only embedding is the identity, and
	\[ \Aut(K/\QQ) = \{1\} \]
	is the trivial group.
\end{example}
So the \emph{reason} this fails is that some of the Galois conjugates of $\cbrt{2}$
don't like in $K$.
In fact, it turns out this is the only reason it might fail,
though the proof is longer I have patience for.
\begin{theorem}[Galois Extensions are Splitting]
	For finite extensions $K/F$, 
	$\left\lvert \Aut(K/F) \right\rvert$ divides $[K:F]$,
	with equality if and only if $K$ is the \emph{splitting field}
	of some separable polynomial with coefficients in $F$.
	\label{thm:Galois_splitting}
\end{theorem}
% The proof is given in an optional section at the end of the chapter.
In short,
\begin{moral}
	Galois extensions $K/F$ are the ones such that $\sigma``(K) \subseteq K$
	for each of the $n$ embeddings $\sigma$.
\end{moral}
So things like $\QQ(\sqrt 2)$ and $\QQ(i)$ work, but $\QQ(\sqrt[3]{2})$ fails,
since the embedding taking $\sqrt[3]{2}$ to a complex number fail to be contained in $\QQ(\sqrt[3]{2})$.

\begin{definition}
	If $K/F$ is a finite extension and $\left\lvert \Aut(K/F) \right\rvert = [K:F]$,
	we say the extension $K/F$ is \vocab{Galois}.
	In that case, we denote $\Aut(K/F)$ by $\Gal(K/F)$ instead
	and call this the \vocab{Galois group} of $K/F$.
\end{definition}

\begin{example}
	[Examples and Non-Examples of Galois extensions]
	\listhack
	\begin{enumerate}[(a)]
		\ii The extension $\QQ(\sqrt2) / \QQ$ is Galois,
		since it's the splitting field of $x^2-2$ over $\QQ$.
		The Galois group has order two, $\sqrt 2 \mapsto \pm \sqrt 2$.
		\ii The extension $\QQ(\sqrt2, \sqrt 3) / \QQ$ is Galois,
		since it's the splitting field of $(x^2-5)^2-6$ over $\QQ$.
		As discussed before, the Galois group is $\Zc 2 \times \Zc 2$.
		\ii The extension $\QQ(\sqrt[3]{2}) / \QQ$ is \emph{not} Galois.
	\end{enumerate}
\end{example}

Here is some more intuition on what $[K:F]$ actually measures: suppose $K$ is a splitting field
of some $(x-\alpha_1) \dots (x-\alpha_n)$, meaning $K = F(\alpha_1, \dots, \alpha_n)$.
Then a permutation $\sigma \in \Aut(K/F)$ is determined by where it sends each $\alpha_i$.
The dimension of $[K:F]$ measures how much ``redundancy'' there is among the $\alpha_i$.
For example, in the case of \[ (x-\sqrt5)(x+\sqrt5) = x^2-5  \]the $\sqrt 5$ and $-\sqrt 5$ were redundant,
in the sense that knowing $\sigma(\sqrt 5)$ tells you $\sigma(-\sqrt 5) = -\sigma(\sqrt 5)$.
But in the $x^3-2$ case, knowing $\sigma(\sqrt[3]{2})$ does \emph{not} tell you where
$\omega\sqrt[3]{2}$ should go; this is reflected in the fact that $[K:F]$ and $\Aut(K/F)$
are both six rather than three.


%\begin{definition}
%	Let $K$ be a field.
%	Then an \vocab{automorphism} of $K$ is a field isomorphism $\sigma : K \to K$
%	(in particular $\sigma(1_K) = 1_K$).
%	The set of all automorphisms is denoted $\Aut(K)$ and it is a group
%	under function composition; we call it the \vocab{automorphism group}.
%\end{definition}
%\begin{example}[Automorphism Group of $\QQ(i)$]
%	The map $\sigma : \QQ(i) \to \QQ(i)$ by $z \mapsto \ol z$ (complex conjugation) 
%	is an automorphism, as is $z \mapsto z$.
%	I claim these are the only such automorphisms, so $\Aut(\QQ(i)) \cong \ZZ/2\ZZ$.
%
%	First, $\sigma$ satisfies Cauchy's Functional Equation when restricted to $\QQ$;
%	since $\sigma(1) = 1$, we derive that $\sigma$ fixes all rational numbers.
%	So once we decide where $i$ goes, we can put
%	\[ \sigma(a+bi) = \sigma(a) + \sigma(b)\sigma(i). \]
%	Since $i^2+1 = 0$, we have $\sigma(i)^2 + 1 = 0$.
%	Hence $\sigma(i) = \pm i$, as desired.
%\end{example}
%The above proof just expresses the idea that
%an automorphism sends elements to their ``conjugates''.

\section{Fundamental Theorem of Galois Theory}
After all this stuff about Galois Theory, I might as well tell you the Fundamental Theorem,
though I won't prove it.
The key idea is to look at \emph{fixed fields}.

\begin{definition}
	Let $K$ be a field and $H$ a group of automorphisms $K \to K$.
	We define the \vocab{fixed field} of $H$, denoted $K^H$, as
	\[ K^H \defeq \left\{ x \in K : \sigma(x) \in K \; \forall \sigma \in H \right\}. \]
\end{definition}
\begin{ques}
	Verify quickly that $K^H$ is actually a field.
\end{ques}

Consider $K = \QQ(\sqrt2, \sqrt3)$ again,
where \[ G = \Gal(K/\QQ) = \{\id, \sigma, \tau, \sigma\tau\} \]
is the Klein four group (where $\sigma(\sqrt2) = -\sqrt 2$ but $\sigma(\sqrt 3) = \sqrt 3$;
$\tau$ goes the other way).
\begin{ques}
	Let $H = \{\id, \sigma\}$. What is $K^H$?
\end{ques}
In that case, the diagram of fields between $\QQ$ and $K$
matches exactly with the subgroups of $G$, as follows:
\begin{center}
\begin{minipage}[t]{4cm}
	\begin{diagram}
		& \QQ(\sqrt2, \sqrt 3) & \\
		\ldLine(1,2) & \dLine & \rdLine(1,2) \\
		\QQ(\sqrt2) & \QQ(\sqrt 6) & \QQ(\sqrt 3) \\
		& \ldLine(1,2) \dLine \rdLine(1,2) & \\
		& \QQ & 
	\end{diagram}
\end{minipage}
\qquad
\begin{minipage}[t]{4cm}
	\begin{diagram}
		& \{\id\} & \\
		\ldLine(1,2) & \dLine & \rdLine(1,2) \\
		\{\id,\tau\} & \;\; \{\id, \sigma\tau\} \;\; & \{\id,\sigma\} \\
		& \ldLine(1,2) \dLine \rdLine(1,2) & \\
		& G & 
	\end{diagram}
\end{minipage}
\end{center}
Here subgroups correspond to fixed field.
That, and much more, holds in general.

\begin{theorem}[Fundamental Theorem of Galois Theory]
	Let $K/F$ be a Galois extension with Galois group $G = \Gal(K/F)$.
	\begin{enumerate}[(a)]
	\ii There is a bijection between field towers $K \subseteq E \subseteq F$ and subgroups $H \subseteq G$:
	\[
		\left\{
		\begin{array}{c}
			K \\ \mid \\ E \\ \mid \\ F
		\end{array}
		\right\}
		\iff
		\left\{
		\begin{array}{c}
			1 \\ \mid \\ H \\ \mid \\ G
		\end{array}
		\right\}
	\]
	The bijection sends $H$ to its fixed field $K^H$, and hence is inclusion reversing.
	\ii Under this bijection, we have $[K:E] = \left\lvert H \right\rvert$ and $[E:F] = [G:H]$.
	\ii $K/E$ is always Galois, and its Galois group is $\Gal(K/E) = H$.
	\ii $E/F$ is Galois if and only if $H$ is normal in $G$. If so, $\Gal(E/F) = G/H$.
	\end{enumerate}
\end{theorem}

%%fakesection Deleted Proof: Galois Extenisons are Splitting
%\section{(Optional) Proof that Galois Extensions are Splitting}
%We prove \Cref{thm:Galois_splitting}.
%First, we extract a useful fragment from the Fundamental Theorem.
%\begin{theorem}[Fixed Field Theorem]
%	\label{thm:fixed_field_theorem}
%	Let $K$ be a field and $G$ a subgroup of $\Aut(K)$.
%	Then $[K:K^G] = \left\lvert G \right\rvert$.
%\end{theorem}
%
%The inequality itself is not difficult:
%\begin{exercise}
%	Show that $[K:F] \ge \Aut(K/F)$,
%	and that equality holds if and only if
%	the set of elements fixed by all $\sigma \in \Aut(K/F)$
%	is exactly $F$.
%	(Use \Cref{thm:fixed_field_theorem}.)
%\end{exercise}
%The equality case is trickier.
%
%The easier direction is when $K$ is a splitting field.
%Assume $K = F(\alpha_1, \dots, \alpha_n)$ is the splitting field of some separable polynomial $p \in F[x]$
%with $n$ distinct roots $\alpha_1, \dots, \alpha_n$ be the distinct roots.
%Adjoin them one by one:
%\begin{diagram}
%	F & \rInj & F(\alpha_1) & \rInj & F(\alpha_1, \alpha_2) & \rInj & \dots & \rInj & K \\
%	\dTo^\id && \dTo^{\tau_1} && \dTo^{\tau_2} && \dots && \dTo_{\tau_n = \sigma} \\
%	F & \rInj & F(\alpha_1) & \rInj & F(\alpha_1, \alpha_2) & \rInj & \dots & \rInj & K \\
%\end{diagram}
%(Does this diagram look familiar?)
%Every map $K \to K$ which fixes $F$ corresponds to an above commutative diagram.
%As before, there are exactly $[F(\alpha_1) : F]$ ways to pick $\tau_1$.
%(You need the fact that the minimal polynomial $p_1$ of $\alpha_1$ is separable for this:
%there needs to be exactly $\deg p_1 = [F(\alpha_1) : F]$ distinct roots to nail $p_1$ into.)
%Similarly, given a choice of $\tau_1$, there are $[F(\alpha_1, \alpha_2) : F(\alpha_1)]$ ways to pick $\tau_2$.
%Multiplying these all together gives the desired $[K:F]$.
%
%\bigskip
%
%Now assume $K/F$ is Galois.
%First, we state the following lemma.
%\begin{lemma}
%	Let $K/F$ be Galois, and $p \in F[x]$ irreducible.
%	If any root of $p$ (in $\ol F$) lies in $K$, then all of them do,
%	and in fact $p$ is separable.
%\end{lemma}
%\begin{proof}
%	Let $\alpha \in K$ be the prescribed root.
%	Consider the set
%	\[ S = \left\{ \sigma(\alpha) \mid \sigma \in \Gal(K/F) \right\}. \]
%	(Note that $\alpha \in S$ since $\Gal(K/F) \ni \id$.)
%	By construction, any $\tau \in \Gal(K/F)$ fixes $S$.
%	So if we construct
%	\[ \tilde p(x) = \prod_{\beta \in S} (x - \beta). \]
%	then by Vieta's Formulas, we find that all the coefficients of $\tilde p$ are fixed by elements of $\sigma$.
%	By the \emph{equality case} we specified in the exercise, it follows that $\tilde p$ has coefficients in $F$!
%	(This is where we use the condition.)
%	Also, by \Cref{lem:root_shuffle}, $\tilde p$ divides $p$.
%
%	Yet $p$ was irreducible, so it is the minimal polynomial of $\alpha$ in $F[x]$,
%	and therefore we must have that $p$ divides $\tilde p$.
%	Hence $p = \tilde p$. Since $\tilde p$ was built to be separable, so is $p$.
%\end{proof}
%Now we're basically done -- pick a basis $\omega_1$, \dots, $\omega_n$ of $K/F$,
%and let $p_i$ be their minimal polynomials; by the above, we don't get any roots outside $K$.
%Consider $P = p_1 \dots p_n$, removing any repeated factors.
%The roots of $P$ are $\omega_1$, \dots, $\omega_n$ and some other guys in $K$.
%So $K$ is the splitting field of $P$.


\section{More Examples of Galois Groups}

\todo{Actually write this section}

$\sqrt2+\sqrt3$

$x^3-2$

$x^p-2$

\section\problemhead
\begin{sproblem}[Galois Group of the Cyclotomic Field]
	Let $p$ be an odd rational prime and $\zeta_p$ a primitive $p$th root of unity.
	Let $K = \QQ(\zeta_p)$.
	Show that \[ \Gal(K/\QQ) \simeq (\ZZ/p\ZZ)^\ast. \]
	\begin{hint}
		Look at the image of $\zeta_p$.
	\end{hint}
	\begin{sol}
		It's just $\Zc{p-1}$, since $\zeta_p$ needs to get sent
		to one (any) of the $p-1$ primitive roots of unity.
	\end{sol}
\end{sproblem}

\begin{problem}[Greek Constructions]
	Prove that the three Greek constructions
	\begin{enumerate}[(a)]
		\ii doubling the cube,
		\ii squaring the circle, and
		\ii trisecting an angle
	\end{enumerate}
	are all impossible.
	(Assume $\pi$ is transcendental.)
	\begin{hint}
		Repeated quadratic extensions have degree $2$, so one can
		only get powers of two.
	\end{hint}
\end{problem}

\begin{problem}
	Show that the only automorphism of $\RR$ is the identity.
	Hence $\Aut(\RR)$ is the trivial group.
	\begin{hint}
		Hint: $\sigma(x^2) = \sigma(x)^2 \ge 0$ plus Cauchy's Functional Equation.
	\end{hint}
\end{problem}

\begin{problem}[Artin's Primitive Element Theorem]
	\yod
	Let $K$ be a number field.
	Show that $K \cong \QQ(\alpha)$ for some $\alpha$.
	\todo{add solution} % \url{http://www.math.cornell.edu/~kbrown/6310/primitive.pdf}}
	\label{prob:artin_primitive_elm}
\end{problem}
