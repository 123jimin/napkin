\chapter{Glossary of notations}
\todo{update this}

\section{General}
\begin{itemize}
	\ii $\forall$: for all
	\ii $\exists$: there exists
	\ii $\sign(\sigma)$: sign of permutation $\sigma$
	\ii $X \implies Y$: $X$ implies $Y$
\end{itemize}
\section{Functions and sets}
\begin{itemize}
	\ii $f\im(S)$ is the image of $f : X \to Y$ for $S \subseteq X$.
	\ii $f\inv(y)$ is the inverse for $f : X \to Y$ when $y \in Y$.
	\ii $f\pre(T)$ is the pre-image for $f : X \to Y$ when $T \subseteq Y$.
	\ii $f \restrict{S}$ is the restriction of $f : X \to Y$ to $S \subseteq X$.
	\ii $f^n$ is the function $f$ applied $n$ times
\end{itemize}

Below are some common sets.
These may also be thought of as groups,
rings, fields etc.\ in the obvious way.
\begin{itemize}
	\ii $\CC$: set of complex numbers
	\ii $\RR$: set of real numbers
	\ii $\NN$: set of positive integers
	\ii $\QQ$: set of rational numbers
	\ii $\ZZ$: set of integers
	\ii $\varnothing$: empty set
\end{itemize}

Some common notation with sets:
\begin{itemize}
	\ii $A \subset B$: $A$ is any subset of $B$
	\ii $A \subseteq B$: $A$ is any subset of $B$
	\ii $A \subseteq B$: $A$ is a \emph{proper} subset of $B$
	\ii $S \times T$: Cartesian product of sets $S$ and $T$
	\ii $S \setminus T$: difference of sets $S$ and $T$
	\ii $S \cup T$: set union of $S$ and $T$
	\ii $S \cap T$: set intersection of $S$ and $T$
	\ii $S \sqcup T$: disjoint union of $S$ and $T$
	\ii $\left\lvert S \right\rvert$: cardinality of $S$
	\ii $S / {\sim}$: if $\sim$ is an equivalence relation on $S$,
	this is the set of equivalence classes
	\ii $x + S$: denotes the set $\{x+s \mid s \in S\}$.
	\ii $xS$: denotes the set $\{xs \mid s \in S\}$.
\end{itemize}

\section{Abstract and linear algebra}
Some common groups/rings/fields:
\begin{itemize}
	\ii $\Zc n$: cyclic group of order $n$
	\ii $\Zm n$: set of units of $\Zc n$.
	\ii $S_n$: symmetric group on $\{1, \dots, n\}$
	\ii $D_{2n}$: dihedral group of order $2n$.
	\ii $0$, $1$: trivial group (depending on context)
	\ii $\FF_p$: integers modulo $p$
\end{itemize}
Notation with groups:
\begin{itemize}
	\ii $1_G$: identity element of the group $G$
	\ii $N \normalin G$: subgroup $N$ is normal in $G$.
	\ii $G/N$: quotient group of $G$ by the normal subgroup $N$
	\ii $Z(G)$: center of group $G$
	\ii $N_G(H)$: normalizer of the subgroup $H$ of $G$
	\ii $G \times H$: product group of $G$ and $H$
	\ii $G \oplus H$: also product group,
	but often used when $G$ and $H$ are abelian
	(and hence we can think of them as $\ZZ$-modules)
	\ii $\Stab_G(x)$: the stabilizer of $x \in X$, if $X$ is acted on by $G$
	\ii $\FixPt g$, the set of fixed points by $g \in G$ (under a group action)
\end{itemize}
Notation with rings:
\begin{itemize}
	\ii $R/I$: quotient of ring $R$ by ideal $I$
	\ii $(a_1, \dots, a_n)$: ideal generated by the $a_i$
	\ii $R^\times$: the group of units of $R$
	\ii $R[x_1, \dots, x_n]$: polynomial ring in $x_i$,
	or ring obtained by adjoining the $x_i$ to $R$
	\ii $F(x_1, \dots, x_n)$: field obtained by adjoining $x_i$ to $F$
	\ii $R^d$: $d$th graded part of a graded (pseudo)ring $R$
\end{itemize}
Linear algebra:
\begin{itemize}
	\ii $V \oplus W$: direct sum
	\ii $V^{\oplus n}$: direct sum of $V$, $n$ times
	\ii $V \otimes W$: tensor product
	\ii $V^{\otimes n}$: tensor product of $V$, $n$ times
	\ii $V^\vee$: dual space
	\ii $T^\vee$: dual map (for $T$ a vector space)
	\ii $T^\dagger$: conjugate transpose (for $T$ a vector space)
	\ii $\left< -,-\right>$: a bilinear form
	\ii $\Mat(V)$: endomorphisms of $V$, i.e.\ $\Hom_k(V,V)$
	\ii $\ee_1$, \dots, $\ee_n$: the ``standard basis'' of $k^{\oplus n}$
\end{itemize}
\section{Quantum computation}
\begin{itemize}
	\ii $\ket{\psi}$: a vector in some vector space $H$
	\ii $\bra{\psi}$: a vector in some vector space $H^\vee$, dual to $\ket{\psi}$.
	\ii $\braket{\phi|\psi}$: evaluation of an element $\bra{\phi} \in H^\vee$ at $\ket{\phi} \in H$.
	\ii $\zup$, $\zdown$: spin $z$-up, spin $z$-down
	\ii $\xup$, $\xdown$: spin $x$-up, spin $x$-down
	\ii $\yup$, $\ydown$: spin $y$-up, spin $y$-down
\end{itemize}

\section{Topology and (complex) analysis}
Common topological spaces:
\begin{itemize}
	\ii $S^1$: the unit circle
	\ii $S^n$: surface of an $n$-sphere (in $\RR^{n+1}$)
	\ii $D^{n+1}$: closed $n+1$ dimensional ball (in $\RR^{n+1}$)
	\ii $\RP^n$: real projective $n$-space
	\ii $\CP^n$: complex projective $n$-space
\end{itemize}
Some topological notation:
\begin{itemize}
	\ii $\partial Y$: boundary of a set $Y$ (in some topological space)
	\ii $X/S$: quotient topology of $X$ by $S \subseteq X$
	\ii $X \times Y$: product topology of spaces $X$ and $Y$
	\ii $X \amalg Y$: disjoint union of spaces $X$ and $Y$
	\ii $X \vee Y$: wedge product of (pointed) spaces $X$ and $Y$
\end{itemize}
Complex analysis:
\begin{itemize}
	\ii $\int_\alpha f \; dz$: contour integral of $f$ along path $\alpha$
	\ii $\Res(f;p)$: the residue of a meromorphic function $f$ at point $p$
	\ii $\Wind(\gamma, p)$: winding number of $\gamma$ around $p$.
\end{itemize}
Algebraic topology:
\begin{itemize}
	\ii $\alpha \simeq \beta$: for paths, this indicates path homotopy
	\ii $\ast$: path concatenation
	\ii $\pi_1(X) = \pi_1(X, x_0)$: the fundamental group of (pointed) space $X$
	\ii $\pi_n(X) = \pi_n(X, x_0)$: the $n$th homotopy group of (pointed) space $X$
	\ii $f_\sharp$: the induced map $\pi_1(X) \to \pi_1(Y)$ of $f : X \to Y$
	\ii $\Delta^n$: the standard $n$-simplex
	\ii $\partial\sigma$: the boundary of a singular $n$-simplex $\sigma$
	\ii $H_n(A_\bullet)$: the $n$th homology group of the chain complex $A_\bullet$
	\ii $H_n(X)$: the $n$th homology group of a space $X$
	\ii $\wt H_n(X)$: the $n$th reduced homology group of $X$
	\ii $H_n(X, A)$: the $n$th relative homology group of $X$ and $A \subseteq X$
	\ii $f_\ast$: the induced map on $H_n(A_\bullet) \to H_n(B_\bullet)$
	of $f : A_\bullet \to B_\bullet$,
	or $H_n(X) \to H_n(Y)$ for $f : X \to Y$
	\ii $\chi(X)$: Euler characteristic of a space $X$
	\ii $H^n(A^\bullet)$: the $n$th cohomology group of a cochain complex $A^\bullet$
	\ii $H^n(A_\bullet; G)$: the $n$th cohomology group of the cochain complex
	obtained by applying $\Hom(-,G)$ to $A_\bullet$
	\ii $H^n(X; G)$: the $n$th cohomology group/ring of $X$ with $G$-coefficients
	\ii $\wt H^n(X; G)$: the $n$th reduced cohomology group/ring of $X$ with $G$-coefficients
	\ii $H^n(X,A ; G)$: the $n$th relative cohomology group/ring of $X$ and $A \subset X$ with $G$-coefficients
	\ii $f^\sharp$: the induced map on $H^n(A^\bullet) \to H^n(B^\bullet)$
	of $f : A^\bullet \to B^\bullet$,
	or $H^n(X) \to H^n(Y)$ for $f : X \to Y$
	\ii $\Ext(-,-)$: the Ext functor
	\ii $\phi \smile \psi$: cup product of cochains $\phi$ and $\psi$
\end{itemize}

\section{Category theory}
Some common categories (in alphabetical order):
\begin{itemize}
	\ii $\catname{Grp}$: category of groups
	\ii $\catname{CRing}$: category of commutative rings
	\ii $\catname{Top}$: category of topological spaces
	\ii $\catname{Top}_\ast$: category of pointed topological spaces
	\ii $\catname{Vect}_k$: category of $k$-vector spaces
	\ii $\catname{FDVect}_k$: category of finite-dimensional vector spaces
	\ii $\catname{Set}$: category of sets
	\ii $\catname{hTop}$: category of topological spaces,
	whose morphisms are homotopy classes of maps
	\ii $\catname{hTop}_\ast$: pointed version of $\catname{hTop}$
	\ii $\catname{hPairTop}$: category of pairs $(X,A)$ with morphisms
	being pair-homotopy equivalence classes
	\ii $\Opens(X)$: the category of open sets of $X$, as a poset
\end{itemize}
Operations with categories:
\begin{itemize}
	\ii $\obj \AA$: objects of the category $\AA$
	\ii $\AA\op$: opposite category
	\ii $\AA \times \BB$: product category
	\ii $[\AA, \BB]$: category of functors from $\AA$ to $\BB$
	\ii $\ker f : \Ker f \to B$: for $f : A \to B$, categorical kernel
	\ii $\coker f : A \to \Coker f$: for $f : A \to B$, categorical cokernel
	\ii $\img f : A \to \Img f$: for $f : A \to B$, categorical image
\end{itemize}

\section{Differential geometry}
\begin{itemize}
	\ii $Df$: total derivative of $f$
	\ii $(Df)_p$: total derivate of $f$ at point $p$
	\ii $\fpartial{f}{e_i}$: $i^{\text{th}}$ partial derivative
	\ii $\alpha_p$: evaluating a $k$-form $\alpha$ at $p$
	\ii $\int_c \alpha$: integration of the differential form $\alpha$ over a cell $c$
	\ii $d\alpha$: exterior derivative of a $k$-form $\alpha$
	\ii $\phi^\ast \alpha$: pullback of $k$-form $\alpha$ by $\phi$
\end{itemize}

\section{Algebraic number theory}
\begin{itemize}
	\ii $\ol \QQ$: ring of algebraic numbers
	\ii $\ol \ZZ$: ring of algebraic integers
	\ii $\ol F$: algebraic closure of a field $F$
	\ii $\NK(\alpha)$: the norm of $\alpha$ in extension $K/\QQ$
	\ii $\TrK(\alpha)$: the trace of $\alpha$ in extension $K/\QQ$
	\ii $\OO_K$: ring of integers in $K$
	\ii $\ka+\kb$: sum of two ideals $\ka$ and $\kb$
	\ii $\ka\kb$: ideal generated by products of elements in ideals $\ka$ and $\kb$
	\ii $\ka \mid \kb$: ideal $\ka$ divides ideal $\kb$
	\ii $\ka\inv$: the inverse of $\ka$ in the ideal group
	\ii $\Norm(I)$: ideal norm
	\ii $\Cl_K$: class group of $K$
	\ii $\Delta_K$: discriminant of number field $K$
	\ii $\mu(\OO_K)$: set of roots of unity contained in $\OO_K$
	\ii $[K:F]$: degree of a field extension
	\ii $\Aut(K/F)$: set of field automorphisms of $K$ fixing $F$
	\ii $\Gal(K/F)$: Galois group of $K/F$
	\ii $D_\kp$: decomposition group of prime ideal $\kp$
	\ii $I_\kp$: inertia group of prime ideal $\kp$
	\ii $\Frob_\kp$: Frobenius element of $\kp$ (element of $\Gal(K/\QQ)$)
	\ii $P_K(\mm)$: ray of principal ideals of a modulus $\mm$
	\ii $I_K(\mm)$: fractional ideals of a modulus $\mm$
	\ii $C_K(\mm)$: ray class group of a modulus $\mm$
	\ii $\left( \frac{L/K}{\bullet} \right)$: the Artin symbol
	\ii $\Ram(L/K)$: primes of $K$ ramifying in $L$
	\ii $\kf(L/K)$: the conductor of $L/K$
\end{itemize}

\section{Representation theory}
\begin{itemize}
	\ii $k[G]$: group algebra
	\ii $V \oplus W$: direct sum of representations $V = (V, \rho_V)$
	and $W = (W, \rho_W)$ of an algebra $A$
	\ii $V^\vee$: dual representation of a representation $V = (V, \rho_V)$
	\ii $\Reg(A)$: regular representation of an algebra $A$
	\ii $\Homrep(V,W)$: algebra of morphisms $V \to W$ of representations
	\ii $\chi_V$: the character $A \to k$ attached to an $A$-representation $V$
	\ii $\Classes(G)$: set of conjugacy classes of $G$
	\ii $\FunCl(G)$: the complex vector space of functions $\Classes(G) \to \CC$
	\ii $V \otimes W$: tensor product of representations $V = (V, \rho_V)$ and $W = (W, \rho_W)$
	of a \emph{group} $G$ (rather than an algebra)
	\ii $\Ctriv$: the trivial representation
	\ii $\Csign$: the sign representation
\end{itemize}

\section{Algebraic geometry}
\begin{itemize}
	\ii $\VV(-)$: vanishing locus of a set or ideal
	\ii $\Aff^n$: $n$-dimensional (complex) affine space
	\ii $\sqrt I$: radical of an ideal $I$
	\ii $\CC[V]$: coordinate ring of an affine variety $V$
	\ii $\OO_V(U)$: ring of rational functions on $U$
	\ii $D(f)$: distinguished open set
	\ii $\CP^n$: complex projective $n$-space (ambient space for projective varieties)
	\ii $(x_0 : \dots : x_n)$: coordinates of projective space
	\ii $U_i$: standard affine charts
	\ii $\Vp(-)$: projective vanishing locus.
	\ii $h_I$, $h_V$: Hilbert function of an ideal $I$ or projective variety $V$
	\ii $f^\ast$: the pullback $\OO_Y \to \OO_X(f\pre(U))$ obtained from $f : X \to Y$
	\ii $\SF_p$: the stalk of a (pre-)sheaf $\SF$ at a point $p$
	\ii $[s]_p:$ the germ of $s \in \SF(U)$ at the point $p$
	\ii $\OO_{X,p}$: shorthand for $(\OO_X)_p$.
	\ii $\SF\sh$: sheafification of pre-sheaf $\SF$
	\ii $\alpha_p : \SF_p \to \SG_p$: morphism of stalks obtained from $\alpha : \SF \to \SG$
	\ii $\mm_{X,p}$: the maximal ideal of $\OO_{X,p}$
	\ii $\Spec R$: the spectrum of a ring $R$
	\ii $S\inv R$: localization of ring $R$ at a set $S$
	\ii $\Proj R$: the projective scheme of a graded ring $S$
\end{itemize}

\section{Set theory}
\begin{itemize}
	\ii $\ZFC$: standard theory of ZFC
	\ii $\ZFC^+$: standard theory of ZFC, plus the sentence
	``there exists a strongly inaccessible cardinal''
	\ii $2^S$ or $\PP(S)$: power set of $S$
	\ii $A \land B$: $A$ and $B$
	\ii $A \lor B$: $A$ or $B$
	\ii $\neg A$: not $A$
	\ii $V$: class of all sets (von Neumann universe)
	\ii $\omega$: the first infinite ordinal, also the set of nonnegative integers
	\ii $V_\alpha$: level of the von Neumann universe
	\ii $\On$: class of ordinals
	\ii $\bigcup A$: the union of elements inside $A$
	\ii $A \approx B$: sets $A$ and $B$ are equinumerous
	\ii $\aleph_\alpha$: the aleph numbers
	\ii $\cof \lambda$: the cofinality of $\lambda$
	\ii $\MM \vDash \phi[b_1, \dots, b_n]$: model $\MM$ satisfies sentence $\phi$
	with parameters $b_1$, \dots, $b_n$
	\ii $\Delta_n$, $\Sigma_n$, $\Pi_n$: levels of the Levy hierarchy
	\ii $\MM_1 \subseteq \MM_2$: $\MM_1$ is a substructure of $\MM_2$
	\ii $\MM_1 \prec \MM_2$: $\MM_1$ is an elementary substructure of $\MM_2$
	\ii $p \parallel q$: elements $p$ and $q$ of a poset $\Po$ are compatible
	\ii $p \perp q$: elements $p$ and $q$ of a poset $\Po$ are incompatible
	\ii $\Name_\alpha$: the hierarchy of $\Po$-names
	\ii $\tau^G$: interpretation of a name $\tau$ by filter $G$
	\ii $M[G]$: the model obtained from a forcing poset $G \subseteq \Po$
	\ii $p \Vdash \varphi(\sigma_1, \dots, \sigma_n)$: $p \in \Po$ forces the sentence $\varphi$
	\ii $\check x$: the name giving an $x \in M$ when interpreted
	\ii $\dot G$: the name giving $G$ when interpreted
\end{itemize}


% Consider adding:
% lim (convergence)
% group presentation
