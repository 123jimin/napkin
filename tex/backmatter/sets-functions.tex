\chapter{Terminology on sets and functions}
\label{ch:sets_functions}
\todo{this chapter still in progress}
This appendix will cover some notions on sets and functions
such as ``bijections'', ``equivalence classes'', and so on.

Remark for experts: I am not dealing with foundational issues in this chapter.
See \Cref{ch:zfc} (and onwards) if that's what you're interested in.
Consequently I will not prove most assertions.

\section{Sets}
A \vocab{set} for us will just be a collection of elements (whatever they may be).
For example, the set $\NN = \{1, 2, 3, 4, \dots\}$ is the positive integers,
and $\ZZ = \{ \dots, -2, -1, 0, 1, 2, \dots\}$ is the set of all integers.
As another example, we have a set of humans:
\[ H = \left\{ x \mid \text{$x$ is a featherless biped} \right\}. \]
(Here the ``$\mid$'' means ``such that''.)

There's also a set with no elements, which we call the \vocab{empty set}.
It's denoted by $\varnothing$.

It's conventional to use capital letters for sets (like $H$),
and lowercase letters for elements of sets (like $x$).

\begin{definition}
We write $x \in S$ to mean ``$x$ is in $S$'', for example $3 \in \NN$.
\end{definition}

\begin{definition}
	If every element of a set $A$ is also in a set $B$,
	then we say $A$ is a \vocab{subset} of $B$,
	and denote this by $A \subseteq B$.
	If moreover $A \neq B$, we say $A$ is a \vocab{proper subset}
	and write $A \subsetneq B$.
	(This is analogous to $\le$ and $<$.)

	Given a set $A$, the set of all subsets is denoted $2^A$
	or $\PP(A)$ and called the \vocab{power set} of $A$.
\end{definition}
\begin{example}
	[Examples of subsets]
	\listhack
	\begin{enumerate}[(a)]
		\ii $\{1,2,3\} \subseteq \NN \subseteq \ZZ$.
		\ii $\varnothing \subseteq A$ for any set $A$. (Why?)
		\ii $A \subseteq A$ for any set $A$.
		\ii If $A = \{1,2\}$ then $2^A =
		\left\{ \varnothing, \{1\}, \{2\}, \{1,2\} \right\}$.
	\end{enumerate}
\end{example}

\begin{definition}
	We write
	\begin{itemize}
		\ii $A \cup B$ for the set of elements in
		\emph{either} $A$ or $B$ (possibly both),
		called the \vocab{union} of $A$ and $B$.
		\ii $A \cap B$ for the set of elements in \emph{both} $A$ and $B$, and
		called the \vocab{intersection} of $A$ and $B$.
		\ii $A \setminus B$ for the set of elements in $A$ but \emph{not} in $B$.
	\end{itemize}
\end{definition}

\begin{example}
	[Examples of set operations]
	Let $A = \{1,2,3\}$ and $B = \{3,4,5\}$. Then
	\begin{align*}
		A \cup B &= \{1,2,3,4,5\} \\
		A \cap B &= \{3\} \\
		A \setminus B &= \{1,2\}.
	\end{align*}
\end{example}

\begin{exercise}
	Convince yourself: for any sets $A$ and $B$,
	we have $A \cap B \subseteq A \subseteq A \cup B$.
\end{exercise}

Here are some commonly recurring sets:
\begin{itemize}
	\ii $\CC$ is the set of complex numbers, like $3.2 + \sqrt 2 i$.
	\ii $\RR$ is the set of real numbers, like $\sqrt 2$ or $\pi$.
	\ii $\NN$ is the set of positive integers, like $5$ or $9$.
	\ii $\QQ$ is the set of rational numbers, like $7/3$.
	\ii $\ZZ$ is the set of integers, like $-2$ or $8$.
\end{itemize}
(These are pronounced in the way you would expect:
``see'', ``are'', ``en'', ``cue'', ``zed''.)

\section{Functions}
Given two sets $A$ and $B$, a \vocab{function} $f$ from $A$ to $B$
is a mapping of every element of $A$ to some element of $B$.

We call $A$ the \vocab{domain} of $f$, and $B$ the \vocab{codomain}.
We write this as $f \colon A \to B$ or $A \taking f B$.
\begin{abuse}
	If the name $f$ is not important, we will often just write $A \to B$.
\end{abuse}
We write $f(a) = b$ or $a \mapsto b$ to signal that $f$ takes $a$ to $b$.

If $B$ has $0$ as an element and $f(a) = 0$,
we often say $a$ is a \vocab{root} or \vocab{zero} of $f$,
and that $f$ \vocab{vanishes} at $a$.

\subsection{Injective / surjective / bijective functions}

\begin{definition}
	A function $f \colon A \to B$ is \vocab{injective}
	if it is ``one-to-one'' in the following sense:
	if $f(a) = f(a')$ then $a = a'$.
	In other words, for any $b \in B$,
	there is \emph{at most} one $a \in A$ such that $f(a) = b$.

	Often, we will write $f \colon A \injto B$ to emphasize this.
\end{definition}
\begin{definition}
	A function $f \colon A \to B$ is \vocab{surjective}
	if it is ``onto'' in the following sense:
	for any $b \in B$ there is \emph{at least} one $a \in A$
	such that $f(a) = b$.

	Often, we will write $f \colon A \surjto B$ to emphasize this.
\end{definition}
\begin{definition}
	A function $f \colon A \to B$ is \vocab{bijective}
	if it is both injective and surjective.
	In other words, for each $b \in B$,
	there is \emph{exactly} one $a \in A$ such that $f(a) = b$.
\end{definition}

\begin{example}
	[Examples of functions]
	By ``human'' I mean ``living featherless biped''.
	\begin{enumerate}[(a)]
		\ii There's a function taking every human to their
		age in years (rounded to the nearest integer).
		This function is \textbf{not injective},
		because for example there are many people with age $20$.
		This function is also \textbf{not surjective}: no one has age $10000$.

		\ii There's a function taking every
		USA citizen to their social security number.
		This is also \textbf{not surjective} (no one has SSN equal to $3$),
		but at least it \textbf{is injective} (no two people have the same SSN).
	\end{enumerate}
\end{example}

\begin{example}
	[Examples of bijections]
	\listhack
	\begin{enumerate}[(a)]
		\ii Let $A = \{1,2,3,4,5\}$ and $B = \{6,7,8,9,10\}$.
		Then the function $f \colon A \to B$ by $a \mapsto a+5$ is a bijection.
		\ii In a classroom with $30$ seats,
		there is exactly one student in every seat.
		Thus the function taking each student to the seat they're in
		is a bijection; in particular, there are exactly $30$ students.
	\end{enumerate}
\end{example}

\begin{remark}
	Assume for convenience that $A$ and $B$ are finite sets.
	Then:
	\begin{itemize}
		\ii If $f \colon A \injto B$ is injective,
		then the size of $A$ is at most the size of $B$.
		\ii If $f \colon A \surjto B$ is surjective,
		then the size of $A$ is at least the size of $B$.
		\ii If $f \colon A \to B$ is a bijection,
		then the size of $A$ equals the size of $B$.
	\end{itemize}
\end{remark}

Now, notice that if $f \colon A \to B$ is a bijection,
then we can ``apply $f$ backwards'':
(for example, rather than mapping each student to the seat they're in,
we map each seat to the student sitting in it).
This is called an \vocab{inverse function};
we denote it $f\inv \colon B \to A$.

\subsection{Images and pre-images}
Let $X \taking f Y$ be a function.

\begin{definition}
	Suppose $T \subseteq Y$.
	The \vocab{pre-image} $f\pre(T)$ is the set of all
	$x \in X$ such that $f(x) \in T$.
	Thus, $f\pre(T)$ is a subset of $X$.
\end{definition}
\begin{example}
	[Examples of pre-image]
	Let $f \colon H \to \ZZ$ be the age function from earlier.
	Then
	\begin{enumerate}[(a)]
		\ii $f\pre(\{13, 14, 15, 16, 17, 18, 19\})$ is the set of teenagers.
		\ii $f\pre(\{0\})$ is the set of newborns.
		\ii $f\pre(\{1000, 1001, 1002, \dots \}) = \varnothing$,
		as I don't think anyone is that old.
	\end{enumerate}
\end{example}

\begin{abuse}
	By abuse of notation, we may abbreviate $f\pre(\{y\})$ to $f\pre(y)$.
	So for example, $f\pre(\{0\})$ above becomes shortened to $f\pre(0)$.
\end{abuse}

The dual notion is:
\begin{definition}
	Suppose $S \subseteq X$.
	The \vocab{image} $f\im(S)$ is the set of
	all things of the form $f(s)$.
\end{definition}
\begin{example}
	[Examples of images]
	Let $A = \{1,2,3,4,5\}$ and $B = \ZZ$.
	Consider a function $f \colon A \to B$ given by
	\[
		f(1) = 17 \quad
		f(2) = 17 \quad
		f(3) = 19 \quad
		f(4) = 30 \quad
		f(5) = 234.
	\]
	\begin{enumerate}[(a)]
		\ii The image $f\im(\{1,2,3\})$ is the set $\{17, 19\}$.
		\ii The image $f\im(A)$ is the set $\{17, 19, 30, 234\}$.
	\end{enumerate}
\end{example}
\begin{ques}
	Suppose $f \colon A \surjto B$ is surjective.
	What is $f\im(A)$?
\end{ques}

\section{Equivalence relations}
Let $X$ be a fixed set now.
A binary relation $\sim$ on $X$ assigns a truth value ``true''
or ``false'' to $x \sim y$ for each $x$ or $y$.
Now an \vocab{equivalence relation} $\sim$ on $X$ is a binary relation
which satisfies the following axioms:
\begin{itemize}
	\ii Reflexive: we have $x \sim x$.
	\ii Symmetric: if $x \sim y$ then $y \sim x$
	\ii Transitive: if $x \sim y$ and $y \sim z$ then $x \sim z$.
\end{itemize}
An \vocab{equivalence class} is then a
set of all things equivalent to each other.
One can show that $X$ becomes partitioned by these equivalence classes:

\begin{example}
	[Example of an equivalence relation]
	Let $\NN$ denote the set of positive integers.
	Then suppose we declare $a \sim b$ if $a$ and $b$ have the same last digit,
	for example $131 \sim 211$, $45 \sim 125$, and so on.

	Then $\sim$ is an equivalence relation.
	It partitions $\NN$ into ten equivalence classes,
	one for each trailing digit.
\end{example}

Often, the set of equivalence classes will be denoted $X/{\sim}$
(pronounced ``$X$ mod sim'').
